\documentclass[11pt]{article}

\usepackage{amsfonts,amssymb,amsmath,amsthm,latexsym,graphics,epsfig,amsfonts}
\usepackage{verbatim,enumerate,array,booktabs,color,bigstrut,prettyref,tikz-cd}
\usepackage{multirow}
\usepackage[all]{xy}
\usepackage[backref]{hyperref}
\usepackage[OT2,T1]{fontenc}
%\usepackage{ctable}
\usepackage{mathtools}

\usepackage{longtable}


\usepackage{mathtools}
\newcommand{\Mod}[1]{\ (\mathrm{mod}\ #1)}
\newcommand{\mathdash}{\relbar\mkern-8mu\relbar}
\newcommand*\circled[2][1.6]{\tikz[baseline=(char.base)]{
    \node[shape=circle, draw, inner sep=1pt, 
        minimum height={\f@size*#1},] (char) {\vphantom{WAH1g}#2};}}
\makeatother



\usepackage{tabularray}
\UseTblrLibrary{amsmath,varwidth}

\usepackage{tabularx}
\usepackage{longtable}
\usepackage{arydshln}


\newcommand\myiso{\stackrel{\mathclap{\normalfont\mbox{\small $p$}}}{-}}
\newcommand\myisot{\stackrel{\mathclap{\normalfont\mbox{\small $3$}}}{-}}

\newcommand{\pref}[1]{\prettyref{#1}}
\newrefformat{eq}{\textup{(\ref{#1})}}
\newrefformat{prty}{\textup{(\ref{#1})}}

\definecolor{mylinkcolor}{rgb}{0.8,0,0}
\definecolor{myurlcolor}{rgb}{0,0,0.8}
\definecolor{mycitecolor}{rgb}{0,0,0.8}
\hypersetup{colorlinks=true,urlcolor=myurlcolor,citecolor=mycitecolor,linkcolor=mylinkcolor,linktoc=page,breaklinks=true}

%\DeclareSymbolFont{cyrletters}{OT2}{wncyr}{m}{n}
%\DeclareMathSymbol{\Sha}{\mathalpha}{cyrletters}{"58}

\addtolength{\textwidth}{4cm} \addtolength{\hoffset}{-2cm}
\addtolength{\marginparwidth}{-2cm}

%\theoremstyle{definition}
\newtheorem{defn}{Definition}[section]
\newtheorem{definition}[defn]{Definition}
\newtheorem{claim}[defn]{Claim}

%\theoremstyle{plain}
\newtheorem{thmA}{Theorem A}
\newtheorem{thmB}{Theorem B}
\newtheorem{thm2}{Theorem}
\newtheorem{prop2}{Proposition}
\newtheorem{note}{Note}

\newtheorem{corollary}[defn]{Corollary}
\newtheorem{lemma}[defn]{Lemma}
\newtheorem{property}[defn]{Property}
\newtheorem{thm}[defn]{Theorem}
\newtheorem{theorem}[defn]{Theorem}
\newtheorem{cor}[defn]{Corollary}
\newtheorem{prop}[defn]{Proposition}
\newtheorem{proposition}[defn]{Proposition}
\newtheorem{thmnn}{Theorem}
\newtheorem{conj}[defn]{Conjecture}

\theoremstyle{definition}
\newtheorem{remarks}{Remarks}
\newtheorem{ack}{Acknowledgements}
\newtheorem{remark}[defn]{Remark}
\newtheorem{question}[defn]{Question}
\newtheorem{example}[defn]{Example}


\newcommand{\Q}{\mathbb Q}
\newcommand{\Qbar}{\overline{\Q}}
\newcommand{\Z}{\mathbb Z}

\newcommand{\modQ}{\,\text{mod}\,(\Q^)^2}

\newcommand{\mysquare}[1]{\tikz{\path[draw] (0,0) rectangle node{\tiny #1} (8pt,8pt) ;}}
\newcommand{\mycircle}[1]{\tikz{\path[draw] (0,0) circle (4pt) node{\tiny #1};}}


%------------------------------------
\newcommand{\Kd}{\operatorname{K}}
\newcommand{\kI}{\operatorname{I}}
\newcommand{\kII}{\operatorname{II}}
\newcommand{\kIII}{\operatorname{III}}
\newcommand{\kIV}{\operatorname{IV}}
%-------------------------------------


\begin{document}
\title{Type $R_4(10)$}
\maketitle



\section{Setting}
The isogeny graphs of type $R_4(10)$ are given by
four isogenous elliptic curves:

\[ \begin{tikzcd}
E_1 \arrow[dash]{r}{5} 
    \arrow[dash]{d}{2} & 
    E_5  \arrow[dash]{d}{2} \\
 E_2 \arrow[dash]{r}{5} & E_{10}   \,.
\end{tikzcd}
\]

\noindent A hauptmodul for $X_0(10)$ is  
$$
t=4 + 2^2 5 \displaystyle{\frac{(\eta(2\tau) \eta(10\tau)^3)}{\eta(\tau)^3 \eta(5\tau)}}\,.
$$
One has
$$
\begin{tblr}{l@{\,=\,}l}
j(E_1) = j(\tau) & 
\displaystyle{\frac{\left(t^6-4 t^5+16 t+16\right)^3}{(t-4) t^5 (t+1)^2}}\,\\
j(E_2) = j(2\tau) & 
\displaystyle{\frac{\left(t^6-4 t^5+256 t+256\right)^3}{(t-4)^2 t^{10} (t+1)}}\,
\\
j(E_{5}) = j(5\tau) & 
\displaystyle{\frac{\left(t^6-4 t^5+240 t^4-480 t^3+1440 t^2-944 t+16\right)^3}{(t-4)^5 t (t+1)^{10}}}\,\\
j(E_{10}) = j(10\tau) & 
\displaystyle{\frac{\left(t^6+236 t^5+1440 t^4+1920 t^3+3840 t^2+256 t+256\right)^3}{(t-4)^{10} t^2 (t+1)^5}}\,
\end{tblr}
$$
and the Fricke involutions of $X_0(10)$ are given by:
$$
W_{10}(t)=4(t + 1)/(t - 4) \,,\qquad 
W_5(t)=(-t + 4)/(t + 1)\,,\qquad
W_2(t)= -4/t\,.
$$


%===== R4(10) =====
%w=(4*t + 4)/(t - 4) --(ord(w)=2)--> d=-10 --> [ 4, 3, 2, 1 ]
%w=(-t + 4)/(t + 1) --(ord(w)=2)--> d=-5 --> [ 3, 4, 1, 2 ]
%w=-4/t --(ord(w)=2)--> d=2 --> [ 2, 1, 4, 3 ]

%For $t$ in $\Q\setminus \{0,-8,-9\}$, the $p$-adic valuations of the Fricke involutions applied to $t$ are:
%
%\begin{tblr}
%{cells={mode=imath},colspec=|c|c|c|c|}
%\hline
%p & v_p(W_2(t)) & v_p(W_3(t)) & v_p(W_6(t))  \\
%\hline
%\neq 2,3 & 0 & 0 & - v_p(t) \\
%\hline
% 3  & v_3(t+9)-v_3(t+8) & 2 + v_3(t+8)-v_3(t+9) & 2-v_3(t) \\
%\hline
% 2  & 3+v_2(t+9)+v_2(t+8) & v_2(t+8)-v_2(t+9) & 3-v_2(t) \\
%\hline
%\end{tblr}

\vskip 0.5truecm

We can (and do) choose Weierstrass equations for $(E_1,E_2,E_{5},E_{10})$ such that their signatures are:


\vskip 0.3truecm

\[
\begin{tblr}{|c|l|}
\hline \SetCell[c=2]{c} R_4(10) \\ \hline
 c_4(E_1) & (t^2 + 4)(t^6 - 4t^5 + 16t + 16)\\
 c_6(E_1) & (t^2 - 2t - 4)(t^2 - 2t + 2)(t^2 + 4)^{2}(t^4 - 2t^3 - 6t^2 - 8t - 4)\\
 \Delta(E_1) & (t - 4)t^{5}(t + 1)^{2}(t^2 + 4)^{3}\\ \hline
 c_4(E_2) & (t^2 + 4)(t^6 - 4t^5 + 256t + 256)\\
 c_6(E_2) & (t^2 - 2t - 4)(t^2 + 4)^{2}(t^2 + 4t + 8)(t^4 - 8t^3 + 24t^2 - 32t - 64)\\
 \Delta(E_2) & (t - 4)^{2}t^{10}(t + 1)(t^2 + 4)^{3}\\ \hline
 c_4(E_5) & (t^2 + 4)(t^6 - 4t^5 + 240t^4 - 480t^3 + 1440t^2 - 944t + 16)\\
 c_6(E_5) & (t^2 - 2t + 2)(t^2 + 4)^{2}(t^2 + 22t - 4)(t^4 - 26t^3 + 66t^2 - 536t - 4)\\
 \Delta(E_5) & (t - 4)^{5}t(t + 1)^{10}(t^2 + 4)^{3}\\ 
 \hline
 c_4(E_{10}) & (t^2 + 4)(t^6 + 236t^5 + 1440t^4 + 1920t^3 + 3840t^2 + 256t + 256)\\
 c_6(E_{10}) & (t^2 + 4)^{2}(t^2 + 4t + 8)(t^2 + 22t - 4)(t^4 - 536t^3 - 264t^2 - 416t - 64)\\
 \Delta(E_{10}) & (t - 4)^{10}t^{2}(t + 1)^{5}(t^2 + 4)^{3}\\ \hline
\end{tblr}
\]

\vskip 0.3truecm

\noindent and, with this choice, the isogeny graph is normalized.

%\pagebreak

With regard to the action of the Fricke involutions 
on the isogeny graph, 
it can be displayed as follows:

%\begin{comment}
%$$
%\begin{array}{llll}
%    E_1\vert\operatorname{Id}=E_1 & E_1\vert \operatorname{W_2}=E_2 &
%     E_1\vert \operatorname{W_3}=E_{5}^{-3} &  E_1\vert \operatorname{W_6}=E_{10}^{-3} \\
%    E_2\vert \operatorname{Id}=E_2 & E_2\vert \operatorname{W_2}=E_1 &
%     E_2\vert \operatorname{W_3}=E_{10}^{-3} &  E_2\vert \operatorname{W_6}=E_{5}^{-3} \\
%    E_{5}\vert \operatorname{Id}=E_{5} & E_{5}\vert \operatorname{W_2}=E_{10} &
%     E_{5}\vert \operatorname{W_3}=E_1^{-3} &  E_{5}\vert \operatorname{W_6}=E_2^{-3} \\
%    E_{10}\vert \operatorname{Id}=E_{10} & E_{10}\vert \operatorname{W_2}=E_{5} &
%     E_{10}\vert \operatorname{W_3}=E_2^{-3} &  E_{10}\vert \operatorname{W_6}=E_1^{-3} \\
%\end{array}
%$$
%\end{comment}
%
\[ 
\begin{tikzcd}
E_1 \arrow[dash]{r}{5} 
    \arrow[dash]{d}{2} & 
    E_{5}  \arrow[dash]{d}{2} \\
 E_2 \arrow[dash]{r}{5} & E_{10}   
\end{tikzcd}
\phantom{\colon W_2}
\hskip 1truecm
\begin{tikzcd}
E_2^2 \arrow[dash]{r}{5} 
    \arrow[dash]{d}{2} & 
    E_{10}^2  \arrow[dash]{d}{2} \\
 E_1^2 \arrow[dash]{r}{5} & E_{5}^2   
\end{tikzcd}
\colon W_2
\]

\[ 
\begin{tikzcd}
E_{5}^{-5} \arrow[dash]{r}{5} 
    \arrow[dash]{d}{2} & 
    E_1^{-5}  \arrow[dash]{d}{2} \\
 E_{10}^{-5} \arrow[dash]{r}{5} & E_2^{-5}   
\end{tikzcd} \colon W_5
\hskip 1truecm
\begin{tikzcd}
E_{10}^{-10} \arrow[dash]{r}{5} 
    \arrow[dash]{d}{2} & 
    E_2^{-10}  \arrow[dash]{d}{2} \\
 E_{5}^{-10} \arrow[dash]{r}{5} & E_1^{-10} 
\end{tikzcd}
\colon W_{10}
\]
where the  
arrows correspond to the dual isogenies.
%===== R4(10) =====
%w=(4*t + 4)/(t - 4) --(ord(w)=2)--> d=-10 --> [ 4, 3, 2, 1 ]
%w=(-t + 4)/(t + 1) --(ord(w)=2)--> d=-5 --> [ 3, 4, 1, 2 ]
%w=-4/t --(ord(w)=2)--> d=2 --> [ 2, 1, 4, 3 ]


\newpage

\section{Kodaira symbols \& Pal coefficients}

\begin{longtblr}
[caption= {$R_4(10)$ data for $p\ne 2,3,5$}]
{cells={mode=imath},hlines,vlines,measure=vbox,
hline{Z}={1-X}{0pt},
vline{1}={Y-Z}{0pt},
colspec=cclclcc}
\SetCell[c=1]{c} R_4(10) &\SetCell[c=6]{c} p\ne 2,3,5  &    & \\
\SetCell[c=1]{c} t & E & 
\SetCell[c=1]{c} \operatorname{sig}_p(E) & u & \Kd_p(E) & \SetCell[c=2]{c} u_p(d)\\
%-------------------------------------------------
\SetCell[r=4]{c} 
m= v_p(t)>0 
& E_1 & (0,0,5m) & 1 & \kI_{5m} & 1 & 1 \\
& E_2 & (0,0,10m) & 1 &\kI_{10m} & 1 & 1\\
& E_{5} & (0,0,m) & 1 &\kI_{m}  & 1& 1\\
& E_{10} & (0,0,2m) & 1 &\kI_{2m}  & 1& 1\\
%-------------------------------------------------
\SetCell[r=4]{c} 
\begin{array}{c}
     v_p(t)=0  \\[3pt]
    m = v_p(t+1)>0 
\end{array}
& E_1 & (0,0,2m) & 1 &\kI_{2m}  & 1& 1 \\
& E_2 & (0,0,m) & 1 &\kI_{m} & 1& 1\\
& E_{5} & (0,0,10m) & 1 &\kI_{10m} & 1& 1\\
& E_{10} & (0,0,5m) & 1 &\kI_{5m}  & 1& 1\\
%-------------------------------------------------
\SetCell[r=4]{c} 
\begin{array}{c}
     v_p(t)=0  \\[3pt]
    m = v_p(t-4)>0
\end{array}
& E_1 & (0,0,m) & 1 &\kI_{m}   & 1& 1\\
& E_2 & (0,0,2m) & 1 &\kI_{2m}  & 1& 1\\
& E_{5} & (0,0,5m) & 1 &\kI_{5m} & 1& 1\\
& E_{10} & (0,0,10m) & 1 &\kI_{10m}  & 1& 1\\
%-------------------------------------------------
\SetCell[r=4]{c} 
\begin{array}{c}
v_p(t)=0  \\[6pt]
v_p(t^2 + 4)=4k 
\end{array}
& E_{1} & (0,2k,0) & p^k & \kI_{0} & 1 & 1\\
& E_{2} & (0,2k,0) & p^k & \kI_{0} & 1 & 1\\
& E_{5} & (0,2k,0) & p^k & \kI_{0} & 1 & 1\\
& E_{10} & (0,2k,0) & p^k & \kI_{0} & 1 & 1\\
%-------------------------------------------------
\SetCell[r=4]{c} 
\begin{array}{c}
v_p(t)=0  \\[6pt]
v_p(t^2 + 4)=4k +1
\end{array}
& E_{1} & (1,2+2k,3) & p^k & \kIII & 1 & 1\\
& E_{2} & (1,2+2k,3) & p^k & \kIII & 1 & 1\\
& E_{5} & (1,2+2k,3) & p^k & \kIII & 1 & 1\\
& E_{10} & (1,2+2k,3) & p^k & \kIII & 1 & 1\\
%-------------------------------------------------
\SetCell[r=4]{c} 
\begin{array}{c}
v_p(t)=0  \\[6pt]
v_p(t^2 + 4)=4k +2
\end{array}
& E_{1} & (2,4+2k,6) & p^k & \kI_{0}^* & p & 1\\
& E_{2} & (2,4+2k,6) & p^k & \kI_{0}^* & p & 1\\
& E_{5} & (2,4+2k,6) & p^k & \kI_{0}^* & p & 1\\
& E_{10} & (2,4+2k,6) & p^k & \kI_{0}^* & p & 1\\
%-------------------------------------------------
\SetCell[r=4]{c} 
\begin{array}{c}
v_p(t)=0  \\[6pt]
v_p(t^2 + 4)=4k +3
\end{array}
& E_{1} & (3,6+2k,9) & p^k & \kIII^* & p & 1\\
& E_{2} & (3,6+2k,9) & p^k & \kIII^* & p & 1\\
& E_{5} & (3,6+2k,9) & p^k & \kIII^* & p & 1\\
& E_{10} & (3,6+2k,9) & p^k & \kIII^* & p & 1\\
%-------------------------------------------------
\SetCell[r=4]{c} 
-m=v_p(t)<0 
& E_1 & (0,0,10m) & p^{-2m} &\kI_{6m}  & 1& 1\\
& E_2 & (0,0,5m) & p^{-2m} &\kI_{5m} & 1& 1\\
& E_{5} & (0,0,2m) & p^{-2m}&\kI_{2m} & 1& 1\\
& E_{10} & (0,0,m) & p^{-2m} &\kI_{m} & 1& 1\\
 \SetCell[c=5,r=2]{c} & & & & & d\equiv 0  & d\not\equiv 0 \\
                      & & & & & \SetCell[c=2]{c} d \Mod p & \\
\end{longtblr}



\newpage


\begin{longtblr}
[caption= {$R_4(10)$ data for $p=3$}]
{cells={mode=imath},hlines,vlines,measure=vbox,
hline{Z}={1-X}{0pt},
vline{1}={Y-Z}{0pt},
colspec=cclclcc}
\SetCell[c=1]{c} R_4(10) &\SetCell[c=6]{c} p=3  &    & \\
\SetCell[c=1]{c} t & E & 
\SetCell[c=1]{c} \operatorname{sig}_3(E) & u & \Kd_3(E) & \SetCell[c=2]{c} u_3(d)\\
%-------------------------------------------------
\SetCell[r=4]{c} 
m= v_3(t)>0 
& E_1 & (0,0,5m) & 1 & \kI_{5m} & 1 & 1 \\
& E_2 & (0,0,10m) & 1 &\kI_{10m} & 1 & 1\\
& E_{5} & (0,0,m) & 1 &\kI_{m}  & 1& 1\\
& E_{10} & (0,0,2m) & 1 &\kI_{2m}  & 1& 1\\
%-------------------------------------------------
\SetCell[r=4]{c} 
\begin{array}{c}
     v_3(t)=0  \\[3pt]
    m = v_p(t+1)>0 
\end{array}
& E_1 & (0,0,2m) & 1 &\kI_{2m}  & 1& 1 \\
& E_2 & (0,0,m) & 1 &\kI_{m} & 1& 1\\
& E_{5} & (0,0,10m) & 1 &\kI_{10m} & 1& 1\\
& E_{10} & (0,0,5m) & 1 &\kI_{5m}  & 1& 1\\
%-------------------------------------------------
\SetCell[r=4]{c} 
\begin{array}{c}
     v_3(t)=0  \\[3pt]
    m = v_p(t-4)>0
\end{array}
& E_1 & (0,0,m) & 1 &\kI_{m}   & 1& 1\\
& E_2 & (0,0,2m) & 1 &\kI_{2m}  & 1& 1\\
& E_{5} & (0,0,5m) & 1 &\kI_{5m} & 1& 1\\
& E_{10} & (0,0,10m) & 1 &\kI_{10m}  & 1& 1\\
%-------------------------------------------------
\SetCell[r=4]{c} 
-m=v_3(t)<0 
& E_1 & (0,0,10m) & 3^{-2m} &\kI_{10m}  & 1& 1\\
& E_2 & (0,0,5m) & 3^{-2m} &\kI_{5m} & 1& 1\\
& E_{5} & (0,0,2m) & 3^{-2m}&\kI_{2m} & 1& 1\\
& E_{10} & (0,0,m) & 3^{-2m} &\kI_{m} & 1& 1\\
 \SetCell[c=5,r=2]{c} & & & & & d\equiv 0  & d\not\equiv 0 \\
                      & & & & & \SetCell[c=2]{c} d \Mod 3 & \\
\end{longtblr}
\newpage


\begin{longtblr}
[caption= {$R_4(10)$ data for $p=5$}]
{cells={mode=imath},hlines,vlines,measure=vbox,
hline{Z}={1-X}{0pt},
vline{1}={Y-Z}{0pt},
colspec=cclclcc}
\SetCell[c=1]{c} R_4(10) &\SetCell[c=6]{c} p=5  &    & \\
\SetCell[c=1]{c} t & E & 
\SetCell[c=1]{c} \operatorname{sig}_5(E) & u & \Kd_5(E) & \SetCell[c=2]{c} u_5(d)\\
%-------------------------------------------------
\SetCell[r=4]{c} 
m= v_5(t)>0 
& E_1 & (0,0,5m) & 1 & \kI_{5m} & 1 & 1 \\
& E_2 & (0,0,10m) & 1 &\kI_{10m} & 1 & 1\\
& E_{5} & (0,0,m) & 1 &\kI_{m}  & 1& 1\\
& E_{10} & (0,0,2m) & 1 &\kI_{2m}  & 1& 1\\
%-------------------------------------------------
\SetCell[r=4]{c} 
\begin{array}{c}
v_5(t)=0  \\
t\equiv 1\,(5)\\
v_5(t^2 + 4)=4k 
\end{array}
& E_{1} & (0,1+2k,0) & 5^k & \kI_{0} & 1 & 1\\
& E_{2} & (0,1+2k,0) & 5^k & \kI_{0} & 1 & 1\\
& E_{5} & (0,2k,0) & 5^k & \kI_{0} & 1 & 1\\
& E_{10} & (0,2k,0) & 5^k & \kI_{0} & 1 & 1\\
%-------------------------------------------------
\SetCell[r=4]{c} 
\begin{array}{c}
v_5(t)=0  \\
t\equiv 1\,(5)\\
v_5(t^2 + 4)=4k +1
\end{array}
& E_{1} & (1,3+2k,3) & 5^k & \kIII & 1 & 1\\
& E_{2} & (1,3+2k,3) & 5^k & \kIII & 1 & 1\\
& E_{5} & (1,2+2k,3) & 5^k & \kIII & 1 & 1\\
& E_{10} & (1,2+2k,3) & 5^k & \kIII & 1 & 1\\
%-------------------------------------------------
\SetCell[r=4]{c} 
\begin{array}{c}
v_5(t)=0  \\
t\equiv 1\,(5)\\
v_5(t^2 + 4)=4k +2
\end{array}
& E_{1} & (2,5+2k,6) & 5^k & \kI_{0}^* & 5 & 1\\
& E_{2} & (2,5+2k,6) & 5^k & \kI_{0}^* & 5 & 1\\
& E_{5} & (2,4+2k,6) & 5^k & \kI_{0}^* & 5 & 1\\
& E_{10} & (2,4+2k,6) & 5^k & \kI_{0}^* & 5 & 1\\
%-------------------------------------------------
\SetCell[r=4]{c} 
\begin{array}{c}
v_5(t)=0  \\
t\equiv 1\,(5)\\
v_5(t^2 + 4)=4k +3
\end{array}
& E_{1} & (3,7+2k,9) & 5^k & \kIII^* & 5 & 1\\
& E_{2} & (3,7+2k,9) & 5^k & \kIII^* & 5 & 1\\
& E_{5} & (3,6+2k,9) & 5^k & \kIII^* & 5 & 1\\
& E_{10} & (3,6+2k,9) & 5^k & \kIII^* & 5 & 1\\
%-------------------------------------------------
\SetCell[r=4]{c} 
\begin{array}{c}
v_5(t)=0  \\
t\equiv 4\,(25)\\
v_5(t-4)=m
\end{array}
& E_{1} & (2,3,m+5) & 1 & \kI_{m-1}^* & 5 & 1\\
& E_{2} & (2,3,2m+4) & 1 & \kI_{2m-2}^* & 5 & 1\\
& E_{5} & (2,3,5m+1) & 5 & \kI_{5m-5}^* & 5 & 1\\
& E_{10} & (2,3,10m-4) & 5 & \kI_{10m-10}^* & 5 & 1\\
%-------------------------------------------------
\SetCell[r=4]{c} 
\begin{array}{c}
v_5(t)=0  \\
t\equiv 9\,(25)
\end{array}
& E_{1} & (2,3,6) & 1 & \kI_{0}^* & 5 & 1\\
& E_{2} & (2,\ge 4,6) & 1 & \kI_{0}^* & 5 & 1\\
& E_{5} & (2,3,6) & 5 & \kI_{0}^* & 5 & 1\\
& E_{10} & (2,\ge 4,6) & 5 & \kI_{0}^* & 5 & 1\\
%-------------------------------------------------
\SetCell[r=4]{c} 
\begin{array}{c}
v_5(t)=0  \\
t\equiv 14,19\,(25)\\
v_5(t^2 + 4)=4k 
\end{array}
& E_{1} & (1,\ge 2,3) & 5^k & \kIII & 1 & 1\\
& E_{2} & (1,\ge 3,3) & 5^k & \kIII & 1 & 1\\
& E_{5} & (1,\ge 2,3) & 5^{k+1} & \kIII & 1 & 1\\
& E_{10} & (1,\ge 2,3) & 5^{k+1} & \kIII & 1 & 1\\
%-------------------------------------------------
\SetCell[r=4]{c} 
\begin{array}{c}
v_5(t)=0  \\
t\equiv 14,19\,(25)\\
v_5(t^2 + 4)=4k +1
\end{array}
& E_{1} & (2,\ge 3,6) & 5^k & \kI_{0}^* & 5 & 1\\
& E_{2} & (2,\ge 3,6) & 5^k & \kI_{0}^* & 5 & 1\\
& E_{5} & (2,\ge 3,6) & 5^{k+1} & \kI_{0}^* & 5 & 1\\
& E_{10} & (2,\ge 3,6) & 5^{k+1} & \kI_{0}^* & 5 & 1\\
%-------------------------------------------------
\SetCell[r=4]{c} 
\begin{array}{c}
v_5(t)=0  \\
t\equiv 14,19\,(25)\\
v_5(t^2 + 4)=4k +2
\end{array}
& E_{1} & (3,\ge 5,9) & 5^k & \kIII^* & 5 & 1\\
& E_{2} & (3,\ge 5,9) & 5^k & \kIII^* & 5 & 1\\
& E_{5} & (3,\ge 5,9) & 5^{k+1} & \kIII^* & 5 & 1\\
& E_{10} & (3,\ge 5,9) & 5^{k+1} & \kIII^* & 5 & 1\\
%-------------------------------------------------
\SetCell[r=4]{c} 
\begin{array}{c}
v_5(t)=0  \\
t\equiv 14,19\,(25)\\
v_5(t^2 + 4)=4k +3
\end{array}
& E_{1} & (0,\ge 1,0) & 5^{k+1} & \kI_{0} & 1 & 1\\
& E_{2} & (0,\ge 1,0) & 5^{k+1} & \kI_{0} & 1 & 1\\
& E_{5} & (0,\ge 1,0) & 5^{k+2} & \kI_{0} & 1 & 1\\
& E_{10} & (0,\ge 1,0) & 5^{k+2} & \kI_{0} & 1 & 1\\
%-------------------------------------------------
\SetCell[r=4]{c} 
\begin{array}{c}
v_5(t)=0  \\
t\equiv 24\,(25)\\
v_5(t+1)=m
\end{array}
& E_{1} & (2,3,2m+4) & 1 & kI_{2m-2}^* & 5 & 1\\
& E_{2} & (2,3,m+5) & 1 & kI_{m-1}^* & 5 & 1\\
& E_{5} & (2,3,10m-4) & 5 & kI_{m-1}^* & 5 & 1\\
& E_{10} & (2,3,5m+1) & 5 & kI_{5m-5}^* & 5 & 1\\
%-------------------------------------------------
\SetCell[r=4]{c} 
-m=v_5(t)<0 
& E_1 & (0,0,10m) & 5^{-2m} &\kI_{10m}  & 1& 1\\
& E_2 & (0,0,5m) & 5^{-2m} &\kI_{5m} & 1& 1\\
& E_{5} & (0,0,2m) & 5^{-2m}&\kI_{2m} & 1& 1\\
& E_{10} & (0,0,m) & 5^{-2m} &\kI_{m} & 1& 1\\
 \SetCell[c=5,r=2]{c} & & & & & d\equiv 0  & d\not\equiv 0 \\
                      & & & & & \SetCell[c=2]{c} d \Mod 5 & \\
\end{longtblr}


\newpage

\begin{longtblr}
[caption = {$R_4(10)$ data for $p$=2}]
{cells = {mode=imath},hlines,vlines,measure=vbox,
hline{Z} = {1-5}{0pt},
vline{1} = {Y-Z}{0pt},
colspec  = cclclccc}
%--------------------------------------
\SetCell[c=1]{c} R_4(10) &\SetCell[c=7]{c} p=2  & & & &  & & \\
\SetCell[c=1]{c} t & E & 
\SetCell[c=1]{c}\operatorname{sig}_2(E) & u & \Kd_2(E) & \SetCell[c=3]{c} u_2(d)   &   &    \\
%--------------------------------------
\SetCell[r=4]{c} m=v_2(t)> 2 
& E_1 & (6,9,5m+8) & 1 &\kI_{5m-2}^* & 1 & \text{$2^*$ or $4^*$}  & 1\\
& E_2 & (6,9,10m-2) & 2 &\kI_{10m-12}^* & 1 &  \text{$2^*$ or $4^*$} & 1\\
& E_5 & (6,9,m+16) & 1 &\kI_{m+6}^* & 1 & \text{$2^*$ or $4^*$}  & 1\\
& E_{10} & (6,9,2m+14) & 2 &\kI_{2m-4}^* & 1 & \text{$2^*$ or $4^*$}  & 1\\
%--------------------------------------
\SetCell[r=4]{c} v_2(t)= 2
& E_1 & (6,9,19) & 1 & \kI_{9}^* & 1 & \text{$2^*$ or $4^*$} &1 \\
& E_2& (6,9,20) & 2 & \kI_{10}^* & 1 &  \text{$2^*$ or $4^*$} & 1\\
& E_5 & (6,9,23) & 1 & \kI_{13}^* & 1 & \text{$2^*$ or $4^*$}  & 1\\
& E_{10} & (6,9,28) & 2 & \kI_{18}^* & 1 &\text{$2^*$ or $4^*$}   & 1\\
%--------------------------------------
\SetCell[r=4]{c} v_2(t)= 1 
& E_1 & (7,11,15) & 1 & \kIII^* & 1 & 2 & 1\\
& E_2 & (5,8,9) & 2 &\kIII & 1 & 1 & 1\\
& E_5 & (7,11,15) & 1 & \kIII & 1 & 2 &1 \\
& E_{10} & (5,8,9) & 2 &\kIII & 1 & 1 & 1\\
%--------------------------------------
\SetCell[r=4]{c} 
\begin{array}{c}
v_2(t)= 0\\
m=v_2(t+1)
\end{array}
& E_1 & (0,0,2m) & 1 & \kI_{2m} & 1 & 2^{-1} & 2^{-1} \\
& E_2 & (0,0,m) & 1 & \kI_{m} & 1 & 2^{-1} & 2^{-1}\\
& E_5 & (0,0,10m) & 1 & \kI_{10m} & 1 & 2^{-1} & 2^{-1}\\
& E_{10} & (0,0,5m) & 1 & \kI_{5m} & 1 & 2^{-1} & 2^{-1}\\
%--------------------------------------
\SetCell[r=4]{c} -m=v_2(t)<0 
& E_1 & (4,6,10m+12) & 2^{-2m-1} &\kI_{10m+4}^* & 1 &1  & 2\\
& E_2 & (4,6,5m+12) & 2^{-2m-1} &\kI_{5m+4}^* & 1 & 1 & 2\\
& E_{5} & (4,6,2m+12) & 2^{-2m-1} &\kI_{2m+4}^* & 1 & 1 &2 \\
& E_{10} & (4,6,m+12) & 2^{-2m-1} &\kI_{m+4}^* & 1 & 1 & 2\\
%----------------------------------------------
 \SetCell[c=5,r=2]{c} & & & & &  d\equiv 1 &  d\equiv 2  & d\equiv 3 \\
                      & & & & & \SetCell[c=3]{c} d \Mod{4} & \\
\end{longtblr}


\newpage

\section{Conclusion}

\begin{prop}
Let 
\[ \begin{tikzcd}
E_1 \arrow[dash]{r}{5} 
    \arrow[dash]{d}{2} & 
    E_{5}  \arrow[dash]{d}{2} \\
 E_2 \arrow[dash]{r}{5} & E_{10}   
\end{tikzcd}
\]
be a $\mathbf{Q}$-isogeny graph of type $R_4(10)$ corresponding to a given $t$ in $\mathbf{Q}\setminus \{-1,\pm 4\}$ as above. 
For every square-free integer $d$, 
the probability of a vertex
to be the Faltings curve (circled)
in the twisted graph 
\[ \begin{tikzcd}
E_1^d \arrow[dash]{r}{5}
    \arrow[dash]{d}{2} & 
    E_{5}^d  \arrow[dash]{d}{2} \\
 E_2^d \arrow[dash]{r}{5} &    \, E_{10}^d
\end{tikzcd}
\]
is given by:
%\newpage
\begin{longtblr}{|c|c|c|c|c|}
\hline
%--------------------------------------
\SetCell[c=2]{c} R_4(10) & & \SetCell[c=1]{c}\text{twisted isogeny graph} & $d$ & \SetCell[c=1]{c}\text{prob} \\
 \hline
  \SetCell[r=2]{c}  v_2(t)\leq 0 &\SetCell[r=1]{c}  \begin{array}{c}  \mbox{$\quad$}\\ v_5(t)\neq 0 \\ \mbox{$\quad$}\end{array} &
\SetCell[r=2]{c}\makecell{%
        \begin{tikzcd}[ampersand replacement=\&]
\circled[0.8]{$E_1^d$} \ar[r] 
   \ar[d]  \& 
    E_{5}^d  \ar[d]  \\
 E_2^d \ar[r]  \&    \, E_{10}^d
        \end{tikzcd}}  
& &  1 \\
\cline{2-2}
  & \SetCell[r=1]{c} 
   \begin{array}{c}
 v_5(t)=0\\
 t\equiv 1\,(5)  \end{array}  & & & \\
%--------------------------------------
\hline
\SetCell[r=2]{c} v_2(t)=1 & \SetCell[r=1]{c} \begin{array}{c}  \mbox{$\quad$}\\ v_5(t)\neq 0 \\ \mbox{$\quad$}\end{array}
& 
\makecell{%
        \begin{tikzcd}[ampersand replacement=\&]
\circled[0.8]{$E_1^d$} \ar[r] 
   \ar[d]  \& 
    E_{5}^d  \ar[d]  \\
 E_2^d \ar[r]  \&    \, E_{10}^d
        \end{tikzcd}} 
     &d\equiv 0\,(2)   & 1/3\\ 
   \cline{2-2}  
&\SetCell[r=1]{c}  \begin{array}{c}
 v_5(t)=0\\
 t\equiv 1\,(5)  \end{array}   &  \makecell{%
        \begin{tikzcd}[ampersand replacement=\&]
E_1^d \ar[d] 
    \& 
    E_{5}^d \ar[l]   \ar[d]  \\
\circled[0.8]{$E_2^d$}  \&    \, E_{10}^d  \ar[l] 
        \end{tikzcd}} 
&d\not\equiv 0\,(2) &  2/3 \\
%--------------------------------------
 \hline
 \SetCell[r=2]{c} v_2(t)>1 & \SetCell[r=1]{c} \begin{array}{c}  \mbox{$\quad$}\\ v_5(t)\neq 0 \\ \mbox{$\quad$}\end{array}&
\SetCell[r=2]{c} \makecell{%
        \begin{tikzcd}[ampersand replacement=\&]
E_1^d \ar[d] 
    \& 
    E_{5}^d \ar[l]   \ar[d]  \\
\circled[0.8]{$E_2^d$}  \&    \, E_{10}^d  \ar[l] 
        \end{tikzcd}} 
& & \SetCell[r=2]{c} 1 \\
\cline{2-2}
  & \SetCell[r=1]{c} 
   \begin{array}{c}
 v_5(t)=0\\
 t\equiv 1\,(5)  \end{array}  & & & \\
%--------------------------------------
\hline
 v_2(t)>1& \SetCell[r=1]{c} 
   \begin{array}{c}
 v_5(t)=0\\
 t\equiv 4\,(5)  \end{array}  &
\makecell{%
        \begin{tikzcd}[ampersand replacement=\&]
E_1^d  \ar[r]
   \& 
    E_{5}^d   \\
E_2^d   \ar[u]  \ar[r]   \&    \,  \circled[0.8]{$E_{10}^d$} \ar[u]
        \end{tikzcd}}  
& &  1 \\
%--------------------------------------
\hline
\SetCell[r=2]{c} v_2(t)=1 & \SetCell[r=2]{c} \SetCell[r=1]{c} 
   \begin{array}{c}
 v_5(t)=0\\
 t\equiv 4\,(5)  \end{array}  
& 
\makecell{%
        \begin{tikzcd}[ampersand replacement=\&]
E_1^d  \ar[r]
   \& 
     \circled[0.8]{$E_{5}^d$}   \\
 E_2^d   \ar[u]  \ar[r]   \&    \, E_{10}^d \ar[u]
        \end{tikzcd}}    
      &d\equiv 0\,(2)  & 1/3\\ 
& &  \makecell{%
        \begin{tikzcd}[ampersand replacement=\&]
E_1^d \& 
    E_{5}^d  \ar[l]  \\
 E_2^d \ar[u]  \&    \circled[0.8]{$E_{10}^d$}\ar[l]\ar[u]
        \end{tikzcd}} 
&d\not\equiv 0\,(2) &  2/3 \\
%--------------------------------------
 \hline
 v_2(t)\le 0 & \SetCell[r=1]{c} 
   \begin{array}{c}
 v_5(t)=0\\
 t\equiv 4\,(5)  \end{array} 
  &
\makecell{%
        \begin{tikzcd}[ampersand replacement=\&]
E_1^d \& 
     \circled[0.8]{$E_{5}^d$}  \ar[l]  \\
 E_2^d \ar[u]  \&    E_{10}^d\ar[l]\ar[u]
        \end{tikzcd}}  
& &  1 \\
\hline
\end{longtblr}

\end{prop}



\noindent{\it Proof.} From the previous tables one gets:

\vskip 0.5truecm


\begin{tblr}{cells={mode=imath},hlines,vlines,measure=vbox}
%\hline
\SetCell[c=1]{c} t &\SetCell[c=1]{c} [u(E)]  & \SetCell[c=1]{c} [u(E)(d)] & \\
%----------------------------------------------------------------
 \SetCell[r=1]{c} v_5(t)\ne 0 & \SetCell[r=2]{c} (1:1:1:1) & \SetCell[r=2]{c} (1:1:1:1) &   \SetCell[r=2]{c}  \\
  \SetCell[r=1]{c}  \begin{array}{c}
 v_5(t)=0\\
 t\equiv 1\,(5)  \end{array} & & & \\
 %----------------------------------------------------------------
 \SetCell[r=1]{c} \begin{array}{c}
 v_5(t)=0\\
 t\equiv 4\,(5)   \end{array}
 & \SetCell[r=1]{c} (1:1:5:5) & \SetCell[r=1]{c} (1:1:1:1) & \\
 %----------------------------------------------------------------
\hline
 %----------------------------------------------------------------
\SetCell[r=1]{c} v_2(t)>1 & \SetCell[r=1]{c} (1:2:1:2) & (1:1:1:1) &   \\
 %----------------------------------------------------------------
\SetCell[r=2]{c} v_2(t)=1 & \SetCell[r=2]{c} (1:2:1:2) & (1:1:1:1) & d\not\equiv 0\,(2)   \\
&  & \SetCell[r=1]{c} (2:1:2:1) & d\equiv 0\,(2)  \\
 %----------------------------------------------------------------
\SetCell[r=1]{c} v_2(t)\leq 0 & \SetCell[r=1]{c} (1:1:1:1) & (1:1:1:1) &  
 %----------------------------------------------------------------
\end{tblr}
\end{document}