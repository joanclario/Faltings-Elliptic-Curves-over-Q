\documentclass[11pt]{article}
%--------------------------------------------------------------------------
\usepackage{amsfonts,amssymb,amsmath,amsthm,latexsym,graphics,epsfig,amsfonts}
\usepackage{verbatim,enumerate,array,booktabs,color,bigstrut,prettyref,tikz-cd}
\usepackage{multirow}
\usepackage[all]{xy}
\usepackage[backref]{hyperref}
\usepackage[OT2,T1]{fontenc}
\usepackage{mathtools}
\usepackage{caption}
\usepackage{longtable}
\usepackage{mathtools}
\usepackage{tabularray}
\UseTblrLibrary{amsmath,varwidth}
\usepackage{tabularx}
\usepackage{longtable}
\usepackage{arydshln}
%--------------------------------------------------------------------------
\newcommand{\Mod}[1]{\ (\mathrm{mod}\ #1)}
\newcommand{\mathdash}{\relbar\mkern-8mu\relbar}
\newcommand*\circled[2][1.6]{\tikz[baseline=(char.base)]{
    \node[shape=circle, draw, inner sep=1pt, 
        minimum height={\f@size*#1},] (char) {\vphantom{WAH1g}#2};}}
\makeatother
%\addtolength{\textwidth}{4cm} 
\addtolength{\textheight}{3.5cm} 
\addtolength{\topmargin}{-2cm}
%\addtolength{\marginparwidth}{-2cm} 
\newcommand\myiso{\stackrel{\mathclap{\normalfont\mbox{\small $p$}}}{-}}
\newcommand\myisot{\stackrel{\mathclap{\normalfont\mbox{\small $3$}}}{-}}
\newcommand{\pref}[1]{\prettyref{#1}}
%------------------------------------
\newcommand{\Kd}{\operatorname{K}}
\newcommand{\kI}{\operatorname{I}}
\newcommand{\kII}{\operatorname{II}}
\newcommand{\kIII}{\operatorname{III}}
\newcommand{\kIV}{\operatorname{IV}}
%-------------------------------------
\newrefformat{eq}{\textup{(\ref{#1})}}
\newrefformat{prty}{\textup{(\ref{#1})}}

\definecolor{mylinkcolor}{rgb}{0.8,0,0}
\definecolor{myurlcolor}{rgb}{0,0,0.8}
\definecolor{mycitecolor}{rgb}{0,0,0.8}
\hypersetup{colorlinks=true,urlcolor=myurlcolor,citecolor=mycitecolor,linkcolor=mylinkcolor,linktoc=page,breaklinks=true}
%\DeclareSymbolFont{cyrletters}{OT2}{wncyr}{m}{n}
%\DeclareMathSymbol{\Sha}{\mathalpha}{cyrletters}{"58}
\addtolength{\textwidth}{4cm} \addtolength{\hoffset}{-2cm}
\addtolength{\marginparwidth}{-2cm}
%\theoremstyle{definition}
\newtheorem{defn}{Definition}[section]
\newtheorem{definition}[defn]{Definition}
\newtheorem{claim}[defn]{Claim}
%\theoremstyle{plain}
\newtheorem{thmA}{Theorem A}
\newtheorem{thmB}{Theorem B}
\newtheorem{thm2}{Theorem}
\newtheorem{prop2}{Proposition}
\newtheorem{note}{Note}
\newtheorem{corollary}[defn]{Corollary}
\newtheorem{lemma}[defn]{Lemma}
\newtheorem{property}[defn]{Property}
\newtheorem{thm}[defn]{Theorem}
\newtheorem{theorem}[defn]{Theorem}
\newtheorem{cor}[defn]{Corollary}
\newtheorem{prop}[defn]{Proposition}
\newtheorem{proposition}[defn]{Proposition}
\newtheorem{thmnn}{Theorem}
\newtheorem{conj}[defn]{Conjecture}
\theoremstyle{definition}
\newtheorem{remarks}{Remarks}
\newtheorem{ack}{Acknowledgements}
\newtheorem{remark}[defn]{Remark}
\newtheorem{question}[defn]{Question}
\newtheorem{example}[defn]{Example}
%-------------------------------------
\newcommand{\Q}{\mathbb Q}
\newcommand{\Qbar}{\overline{\Q}}
\newcommand{\Z}{\mathbb Z}
\newcommand{\modQ}{\,\text{mod}\,(\Q^)^2}
%-------------------------------------
\newcommand{\mysquare}[1]{\tikz{\path[draw] (0,0) rectangle node{\tiny #1} (8pt,8pt) ;}}
\newcommand{\mycircle}[1]{\tikz{\path[draw] (0,0) circle (4pt) node{\tiny #1};}}
%-------------------------------------
\begin{document}
\title{Type $R_4(21)$}
\date{\today}
\maketitle
%-------------------------------------
\section{Setting}
The isogeny graphs of type $R_4(21)$ are given by four isogenous elliptic curves:


\[ \begin{tikzcd}
E_1 \arrow[dash]{r}{7} 
    \arrow[dash]{d}{3} & 
    E_5  \arrow[dash]{d}{3} \\
 E_3 \arrow[dash]{r}{7} & E_{21}   \,.
\end{tikzcd}
\]


The modular curve $X_0(21)$ is elliptic of rank $0$. Its rational points are: XXX four rational cusps and two rational CM points $\tau,\tau'\in\mathbb H$. Their corresponding $j$-invariants are:
$$
j(\tau)=\frac{3^{3}\cdot 5^{3}}{2},\qquad 
j(\tau')=\frac{-3^{2}\cdot 5^{6}}{2^{3}},\quad 
j(\tau'')=\frac{-3^{3}\cdot 5^{3}\cdot 383^{3}}{2^{7}},\quad
j(\tau''')=\frac{-3^{2}\cdot 5^{3}\cdot 101^{3}}{2^{21}}
$$
%We have XXX $j(21\tau)=j(\tau')$.XXX

We can (and do) choose Weierstrass equations 
\vskip 0.5truecm
\begin{tblr}
{cells={mode=imath},hlines,vlines,measure=vbox,
colspec=clll}
E & \text{Minimal Weierstrass model} & j(E) &\text{label}\\
E_1 &  y^2 = x^3 + 45x + 18 & \frac{+3^{3}\cdot 5^{3}}{2} & \texttt{1296k1}\\
E_3 & y^2 = x^3 - 675 x + 7074  &  \frac{-3^{2}\cdot 5^{6}}{2^{3}} & \texttt{1296k2}\\
E_{7} & y^2 = x^3 - 17235 x - 870894  & \frac{-3^{3}\cdot 5^{3}\cdot 383^{3}}{2^{7}}  & \texttt{1296k3}\\
E_{21} & y^2 = x^3 - 13635x - 1244862 & \frac{-3^{2}\cdot 5^{3}\cdot 101^{3}}{2^{21}}  & \texttt{1296k4}\\
\end{tblr}
\vskip 0.5truecm

so that one has:
\vskip 0.5truecm
\begin{tblr}{
vline{1,2,3}={1-4,6-9,11-15,16-19}{solid},
cells={mode=imath},colspec=cll|l|l|}
\hline 
 E  & E_1 & E_3 & E_7 & E_{21}\\
 \hline 
c_4(E) & -2^{4}\cdot 3^{3}\cdot 5 &  2^{4}\cdot 3^{4}\cdot 5^{2} & 2^{4}\cdot 3^{3}\cdot 5\cdot 383 & 2^{4}\cdot 3^{4}\cdot 5\cdot 101
\\
c_6(E) &-2^{6}\cdot 3^{5} & -2^{6}\cdot 3^{6}\cdot 131 & 2^{6}\cdot 3^{5}\cdot 48383 &  2^{6}\cdot 3^{6}\cdot 23053
\\  
\Delta(E) & -2^{13}\cdot 3^{6} & -2^{15}\cdot 3^{10} & -2^{19}\cdot 3^{6} &  -2^{33}\cdot 3^{10}
\\
\hline
\end{tblr}



\vskip 0.5truecm
\noindent and, with this choice, the isogeny graph is normalized. 

%\newpage 

We have that the Faltings curve (circled) in the graph is
%\vskip 0.5truecm
$$
\makecell{%
        \begin{tikzcd}[ampersand replacement=\&]
\circled[0.8]{$E_1$} \ar[r] 
   \ar[d]  \& 
    E_7  \ar[d]  \\
 E_3 \ar[r]  \&    \, E_{21}
        \end{tikzcd}}  
$$
\vskip 0.5truecm

Then any $\mathbb Q$-isogeny class of type $R_4(21)$ is isomorphic to 

\[ \begin{tikzcd}
E_1 \arrow[dash]{r} 
    \arrow[dash]{d} & 
    E_7  \arrow[dash]{d} \\
 E_3 \arrow[dash]{r} & E_{21}   \,.
\end{tikzcd}
\]



\section{Signatures, minimal models, Kodaira symbols, and Pal values}


$$
\begin{tblr}
%[caption = {$R_4(15)$ data for $p$=2}]
{cells = {mode=imath},hlines,vlines,measure=vbox,
hline{Z} = {1-3}{0pt},
vline{1} = {Y-Z}{0pt},
colspec  = clclccc}
%----------------------------------------------
\SetCell[c=6]{c} p=2   & & & & \\ 
 E & \SetCell[c=1]{c} \operatorname{sig}_2(E)  & \SetCell[c=1]{c} \Kd_2(E) & \SetCell[c=3]{c} u_2(d)  \\
%----------------------------------------------
 E_1 & (4,6,13)  & \kI_{5}^* & 1 & 1 & 2 \\
 E_3 & (4,6,15)  & \kI_{7}^* & 1 & 1 & 2 \\
 E_7 & (4,6,19)  & \kI_{11}^* & 1 & 1 & 2 \\
 E_{21} &(4,6,33)   & \kI_{25}^* & 1 & 1 & 2\\
%----------------------------------------------
 \SetCell[c=3,r=2]{c}   & & &  d\equiv 1 &  d\equiv 2  & d\equiv 3 \\
                        & & & \SetCell[c=3]{c} d \Mod{4}   \\
\end{tblr}
$$

\vskip 0.75truecm


$$
\begin{tblr}
%[caption = {$R_4(15)$ data for $p$=7}]
{cells = {mode=imath},hlines,vlines,measure=vbox,
hline{Z} = {1-3}{0pt},
vline{1} = {Y-Z}{0pt},
colspec  = clclcc}
%----------------------------------------------
\SetCell[c=5]{c} p=3  &  & &  \\ 
 E & \SetCell[c=1]{c} \operatorname{sig}_{3}(E)  & \SetCell[c=1]{c} \Kd_{3}(E) & \SetCell[c=2]{c} u_{3}(d)  \\
%----------------------------------------------
 E_1 & (3,5,6)  & \kIV & 1 & 1  \\
 E_3 & (4,6,10)   & \kIV^*& 3 & 1\\
 E_7 & (3,5,6)  & \kIV & 1 & 1  \\
 E_{21} & (4,6,10)   & \kIV^* & 3 & 1  \\
%----------------------------------------------
 \SetCell[c=3,r=2]{c}  &  & &  d\equiv 0 &  d\not\equiv 0  \\
                       &  & & \SetCell[c=2]{c} d \Mod{3}   \\
\end{tblr}
%%

\vskip 0.75truecm

\newpage
\section{Conclusion}
\begin{prop}
For every square-free integer $d$, 
the probability of a vertex
to be the Faltings curve (circled)
in the twisted isogeny graph 
$$  \begin{tikzcd}
E^d_1 \arrow[dash]{r}{7} 
    \arrow[dash]{d}{3} & 
    E^d_7  \arrow[dash]{d}{3} \\
 E^d_3 \arrow[dash]{r}{7} & E^d_{21}   \,.
\end{tikzcd}
$$
is given by:
\vskip 0.75truecm

\begin{tblr}{|c|c|c|}
\hline
 \SetCell[c=1]{c}\text{twisted isogeny graph} & \SetCell[c=1]{c}\text{condition} & \SetCell[c=1]{c}\text{prob} \\
 \hline
\makecell{%
        \begin{tikzcd}[ampersand replacement=\&]
\circled[0.8]{$E_1^d$} \ar[r] 
   \ar[d]  \& 
    E_7^d  \ar[d]  \\
 E_3^d \ar[r]  \&    \, E_{21}^d
        \end{tikzcd}} 
        & d\,\not\equiv 0\,(3)& 3/4\\ 
\hline
   \makecell{%
        \begin{tikzcd}[ampersand replacement=\&]
E_1^d \ar[d] 
    \& 
    E_7^d \ar[l]   \ar[d]  \\
\circled[0.8]{$E_3^d$}  \&    \, E_{21}^d  \ar[l] 
        \end{tikzcd}} 
&  d\,\equiv 0\,(3) & 1/4 \\
\hline
\end{tblr}



\end{prop}

\vskip 0.35truecm

\noindnet{\it Proof.} From the previous tables one gets:

\vskip 0.5truecm

\begin{tblr}{cells={mode=imath},hlines,vlines,measure=vbox}
%-------------------------------------------------
\SetCell[c=1]{c} [u(E)(d)] & \SetCell[c=1]{c} d & \SetCell[c=1]{c}\text{Prob}\\
%-------------------------------------------------
\SetCell[r=1]{c} \left(1:\frac{1}{3}:\frac{1}{7}:\frac{1}{21}\right) & d\not\equiv 0\,(3)&\SetCell[r=2]{c} \left(\frac{3}{4},0,\frac{1}{4},0\right) \\
 \SetCell[r=1]{c} \left(1:3:\frac{1}{7}:\frac{3}{7}\right) & d\equiv 0\,(3)& \\
%-------------------------------------------------
\end{tblr}

\end{document}


