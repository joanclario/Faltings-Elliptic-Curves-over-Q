\documentclass[11pt]{article}
%--------------------------------------------------------------------------
\usepackage{amsfonts,amssymb,amsmath,amsthm,latexsym,graphics,epsfig,amsfonts}
\usepackage{verbatim,enumerate,array,booktabs,color,bigstrut,prettyref,tikz-cd}
\usepackage{multirow}
\usepackage[all]{xy}
\usepackage[backref]{hyperref}
\usepackage[OT2,T1]{fontenc}
\usepackage{mathtools}
\usepackage{caption}
\usepackage{longtable}
\usepackage{mathtools}
\usepackage{tabularray}
\UseTblrLibrary{amsmath,varwidth}
\usepackage{tabularx}
\usepackage{longtable}
\usepackage{arydshln}
%--------------------------------------------------------------------------
\newcommand{\Mod}[1]{\ (\mathrm{mod}\ #1)}
\newcommand{\mathdash}{\relbar\mkern-8mu\relbar}
\newcommand*\circled[2][1.6]{\tikz[baseline=(char.base)]{
    \node[shape=circle, draw, inner sep=1pt, 
        minimum height={\f@size*#1},] (char) {\vphantom{WAH1g}#2};}}
\makeatother
%\addtolength{\textwidth}{4cm} 
\addtolength{\textheight}{3.5cm} 
\addtolength{\topmargin}{-2cm}
%\addtolength{\marginparwidth}{-2cm} 
\newcommand\myiso{\stackrel{\mathclap{\normalfont\mbox{\small $p$}}}{-}}
\newcommand\myisot{\stackrel{\mathclap{\normalfont\mbox{\small $3$}}}{-}}
\newcommand{\pref}[1]{\prettyref{#1}}
%------------------------------------
\newcommand{\Kd}{\operatorname{K}}
\newcommand{\kI}{\operatorname{I}}
\newcommand{\kII}{\operatorname{II}}
\newcommand{\kIII}{\operatorname{III}}
\newcommand{\kIV}{\operatorname{IV}}
%-------------------------------------
\newrefformat{eq}{\textup{(\ref{#1})}}
\newrefformat{prty}{\textup{(\ref{#1})}}

\definecolor{mylinkcolor}{rgb}{0.8,0,0}
\definecolor{myurlcolor}{rgb}{0,0,0.8}
\definecolor{mycitecolor}{rgb}{0,0,0.8}
\hypersetup{colorlinks=true,urlcolor=myurlcolor,citecolor=mycitecolor,linkcolor=mylinkcolor,linktoc=page,breaklinks=true}
%\DeclareSymbolFont{cyrletters}{OT2}{wncyr}{m}{n}
%\DeclareMathSymbol{\Sha}{\mathalpha}{cyrletters}{"58}
\addtolength{\textwidth}{4cm} \addtolength{\hoffset}{-2cm}
\addtolength{\marginparwidth}{-2cm}
%\theoremstyle{definition}
\newtheorem{defn}{Definition}[section]
\newtheorem{definition}[defn]{Definition}
\newtheorem{claim}[defn]{Claim}
%\theoremstyle{plain}
\newtheorem{thmA}{Theorem A}
\newtheorem{thmB}{Theorem B}
\newtheorem{thm2}{Theorem}
\newtheorem{prop2}{Proposition}
\newtheorem{note}{Note}
\newtheorem{corollary}[defn]{Corollary}
\newtheorem{lemma}[defn]{Lemma}
\newtheorem{property}[defn]{Property}
\newtheorem{thm}[defn]{Theorem}
\newtheorem{theorem}[defn]{Theorem}
\newtheorem{cor}[defn]{Corollary}
\newtheorem{prop}[defn]{Proposition}
\newtheorem{proposition}[defn]{Proposition}
\newtheorem{thmnn}{Theorem}
\newtheorem{conj}[defn]{Conjecture}
\theoremstyle{definition}
\newtheorem{remarks}{Remarks}
\newtheorem{ack}{Acknowledgements}
\newtheorem{remark}[defn]{Remark}
\newtheorem{question}[defn]{Question}
\newtheorem{example}[defn]{Example}
%-------------------------------------
\newcommand{\Q}{\mathbb Q}
\newcommand{\Qbar}{\overline{\Q}}
\newcommand{\Z}{\mathbb Z}
\newcommand{\modQ}{\,\text{mod}\,(\Q^)^2}
%-------------------------------------
\newcommand{\mysquare}[1]{\tikz{\path[draw] (0,0) rectangle node{\tiny #1} (8pt,8pt) ;}}
\newcommand{\mycircle}[1]{\tikz{\path[draw] (0,0) circle (4pt) node{\tiny #1};}}
%-------------------------------------
\begin{document}
\title{Type $R_4(14)$}
\date{\today}
\maketitle
%-------------------------------------
\section{Setting}
The isogeny graphs of type $R_4(14)$ are given by four isogenous elliptic curves:


\[ \begin{tikzcd}
E_1 \arrow[dash]{r}{7} 
    \arrow[dash]{d}{2} & 
    E_7  \arrow[dash]{d}{2} \\
 E_2 \arrow[dash]{r}{7} & E_{14}   \,.
\end{tikzcd}
\]


%\noindent The modular curve $X_0(14)$ is isomorphic to the elliptic curve \textt{14a.6} given by the Weierstrass equation $y^2+xy+y=x^3+4x-6$. It has rank $0$ over $\Q$; in fact, $X_0(14)(\Q)$ is isomorphic to $\Z/6\Z$.  The forgetful $j$-map is given by:
%$$
%j \colon X_0(14) \longrightarrow \mathbf{P}^1 \,,
%(x,y) \mapsto x 
%$$

The modular curve $X_0(14)$ is elliptic of rank $0$. Its rational points are: four rational cusps and two rational CM points $\tau_,\tau'\in\mathbb H$. Their corresponding $j$-invariants are:
$$
j(\tau)=-3^5\cdot 5^{3},\qquad j(\tau')=3^5\cdot 5^{3}\cdot 17^{3}.
$$
We have $j(14\tau)=j(\tau')$.

%Its rational points correspond to the following pair of $j$-invariants:
%$$
%(j_1,j_{14})=(-3^5\cdot 5^{3},3^5\cdot 5^{3}\cdot %17^{3}).
%$$
We can (and do) choose Weierstrass equations 
\vskip 0.5truecm
\begin{tblr}
{cells={mode=imath},hlines,vlines,measure=vbox,
colspec=clll}
E & \text{Minimal Weierstrass model} & j(E) &\text{label}\\
\hline
E_1 & y^2+xy=x^3-x^2-2x-1 & -3^5\cdot 5^{3}& \texttt{49a1}\\
E_{2} & y^2+xy=x^3-x^2-37x-78 & 3^5\cdot 5^{3}\cdot 17^{3} & \texttt{49a2}\\
E_7 & y^2+xy=x^3-x^2-107x+552 & -3^5\cdot 5^{3} & \texttt{49a3}\\
E_{14} & y^2+xy=x^3-x^2-1822x+30393 & 3^5\cdot 5^{3}\cdot 17^{3} & \texttt{49a4}\\
\end{tblr}
\vskip 0.5truecm

so that one has:
\vskip 0.5truecm
\begin{tblr}{
vline{1,2,3}={1-4,6-9,11-14,16-19}{solid},
cells={mode=imath},colspec=cll|l|l|}
\hline 
 E  & E_1 & E_2 & E_7 & E_{14}\\
 \hline 
c_4(E) & 3\cdot 5\cdot 7 & 3\cdot 5\cdot 7\cdot 17& 3\cdot 5\cdot 7^{3}& 3\cdot 5\cdot 7^{3}\cdot 17
\\
c_6(E) &3^{3}\cdot 7^{2} & 3^{4}\cdot 7^{2}\cdot 19  & -3^{3}\cdot 7^{5}  & -3^{4}\cdot 7^{5}\cdot 19
\\  
\Delta(E) & -7^{3}  & 7^{3} & -7^{9} & 7^{3}
\\
\hline
\end{tblr}

\vskip 0.5truecm
\noindent and, with this choice, the isogeny graph is normalized. 

%\newpage 

We have that the Faltings curve (circled) in the graph is
%\vskip 0.5truecm
$$
\makecell{%
        \begin{tikzcd}[ampersand replacement=\&]
\circled[0.8]{$E_1$} \ar[r] 
   \ar[d]  \& 
    E_7  \ar[d]  \\
 E_2 \ar[r]  \&    \, E_{14}
        \end{tikzcd}}  
$$
\vskip 0.5truecm

Then any $\mathbb Q$-isogeny class of type $R_4(14)$ is isomorphic to 

\[ \begin{tikzcd}
E_1 \arrow[dash]{r}{7} 
    \arrow[dash]{d}{2} & 
    E_7  \arrow[dash]{d}{2} \\
 E_2 \arrow[dash]{r}{7} & E_{14}   \,.
\end{tikzcd}
\]

Note that $E_1, E_2,E_7$ and $E_{14}$ have complex multiplication by an order in the ring of integers of $\mathbf{Q}(\sqrt{-7})$ and $E_7=E_1^{-7}$, $E_{14}=E_2^{-7}$. 


\section{Signatures, minimal models, Kodaira symbols, and Pal values}

\vskip 0.3truecm

$$
\begin{tblr}
%[caption = {$R_4(14)$ data for $p$=7}]
{cells = {mode=imath},hlines,vlines,measure=vbox,
hline{Z} = {1-3}{0pt},
vline{1} = {Y-Z}{0pt},
colspec  = clclcc}
%----------------------------------------------
\SetCell[c=5]{c} p=7  & & &  \\ 
 E & \SetCell[c=1]{c} \operatorname{sig}_{7}(E)  & \SetCell[c=1]{c} \Kd_{7}(E) & \SetCell[c=2]{c} u_{7}(d)  \\
%----------------------------------------------
 E_1 & (1,2,3)  & \kIII & 1 & 1  \\
 E_{2} &(1,2,3)   & \kIII & 1 & 1 \\
 E_7 & (3,5,9)   & \kIII^*& 7 & 1\\
 E_{14} & (3,5,9)   & \kIII^* & 7 & 1  \\
%----------------------------------------------
 \SetCell[c=3,r=2]{c}  &  & &  d\equiv 0 &  d\not\equiv 0  \\
                       &  & & \SetCell[c=2]{c} d \Mod{7}   \\
\end{tblr}
%%

\vskip 0.75truecm

\newpage
\section{Conclusion}
\begin{prop}
%Let 
%$  \begin{tikzcd}
%E_1 \arrow[dash]{r}{7} 
%    \arrow[dash]{d}{2} & 
%    E_7  \arrow[dash]{d}{2} \\
% E_2 \arrow[dash]{r}{7} & E_{14}   \,.
%\end{tikzcd}
%$
%be the above $\mathbf{Q}$-isogeny graph of type $L_{2}(11)$. 
For every square-free integer $d$, 
the probability of a vertex
to be the Faltings curve (circled)
in the twisted isogeny graph 
$$  \begin{tikzcd}
E^d_1 \arrow[dash]{r}{7} 
    \arrow[dash]{d}{2} & 
    E^d_7  \arrow[dash]{d}{2} \\
 E^d_2 \arrow[dash]{r}{7} & E^d_{14}   \,.
\end{tikzcd}
$$
is given by:
\vskip 0.75truecm

\begin{tblr}{|c|c|c|}
\hline
 \SetCell[c=1]{c}\text{twisted isogeny graph} & \SetCell[c=1]{c}\text{condition} \SetCell[c=1]{c}\text{prob} \\
 \hline
\makecell{%
        \begin{tikzcd}[ampersand replacement=\&]
\circled[0.8]{$E_1^d$} \ar[r] 
   \ar[d]  \& 
    E_7^d  \ar[d]  \\
 E_2^d \ar[r]  \&    \, E_{14}^d
        \end{tikzcd}} 
        & d\not\equiv 0\,(7)& 7/8\\ 
\hline
   \makecell{%
        \begin{tikzcd}[ampersand replacement=\&]
E_1^d \ar[d] 
    \& 
    \circled[0.8]{$E_7^d$} \ar[l]   \ar[d]  \\
E_2^d  \&    \, E_{14}^d  \ar[l] 
        \end{tikzcd}} 
&  d\equiv 0\,(7) & 1/8 \\
\hline
\end{tblr}



\end{prop}

\vskip 0.35truecm

\noindnet{\it Proof.} From the previous tables one gets:

\vskip 0.5truecm

\begin{tblr}{cells={mode=imath},hlines,vlines,measure=vbox}
%-------------------------------------------------
\SetCell[c=1]{c} [u(E)(d)] & \SetCell[c=1]{c} d & \SetCell[c=1]{c}\text{Prob}\\
%-------------------------------------------------
 \SetCell[r=1]{c} \left(1:\frac{1}{2}:\frac{1}{7}:\frac{1}{14}\right) & d\not\equiv 0\,(7)&\SetCell[r=2]{c} \left(\frac{7}{8},0,\frac{1}{8},0\right) \\
  \SetCell[r=1]{c} \left(1:\frac{1}{2}:7:\frac{7}{2}\right) & d\equiv 0\,(7)& \\
%-------------------------------------------------
\end{tblr}

\end{document}


