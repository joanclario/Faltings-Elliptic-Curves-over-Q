\documentclass[11pt]{article}
%--------------------------------------------------------------------------
\usepackage{amsfonts,amssymb,amsmath,amsthm,latexsym,graphics,epsfig,amsfonts}
\usepackage{verbatim,enumerate,array,booktabs,color,bigstrut,prettyref,tikz-cd}
\usepackage{multirow}
\usepackage[all]{xy}
\usepackage[backref]{hyperref}
\usepackage[OT2,T1]{fontenc}
\usepackage{mathtools}
\usepackage{caption}
\usepackage{longtable}
\usepackage{mathtools}
\usepackage{tabularray}
\UseTblrLibrary{amsmath,varwidth}
\usepackage{tabularx}
\usepackage{longtable}
\usepackage{arydshln}
%--------------------------------------------------------------------------
\newcommand{\Mod}[1]{\ (\mathrm{mod}\ #1)}
\newcommand{\mathdash}{\relbar\mkern-8mu\relbar}
\newcommand*\circled[2][1.6]{\tikz[baseline=(char.base)]{
    \node[shape=circle, draw, inner sep=1pt, 
        minimum height={\f@size*#1},] (char) {\vphantom{WAH1g}#2};}}
\makeatother
%\addtolength{\textwidth}{4cm} 
\addtolength{\textheight}{3.5cm} 
\addtolength{\topmargin}{-2cm}
%\addtolength{\marginparwidth}{-2cm} 
\newcommand\myiso{\stackrel{\mathclap{\normalfont\mbox{\small $p$}}}{-}}
\newcommand\myisot{\stackrel{\mathclap{\normalfont\mbox{\small $3$}}}{-}}
\newcommand{\pref}[1]{\prettyref{#1}}
%------------------------------------
\newcommand{\Kd}{\operatorname{K}}
\newcommand{\kI}{\operatorname{I}}
\newcommand{\kII}{\operatorname{II}}
\newcommand{\kIII}{\operatorname{III}}
\newcommand{\kIV}{\operatorname{IV}}
%-------------------------------------
\newrefformat{eq}{\textup{(\ref{#1})}}
\newrefformat{prty}{\textup{(\ref{#1})}}

\definecolor{mylinkcolor}{rgb}{0.8,0,0}
\definecolor{myurlcolor}{rgb}{0,0,0.8}
\definecolor{mycitecolor}{rgb}{0,0,0.8}
\hypersetup{colorlinks=true,urlcolor=myurlcolor,citecolor=mycitecolor,linkcolor=mylinkcolor,linktoc=page,breaklinks=true}
%\DeclareSymbolFont{cyrletters}{OT2}{wncyr}{m}{n}
%\DeclareMathSymbol{\Sha}{\mathalpha}{cyrletters}{"58}
\addtolength{\textwidth}{4cm} \addtolength{\hoffset}{-2cm}
\addtolength{\marginparwidth}{-2cm}
%\theoremstyle{definition}
\newtheorem{defn}{Definition}[section]
\newtheorem{definition}[defn]{Definition}
\newtheorem{claim}[defn]{Claim}
%\theoremstyle{plain}
\newtheorem{thmA}{Theorem A}
\newtheorem{thmB}{Theorem B}
\newtheorem{thm2}{Theorem}
\newtheorem{prop2}{Proposition}
\newtheorem{note}{Note}
\newtheorem{corollary}[defn]{Corollary}
\newtheorem{lemma}[defn]{Lemma}
\newtheorem{property}[defn]{Property}
\newtheorem{thm}[defn]{Theorem}
\newtheorem{theorem}[defn]{Theorem}
\newtheorem{cor}[defn]{Corollary}
\newtheorem{prop}[defn]{Proposition}
\newtheorem{proposition}[defn]{Proposition}
\newtheorem{thmnn}{Theorem}
\newtheorem{conj}[defn]{Conjecture}
\theoremstyle{definition}
\newtheorem{remarks}{Remarks}
\newtheorem{ack}{Acknowledgements}
\newtheorem{remark}[defn]{Remark}
\newtheorem{question}[defn]{Question}
\newtheorem{example}[defn]{Example}
%-------------------------------------
\newcommand{\Q}{\mathbb Q}
\newcommand{\Qbar}{\overline{\Q}}
\newcommand{\Z}{\mathbb Z}
\newcommand{\modQ}{\,\text{mod}\,(\Q^)^2}
%-------------------------------------
\newcommand{\mysquare}[1]{\tikz{\path[draw] (0,0) rectangle node{\tiny #1} (8pt,8pt) ;}}
\newcommand{\mycircle}[1]{\tikz{\path[draw] (0,0) circle (4pt) node{\tiny #1};}}
%-------------------------------------
\begin{document}
\title{Type $L_2(17)$}
\date{\today}
\maketitle
%-------------------------------------
\section{Setting}
The isogeny graphs of type $L_2(17)$ are given by two isogenous elliptic curves:

\[ 
\begin{tikzcd}
E_1 \arrow[dash]{r}{17} & E_{17}\,.
\end{tikzcd}
\]

\noindent The modular curve $X_0(17)$ is elliptic of rank $0$. Its rational points are: two rational cusps and two non-cuspidal non-CM points $\tau,\tau'\in\mathbb H$. Their corresponding $j$-invariants are:
$$
j(\tau)=\frac{-17\cdot 373^{3}}{2^{17}},\qquad j(\tau')=\frac{-17^{2}\cdot 101^{3}}{2}.
$$
We have $j(17\tau)=j(\tau')$.

\

We can (and do) choose Weierstrass equations:
\vskip 0.5truecm
\begin{tblr}
{cells={mode=imath},hlines,vlines,measure=vbox,
colspec=clll}
E & \text{Minimal Weierstrass model} & j(E) &\text{label}\\
\hline
E_{1} & y^2 + xy = x^3 + x^2 - 660x - 7600 & \frac{-17\cdot 373^{3}}{2^{17}} & \texttt{14450n1}\\
E_{17} & y^2 + xy = x^3 + x^2 - 878710x + 316677750  & \frac{-17^{2}\cdot 101^{3}}{2} & \texttt{14450n2}\\
\end{tblr}
\vskip 0.5truecm

so that one has:
\vskip 0.5truecm

\begin{tblr}
{cells={mode=imath},hlines,vlines,measure=vbox,
colspec=cll}
%-------------------------------------------------
 E  & E_{1}& E_{17}\\
c_4(E) & 5\cdot 17\cdot 373 & 5\cdot 17^{4}\cdot 101\\
c_6(E) & 5^{2}\cdot 17\cdot 14891 &-5^{2}\cdot 17^{5}\cdot 7717 \\  
\Delta(E) & -2^{17}\cdot 5^{3}\cdot 17^{2}  & -2\cdot 5^{3}\cdot 17^{10}\\
\end{tblr}


\vskip 0.5truecm
\noindent and, with this choice, the isogeny graph is normalized. 

\

We have that the Faltings curve (circled)
in the graph is
\[
\begin{tblr}[mode=imath]{|c|c|}
\hline
 \circled[0.8]{$E_{1}$} \longrightarrow E_{17} \\
\hline
\end{tblr}
\]
Note that any $\mathbb Q$-isogeny class of elliptic curves of type $L_2(17)$ is isomorphic to $ 
\begin{tikzcd}
E_{1} \arrow[dash]{r}{17}  & E_{17} \,.
\end{tikzcd}
\newpage 


\section{Signatures, minimal models, Kodaira symbols, and Pal values}

\vskip 0.3truecm

$$
\begin{tblr}
%[caption = {$L_2(17)$ data for $p$=2}]
{cells = {mode=imath},hlines,vlines,measure=vbox,
hline{Z} = {1-3}{0pt},
vline{1} = {Y-Z}{0pt},
colspec  = clclccc}
%----------------------------------------------
\SetCell[c=7]{c} p=2  &  & & & \\ 
 E & \SetCell[c=1]{c} \operatorname{sig}_2(E)  & \SetCell[c=1]{c} \Kd_2(E) & \SetCell[c=3]{c} u_2(d)  \\
%----------------------------------------------
 E_1 & (0,0,17) & \kI_{17} & 1 & 1/2 & 1/2 \\
 E_{17} &(0,0,1)   & \kI_{1} & 1 & 1/2 & 1/2\\
%----------------------------------------------
 \SetCell[c=3,r=2]{c}  & & &  d\equiv 1 &  d\equiv 2  & d\equiv 3 \\
                       & & & \SetCell[c=3]{c} d \Mod{4}   \\
\end{tblr}
$$

\vskip 0.75truecm

$$
\begin{tblr}
%[caption = {$L_2(17)$ data for $p$=5}]
{cells = {mode=imath},hlines,vlines,measure=vbox,
colspec  = cllccc}
%----------------------------------------------
\SetCell[c=4]{c} p=5  & & & \\ 
 E & \SetCell[c=1]{c} \operatorname{sig}_5(E)  & \SetCell[c=1]{c} \Kd_5(E) & \SetCell[c=1]{c} u_5(d)  \\
%----------------------------------------------
 E_1 & (1,2,3)  & \kIII & 1  \\
 E_{17} & (1,2,3)  & \kIII & 1  \\
%----------------------------------------------
 \end{tblr}
$$

\vskip 0.75truecm

$$
\begin{tblr}
%[caption = {$L_2(17)$ data for $p$=11}]
{cells = {mode=imath},hlines,vlines,measure=vbox,
hline{Z} = {1-3}{0pt},
vline{1} = {Y-Z}{0pt},
colspec  = cllcc}
%----------------------------------------------
\SetCell[c=5]{c} p=17   & & &  \\ 
 E & \SetCell[c=1]{c} \operatorname{sig}_{17}(E)  & \SetCell[c=1]{c} \Kd_{17}(E) & \SetCell[c=2]{c} u_{17}(d)  \\
%----------------------------------------------
 E_{1} & (1,1,2)  & \kII & 1 & 1  \\
 E_{17} &(4,5,10)   & \kII^* & 17 & 1 \\
 %----------------------------------------------
 \SetCell[c=3,r=2]{c}   & & &  d\equiv 0 &  d\not\equiv 0  \\
                        & & & \SetCell[c=2]{c} d \Mod{17}   \\
\end{tblr}
$$

\vskip 0.75truecm

%\newpage
\section{Conclusion}

\begin{prop}
%Let 
%$ 
%\begin{tikzcd}
%E_1 \arrow[dash]{r}{11}  & E_{11} 
%\end{tikzcd}
%$
%be a $\mathbf{Q}$-isogeny graph of type $L_{2}(11)$. 
For every square-free integer $d$, 
the probability of a vertex
to be the Faltings curve (circled)
in the twisted isogeny graph 
$
\begin{tikzcd} 
E_{1}^d \arrow[dash]{r}{17}  & E_{17}^d
\end{tikzcd}
$ 
is given by:

\[
\begin{tblr}[mode=imath]{|c|c|c|}
\hline
\text{twisted isogeny graph} & \text{condition}  &\text{Prob} \\
\hline
 \circled[0.8]{$E_{1}^d$} \longrightarrow E_{17}^d  & d\not\equiv 0\,(17) & 17/18 \\
\hline
 E_{1}^d \longrightarrow \circled[0.8]{$E_{17}^d$}  &  d\equiv 0\,(17) & 1/18 \\
\hline
\end{tblr}
\]



\end{prop}

\vskip 0.35truecm

\noindnet{\it Proof.} From the previous tables one gets:

\vskip 0.5truecm

\begin{tblr}{cells={mode=imath},hlines,vlines,measure=vbox}
%-------------------------------------------------
 \SetCell[c=1]{c} [u(E)(d)] & \SetCell[c=1]{c} d & \SetCell[c=1]{c}\text{Prob}\\
%-------------------------------------------------
\SetCell[r=1]{c} (1:1) & d\not\equiv 0\,(17)&\SetCell[r=2]{c} \left(\frac{17}{18},\frac{1}{18}\right) \\
  \SetCell[r=1]{c} (1:17) & d\equiv 0\,(17)& \\
%-------------------------------------------------
\end{tblr}

\end{document}



