\documentclass[11pt]{article}
%--------------------------------------------------------------------------
\usepackage{amsfonts,amssymb,amsmath,amsthm,latexsym,graphics,epsfig,amsfonts}
\usepackage{verbatim,enumerate,array,booktabs,color,bigstrut,prettyref,tikz-cd}
\usepackage{multirow}
\usepackage[all]{xy}
\usepackage[backref]{hyperref}
\usepackage[OT2,T1]{fontenc}
\usepackage{mathtools}
\usepackage{caption}
\usepackage{longtable}
\usepackage{mathtools}
\usepackage{tabularray}
\UseTblrLibrary{amsmath,varwidth}
\usepackage{tabularx}
\usepackage{longtable}
\usepackage{arydshln}
%--------------------------------------------------------------------------
\newcommand{\Mod}[1]{\ (\mathrm{mod}\ #1)}
\newcommand{\mathdash}{\relbar\mkern-8mu\relbar}
\newcommand*\circled[2][1.6]{\tikz[baseline=(char.base)]{
    \node[shape=circle, draw, inner sep=1pt, 
        minimum height={\f@size*#1},] (char) {\vphantom{WAH1g}#2};}}
\makeatother
%\addtolength{\textwidth}{4cm} 
\addtolength{\textheight}{3.5cm} 
\addtolength{\topmargin}{-2cm}
%\addtolength{\marginparwidth}{-2cm} 
\newcommand\myiso{\stackrel{\mathclap{\normalfont\mbox{\small $p$}}}{-}}
\newcommand\myisot{\stackrel{\mathclap{\normalfont\mbox{\small $3$}}}{-}}
\newcommand{\pref}[1]{\prettyref{#1}}
%------------------------------------
\newcommand{\Kd}{\operatorname{K}}
\newcommand{\kI}{\operatorname{I}}
\newcommand{\kII}{\operatorname{II}}
\newcommand{\kIII}{\operatorname{III}}
\newcommand{\kIV}{\operatorname{IV}}
%-------------------------------------
\newrefformat{eq}{\textup{(\ref{#1})}}
\newrefformat{prty}{\textup{(\ref{#1})}}

\definecolor{mylinkcolor}{rgb}{0.8,0,0}
\definecolor{myurlcolor}{rgb}{0,0,0.8}
\definecolor{mycitecolor}{rgb}{0,0,0.8}
\hypersetup{colorlinks=true,urlcolor=myurlcolor,citecolor=mycitecolor,linkcolor=mylinkcolor,linktoc=page,breaklinks=true}
%\DeclareSymbolFont{cyrletters}{OT2}{wncyr}{m}{n}
%\DeclareMathSymbol{\Sha}{\mathalpha}{cyrletters}{"58}
\addtolength{\textwidth}{4cm} \addtolength{\hoffset}{-2cm}
\addtolength{\marginparwidth}{-2cm}
%\theoremstyle{definition}
\newtheorem{defn}{Definition}[section]
\newtheorem{definition}[defn]{Definition}
\newtheorem{claim}[defn]{Claim}
%\theoremstyle{plain}
\newtheorem{thmA}{Theorem A}
\newtheorem{thmB}{Theorem B}
\newtheorem{thm2}{Theorem}
\newtheorem{prop2}{Proposition}
\newtheorem{note}{Note}
\newtheorem{corollary}[defn]{Corollary}
\newtheorem{lemma}[defn]{Lemma}
\newtheorem{property}[defn]{Property}
\newtheorem{thm}[defn]{Theorem}
\newtheorem{theorem}[defn]{Theorem}
\newtheorem{cor}[defn]{Corollary}
\newtheorem{prop}[defn]{Proposition}
\newtheorem{proposition}[defn]{Proposition}
\newtheorem{thmnn}{Theorem}
\newtheorem{conj}[defn]{Conjecture}
\theoremstyle{definition}
\newtheorem{remarks}{Remarks}
\newtheorem{ack}{Acknowledgements}
\newtheorem{remark}[defn]{Remark}
\newtheorem{question}[defn]{Question}
\newtheorem{example}[defn]{Example}
%-------------------------------------
\newcommand{\Q}{\mathbb Q}
\newcommand{\Qbar}{\overline{\Q}}
\newcommand{\Z}{\mathbb Z}
\newcommand{\modQ}{\,\text{mod}\,(\Q^)^2}
%-------------------------------------
\newcommand{\mysquare}[1]{\tikz{\path[draw] (0,0) rectangle node{\tiny #1} (8pt,8pt) ;}}
\newcommand{\mycircle}[1]{\tikz{\path[draw] (0,0) circle (4pt) node{\tiny #1};}}
%-------------------------------------
\begin{document}
\title{Type $L_2(11)$}
\date{\today}
\maketitle
%-------------------------------------
\section{Setting}
The isogeny graphs of type $L_2(11)$ are given by two isogenous elliptic curves:

\[ 
\begin{tikzcd}
E_1 \arrow[dash]{r}{11} & E_{11}\,.
\end{tikzcd}
\]

\noindent The modular curve $X_0(11)$ is an elliptic curve of rank $0$ over the rationals. More precisely, we can choose the Weierstrass model $y^2 + y = x^3 - x^2 - 10\, x - 20$ for $X_0(11)$. 
The $j$-forgetful map $j\colon X_0(11) \to X_0(1)$ is given by
$$
j=
\frac{P(x)+y\,Q(x)}
   {\left(-17
   x^2-x y-243 x-105
   y+859\right)^3}
$$
with
$$
\begin{array}{ll}
 P(x) =   & 
 x^8+160170 x^7 +22013817 x^6 -1234891244 x^5    +18403682346 x^4  -145947253957x^3\\[6pt]
     & +1422949497947x^2 +5880426893238 x+7325611514413 \\[6pt]
Q(x)=&
   -692 x^6 
   -12510792 x^5  
   +815793738 x^4
   -17947463042x^3 
   +112966993208 x^2 \\[6pt]
   &
   +491634446704 x
   -468196759663
\end{array}
$$
One has
$$
X_0(11)(\Q)=
\{
(0 : 1 : 0), (5 : -6 : 1), (5 : 5 : 1), (16 : -61 : 1), (16 : 60 : 1)\}
$$
and thus:
$$
j((0 : 1 : 0))= \infty\,,\quad
j((16 : -61 : 1))= \infty\,,
$$
$$
j((16 : 60 : 1))= -11^2 \,,\quad
j((5 : -6 : 1))= - 11 \cdot 131^3 \,,\quad
j((5 : 5 : 1))= - 2^{15} \,.
$$
Its rational points are: two rational cusps
$(\infty)=(0:1:0)$, $(0)=(16:-61:1)$, one rational CM point $(5 : 5 : 1)$ that corresponds to a $\tau_b=\frac{1}{2}+\frac{\sqrt{-11}}{2\cdot 11}\in\mathbb H$ and two non-cuspidal non-CM points 
$(16 : 60 : 1)$ and 
$(5 : -6 : 1)$ that correspond to
$\tau_a=0.5+0.09227...i$, and $\tau'_a=0.5+0.24630...i\in\mathbb H$. That is
$$
j(\tau_b)=-2^{15},\qquad j(\tau_a)=-11^{2},\qquad j(\tau'_a)=-11\cdot 131^{3},
$$
and we have $j(11\tau_b)=j(\tau_b)$ and $j(11\tau_a)=j(\tau'_a)$.

\

We choose Weierstrass equations:
\vskip 0.5truecm
\begin{tblr}
{cells={mode=imath},hlines,vlines,measure=vbox,
colspec=clll}
E & \text{Minimal Weierstrass model} & j(E) &\text{label}\\
\hline
E_{1_a} & y^2+xy+y=x^3+x^2-30x-76 & -11\cdot 131^{3} & \texttt{121a1}\\
E_{11_a} & y^2+xy+y=x^3+x^2-305x+7888 & -11^{2} & \texttt{121a2}\\
\hline
E_{1_b} & y^2+y=x^3-x^2-7x+10 & -2^{15} & \texttt{121b1}\\
E_{11_b} & y^2+y=x^3-x^2-887x-10143 & -2^{15} & \texttt{121b2}\\
\end{tblr}
\vskip 0.5truecm

so that one has:
\vskip 0.5truecm

\begin{tblr}{cc}
\begin{tblr}
{cells={mode=imath},hlines,vlines,measure=vbox,
colspec=cll}
%-------------------------------------------------
 E  & E_{1_a}& E_{11_a}\\
c_4(E) & 11\cdot 131 & 11^{4}\\
c_6(E) & 11\cdot 4973 & -11^{5}\cdot 43\\  
\Delta(E) & -11^{2} & -11^{10}\\
\end{tblr}
&
\begin{tblr}
{cells={mode=imath},hlines,vlines,measure=vbox,
colspec=cll}
 E  & E_{1_b} &  E_{11_b} \\
\hline 
c_4(E) & 2^{5}\cdot 11
 & 2^{5}\cdot 11^{3}\\
c_6(E) & -2^{3}\cdot 7\cdot 11^{2}
 & 2^{3}\cdot 7\cdot 11^{5}  \\  
\Delta(E) & -11^{3}
 & -11^{9} \\
\end{tblr}
\end{tblr}

\vskip 0.5truecm
\noindent and, with this choice, the
corresponding isogenies are normalized. 

\

For $k\in \{a,b\}$, we have that the Faltings curve (circled)
in the graph is
\[
\begin{tblr}[mode=imath]{|c|c|}
\hline
 \circled[0.8]{$E_{1_k}$} \longrightarrow E_{11_k} \\
\hline
% \circled[0.8]{$E'_1$} \longrightarrow E'_{11} \\
%\hline
\end{tblr}
\]
Note that any $\mathbb Q$-isogeny class of elliptic curves of type $L_2(11)$ is isomorphic to $ 
\begin{tikzcd}
E_{1_k} \arrow[dash]{r}{11}  & E_{11_k} 
\end{tikzcd}
$ for some  $k\in \{a,b\}$.
%or to $ 
%\begin{tikzcd}
%E'_1 \arrow[dash]{r}{11}  & E'_{11} 
%\end{tikzcd}
%$.

Note that $E_{1_b}$ and $E_{11_b}$ have complex multiplication by the ring of integers of $\mathbf{Q}(\sqrt{-11})$ and $E_{11_b}=E_{1_b}^{-11}$.


\section{Signatures, minimal models, Kodaira symbols, and Pal values}

\vskip 0.3truecm

$$
\begin{tblr}
%[caption = {$L_2(11)$ data for $p$=11}]
{cells = {mode=imath},hlines,vlines,measure=vbox,
hline{Z} = {1-3}{0pt},
vline{1} = {Y-Z}{0pt},
colspec  = cllcc}
%----------------------------------------------
\SetCell[c=5]{c} p=11   & & &  \\ 
 E & \SetCell[c=1]{c} \operatorname{sig}_{11}(E)  & \SetCell[c=1]{c} \Kd_{11}(E) & \SetCell[c=2]{c} u_{11}(d)  \\
%----------------------------------------------
 E_{1_a} & (1,1,2)  & \kII & 1 & 1  \\
 E_{11_a} &(4,5,10)   & \kII^* & 11 & 1 \\
 %----------------------------------------------
\hline
 E_{1_b} & (1,2,3)   & \kIII& 1 & 1\\
 E_{11_b} & (3,5,9)   & \kIII^* & 11 & 1  \\
%----------------------------------------------
 \SetCell[c=3,r=2]{c}   & & &  d\equiv 0 &  d\not\equiv 0  \\
                        & & & \SetCell[c=2]{c} d \Mod{11}   \\
\end{tblr}
$$

\vskip 0.75truecm

\newpage

\section{Conclusion}

\begin{prop}
%Let 
%$ 
%\begin{tikzcd}
%E_1 \arrow[dash]{r}{11}  & E_{11} 
%\end{tikzcd}
%$
%be a $\mathbf{Q}$-isogeny graph of type $L_{2}(11)$. 
Let $k\in\{a,b\}$. For every square-free integer $d$, 
the probability of a vertex
to be the Faltings curve (circled)
in the twisted isogeny graph 
$
\begin{tikzcd} 
E_{1_k}^d \arrow[dash]{r}{11}  & E_{11_k}^d
\end{tikzcd}
$ 
is given by:

\[
\begin{tblr}[mode=imath]{|c|c|c|}
\hline
\text{twisted isogeny graph} & \text{condition}  &\text{Prob} \\
\hline
 \circled[0.8]{$E_{1_k}^d$} \longrightarrow E_{11_k}^d  & d\not\equiv 0\,(11) & 11/12 \\
\hline
 E_{1_k}^d \longrightarrow \circled[0.8]{$E_{11_k}^d$}  &  d\equiv 0\,(11) & 1/12 \\
\hline
\end{tblr}
\]



\end{prop}

\vskip 0.35truecm

\noindnet{\it Proof.} From the previous tables one gets:

\vskip 0.5truecm

\begin{tblr}{cells={mode=imath},hlines,vlines,measure=vbox}
%-------------------------------------------------
 \SetCell[c=1]{c} [u(E)(d)] & \SetCell[c=1]{c} d & \SetCell[c=1]{c}\text{Prob}\\
%-------------------------------------------------
 \SetCell[r=1]{c} (1:1) & d\not\equiv 0\,(11)&\SetCell[r=2]{c} \left(\frac{11}{12},\frac{1}{12}\right) \\
  \SetCell[r=1]{c} (1:11) & d\equiv 0\,(11)& \\
%-------------------------------------------------
\end{tblr}

\end{document}


\begin{longtblr}
[caption = {$L_2(11)$ data for $p$=2}]
{cells = {mode=imath},hlines,vlines,measure=vbox,
hline{Z} = {1-4}{0pt},
vline{1} = {Y-Z}{0pt},
colspec  = clclccc}
%----------------------------------------------
\SetCell[c=7]{c} p=2  & & & & & \\ 
 E & \SetCell[c=1]{c} \operatorname{sig}_2(E) & u & \SetCell[c=1]{c} \Kd_2(E) & \SetCell[c=3]{c} u_2(d)  \\
%----------------------------------------------
 E_1 & (0,0,0) & 1 & I_{0} & 1 & 1/2 & 1/2 \\
 E_{11} &(0,0,0)  & 1 & I_{0} & 1 & 1/2 & 1/2\\
 %----------------------------------------------
\hline
 E'_1 & (0,0,0)  & 1 & I_{0}& 1 & 1/2 & 1/2\\
 E'_{11} & (0,0,0)  & 1 & I_{0} & 1 & 1/2 & 1/2 \\
%----------------------------------------------
 \SetCell[c=4,r=2]{c}  & & & &  d\equiv 1 &  d\equiv 2  & d\equiv 3 \\
                       & & & & \SetCell[c=3]{c} d \Mod{4}   \\
\end{longtblr}

