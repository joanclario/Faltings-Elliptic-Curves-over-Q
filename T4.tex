\documentclass[11pt]{article}
\usepackage{amsfonts,amssymb,amsmath,amsthm,latexsym,graphics,epsfig,amsfonts}
\usepackage{verbatim,enumerate,array,booktabs,color,bigstrut,prettyref,tikz-cd}
\usepackage{multirow}
\usepackage[all]{xy}
\usepackage[backref]{hyperref}
\usepackage[OT2,T1]{fontenc}
%\usepackage{ctable}
\usepackage{mathtools}

\usepackage{graphicx}


\usepackage{longtable}


\usepackage{mathtools}
\newcommand{\Mod}[1]{\ (\mathrm{mod}\ #1)}
\newcommand{\mathdash}{\relbar\mkern-8mu\relbar}

\newcommand*\circled[2][1.6]{\tikz[baseline=(char.base)]{
    \node[shape=circle, draw, inner sep=1pt, 
        minimum height={\f@size*#1},] (char) {\vphantom{WAH1g}#2};}}
\makeatother



\usepackage{tabularray}
\UseTblrLibrary{amsmath,varwidth}

\usepackage{tabularx}
\usepackage{longtable}
\usepackage{arydshln}


\newcommand\myiso{\stackrel{\mathclap{\normalfont\mbox{\small $p$}}}{-}}
\newcommand\myisot{\stackrel{\mathclap{\normalfont\mbox{\small $3$}}}{-}}

\newcommand{\pref}[1]{\prettyref{#1}}
\newrefformat{eq}{\textup{(\ref{#1})}}
\newrefformat{prty}{\textup{(\ref{#1})}}

\definecolor{mylinkcolor}{rgb}{0.8,0,0}
\definecolor{myurlcolor}{rgb}{0,0,0.8}
\definecolor{mycitecolor}{rgb}{0,0,0.8}
\hypersetup{colorlinks=true,urlcolor=myurlcolor,citecolor=mycitecolor,linkcolor=mylinkcolor,linktoc=page,breaklinks=true}

%\DeclareSymbolFont{cyrletters}{OT2}{wncyr}{m}{n}
%\DeclareMathSymbol{\Sha}{\mathalpha}{cyrletters}{"58}

\addtolength{\textwidth}{4cm} \addtolength{\hoffset}{-2cm}
\addtolength{\marginparwidth}{-2cm}

%\theoremstyle{definition}
\newtheorem{defn}{Definition}[section]
\newtheorem{definition}[defn]{Definition}
\newtheorem{claim}[defn]{Claim}

%\theoremstyle{plain}
\newtheorem{thmA}{Theorem A}
\newtheorem{thmB}{Theorem B}
\newtheorem{thm2}{Theorem}
\newtheorem{prop2}{Proposition}
\newtheorem{note}{Note}

\newtheorem{corollary}[defn]{Corollary}
\newtheorem{lemma}[defn]{Lemma}
\newtheorem{property}[defn]{Property}
\newtheorem{thm}[defn]{Theorem}
\newtheorem{theorem}[defn]{Theorem}
\newtheorem{cor}[defn]{Corollary}
\newtheorem{prop}[defn]{Proposition}
\newtheorem{proposition}[defn]{Proposition}
\newtheorem{thmnn}{Theorem}
\newtheorem{conj}[defn]{Conjecture}

\theoremstyle{definition}
\newtheorem{remarks}{Remarks}
\newtheorem{ack}{Acknowledgements}
\newtheorem{remark}[defn]{Remark}
\newtheorem{question}[defn]{Question}
\newtheorem{example}[defn]{Example}


\newcommand{\Q}{\mathbb Q}
\newcommand{\Qbar}{\overline{\Q}}
\newcommand{\Z}{\mathbb Z}

\newcommand{\modQ}{\,\text{mod}\,(\Q^)^2}

\newcommand{\mysquare}[1]{\tikz{\path[draw] (0,0) rectangle node{\tiny #1} (8pt,8pt) ;}}
\newcommand{\mycircle}[1]{\tikz{\path[draw] (0,0) circle (4pt) node{\tiny #1};}}


%------------------------------------
\newcommand{\Kd}{\operatorname{K}}
\newcommand{\kI}{\operatorname{I}}
\newcommand{\kII}{\operatorname{II}}
\newcommand{\kIII}{\operatorname{III}}
\newcommand{\kIV}{\operatorname{IV}}
%-------------------------------------



\begin{document}
\title{Type $T_4$}
\date{\today}
\maketitle
%\section{Setting}
The isogeny graphs of type $T_4$ are given by
four isogenous elliptic curves:

\[ \begin{tikzcd}
& E_1 \ar[dash,d,"2"] & \\
& E_2 \ar[dash,ld,swap,"2"] \ar[dash,rd,"2"] & \\
E_4   & &  E_{12}   \,.
\end{tikzcd}
\]
 
%For $i=1,2,4,3/2$:
%$$
%E_i:y^2+A_i x+B_i,\
%$$
%
%
%$$
%\begin{tblr}[mode=dmath]{|c|l|}
% \hline 
%	A_1 & 	\\
%	B_1 &  \\
%\hline
%	A_2 & 	\\
%	B_2 &  \\
%\hline
%	A_4 & 	\\
%	B_4 &  \\
%\hline
%	A_{1,1/2} & 	\\
%	B_{1,1/2} &  \\
%\hline
%\end{tblr}
%$$

\noindent A hauptmodul for $X_0(4)$ is  
$$
t = 2^8 \displaystyle{\left(\frac{\eta(4\tau)}{\eta(\tau)}\right)^8}\,.
$$
One has
$$
\begin{tblr}{l@{\,=\,}l}
j_1= j(E_1) = j(\tau) & 
t^{-1} \cdot (t + 16)^{-1} \cdot (t^{2} + 16 t + 16)^{3}\\
j_2 = j(E_2) = j(2\tau) & 
t^{-2} \cdot (t + 16)^{-2} \cdot (t^{2} + 16 t + 256)^{3}
\\
j_4 = j(E_4) = j(4\tau) & 
t^{-4} \cdot (t + 16)^{-1} \cdot (t^{2} + 256 t + 4096)^{3}\\
j_{12} = j(E_{12}) = j(\tau+1/2) & 
\left(-1\right) \cdot (t + 16)^{-4} \cdot t^{-1} \cdot (t^{2} - 224 t + 256)^{3}\,,
\end{tblr}
$$
and the subgroup of $\operatorname{Aut} X_0(4)$ that fixes the set of vertices of the graph is
isomorphic to the symmetric group $\mathcal{S}_3$ with elements:

$$
\begin{tblr}{l@{\,=\,}lcc}
    \SetCell[c=3]{r} \text{permutation} & & & \text{order}  \\
   \operatorname{id}(t) & t  &  (j_1,j_2,j_4,j_{12}) & 1 \\
   \sigma(t) & -256/(t+16) & (j_{12},j_2,j_1,j_4) & 3 \\
   \sigma^2(t) & -16\,(t+16)/t & (j_4,j_2,j_{12},j_1) & 3 \\ 
   \tau(t) & 256/t & (j_4,j_2,j_4,j_{12}) & 2 \\
   \sigma \tau(t) & -(t+16) &   (j_1,j_2,j_4,j_{12})               & 2 \\
   \sigma^2 \tau(t) & -16\, t/(t+16) &  (j_{12},j_2,j_4,j_1)                 & 2 \\
\end{tblr}
$$

%------------------------------------------------------
\begin{comment}

For $t$ in $\Q\setminus \{0,-16\}$, the $p$-adic valuations of the Fricke involutions applied to $t$ are:

\begin{tblr}
{cells={mode=imath},colspec=|c|c|c|c|}
\hline
p & v_p(W_2(t)) & v_p(W_3(t)) & v_p(W_6(t))  \\
\hline
\neq 2,3 & 0 & 0 & - v_p(t) \\
\hline
 3  & v_3(t+9)-v_3(t+8) & 2 + v_3(t+8)-v_3(t+9) & 2-v_3(t) \\
\hline
 2  & 3+v_2(t+9)+v_2(t+8) & v_2(t+8)-v_2(t+9) & 3-v_2(t) \\
\hline
\end{tblr}

\end{comment}
%------------------------------------------------------

\vskip 0.5truecm

We can (and do) choose Weierstrass equations for $(E_1,E_2,E_4,E_{12})$ such that their signatures are:

$$
\begin{tblr}
{cells={mode=imath},colspec=|c|l|}
 \hline 
	c_4(E_1) & 	\left(t^2+16 \, t+16\right)\\
	c_6(E_1) & (t+8) \left(t^2+16\, t-8\right)\\
	\Delta(E_1) & t\, (t+16)\\
\hline
	c_4(E_2) & 	\left(t^2+16 \, t+256\right) \\
	c_6(E_2) & (t-16) (t+8) (t+32) \\
	\Delta(E_2) &  t^2 (t+16)^2\\
\hline
	c_4(E_4) & 	 \left(t^2+256 \, t+4096\right) \\
	c_6(E_4) & (t+32) \left(t^2-512\, t-8192\right) \\
	\Delta(E_4) &  t^4 (t+16) \\
\hline
	c_4(E_{12}) &  \left(t^2-224\, t+256\right)	\\
	c_6(E_{12}) & (t-16) \left(t^2+544\, t+256\right) \\
	\Delta(E_{12}) & - t\,  (t+16)^4\\
\hline
\end{tblr}
$$


\newpage

%\section{Kodaira symbols \& Pal coefficients}

\begin{longtblr}
[caption = {$T_4$ data for $p\neq 2$}]
{cells = {mode=imath},hlines,vlines,measure=vbox,
hline{Z} = {1-5}{0pt},
vline{1} = {Y-Z}{0pt},
colspec  = cclclcc}
%--------------------------------------
\SetCell[c=1]{c} T_4 &\SetCell[c=6]{c} p\neq 2  & & & & & \\ t & E & 
\SetCell[c=1]{c} \operatorname{sig}_p(E) & u & \Kd_p(E) & \SetCell[c=2]{c} u_p(d)   \\
%--------------------------------------
\SetCell[r=4]{c} 
m= v_p(t)>0 
& E_1 & (0,0,m) & 1 &  \kI_m & 1 & 1\\
& E_2 & (0,0,2m) & 1 &  \kI_{2m} & 1 & 1 \\
& E_4 & (0,0,4m) & 1 &  \kI_{4m}  & 1& 1 \\
& E_{12} & (0,0,m) & 1 &  \kI_{m}  & 1& 1 \\
\SetCell[r=4]{c} 
\begin{array}{c}
     v_p(t)=0  \\[3pt]
    m = v_p(t+16)>0 
\end{array}
& E_1 & (0,0,m) & 1 &  \kI_{m}  & 1 & 1 \\
& E_2 & (0,0,2m) & 1 &  \kI_{2m} & 1& 1 \\
& E_4 & (0,0,m) & 1 &  \kI_{m} & 1& 1 \\
& E_{12} & (0,0,4m) & 1 &  \kI_{4m}  & 1& 1 \\
\SetCell[r=4]{c} 
\begin{array}{c}
     -m= v_p(t)<0  \\[3pt]
    m \text{ odd} 
\end{array}
& E_1 & (2,3,4m+6) & p^{-(m+1)/2} & \kI_{4m}^*  & p& 1 \\
& E_2 & (2,3,2m+6) & p^{-(m+1)/2} &  \kI_{2m}^* & p& 1 \\
& E_4 & (2,3,m+6) & p^{-(m+1)/2}&  \kI_{m}^* & p& 1 \\
& E_{12} & (2,3,m+6) & p^{-(m+1)/2} &  \kI_{m}^* & p& 1 \\
\SetCell[r=4]{c} 
\begin{array}{c}
     -m= v_p(t)<0  \\[3pt]
    m \text{ even} 
\end{array}
& E_1 & (0,0,4m) & p^{-m/2} & \kI_{4m} & 1& 1 \\
& E_2 & (0,0,2m) & p^{-m/2} &  \kI_{2m} & 1& 1 \\
& E_4 & (0,0,m) & p^{-m/2}&  \kI_{m} & 1& 1 \\
& E_{12} & (0,0,m) & p^{-m/2} &  \kI_{m} & 1& 1 \\
%-------------------------------------------------
 \SetCell[c=5,r=2]{c} & & & & & d\equiv 0  & d\not\equiv 0 \\
                      & & & & & \SetCell[c=2]{c} d \Mod p & \\
\end{longtblr}

\newpage


\begin{longtblr}
[caption = {$T_4$ data for $p$=2}]
{cells = {mode=imath},hlines,vlines,measure=vbox,
hline{Z} = {1-5}{0pt},
vline{1} = {Y-Z}{0pt},
colspec  = cclclccc}
%--------------------------------------
\SetCell[c=1]{c} T_4 &\SetCell[c=7]{c} p=2  & & & &  & & \\
\SetCell[c=1]{c} t & E & 
\SetCell[c=1]{c}\operatorname{sig}_2(E) & u & \Kd_2(E) & \SetCell[c=3]{c} u_2(d) & & \\
%-------------------------------------------
\SetCell[r=4]{c} 
m= v_2(t)>7 
& E_1 & (0,0,m-8) & 2 &  \kI_{m-8} & 1 & 2^{-1} & 2^{-1}\\
& E_2 & (0,0,2(m-8)) & 2^2 &  \kI_{2(m-8)} & 1 &  2^{-1} & 2^{-1} \\
& E_4 & (0,0,4(m-8)) & 2^3 &  \kI_{4(m-8)}  & 1&  2^{-1} & 2^{-1} \\
& E_{12} & (0,0,m-8) & 2^2 &  \kI_{m-8}  & 1&  2^{-1} & 2^{-1} \\
\SetCell[r=4]{c} 
v_2(t)=7 
& E_1 & (4,6,11) & 1 &  \kII^* & 1& 1 & 1 \\
& E_2 & (4,6,10) & 2 &  \kIII^* & 1 & 1 & 1 \\
& E_4 & (4,6,8) & 2^2 &  \kI_1^*  & 1& 1 & 1 \\
& E_{12} & (4,6,11) & 2 &  \kII^*  & 1& 1 & 1 \\
\SetCell[r=4]{c} 
v_2(t)=6 
& E_1 & (4,6,10) & 1 &  \kIII^* & 1& 1 & 1 \\
& E_2 & (4,6,8) & 2 &  \kI_1^* & 1 & 1 & 1 \\
& E_4 & (5,5,4) & 2^2 &  \kIII  & 1& 1 & 1 \\
& E_{12} & (4,6,10) & 2 &  \kIII^*  & 1& 1 & 1 \\
\SetCell[r=4]{c} 
v_2(t)=5 
& E_1 & (4,6,9) & 1 &  \kI_0^* & 1& 1 & 1 \\
& E_2 & (4,\ge 7,6) & 2 &  \kIII & 1 & 1 & 1 \\
& E_4 & (6,\ge 10,12) & 2 &  \kI_3^*  & 1&  2 & 1 \\
& E_{12} & (4,6,9) & 2 &  \kI_0^*  & 1& 1 & 1 \\
%----------------------------------------------
\SetCell[r=4]{c} 
\begin{array}{c}
v_2(t)=4  \\
t/2^4\equiv 1,-3\,(8)
\end{array}
& E_1 & (4,6,9) & 1 &  \kI_0^* & 1& 1 & 1 \\
& E_2 & (4,\ge 7,6) & 2 &  \kIII & 1 & 1 & 1 \\
& E_4 & (4,6,9) & 2 &  \kI_0^*  & 1& 1 & 1 \\
& E_{12} & (6,\ge 10,12) & 2 &  \kI_3^*  & 1& 2 & 1 \\
%----------------------------------------------

\SetCell[r=4]{c} 
\begin{array}{c}
v_2(t)=4 \\[3pt]
t/2^4\equiv -1\,(8)\\[3pt]
v_2(t+16)=m+4
\end{array}
& E_1 & (4,6,8+m) & 1 &  \kI_{m}^* & 1& 1 & \begin{array}{c}
\text{$2$ if $m\ge 4$}\\
\text{$1$ if $m<4$}
\end{array}\\
& E_2 & (4,6,2m+4) & 2 &  \kI_{2m-4}^* & 1 & 1 &
\begin{array}{c}
\text{$2$ if $m\ge 4$}\\
\text{$1$ if $m<4$}
\end{array} \\
& E_4 & (4,6,8+m) & 2 &  \kI_{m}^*  & 1& 1 & \begin{array}{c}
\text{$2$ if $m\ge 4$}\\
\text{$1$ if $m<4$}
\end{array} \\
& E_{12} & (4,6,4m-4) & 2^2 &  \kI_{4m-12}^*  & 1& 1 & \begin{array}{c}
\text{$2$ if $m\ge 4$}\\
\text{$1$ if $m<4$}
\end{array} \\
%----------------------------------------------

\SetCell[r=4]{c} 
\begin{array}{c}
v_2(t)=4  \\
t/2^4\equiv 3\,(8)
\end{array}
& E_1 & (4,6,10) & 1 &  \kI_2^* & 1& 1 & 1 \\
& E_2 & (4,6,8) & 2 &  \kI_0^* & 1 & 1 & 1 \\
& E_4 & (4,6,10) & 2 &  \kI_2^*  & 1& 1 & 1 \\
& E_{12} & (5,5,4) & 2^2 &  \kII  & 1& 1 & 1 \\
%----------------------------------------------

\SetCell[r=4]{c} 
v_2(t)=3
& E_1 & (4,\ge 7,6) & 1 &  \kII & 1& 1 & 1 \\
& E_2 & (6,\ge 10,12) & 1 &  \kI_2^* & 1 & 2 & 1 \\
& E_4 & (6,9,15) & 1 &  \kI_5^*  & 1& 2 & 1 \\
& E_{12} & (6,9,15) & 1 &  \kI_5^*  & 1& 2 & 1 \\
%----------------------------------------------
\SetCell[r=4]{c} 
\begin{array}{c}
v_2(t)=2\\
t/2^2\equiv 1\,(4)
\end{array}
& E_1 & (5,5,4) & 1 &  \kII & 1& 1 & 1 \\
& E_2 & (4,6,8) & 1 &  \kI_0^* & 1 & 1 & 1 \\
& E_4 & (4,6,10) & 1 &  \kI_2^*  & 1& 1 & 1 \\
& E_{12} & (4,6,10) & 1 &  \kI_2^*  & 1& 1 & 1 \\
%----------------------------------------------
\SetCell[r=4]{c} 
\begin{array}{c}
v_2(t)=2\\
t/2^2\equiv 3\,(4)
\end{array}
& E_1 & (5,5,4) & 1 &  \kIII & 1& 1 & 1 \\
& E_2 & (4,6,8) & 1 &  \kI_1^* & 1 & 1 & 1 \\
& E_4 & (4,6,10) & 1 &  \kIII^*  & 1& 1 & 1 \\
& E_{12} & (4,6,10) & 1 &  \kIII^*  & 1& 1 & 1 \\
%----------------------------------------------
\SetCell[r=4]{c} 
v_2(t)=1
& E_1 & (6,9,14) & 2^{-1} &  \kI_4^* & 1& 2 & 1 \\
& E_2 & (6,9,16) & 2^{-1} &  \kI_6^* & 1 & 2 & 1 \\
& E_4 & (6,9,17) & 2^{-1} &  \kI_7^*  & 1& 2 & 1 \\
& E_{12} & (6,9,17) & 2^{-1} &  \kI_7^*  & 1& 2 & 1 \\
%----------------------------------------------
\SetCell[r=4]{c} 
\begin{array}{c}
     -2m= v_p(t)\le 0  \\[3pt]
   2^{2m} t \equiv 1 \, (4) 
\end{array}
& E_1 & (4,6,12+8m) & 2^{-(m+1)} & \kI_{4+8m}^* & 1& 1 & 2 \\
& E_2 & (4,6,12+4m) & 2^{-(m+1)} &  \kI_{4+4m}^* & 1& 1 & 2 \\
& E_4 & (4,6,12+2m) & 2^{-(m+1)} &  \kI_{4+2m}^* & 1& 1 & 2 \\
& E_{12} & (4,6,12+2m) & 2^{-(m+1)} &  \kI_{4+2m}^* & 1& 1 & 2 \\
\SetCell[r=4]{c} 
\begin{array}{c}
     -2m= v_p(t)\le 0  \\[3pt]
   2^{2m} t \equiv 3 \, (4) 
\end{array}
& E_1 & (0,0,8m) & 2^{-m} & \kI_{8m} & 1&  2^{-1} & 2^{-1} \\
& E_2 & (0,0,4m) & 2^{-m} &  \kI_{4m} & 1&  2^{-1} & 2^{-1} \\
& E_4 & (0,0,2m) & 2^{-m} &  \kI_{2m} & 1&  2^{-1} & 2^{-1} \\
& E_{12} & (0,0,2m) & 2^{-m} &  \kI_{2m} & 1&  2^{-1} & 2^{-1} \\
%----------------------------------------------
\SetCell[r=4]{c} 
     -(2m+1)= v_p(t)<0 
& E_1 & (6,9,8m+22) & 2^{-(m+2)} & \kI_{8m+12}^*  & 1& 2^2 & 1 \\
& E_2 & (6,9,4m+20) & 2^{-(m+2)} &  \kI_{4m+10}^* & 1& 2^2 & 1 \\
& E_4 & (6,9,2m+19) & 2^{-(m+2)}&  \kI_{2m+9}^* & 1& 2^2 & 1 \\
& E_{12} & (6,9,2m+19) & 2^{-(m+2)} &  \kI_{2m+9}^* & 1& 2^2 & 1 \\
%----------------------------------------------
 \SetCell[c=5,r=2]{c} & & & & &  d\equiv 1 &  d\equiv 2  & d\equiv 3 \\
                      & & & & & \SetCell[c=3]{c} d \Mod{4} & \\
\end{longtblr}

\vskip 0.3truecm

\newpage
%\section{Conclusion}

\begin{prop}
Let 
\[ \begin{tikzcd}
& E_1 \ar[dash,d,"2"] & \\
& E_2 \ar[dash,ld,swap,"2"] \ar[dash,rd,"2"] & \\
E_4   & &  E_{12}   
\end{tikzcd}
\]
be a $\mathbf{Q}$-isogeny graph of type $T_4$ corresponding to a given $t$ in $\mathbf{Q}$, $t
\ne 0,-16$. For every square-free integer $d$, 
the probability of a vertex
to be the Faltings curve (circled)
in the twisted isogeny graph 
\[ \begin{tikzcd}
& E^d_1 \ar[dash,d,"2"] & \\
& E^d_2 \ar[dash,ld,swap,"2"] \ar[dash,rd,"2"] & \\
E^d_4   & &  E^d_{12}   
\end{tikzcd}
\] 
is given by:

\newpage

\begin{longtblr}
[caption = La Pollaaa]
{cells = {mode=imath},hlines,vlines,measure=vbox,colspec=cccc}
%--------------------------------------
 T_4 & \text{twisted isogeny graph} & d & \text{Prob} \\
%--------------------------------------
v_2(t)\ge 6 &
\scalebox{.6}{
        \begin{tikzcd}[ampersand replacement=\&]
                \& E_1^d \& \\
                \& E_2^d \ar[u]\ar[dr] \&  \\
                \circled[0.8]{$E_4^d$} \ar[ur]  \&  \&  \, E_{12}^d
        \end{tikzcd}} & & 1 \\
%--------------------------------------
\SetCell[r=2]{c} v_2(t)=5 &
\scalebox{.6}{
        \begin{tikzcd}[ampersand replacement=\&]
                \& E_1^d \& \\
                \& E_2^d  \ar[u]\ar[dr] \&  \\
                \circled[0.8]{$E_4^d$} \ar[ur]  \&  \&  \, E_{12}^d
        \end{tikzcd}} &d\equiv 2\,(4) & 1/3\\
& 
\scalebox{.6}{
        \begin{tikzcd}[ampersand replacement=\&]
                \& E_1^d \& \\
                \& \circled[0.8]{$E_2^d$}  \ar[u]\ar[dr]\ar[dl] \&  \\
                E_4^d \&  \&  \, E_{12}^d
        \end{tikzcd}} & d\not\equiv 2\,(4) & 2/3\\
%--------------------------------------
\SetCell[r=2]{c} 
\begin{array}{c}
    v_2(t)=4  \\
    t/2^4\equiv 1,5\,(8)
\end{array} &
\scalebox{.6}{
        \begin{tikzcd}[ampersand replacement=\&]
                \& E_1^d \& \\
                \& E_2^d\ar[dl]  \ar[u] \&  \\
                 E_4^d   \&  \&  \, \circled[0.8]{$E_{12}^d$} \ar[ul]
        \end{tikzcd}} & d\equiv 2\,(4) & 1/3\\
& 
\scalebox{.6}{
        \begin{tikzcd}[ampersand replacement=\&]
                \& E_1^d \& \\
                \& \circled[0.8]{$E_2^d$}  \ar[u]\ar[dr]\ar[dl] \&  \\
                 E_4^d \&  \&  \, E_{12}^d
        \end{tikzcd}} & d\not\equiv 2\,(4) & 2/3\\
%--------------------------------------
\begin{array}{c}
    v_2(t)=4  \\
    t/2^4\equiv 3,7\,(8)
\end{array} &
\scalebox{.6}{
        \begin{tikzcd}[ampersand replacement=\&]
                \& E_1^d \& \\
                \& E_2^d  \ar[u]\ar[dl] \&  \\
                 E_4^d \&  \&  \, \circled[0.8]{$E_{12}^d$}\ar[ul]
        \end{tikzcd}}  & & 1\\
%--------------------------------------
\SetCell[r=2]{c}
v_2(t)=3  & 
\scalebox{.6}{
        \begin{tikzcd}[ampersand replacement=\&]
                \& E_1^d \& \\
                \& \circled[0.8]{$E_2^d$}  \ar[u]\ar[dr]\ar[dl] \&  \\
                 E_4^d \&  \&  \, E_{12}^d
        \end{tikzcd}}
         & d\equiv 2\,(4) & 1/3\\
&
\scalebox{.6}{
        \begin{tikzcd}[ampersand replacement=\&]
                \& \circled[0.8]{$E_1^d$}\ar[d] \& \\
                \& E_2^d\ar[dl]  \ar[dr] \&  \\
                E_4^d   \&  \&  \, E_{12}^d 
        \end{tikzcd}}  & d\not\equiv 2\,(4) & 2/3\\
%--------------------------------------
v_2(t)\le 2 &
\scalebox{.6}{
        \begin{tikzcd}[ampersand replacement=\&]
                \& \circled[0.8]{$E_1^d$}\ar[d] \& \\
                \& E_2^d\ar[dl]  \ar[dr] \&  \\
                 E_4^d   \&  \&  \, E_{12}^d 
        \end{tikzcd}} & & 1\\
%--------------------------------------
\end{longtblr}

\end{prop}


\vskip 0.35truecm

\noindent{\it Proof.} From the previous tables one gets:

\vskip 0.5truecm


\begin{tblr}{cells={mode=imath},hlines,vlines,measure=vbox}
%-------------------------------------------------
\SetCell[c=1]{c} t &\SetCell[c=1]{c} [u(E)]  & \SetCell[c=1]{c} [u(E)(d)] & \SetCell[c=1]{c} d & \SetCell[c=1]{c}\text{Prob}\\
%-------------------------------------------------
\SetCell[r=1]{c} v_2(t)\ge 6 & \SetCell[r=1]{c} (1:2:2^2:1) & (1:1:1:1) &  & \SetCell[r=1]{c} (0,0,1,0)\\
%-------------------------------------------------
\SetCell[r=2]{c} v_2(t)=5 & \SetCell[r=2]{c} (1:2:2:2) & (1:1:1:1) & d\not\equiv 2\,(4)&\SetCell[r=2]{c} \left(0,\frac{1}{3},\frac{2}{3},0\right) \\
& & \SetCell[r=1]{c} (1:1:2:1) & d\equiv 2\,(4)& \\
%-------------------------------------------------
\SetCell[r=2]{c}
\begin{array}{c}
v_2(t)=4  \\
t/2^4\equiv 1,-3\,(8)
\end{array}
& \SetCell[r=2]{c} (1:2:2:2) & (1:1:1:1) & d\not\equiv 2\,(4)&\SetCell[r=2]{c} \left(0,\frac{2}{3},0,\frac{1}{3}\right) \\
& & \SetCell[r=1]{c} (1:1:1:2) & d\equiv 2\,(4)& \\
%-------------------------------------------------
\SetCell[r=2]{c}
\begin{array}{c}
v_2(t)=4  \\
t/2^4\equiv -1,3\,(8)
\end{array}
& \SetCell[r=2]{c} (1:2:2:2^2) & \SetCell[r=2]{c} (1:1:1:1) & \SetCell[r=2]{c} &\SetCell[r=2]{c} \left(0,0,0,1\right) \\
 & & & & \\
%-------------------------------------------------
\SetCell[r=2]{c} v_2(t)=3 & \SetCell[r=2]{c} (1:1:1:1) & (1:2:2:2) & d\not\equiv 2\,(4)&\SetCell[r=2]{c} \left(\frac{2}{3},\frac{1}{3},0,0\right) \\
& & \SetCell[r=1]{c} (1:1:1:1) & d\equiv 2\,(4)& \\
%-------------------------------------------------
\SetCell[r=1]{c} v_2(t)\le 2 & \SetCell[r=1]{c} (1:1:1:1) & (1:1:1:1) &  & \SetCell[r=1]{c} (1,0,0,0)\\
%-------------------------------------------------
\end{tblr}

\vskip 1.8truecm





\[ \begin{tikzcd}
& \omega\langle 1,\tau \rangle \ar[dash,d] & \\
& \frac{1}{2}\omega\langle 1,2\,\tau \rangle \ar[dash,ld] \ar[dash,rd] & \\
\frac{1}{4}\,\omega\langle 1,4\,\tau \rangle   & &  \frac{1}{2}\,\omega\langle 1,\tau+1/2 \rangle   \,.
\end{tikzcd}
\]

\vskip 0.8truecm

\[ \begin{tikzcd}
& V \ar[dash,d] & \\
& \frac{1}{2}\,V \ar[dash,ld] \ar[dash,rd] & \\
\frac{1}{4}\,V  & &  \frac{1}{4}\,V   \,.
\end{tikzcd}
\]

\vskip 0.8truecm

\[ \begin{tikzcd}
& V u_1^2 \ar[dash,d] & \\
& \frac{1}{2}\,V  u_2^2\ar[dash,ld] \ar[dash,rd] & \\
\frac{1}{4}\,V  u_4^2 & &  
\frac{1}{4}\,V u_{12}^2  \,.
\end{tikzcd}
\]

\vskip 0.8truecm

\[ \begin{tikzcd}
& V u_1^2 
\displaystyle{\frac{u_1(d)^2}{|d|}}
\ar[dash,d] & \\
& \frac{1}{2}\,V  u_2^2
\displaystyle{\frac{u_2(d)^2}{|d|} }
\ar[dash,ld] \ar[dash,rd] & \\
\frac{1}{4}\,V  u_4^2 
\displaystyle{\frac{u_4(d)^2}{|d|}} & &  
\frac{1}{4}\,V u_{12}^2 
\displaystyle{\frac{u_{12}(d)^2}{|d|}\,.}
\end{tikzcd}
\]

\end{document}
