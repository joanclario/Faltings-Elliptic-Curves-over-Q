\documentclass[11pt]{article}
%--------------------------------------------------------------------------
\usepackage{amsfonts,amssymb,amsmath,amsthm,latexsym,graphics,epsfig,amsfonts}
\usepackage{verbatim,enumerate,array,booktabs,color,bigstrut,prettyref,tikz-cd}
\usepackage{multirow}
\usepackage[all]{xy}
\usepackage[backref]{hyperref}
\usepackage[OT2,T1]{fontenc}
\usepackage{mathtools}
\usepackage{caption}
\usepackage{longtable}
\usepackage{mathtools}
\usepackage{tabularray}
\UseTblrLibrary{amsmath,varwidth}
\usepackage{tabularx}
\usepackage{longtable}
\usepackage{arydshln}
%--------------------------------------------------------------------------
\newcommand{\Mod}[1]{\ (\mathrm{mod}\ #1)}
\newcommand{\mathdash}{\relbar\mkern-8mu\relbar}
\newcommand*\circled[2][1.6]{\tikz[baseline=(char.base)]{
    \node[shape=circle, draw, inner sep=1pt, 
        minimum height={\f@size*#1},] (char) {\vphantom{WAH1g}#2};}}
\makeatother
%\addtolength{\textwidth}{4cm} 
\addtolength{\textheight}{3.5cm} 
\addtolength{\topmargin}{-2cm}
%\addtolength{\marginparwidth}{-2cm} 
\newcommand\myiso{\stackrel{\mathclap{\normalfont\mbox{\small $p$}}}{-}}
\newcommand\myisot{\stackrel{\mathclap{\normalfont\mbox{\small $3$}}}{-}}
\newcommand{\pref}[1]{\prettyref{#1}}
%------------------------------------
\newcommand{\Kd}{\operatorname{K}}
\newcommand{\kI}{\operatorname{I}}
\newcommand{\kII}{\operatorname{II}}
\newcommand{\kIII}{\operatorname{III}}
\newcommand{\kIV}{\operatorname{IV}}
%-------------------------------------
\newrefformat{eq}{\textup{(\ref{#1})}}
\newrefformat{prty}{\textup{(\ref{#1})}}

\definecolor{mylinkcolor}{rgb}{0.8,0,0}
\definecolor{myurlcolor}{rgb}{0,0,0.8}
\definecolor{mycitecolor}{rgb}{0,0,0.8}
\hypersetup{colorlinks=true,urlcolor=myurlcolor,citecolor=mycitecolor,linkcolor=mylinkcolor,linktoc=page,breaklinks=true}
%\DeclareSymbolFont{cyrletters}{OT2}{wncyr}{m}{n}
%\DeclareMathSymbol{\Sha}{\mathalpha}{cyrletters}{"58}
\addtolength{\textwidth}{4cm} \addtolength{\hoffset}{-2cm}
\addtolength{\marginparwidth}{-2cm}
%\theoremstyle{definition}
\newtheorem{defn}{Definition}[section]
\newtheorem{definition}[defn]{Definition}
\newtheorem{claim}[defn]{Claim}
%\theoremstyle{plain}
\newtheorem{thmA}{Theorem A}
\newtheorem{thmB}{Theorem B}
\newtheorem{thm2}{Theorem}
\newtheorem{prop2}{Proposition}
\newtheorem{note}{Note}
\newtheorem{corollary}[defn]{Corollary}
\newtheorem{lemma}[defn]{Lemma}
\newtheorem{property}[defn]{Property}
\newtheorem{thm}[defn]{Theorem}
\newtheorem{theorem}[defn]{Theorem}
\newtheorem{cor}[defn]{Corollary}
\newtheorem{prop}[defn]{Proposition}
\newtheorem{proposition}[defn]{Proposition}
\newtheorem{thmnn}{Theorem}
\newtheorem{conj}[defn]{Conjecture}
\theoremstyle{definition}
\newtheorem{remarks}{Remarks}
\newtheorem{ack}{Acknowledgements}
\newtheorem{remark}[defn]{Remark}
\newtheorem{question}[defn]{Question}
\newtheorem{example}[defn]{Example}
%-------------------------------------
\newcommand{\Q}{\mathbb Q}
\newcommand{\Qbar}{\overline{\Q}}
\newcommand{\Z}{\mathbb Z}
\newcommand{\modQ}{\,\text{mod}\,(\Q^)^2}
%-------------------------------------
\newcommand{\mysquare}[1]{\tikz{\path[draw] (0,0) rectangle node{\tiny #1} (8pt,8pt) ;}}
\newcommand{\mycircle}[1]{\tikz{\path[draw] (0,0) circle (4pt) node{\tiny #1};}}
%-------------------------------------
\begin{document}
\title{Type $L_3(9)$}
\date{\today}
\maketitle
%-------------------------------------
%\section{Setting}
\noindent 
The isogeny graphs of type $L_3(9)$ are given by
three isogenous elliptic curves:

\[ 
\begin{tikzcd}
E_1 \arrow[dash]{r}{3} & E_3  \arrow[dash]{r}{3} & E_9   \,.
\end{tikzcd}
\]

\noindent A hauptmodul of $X_0(9)$ is  
$$t(\tau)= 3^3 \left( \frac{\eta(9\tau)}{\eta(\tau)}\right)^3\,.$$ 
Letting $t=t(\tau)$, one can write

$$
\begin{tblr}{l@{\,=\,}l}
j(E_1) = j(\tau) & 
\displaystyle{\frac{(t + 3)^{3}(t^{3} + 9 \, t^{2} + 27 \, t + 3)^{3}}{t(t^{2} + 9 \, t + 27)}}\\[6pt]
j(E_3) = j(3\tau) & 
\displaystyle{\frac{(t + 3)^{3}(t + 9)^{3}}{t^3(t^{2} + 9\,  t + 27)^{3} }}
\\[6pt]
j(E_9) = j(9\tau) & 
\displaystyle{\frac{ (t + 9)^{3}(t^{3} + 243 \, t^{2} + 2187\,  t + 6561)^{3}}{t^9(t^{2} + 9\,  t + 27) }}\,,\\[6pt]
\end{tblr}
$$
and the Fricke involution of $X_0(9)$ is given by $W_9(t)= 3^3/t$.
We can (and do) choose Weierstrass equations for $(E_1,E_3,E_9)$ with signatures:


\[
\begin{tblr}{|c|l|}
\hline \SetCell[c=2]{c} L_3(9) \\ \hline
c_4(E_1) & 
(t + 3)  (t^{3} + 9\,  t^{2} + 27\,  t + 3)\\
c_6(E_1) & 
t^{6} + 18\,  t^{5} + 135\,  t^{4} + 504 \, t^{3} + 891\,  t^{2} + 486\,  t - 27\\  
\Delta(E_1) & 
t  (t^{2} + 9\,  t + 27)\\
\hline
c_4(E_3) & 
(t + 3)  (t + 9) (t^{2} + 27) \\
c_6(E_3) & 
(t^{2} - 27) (t^{4} + 18\,  t^{3} + 162\,  t^{2} + 486\,  t + 729) \\  
\Delta(E_3) & 
t^{3}  (t^{2} + 9 \, t + 27)^{3} \\
\hline
c_4(E_9) & 
(t + 9)  (t^{3} + 243\,  t^{2} + 2187\,  t + 6561) \\
c_6(E_9) & 
t^{6} - 486\,  t^{5} - 24057\,  t^{4} - 367416\,  t^{3} - 2657205\,  t^{2} - 9565938\,  t - 14348907\\  
\Delta(E_9) & 
 t^{9}  (t^{2} + 9\,  t + 27)\\
\hline
\end{tblr}
\]

\vskip 0.5truecm

\noindent and, with this choice, the isogeny graph is normalized. The involution $W_9$ acts on the
isogeny graphs of type $L_3(9)$ as:

\[ 
\begin{tikzcd}
E_9^{-3} \arrow[dash]{r}{3} & E_3^{-3}  \arrow[dash]{r}{3} & E_1^{-3}  \,.
\end{tikzcd}
\]

\noindent 
The signatures, minimal models, Kodaira symbols, and Pal values are as follows:

\vskip 0.3truecm

\begin{longtblr}
[caption= $L_3(9)$ data for $p\neq 2${,} $3$]
{cells={mode=imath},hlines,vlines,measure=vbox,
colspec=cclclccc}
%-------------------------------------------------
L_3(9) & \SetCell[c=5]{c} p\ne 2,3 & & & & \\
t & E & \SetCell[c=1]{c} \operatorname{sig}_p(E) & u & \Kd_p(E) & u_p(d) \\
%-------------------------------------------------
\SetCell[r=3]{c} m=v_p(t)> 0 
& E_1 & (0,0,m) & 1 & \kI_m & 1\\
& E_3 & (0,0,3m) & 1 & \kI_{3m}& 1 \\
& E_9 & (0,0,9m) & 1 & \kI_{9m} & 1\\
%-------------------------------------------------
\SetCell[r=3]{c} 
\begin{array}{c}
v_p(t)=0  \\[6pt]
m=v_p(t^2+9\,t+27) \geq 0 
\end{array}
& E_1 & (0,0,m) & 1 & \kI_{m} & 1 \\
& E_3 & (0,0,3m) & 1 & \kI_{3m} & 1 \\
& E_9 & (0,0,m) & 1 & \kI_{m} & 1 \\
%-------------------------------------------------
\SetCell[r=3]{c} -m=v_p(t) < 0 
& E_1 & (0,0,9m) & p^{-m} & \kI_{9m} & 1  \\
& E_3 & (0,0,3m) & p^{-m} & \kI_{3m} & 1 \\
& E_9 & (0,0,m) & p^{-m} & \kI_{m} & 1 \\
\end{longtblr}


\vskip 0.75truecm


\begin{longtblr}
[caption= {$L_3(9)$ data for $p$=3}]
{cells={mode=imath},hlines,vlines,measure=vbox,
hline{Z}={1-X}{0pt},
vline{1}={Y-Z}{0pt},
colspec=cclclcc}
%-------------------------------------------------
L_3(9) &\SetCell[c=6]{c} p=3  & & & & & \\
t & E & \SetCell[c=1]{c} \operatorname{sig}_3(E) & u & \Kd_3(E) & \SetCell[c=2]{c} u_3(d)  & \\
%-------------------------------------------------
\SetCell[r=3]{c} m=v_3(t)\ge 3 
& E_1 & (2,3,m+3) & 1 & {\kI}^*_{m-3} & 3 & 1\\
& E_3 & (2,3,3m-3) & 3^ & {\kI}^*_{3(m-3)} & 3 & 1 \\
& E_9 & (2,3,9m-21) & 3^{2} & {\kI}^*_{9(m-3)} & 3 & 1 \\
%-------------------------------------------------
\SetCell[r=3]{c} 
     v_3(t)=2   
& E_1 & (2,3,5) & 1 & \kIV  & 1 & 1\\
& E_3 & (\ge 2,3,3) & 3 & \kII  & 1 & 1\\
& E_9 & (\ge 4,6,9) & 3 & {\kIV}^* & 3 & 1 \\
%-------------------------------------------------
\SetCell[r=3]{c} 
     v_3(t)=1  
& E_1 & (\ge 2,3,3) & 1 & \kII  & 1 & 1\\
& E_3 & (\ge 4,6,9) & 1 & {\kIV}^* & 3 & 1 \\
& E_9 & (4,6,11) & 1 & {\kII}^*  & 3 & 1\\
%-------------------------------------------------
\SetCell[r=3]{c} 
    -m=v_3(t)\le 0   
& E_1 & (0,0,9m) & 3^{-m} & \kI_{9m}  & 1 & 1 \\
& E_3 & (0,0,3m) & 3^{-m} & \kI_{3m}  & 1 & 1 \\
& E_9 & (0,0,m) & 3^{-m} & \kI_{m}  & 1 & 1 \\
%-------------------------------------------------
 \SetCell[c=5,r=2]{c} & & & & & d\equiv 0  & d\not\equiv 0 \\
                      & & & & & \SetCell[c=2]{c} d \Mod 3 & \\
\end{longtblr}

\vskip 0.75truecm

\begin{longtblr}
[caption = {$L_3(9)$ data for $p$=2}]
{cells = {mode=imath},hlines,vlines,measure=vbox,
hline{Z} = {1-5}{0pt},
vline{1} = {Y-Z}{0pt},
colspec  = cclclccc}
%----------------------------------------------
L_3(9) & \SetCell[c=7]{c} p=2  & & & & & \\ 
t & E & \SetCell[c=1]{c} \operatorname{sig}_2(E) & u & \SetCell[c=1]{c} \Kd_2(E) & \SetCell[c=3]{c} u_2(d)  \\
%----------------------------------------------
\SetCell[r=3]{c} m=v_2(t)>0 
& E_1 & (4,6,m+12) & 2^{-1} & \kI_{m+4}^* & 1 & 1 & 2 \\
& E_3 & (4,6,3m+12) & 2^{-1} & \kI_{3m+4}^* & 1 & 1 & 2\\
& E_9 & (4,6,9m+12) & 2^{-1} & \kI_{9m+4}^*& 1 & 1 & 2\\
%----------------------------------------------
\SetCell[r=3]{c} v_2(t)=0 
& E_1 & (\geq 8,9,12) & 2^{-1} & \kII^* & 1 & 2 & 2 \\
& E_3 & (\geq 8,9,12) & 2^{-1} & \kII^* & 1 & 2 & 2\\
& E_9 & (\geq 8,9,12) & 2^{-1} & \kII^*& 1 & 2 & 2\\
%----------------------------------------------
\SetCell[r=3]{c} -m=v_2(t)<0 
& E_1 & (4,6,9m+12) & 2^{-m-1} & \kI_{9m+4}^* & 1 & 1 & 2\\
& E_3 & (4,6,3m+12) & 2^{-m-1}  & \kI_{3m+4}^*& 1 & 1 & 2\\
& E_9 & (4,6,m+12) & 2^{-m-1}  & \kI_{m+4}^*& 1 & 1 & 2 \\
%----------------------------------------------
 \SetCell[c=5,r=2]{c} & & & & &  d\equiv 1 &  d\equiv 2  & d\equiv 3 \\
                      & & & & & \SetCell[c=3]{c} d \Mod{4} & \\
\end{longtblr}

\newpage
%\section{Conclusion}

\begin{prop}
Let 
$ 
\begin{tikzcd}
E_1 \arrow[dash]{r}{3}  & E_3 \arrow[dash]{r}{3} & E_9 
\end{tikzcd}
$
be a $\mathbf{Q}$-isogeny graph of type $L_3(9)$ corresponding to a given $t$ in $\mathbf{Q}^*$. For every square-free integer $d$, 
the probability of a vertex
to be the Faltings curve (circled)
in the twisted isogeny graph 
$
\begin{tikzcd} 
E_1^d \arrow[dash]{r}{3}  & E_3^d\arrow[dash]{r}{3} & E_9^d 
\end{tikzcd}
$ 
is given by:

\[
\begin{tblr}[mode=imath]{|c|c|c|c|c|}
\hline
 L_3(9) & \text{twisted isogeny graph} & d & \text{Prob} \\
\hline
 \SetCell[r=1]{c} v_3(t)\leq 0 & \circled[0.8]{$E_1^d$} \longrightarrow E_3^d\longrightarrow E_9^d  & & 1 \\
\hline
\SetCell[r=2]{c} v_3(t)=1   
& \circled[0.8]{$E_1^d$} \longrightarrow E_3^d\longrightarrow E_9^d & d\not\equiv 0\,(3) & 3/4\\ 
&  E_1^d \longleftarrow \circled[0.8]{$E_3^d$} \longrightarrow E_9^d &d\equiv 0\,(3) &  1/4 \\
\hline
\SetCell[r=2]{c}
v_3(t)=2   
 &  E_1^d \longleftarrow \circled[0.8]{$E_3^d$} \longrightarrow E_9^d &d\not\equiv 0\,(3) & 3/4 \\
 &  E_1^d \longleftarrow E_3^d\longleftarrow \circled[0.8]{$E_9^d$} &d\equiv 0\,(3) & 1/4 \\
\hline
 \SetCell[r=1]{c} v_3(t)\ge 3   &  E_1^d \longleftarrow  E_3^d\longleftarrow  \circled[0.8]{$E_9^d$} & & 1 \\
\hline
\end{tblr}
\]



\end{prop}

\vskip 0.35truecm

\noindnet{\it Proof.} From the previous tables one gets:

\vskip 0.5truecm

\begin{tblr}{cells={mode=imath},hlines,vlines,measure=vbox}
%-------------------------------------------------
\SetCell[c=1]{c} t &\SetCell[c=1]{c} [u(E)]  & \SetCell[c=1]{c} [u(E)(d)] & \SetCell[c=1]{c} d & \SetCell[c=1]{c}\text{Prob}\\
%-------------------------------------------------
\SetCell[r=1]{c} v_3(t)\le 0 & \SetCell[r=1]{c} (1:1:1) & (1:1:1) &  & \SetCell[r=1]{c} (1,0,0)\\
%-------------------------------------------------
\SetCell[r=2]{c} v_3(t)=1 & \SetCell[r=2]{c} (1:1:1) & (1:1:1) & d\not\equiv 0\,(3)&\SetCell[r=2]{c} 
\displaystyle{\left(\frac{3}{4},\frac{1}{4},0\right)} \\
& & \SetCell[r=1]{c} (1:3:3) & d\equiv 0\,(3)& \\
%-------------------------------------------------
\SetCell[r=2]{c}  v_3(t)=2 & \SetCell[r=2]{c} (1:3:3) & (1:1:1) & d\not\equiv 0\,(3)&\SetCell[r=2]{c} 
\displaystyle{\left(0,\frac{3}{4},\frac{1}{4}\right)} \\
& & \SetCell[r=1]{c} (1:1:3) & d\equiv 0\,(3)& \\
%-------------------------------------------------
v_3(t)\ge 3 & (1:3:3^{2}) & (1:1:1) & & (0,0,1) \\
%-------------------------------------------------
\end{tblr}

\end{document}


\vskip 1truecm


\begin{tblr}{cells={mode=imath},hlines,vlines,measure=vbox}
%\hline
\SetCell[c=1]{c} t &\SetCell[c=1]{c} u(E_1,E_3,E_9)  & \SetCell[c=1]{c} u_p(E_1,E_3,E_9)(d) &  & \text{Probability}\\
\hline
\SetCell[r=2]{c} v_2(t)>0 & \SetCell[r=2]{c} (2^{-1},2^{-1},2^{-1}) & (1,1,1) & d\equiv 1,2\,(4) & \SetCell[r=6]{c} \\
&  & (2,2,2) & d\equiv 3\,(4) & \\
\SetCell[r=2]{c} v_2(t)=0 & \SetCell[r=2]{c} (2^{-1},2^{-1},2^{-1}) & (1,1,1) & d\equiv 1\,(4) & \\
&  & (2,2,2) & d\equiv 2,3\,(4) & \\
\SetCell[r=2]{c} v_2(t)=-m<0 & \SetCell[r=2]{c} (2^{-m-1},2^{-m-1},2^{-m-1}) & (1,1,1) & d\equiv 1,2\,(4) & \\
&  & (2,2,2) & d\equiv 3\,(4) & \\
\hline
\SetCell[r=2]{c} v_3(t)\ge 3 & \SetCell[r=2]{bg=teal2,fg=white,c} (1,3,3^{2}) & (1,1,1) & d\not\equiv 0\,(3)& \SetCell[r=2]{c} \left(0,0,1\right) \\
& & (3,3,3) & d\equiv 0\,(3)& \\
\SetCell[r=2]{c}  v_3(t)=2 & \SetCell[r=2]{bg=teal2,fg=white,c} (1,3,3) & (1,1,1) & d\not\equiv 0\,(3)&\SetCell[r=2]{c} \left(0,\frac{3}{4},\frac{1}{4}\right) \\
& & \SetCell[r=1]{bg=teal2,fg=white,c} (1,1,3) & d\equiv 0\,(3)& \\
\SetCell[r=2]{c} v_3(t)=1 & \SetCell[r=2]{c} (1,1,1) & (1,1,1) & d\not\equiv 0\,(3)&\SetCell[r=2]{c} \left(\frac{3}{4},\frac{1}{4},0\right) \\
& & \SetCell[r=1]{bg=teal2,fg=white,c} (1,3,3) & d\equiv 0\,(3)& \\
\SetCell[r=1]{c} v_3(t)=-m\le 0 & \SetCell[r=1]{c} (3^{-m},3^{-m},3^{-m}) & (1,1,1) &  & \SetCell[r=1]{c} (1,0,0)\\
\hline
\SetCell[r=1]{c} v_p(t)\ge 0 &  \SetCell[c=1]{c} (1,1,1) & \SetCell[r=2]{c} (1,1,1) & \SetCell[r=2]{c}  & \SetCell[r=2]{c}\\
\SetCell[r=1]{c} v_p(t)=-m<0 & \SetCell[r=1]{c} (p^{-m},p^{-m},p^{-m}) & &  & \\
\end{tblr}



\newpage
\section{OLD VERSION}
{\bf Notation:} 
Let be $E:y^2=x^3+Ax+B$ an elliptic curve:
$$
\begin{array}{l}
p\operatorname{-sig}(E)=(\nu_p(c_4(E)),\nu_p(c_6(E)),\nu_p(\Delta(E)))\\[2mm]
P_2(E)=\left(\frac{c_6(E)}{3^3}\right)^2+2-3 \frac{c_4(E)}{3^2}\\[2mm]
P_5(E)=\left(\frac{c_6(E)}{3^6}\right)^2+2-3 \frac{c_4(E)}{3^4}\\[2mm]
c_4(E)=-2^4 3 A,\\
c_6(E)=-2^5 3^3B,\\
\Delta(E)=2^6 3^3(c_4^3-c_6^2).\\
W_9(t)=\frac{3 (t+6)}{t-3}.
\end{array}
$$

The isogeny graphs of type $L_3(9)$ are given by
three isogenous elliptic curves:

\[ \begin{tikzcd}
E_1 \arrow[dash]{r}{3} & E_3  \arrow[dash]{r}{3} & E_9   \,.
\end{tikzcd}
\]
For $i=1,3,9$:
$$
E_i:y^2+A_i x+B_i,\
$$


$$
\begin{tblr}[mode=dmath]{|c|l|}
 \hline 
	A_1 & 	-3 t \left(t^3-24\right)\\
	B_1 & 2 \left(t^6-36 t^3+216\right)\\
\hline
\hline
    A_3 &  - 3 t (t+6) \left(t^2-6 t+36\right)\\
	B_3 &   2 \left(t^2-6 t-18\right) \left(t^4+6 t^3+54 t^2-108 t+324\right)\\
\hline
\hline
	A_9 & -3 (t+6) \left(t^3+234 t^2+756 t+2160\right) \\[1mm]
	B_9 & 2 \left(t^6-504 t^5-16632 t^4-123012 t^3-517104 t^2-1143072 t-1475496\right)   \\
	\hline
\end{tblr}
$$

$$
\begin{tblr}[mode=dmath]{|c|l|}
 \hline 
	c_4(E_1) & 	2^{4}3^2 t \left(t^3-24\right)\\
	c_6(E_1) & -2^{6}3^3 \left(t^6-36 t^3+216\right)\\
	\Delta(E_1) & 2^{12}3^6 (t-3) \left(t^2+3 t+9\right)\\
\hline
\hline
    c_4(E_3) &	   2^4 3^2 t (t+6) \left(t^2-6 t+36\right)\\
	c_6(E_3) &   -2^6 3^3 \left(t^2-6 t-18\right) \left(t^4+6 t^3+54 t^2-108 t+324\right)\\
	\Delta(E_3) &  2^{12} 3^6 (t-3)^3 \left(t^2+3 t+9\right)^3  \\
\hline
\hline
	c_4(E_9) & 2^4 3^2 (t+6) \left(t^3+234 t^2+756 t+2160\right) \\[1mm]
	c_6(E_9) & -2^6 3^3 \left(t^6-504 t^5-16632 t^4-123012 t^3-517104 t^2-1143072 t-1475496\right)   \\
	\Delta(E_9) &  2^{12} 3^6 (t-3)^9 \left(t^2+3 t+9\right)  \\
	\hline
\end{tblr}
$$

\begin{tblr}{cells={mode=imath},hlines,vlines,measure=vbox}
%\hline
\SetCell[c=5]{c} p=3 & & & & \\
\hline
\SetCell[c=1]{c} t & E & 3\operatorname{-sig}(E) & u & 3\operatorname{-Kod}(E) \\
\hline
\SetCell[r=3]{c} m=v_3(t)> 1 
& E_1 & (m+3,6,9) & 1 & IV^* \\
& E_3 & (m+1,3,3) & 3^{-1} & II \\
& E_9 & (2,3,5) & 3^{-1} & IV \\
\hline
\SetCell[r=3]{c} -m=v_3(t) \le 0 
& E_1 & (2,3,6+9m) & 3^{m} & I_{9m}^* \\
& E_3 & (2,3,6+3m) & 3^{m} & I_{3m}^* \\
& E_9 & (2,3,6+m) & 3^{m} & I_{3m}^* \\
\hline
\SetCell[r=3]{c} 
\begin{array}{c}
v_3(t)= 1 \\[3pt] 
v_3(t/3-1)= 1
\end{array}
& E_1 & (4,6,11) & 1 & II^* \\
& E_3 & (\ge 4,6,9) & 3^{-1}& IV^* \\
& E_9 & (\ge 2,3,3) & 3^{-2} & II \\
\hline
\SetCell[r=3]{c} 
\begin{array}{c}
v_3(t)= 1 \\[3pt] 
v_3(t/3-1)\ge 1\\[3pt] 
m=-v_3(W_9(t))
\end{array}
& E_1 & (0,0,m) & 3^{-1}& I_m \\
& E_3 & (0,0,3m) & 3^{-2} & I_{3m} \\
& E_9 & (0,0,9m) & 3^{-3} & I_{9m} \\
\hline
\SetCell[r=3]{c} 
\begin{array}{c}
v_3(t)= 1 \\[3pt] 
v_3(t/3-2)\ge 1
\end{array}
& E_1 & (4,6,9) & 1 & IV^*  \\
& E_3 & (\ge 2,3,3) &3^{-1} & II \\
& E_9 & (2,3,5) &3^{-1} & IV \\
%\hline
\end{tblr}


\


\begin{tblr}{cells={mode=imath},hlines,vlines,measure=vbox}
%\hline
\SetCell[c=5]{c} p\ne 2,3 & & & & \\
\hline
\SetCell[c=1]{c} t & E & p\operatorname{-sig}(E) & u & p\operatorname{-Kod}(E) \\
\hline
\SetCell[r=3]{c} m=v_p(t)\ge 0 
& E_1 & (m,0,0) & 1 & I_0 \\
& E_3 & (m,0,0) & 1 & I_0 \\
& E_9 & (m,0,0) & 1 & I_0 \\
\hline
\SetCell[r=3]{c} -m=v_3(t) < 0 
& E_1 & (0,0,9m) & p^{m} & I_{9m} \\
& E_3 & (0,0,m) & p^{m} & I_{3m} \\
& E_9 & (0,0,m) & p^{m} & I_{m} \\
%\hline
\end{tblr}

\


\begin{tblr}{cells={mode=imath},hlines,vlines,measure=vbox}
%\hline
\SetCell[c=5]{c} p=2 & & & & \\
\hline
\SetCell[c=1]{c} t & E & 2\operatorname{-sig}(E) & u & 2\operatorname{-Kod}(E) \\
\hline
\SetCell[r=3]{c} m=v_2(t)>0 
& E_1 & (m+7,9,12) & 1 & II^* \\
& E_3 & (m+7,9,12) & 1 & II^* \\
& E_9 & (\ge 9,9,12) & 1 & II^* \\
\hline
\SetCell[r=3]{c} v_2(t) = 0 \quad s?
& E_1 & (4,6,12+s) & 1 & I^*_{4+s} \\
& E_3 & (4,6,12+s) & 1 & I^*_{4+s} \\
& E_9 & (4,6,12+s) & 1 & I^*_{4+s} \\
\hline
\SetCell[r=3]{c} -m=v_2(t) < 0 
& E_1 & (4,6,12+9 m) & 2^{m} & I^*_{4+9m} \\
& E_3 & (4,6,12+3 m) & 2^{m} & I^*_{4+3m} \\
& E_9 & (4,6,12+ m) & 2^{m} & I^*_{4+m} \\
%\hline
\end{tblr}

\newpage


\begin{tblr}{cells={mode=imath},hlines,vlines,measure=vbox}
%\hline
L_3(9) &\SetCell[c=4]{c} p=3  & & & & \SetCell[c=2]{c} u_3(d)  & \\
\hline
\SetCell[c=1]{c} t & E & 3\operatorname{-sig}(E) & u & 3\operatorname{-Kod}(E) & d\equiv 0\pmod 3 & d\not\equiv 0\pmod 3  \\
\hline
\SetCell[r=3]{c} m=v_3(t)> 1 
& E_1 & (m+3,6,9) & 1 & IV^* & 3 & 1 \\
& E_3 & (m+1,3,3) & 3^{-1} & II  & 1 & 1\\
& E_9 & (2,3,5) & 3^{-1} & IV & 1 & 1\\
\hline
\SetCell[r=3]{c} -m=v_3(t) \le 0 
& E_1 & (2,3,6+9m) & 3^{m} & I_{9m}^*  & 3 & 1\\
& E_3 & (2,3,6+3m) & 3^{m} & I_{3m}^*  & 3 & 1\\
& E_9 & (2,3,6+m) & 3^{m} & I_{3m}^*  & 3 & 1\\
\hline
\SetCell[r=3]{c} 
\begin{array}{c}
v_3(t)= 1 \\[3pt] 
v_3(t/3-1)= 1
\end{array}
& E_1 & (4,6,11) & 1 & II^*  & 3 & 1\\
& E_3 & (\ge 4,6,9) & 3^{-1}& IV^*  & 3 & 1\\
& E_9 & (\ge 2,3,3) & 3^{-2} & II & 1 & 1\\
\hline
\SetCell[r=3]{c} 
\begin{array}{c}
v_3(t)= 1 \\[3pt] 
v_3(t/3-1)\ge 1\\[3pt] 
m=-v_3(W_9(t))
\end{array}
& E_1 & (0,0,m) & 3^{-1}& I_m & 1 & 1\\
& E_3 & (0,0,3m) & 3^{-2} & I_{3m} & 1 & 1\\
& E_9 & (0,0,9m) & 3^{-3} & I_{9m} & 1 & 1\\
\hline
\SetCell[r=3]{c} 
\begin{array}{c}
v_3(t)= 1 \\[3pt] 
v_3(t/3-2)\ge 1
\end{array}
& E_1 & (4,6,9) & 1 & IV^*   & 3 & 1\\
& E_3 & (\ge 2,3,3) &3^{-1} & II & 1 & 1\\
& E_9 & (2,3,5) &3^{-1} & IV & 1 & 1 \\
%\hline
\end{tblr}

\


\begin{tblr}{cells={mode=imath},hlines,vlines,measure=vbox}
%\hline
L_3(9) & \SetCell[c=4]{c} p\ne 2,3 & & & & u_p(d)\\
\hline
\SetCell[c=1]{c} t & E & p\operatorname{-sig}(E) & u & p\operatorname{-Kod}(E) & d \\
\hline
\SetCell[r=3]{c} m=v_p(t)\ge 0 
& E_1 & (m,0,0) & 1 & I_0 & 1\\
& E_3 & (m,0,0) & 1 & I_0 & 1\\
& E_9 & (m,0,0) & 1 & I_0& 1 \\
\hline
\SetCell[r=3]{c} -m=v_3(t) < 0 
& E_1 & (0,0,9m) & p^{m} & I_{9m} & 1\\
& E_3 & (0,0,m) & p^{m} & I_{3m} & 1\\
& E_9 & (0,0,m) & p^{m} & I_{m}& 1 \\
%\hline
\end{tblr}

\


\begin{tblr}{cells={mode=imath},hlines,vlines,measure=vbox}
%\hline
L_3(9) &\SetCell[c=4]{c} p=2  & & & & \SetCell[c=3]{c} u_2(d)  & \\
\hline
\SetCell[c=1]{c} t & E & 3\operatorname{-sig}(E) & u & 3\operatorname{-Kod}(E) & d\equiv 1\pmod 4 & d\equiv 2\pmod 4 & d\equiv 3\pmod 4  \\
\hline
\SetCell[r=3]{c} m=v_2(t)>0 
& E_1 & (m+7,9,12) & 1 & II^* & 1 & 2 & 2\\
& E_3 & (m+7,9,12) & 1 & II^*  & 1 & 2 & 2\\
& E_9 & (\ge 9,9,12) & 1 & II^* & 1 & 2 & 2 \\
\hline
\SetCell[r=3]{c} v_2(t) = 0 \quad s?
& E_1 & (4,6,12+s) & 1 & I^*_{4+s} & 1 & 1 & 2  \\
& E_3 & (4,6,12+s) & 1 & I^*_{4+s}& 1 & 1 & 2 \\
& E_9 & (4,6,12+s) & 1 & I^*_{4+s}& 1 & 1 & 2 \\
\hline
\SetCell[r=3]{c} -m=v_2(t) < 0 
& E_1 & (4,6,12+9 m) & 2^{m} & I^*_{4+9m} & 1 & 1 & 2\\
& E_3 & (4,6,12+3 m) & 2^{m} & I^*_{4+3m}& 1 & 1 & 2 \\
& E_9 & (4,6,12+ m) & 2^{m} & I^*_{4+m}& 1 & 1 & 2 \\
%\hline
\end{tblr}

\newpage

\begin{tblr}{cells={mode=imath},hlines,vlines,measure=vbox}
%\hline
\nu_3(t) & \nu_2(t) & (u_d(E_1),u_d(E_3),u_d(E_9)) & d\pmod{12} \\
\hline 
\SetCell[r=8]{c} \nu_3(t) \le 0 &  \SetCell[r=4]{c} \nu_2(t)> 0 & (1,1,1) & 1,5\\
 & & (2,2,2) & 2,7,10,11 \\
 & & (3,3,3) & 9\\
 & &  (6,6,6) & 3,6\\
  & \SetCell[r=4]{c} \nu_2(t) \le 0 & (1,1,1) & 1,2,5,10\\
  & & (2,2,2) & 7,11\\
 & &  (3,3,3) & 6,9\\
 & &  (6,6,6) & 3\\
 \hline
\SetCell[r=8]{c} \nu_3(t)  > 1 &  \SetCell[r=4]{c} \nu_2(t)  > 0 & (1,1,1) & 1,5\\
 & & (2,2,2) & 2,7,10,11 \\
 & & (3,1,1) & 9\\
 & &  (6,2,2) & 3,6\\
  & \SetCell[r=4]{c} \nu_2(t) \le 0 & (1,1,1) & 1,2,5,10\\
  & & (2,2,2) & 7,11\\
 & &  (3,1,1) & 6,9\\
 & &  (6,2,2) & 3\\
  \hline
\SetCell[r=8]{c}  
\begin{array}{c}
v_3(t)= 1 \\[3pt] 
v_3(t/3-1)= 1
\end{array} &  \SetCell[r=4]{c} \nu_2(t)  > 0 & (1,1,1) & 1,5\\
 & & (2,2,2) & 2,7,10,11 \\
 & & (3,3,1) & 9\\
 & &  (6,6,2) & 3,6\\
  & \SetCell[r=4]{c} \nu_2(t) \le 0 & (1,1,1) & 1,2,5,10\\
  & & (2,2,2) & 7,11\\
 & &  (3,3,1) & 6,9\\
 & &  (6,6,2) & 3\\
\hline
\SetCell[r=4]{c}  
\begin{array}{c}
v_3(t)= 1 \\[3pt] 
v_3(t/3-1)\ge 1\\[3pt] 
m=-v_3(W_9(t))
\end{array}&  \SetCell[r=2]{c} \nu_2(t)  > 0 & (1,1,1) & 1,5,9\\
 & & (2,2,2) & 2,3,6,7,10,11 \\
  & \SetCell[r=2]{c} \nu_2(t) \le 0 & (1,1,1) & 1,2,5,6,9,10\\
  & & (2,2,2) & 3,7,11\\
 \hline
\SetCell[r=8]{c}  
\begin{array}{c}
v_3(t)= 1 \\[3pt] 
v_3(t/3-2)\ge 1
\end{array}&  \SetCell[r=4]{c} \nu_2(t)  > 0 & (1,1,1) & 1,5\\
 & & (2,2,2) & 2,7,10,11 \\
 & & (3,3,1) & 9\\
 & &  (6,6,2) & 3,6\\
  & \SetCell[r=4]{c} \nu_2(t) \le 0 & (1,1,1) & 1,2,5,10\\
  & & (2,2,2) & 7,11\\
 & &  (3,1,1) & 6,9\\
 & &  (6,2,2) & 3\\
 \end{tblr}\newpage

%%%%%%%%%%%%%%%%%%%%%%%%%%%%%%%%%%%%
\section{Proof $p=3$}
$$
t=3^n u,\qquad u\in \Z_3^*
$$
\subsection{$E_1$}
$$
	\begin{array}{lll}
	c_4(E_1)=	2^4\ 3^{n+3} u \left(3^{3 n-1} u^3-2^3\right),\\[1mm]
	c_6(E_1)=-2^6 3^6 \left(3^{6 n-3} u^6-2^2 3^{3 n-1} u^3+2^3\right),\\[1mm]
	\Delta(E_1)=2^{12} 3^9 \left(3^{n-1} u-1\right) \left(3^{2 n-2} u^2+3^{n-1} u+1\right)
	\end{array}
	$$
	
\fbox{$n>1$}  	$\gamma_3(E_1)=(n+3,6,9)$ $\stackrel{Papa}{\longrightarrow}$ $3$-$\text{Kod}(E_1)=\text{IV}^*$ since $P_5(E_1)\equiv 3 \!\!\pmod 9$.
 %or $\text{III}^*$:

\

\fbox{$n<0$} $n=-m$, $m>0$:
$$
	\begin{array}{lll}
	c_4(E_1)=	 2^4 3^{2-4 m} u \left(u^3-2^3 3^{3 m+1}\right)            ,\\[1mm]
	c_6(E_1)=  -2^6 3^{3-6 m} \left(-2^2 3^{3 m+2} u^3+2^3 3^{6 m+3}+u^6\right)           .\\[1mm]
	\Delta(E_1)=   2^{12} 3^{6-3 m} \left(u-3^{m+1}\right) \left(3^{m+1} u+3^{2 m+2}+u^2\right)         .
	\end{array}
$$
Change $U=3^m$ $\longrightarrow$ $\gamma_3(E_1)=(2,3,6+9 m)$ $\stackrel{Papa}{\longrightarrow}$ $3$-$\text{Kod}(E_1)=\text{I}_{9m}^*$.
 
 \
 
 \fbox{$n=0$} 
 $$
	\begin{array}{lll}
	c_4(E_1)=	 2^4 3^2 u \left(u^3-3\ 2^3\right)            ,\\[1mm]
	c_6(E_1)=    -2^6 3^3 \left(u^6-2^2 3^2 u^3+2^3 3^3\right)         .\\[1mm]
	\Delta(E_1)=    2^{12} 3^6 (u-3) \left(u^2+3 u+3^2\right)        .
	\end{array}
$$
$\gamma_3(E_1)=(2,3,6)$ $\stackrel{Papa}{\longrightarrow}$ $3$-$\text{Kod}(E_1)=\text{I}_{0}^*$.

\

 \fbox{$n=1\,\, \& \,\, u\equiv 1\pmod 3$}  $u=1+3^m v$, $m>0$, $v\in\mathbb Z_3$ ($v=0\Longleftrightarrow t=3\Longleftrightarrow$ $E_1$ singular).

$$
	\begin{array}{lll}
	c_4(E_1)=	  2^4 3^4 \left(3^m v+1\right) \left(3^{3 m+2} v^3+3^{2 m+3} v^2+3^{m+3} v+1\right)           ,\\[1mm]
	c_6(E_1)=    -2^6 3^6 \left(3^{6 m+3} v^6+2\ 3^{5 m+4} v^5+5\ 3^{4 m+4} v^4+56\ 3^{3 m+2} v^3+11\ 3^{2 m+3} v^2+2\ 3^{m+3} v-1\right)         .\\[1mm]
	\Delta(E_1)=  2^{12} 3^{m+10} v \left(3^{2 m-1} v^2+3^m v+1  \right)     .
	\end{array}
$$
$m=1$ $\Longrightarrow$ $\gamma_3(E_1)=(4,6,11)$ $\stackrel{Papa}{\longrightarrow}$ $3$-$\text{Kod}(E_1)=\text{II}^*$.

\noindent $m>1$ $\Longrightarrow$  Change $U=1/3$: $\Longrightarrow$ $\gamma_3(E_1)=(0,0,m-2)$ $\stackrel{Papa}{\longrightarrow}$ $3$-$\text{Kod}(E_1)=\text{I}_{m-2}$, $\nu_3(W_3(t))=2-m$.

\

 \fbox{$n=1\,\, \& \,\, u\equiv 2\pmod 3$}  $u=2+3^m v$, $m>0$, $v\in\mathbb Z_3\cup \{0\}$.


$$
	\begin{array}{l}
	c_4(E_1)=	  2^4 3^4 \left(3^m v+2\right) \left(3^{3 m+2} v^3+2\ 3^{2 m+3} v^2+2^2 3^{m+3} v+2^6\right)           ,\\[1mm]
	c_6(E_1)=  -2^6 3^6 \left(3^{6 m+3} v^6+4\ 3^{5 m+4} v^5+5\ 2^2 3^{4 m+4} v^4+7\ 17\ 2^2 3^{3 m+2} v^3+\right. \\[1mm]
	\qquad\qquad\qquad \left.+29\ 2^3 3^{2 m+3} v^2+11\ 2^4 3^{m+3} v+181\ 2^3\right)           \\[1mm]
	\Delta(E_1)=   2^{12} 3^9 \left(3^m v+1\right) \left(3^{2 m} v^2+5\ 3^m v+7\right)         .
	\end{array}
$$
$\gamma_3(E_1)=(4,6,9)$ $\stackrel{Papa}{\longrightarrow}$ $3$-$\text{Kod}(E_1)=\text{IV}^*$ since $P_5(E_1)\equiv 6 \!\!\pmod 9$.

%%%%%%%%%%%%%%%%%%%%%%%%%%%%%%%%%%%%%%%%%%%%%%%%%%
%%%%%%%%%%%%%%%%%%%%%%%%%%%%%%%%%%%%%%%%%%%%%%%%%%%%%%%

\subsection{$E_3$}
$$
	\begin{array}{lll}
	c_4(E_3)=	2^4 3^{n+5} u \left(3^{n-1} u+2\right) \left(3^{2 n-2} u^2-2\ 3^{n-1} u+2^2\right)             ,\\[1mm]
	c_6(E_3)=  -2^6 3^9 \left(3^{2 n-2} u^2-2\ 3^{n-1} u-2\right) \left(3^{4 n-4} u^4+2\ 3^{3 n-3} u^3+2\ 3^{2 n-1} u^2-2^2 3^{n-1} u+2^2\right)           .\\[1mm]
	\Delta(E_3)=  2^{12} 3^{15} \left(3^{n-1} u-1\right)^3 \left(3^{2 n-2} u^2+3^{n-1} u+1\right)^3          .
	\end{array}
$$


\noindent \fbox{$n>1$} Change $U=1/3$: 


$\gamma_3(E_3)=(n+1,3,3)$ $\stackrel{Papa}{\longrightarrow}$ $3$-$\text{Kod}(E_3)=\text{II}$ since $P_2(E_3)\equiv 3 \!\!\pmod 9$.

\


\noindent \fbox{$n<0$} $n=-m$, $m>0$:
$$
	\begin{array}{lll}
	c_4(E_3)=2^4 3^{2-4 m} u \left(2\ 3^{m+1}+u\right) \left(-2\ 3^{m+1} u+2^2 3^{2 m+2}+u^2\right)	             ,\\[1mm]
	c_6(E_3)=   -2^6 3^{3-6 m} \left(-2\ 3^{m+1} u-2\ 3^{2 m+2}+u^2\right) \left(2\ 3^{m+1} u^3+2\ 3^{2 m+3} u^2-4\ 3^{3 m+3} u+2^2 3^{4 m+4}+u^4\right)          .\\[1mm]
	\Delta(E_3)=     2^{12} 3^{6-9 m} \left(u-3^{m+1}\right)^3 \left(3^{m+1} u+3^{2 m+2}+u^2\right)^3       .
	\end{array}
$$
Change $U=3^m$ $\longrightarrow$ $\gamma_3(E_3)=(2,3,6+3 m)$ $\stackrel{Papa}{\longrightarrow}$ $3$-$\text{Kod}(E_3)=\text{I}_{3m}^*$.
 
 \
 
 \noindent \fbox{$n=0$} 
 $$
	\begin{array}{lll}
	c_4(E_3)=	2^4 3^2 u (u+2\ 3) \left(u^2-2\ 3 u+2^2 3^2\right)             ,\\[1mm]
	c_6(E_3)=  -2^6 3^3 \left(u^2-2\ 3 u-2\ 3^2\right) \left(u^4+2\ 3 u^3+2\ 3^3 u^2-2^2 3^3 u+2^2 3^4\right)           .\\[1mm]
	\Delta(E_3)=      2^{12} 3^6 (u-3)^3 \left(u^2+3 u+9\right)^3      .
	\end{array}
$$
$\gamma_3(E_3)=(2,3,6)$ $\stackrel{Papa}{\longrightarrow}$ $3$-$\text{Kod}(E_3)=\text{I}_{0}^*$.

\


\noindent \fbox{$n=1\,\, \& \,\, u\equiv 1\pmod 3$}  $u=1+3^m v$, $m>0$, $v\in\mathbb Z_3$ ($v=0\Longleftrightarrow t=3\Longleftrightarrow$ $E_3$ singular).


$$
	\begin{array}{lll}
	c_4(E_3)=	 2^4 3^8 \left(3^m v+1\right) \left(3^{m-1} v+1\right) \left(3^{2 m-1} v^2+1\right)            ,\\[1mm]
	c_6(E_3)=  -2^6 3^{12} \left(3^{2 m-1} v^2-1\right) \left(3^{4 m-2} v^4+2\ 3^{3 m-1} v^3+2\ 3^{2 m} v^2+2\ 3^m v+1\right)           .\\[1mm]
	\Delta(E_3)=   2^{12} 3^{3 m+18} v^3 \left(3^{2 m-1} v^2+3^m v+1\right)^3         .
	\end{array}
$$
$m=1$: $U=1/3$ $\Longrightarrow$ $\gamma_3(E_3)=(\ge 4,6,9)$ $\stackrel{Papa}{\longrightarrow}$ $3$-$\text{Kod}(E_3)=\text{IV}^*$ since $P_5(E_3)\equiv 3v^2 \!\!\pmod 9$.

\noindent $m>1$: $U=1/9$  $\Longrightarrow$ $\gamma_3(E_3)=(0,0,3m-6)$ $\stackrel{Papa}{\longrightarrow}$ $3$-$\text{Kod}(E_3)=\text{I}_{3(m-2)}$, $\nu_3(W_3(t))=2-m$.



\

\noindent \fbox{$n=1\,\, \& \,\, u\equiv 2\pmod 3$}  $u=2+3^m v$, $m>0$, $v\in\mathbb Z_3\cup \{0\}$.


$$
	\begin{array}{lll}
	c_4(E_3)=2^4 3^6 \left(3^m w+2\right) \left(3^m w+4\right) \left(3^{2 m} w^2+2\ 3^m w+4\right)	             ,\\[1mm]
	c_6(E_3)= -2^6 3^9 \left(3^{2 m} w^2+2\ 3^m w-2\right) \left(3^{4 m} w^4+2\ 5\ 3^{3 m} w^3+2\ 7\ 3^{2 m+1} w^2+19\ 2^2 3^m w+13\ 2^2\right)            .\\[1mm]
	\Delta(E_3)=   2^{12} 3^{15} \left(3^m w+1\right)^3 \left(3^{2 m} w^2+5\ 3^m w+7\right)^3         .
	\end{array}
$$
$m=1$: $U=1/3$ $\Longrightarrow$ $\gamma_3(E_3)=(\ge 2,3,3)$ $\stackrel{Papa}{\longrightarrow}$ $3$-$\text{Kod}(E_3)=\text{II}$ since $P_2(E_3)\equiv 3 \!\!\pmod 9$.

\noindent $m>1$: $U=1/3$ $\Longrightarrow$ $\gamma_3(E_3)=( 2,3,3)$ $\stackrel{Papa}{\longrightarrow}$ $3$-$\text{Kod}(E_3)=\text{II}$ since $P_2(E_3)\equiv 2 \!\!\pmod 9$.

\

\


\subsection{$E_9$}

$$
	\begin{array}{lll}
	c_4(E_9)=2^4 3^6 \left(3^{n-1} u+2\right) \left(3^{3 n-3} u^3+2\ 13\ 3^{2 n-1} u^2+7\ 2^2 3^n u+5\ 2^4\right)	             ,\\[1mm]
	c_6(E_9)=  -2^6 3^9 \left(3^{6 n-6} u^6-2^3 7\ 3^{5 n-4} u^5-2^3 7\ 11\ 3^{4 n-3} u^4-2^2 17\ 67\ 3^{3 n-3} u^3-2^4 7\ 19\ 3^{2 n-1} u^2-2^5 7^2 3^n u+11\ 23 \left(-2^3\right)\right)           .\\[1mm]
	\Delta(E_9)= 2^{12} 3^{17} \left(3^{n-1} u-1\right)^9 \left(3^{2 n-2} u^2+3^{n-1} u+1\right)        .	
		\end{array}
$$
\noindent \fbox{$n>1$} Change $U=1/3$: 
$\gamma_3(E_9)=(2,3,5)$ $\stackrel{Papa}{\longrightarrow}$ $3$-$\text{Kod}(E_9)=\text{IV}$.

\


\noindent \fbox{$n<0$} $n=-m$, $m>0$:
$$
	\begin{array}{lll}
	c_4(E_9)=2^4 3^{2-4 m} \left(2\ 3^{m+1}+u\right) \left(2\ 13\ 3^{m+2} u^2+7\ 2^2 3^{2 m+3} u+5\ 2^4 3^{3 m+3}+u^3\right)	             ,\\[1mm]
	c_6(E_9)=   2^6 3^{3-6 m} \left(7\ 2^3 3^{m+2} u^5+7\ 11\ 2^3 3^{2 m+3} u^4+17\ 67\ 2^2 3^{3 m+3} u^3+7\ 19\ 2^4 3^{4 m+5} u^2+2^5 7^2 3^{5 m+6} u+11\ 23\ 2^3 3^{6 m+6}-u^6\right)          .\\[1mm]
	\Delta(E_9)=  -2^{12}  3^{6-11 m} \left(3^{m+1}-u\right)^9 \left(3^{m+1} u+3^{2 m+2}+u^2\right)       .	
		\end{array}
$$
Change $U=3^m$ $\Longrightarrow$ $\gamma_3(E_9)=(2,3,6+ m)$ $\stackrel{Papa}{\longrightarrow}$ $3$-$\text{Kod}(E_9)=\text{I}_{m}^*$.
 
 \
 
 \noindent \fbox{$n=0$} 
 $$
	\begin{array}{l}
		c_4(E_9)=2^4 3^2 (u+6) \left(u^3+2\ 13\ 3^2 u^2+7\ 2^2 3^3 u+5\ 2^4 3^3\right)	             ,\\[1mm]
	c_6(E_9)=  -2^6 3^3 \left(u^6-2^3 3^2 7 u^5-2^3 3^3 7\ 11 u^4-2^2 3^3 17\ 67 u^3-2^4 3^5 7\ 19 u^2-2^5 3^6 7^2 u+11\ 23 \left(-2^3\right) 3^6\right)           .\\[1mm]
	\Delta(E_9)=  	2^{12} 3^6 (u-3)^9 \left(u^2+3 u+9\right).
		\end{array}
$$
$\gamma_3(E_9)=(2,3,6)$ $\stackrel{Papa}{\longrightarrow}$ $3$-$\text{Kod}(E_9)=\text{I}_{0}^*$.

\


\noindent \fbox{$n=1\,\, \& \,\, u\equiv 1\pmod 3$}  $u=1+3^m v$, $m>0$, $v\in\mathbb Z_3$ ($v=0\Longleftrightarrow t=3\Longleftrightarrow$ $E_9$ singular).

\noindent $m=1$:

$$
	\begin{array}{lll}
	c_4(E_9)=2^4 3^2 (v+1) \left(v^3+27 v^2+27 v+9\right)	             ,\\[1mm]
	c_6(E_9)= -2^6 3^3 \left(v^6-2\ 3^3 v^5-3^3 11 v^4-2^3 3^2 7 v^3-3^4 5 v^2-2\ 3^4 v-3^3\right)            .\\[1mm]
	\Delta(E_9)=  	2^{12} 3^3 v^9 \left(3 v^2+3 v+1\right).
	\end{array}
$$
$U=1/9$ $\Longrightarrow$ $\gamma_3(E_9)=(\ge 2,3,3)$ $\stackrel{Papa}{\longrightarrow}$ $3$-$\text{Kod}(E_9)=\text{II}$ since $P_2(E_9)\equiv v^{12}+6 v^4+6 v^3+2 \!\!\pmod 9$, and $v=1,2$.

\noindent $m>1$: 
$$
	\begin{array}{lll}
	c_4(E_9)=2^4 3^{12} \left(3^{m-1} v+1\right) \left(3^{3 m-5} v^3+3^{2 m-1} v^2+3^m v+1\right)	             ,\\[1mm]
	c_6(E_9)= -2^6 3^{18} \left(3^{6 m-9} v^6-2\ 3^{5 m-5} v^5-11\ 3^{4 m-4} v^4-56\ 3^{3 m-4} v^3-5\ 3^{2 m-1} v^2-2\ 3^m v-1\right)           .\\[1mm]
	\Delta(E_9)= 2^{12} 3^{9 m+18} v^9 \left(3^{2 m-1} v^2+3^m v+1\right) 	.
	\end{array}
$$
$U=1/27$  $\Longrightarrow$ $\gamma_3(E_9)=(0,0,9m-18)$ $\stackrel{Papa}{\longrightarrow}$ $3$-$\text{Kod}(E_9)=\text{I}_{9(m-2)}$, $\nu_3(W_3(t))=2-m$.



\

\noindent \fbox{$n=1\,\, \& \,\, u\equiv 2\pmod 3$}  $u=2+3^m v$, $m>0$, $v\in\mathbb Z_3\cup \{0\}$.


$$
	\begin{array}{lll}
	c_4(E_9)=	2^4 3^6 \left(3^m v+4\right) \left(3^{3 m} v^3+7\ 2^2 3^{2 m+1} v^2+17\ 2^3 3^{m+1} v+71\ 2^3\right)             ,\\[1mm]
	c_6(E_9)= -2^6 3^9 \left(3^{6 m} v^6-2^2 13\ 3^{5 m+1} v^5-2^2 17^2 3^{4 m+1} v^4-2^2 5^2 7\ 37\ 3^{3 m} v^3-2^3 3803\ 3^{2 m+1} v^2-2^4 29\ 113\ 3^{m+1} v+13537 (-1) 2^3\right)            ,\\[1mm]
	\Delta(E_9)=2^{12} 3^{17} \left(3^m v+1\right)^9 \left(3^{2 m} v^2+5\ 3^m v+7\right).  
	\end{array}
$$
$U=1/3$ $\Longrightarrow$ $\gamma_3(E_9)=( 2,3,5)$ $\stackrel{Papa}{\longrightarrow}$ $3$-$\text{Kod}(E_9)=\text{IV}$.

%%%%%%%%%%%%%%%%%%%%%%%%%%%%%%%%%%%%%%%%%%%%%%%%%%%
\newpage
\subsection{Proof $p\ne 2,3$}

$$
t=u p^n,\qquad u\in \Z_p^*
$$
\subsection{$E_1$}
$$
\begin{array}{l}
	c_4(E_1)=	  2^4 3^2 u p^n \left(u^3 p^{3 n}-2^3 3\right)           ,\\[1mm]
	c_6(E_1)=  -2^6 3^3 \left(-36 u^3 p^{3 n}+u^6 p^{6 n}+2^3 3^3\right)           .\\[1mm]
	\Delta(E_1)= 2^{12} 3^6 \left(u p^n-3\right) \left(u^2 p^{2 n}+3 u p^n+3^2\right)           .
	\end{array}
$$
\fbox{$n>1$}  	$\gamma_p(E_1)=(n,0,0)$ $\stackrel{Papa}{\longrightarrow}$ $p$-$\text{Kod}(E_1)=\text{I}_0$.

\

\noindent \fbox{$n=0$}  TODO
 $$
	\begin{array}{l}
	c_4(E_1)=	 2^4 3^2 u \left(u^3-3\ 2^3\right)            ,\\[1mm]
	c_6(E_1)=    -2^6 3^3 \left(u^6-2^2 3^2 u^3+2^3 3^3\right)         .\\[1mm]
	\Delta(E_1)=    2^{12} 3^6 (u-3) \left(u^2+3 u+3^2\right)        .
	\end{array}
$$
%$\gamma_p(E_1)=(0,0,0)$ $\stackrel{Papa}{\longrightarrow}$ $p$-$\text{Kod}(E_1)=\text{I}_{0}$.

\

\noindent \fbox{$n<0$} $n=-m$, $m>0$:
$$
	\begin{array}{l}
	c_4(E_1)=	   2^4 3^2 u p^{-4 m} \left(u^3-24 p^{3 m}\right)        ,\\[1mm]
	c_6(E_1)=     -2^6 3^3 p^{-6 m} \left(-36 u^3 p^{3 m}+216 p^{6 m}+u^6\right)        .\\[1mm]
	\Delta(E_1)=    2^{12} 3^6 p^{-3 m} \left(u-3 p^m\right) \left(3 u p^m+9 p^{2 m}+u^2\right)      .
	\end{array}
$$
Change $U=p^m$: $\gamma_p(E_1)=(0,0,9m)$ $\stackrel{Papa}{\longrightarrow}$ $p$-$\text{Kod}(E_1)=\text{I}_{9m}$.
 
\newpage


\SetTblrInner{rowsep=2pt}
\[
\begin{tblr}[mode=imath]{|c|c|c|l|}
\hline
 L_3(9) & (\lambda,\mu) 
 &  3\operatorname{-Kod}(E_1,E_3,E_9) 
 &  \SetCell[c=1]{c} (u_1(d),u_3(d),u_9(d))  \\
 \hline
 \SetCell[r=5]{c}
 v_3(t)\leq 0 &
  \SetCell[r=5]{c}
 (1/3,1/3) & 
  \SetCell[r=5]{c} 
 {(I^*_{9n},I^*_{3n}, I^*_n) \\[5pt] $n=-v_3(t)$} 
  & \SetCell[r=1,c=1]{c}(u,u,u)  \\ 
   & & &  1 \text{ if } d \equiv 1,5 \, (12) \\
   & & &  2 \text{ if } d \equiv 2,7,10,11\, (12) \\
   & & &  3 \text{ if } d \equiv 9 \, (12) \\ 
   & & &  6 \text{ if } d \equiv 3,6 \, (12) \\
\hline
 \SetCell[r=13]{c}
 {
 \multirow{2}{c}{} & {v_3(t)=1} \\[5pt]
  {t/3 \equiv 1 \, (3)} 
 }
 &
 \SetCell[r=13]{c}
 (1,1) & 
 (I_n,I_{3n},I_{9n}) & \SetCell[r=1,c=1]{c}(u,u,u)  \\
 & & n=-v_3(W_9(t)) & 1 \text{ if } d\equiv 1,5,9 \, (12) \\
 & &  
 t/3\equiv 1 \, (9) & 
 2 \text{ if } d\equiv 2,3,6,7,10,11 \, (12)
 \\
 \hline  
 & & \SetCell[r=10]{t} 
 {(II^*,IV^*,II)\\[5pt] $t/3\not\equiv 1 \, (9)$} & 
 \SetCell[r=1,c=1]{c}(u,u,u)  \text{ if } d\not \equiv 0 \, (3)\\
 & & & 
 1 \text{ if } d\equiv 1,2,5,10 \, (12) \,,
 t/3\equiv 4 \, (9)
 \\
 & & & 
 2 \text{ if } d\equiv 7,11 \, (12) \,,
 t/3\equiv 4 \, (9)
 \\
& & & 
 1 \text{ if } d\equiv 1,5 \, (12) \,,
 t/3\equiv 7 \, (9)
 \\
 & & & 
 2 \text{ if } d\equiv 2,7,10,11 \, (12) \,,
 t/3\equiv 7 \, (9)
 \\
  \hline
 & & &  \SetCell[r=1,c=1]{c}(3u,3u,u)  \text{ if } d\equiv 0 \, (3)\\
 & & & 
 1 \text{ if } d\equiv 6,9 \, (12) \,,
 t/3\equiv 4 \, (9)
 \\
 & & & 
 2 \text{ if } d\equiv 3 \, (12) \,,
 t/3\equiv 4 \, (9)
 \\
& & & 
 1 \text{ if } d\equiv 9 \, (12) \,,
 t/3\equiv 7 \, (9)
 \\
 & & & 
 2 \text{ if } d\equiv 3,6 \, (12) \,,
 t/3\equiv 7 \, (9)
 \\
 \hline
  \SetCell[r=6]{c,3.5cm}
   {
 \multirow{2}{c}{}
 v_3(t)=1 \,, t/3 \equiv 2 \, (3) & \\[4pt]
 \text{or} & \\[4pt]
v_3(t)\geq 2 & \\
 }
  &
 \SetCell[r=6]{c}
 (1,1/3) & 
  \SetCell[r=6]{c}
 (IV^*,II,IV) & \SetCell[r=1,c=1]{c}(u,u,u) \text{ if } d\not\equiv 0 \, (3)  \\
    & & & 1 \text{ if } d\equiv 1,2,5,10 \, (12) \\
    & & & 2 \text{ if } d\equiv 7,11 \, (12)\\
 \hline
 & & & \SetCell[r=1,c=1]{c}(3u,u,u) \text{ if } d\equiv 0 \, (3)  \\
     & & & 1 \text{ if } d\equiv 6,9 \, (12) \\
    & & & 2 \text{ if } d\equiv 3 \, (12)\\
\hline
\end{tblr}
\]

\vskip 0.7truecm

\noindent{\bf  Corollary 1.}
Let 
$ \begin{tikzcd}
E_1 \arrow[dash]{r}{3}  & E_3 \arrow[dash]{r}{3} & E_9 
\end{tikzcd}
$
be the $\mathbf{Q}$-isogeny graph of type $L_3(9)$ corresponding to a given $t$ in $\mathbf{Q}$, $t\neq 3$. Let $d$ be a square-free integer. Then, the orientation of graph the
twisted graph $ \begin{tikzcd}
E_1^d \arrow[dash]{r}{3}  & E_3^d\arrow[dash]{r}{3} & E_9^d 
\end{tikzcd}
$ is given by:

\[
\begin{tblr}[mode=math]{|c|c|c|c|}
\hline
 L_3(9) 
 &  3\operatorname{-Kod}(E_1,E_3,E_9) 
 &  
E_1^d \mathdash 
E_3^d\mathdash 
E_9^d 
 & \text{prob} \\
 \hline
 \SetCell[r=2]{c}
v_3(t)\leq 0 &
(I^*_{9n},I^*_{3n}, I^*_n) & \SetCell[r=2]{c}
\circled[0.8]{$E_1^d$} \longrightarrow 
E_3^d\longrightarrow 
E_9^d  
&  \SetCell[r=2]{c}1
\\
& n=-v_3(t) & & \\
\hline
 \SetCell[r=5]{c}
 {
 \multirow{2}{c}{} & {v_3(t)=1} \\[5pt]
                     {t/3 \equiv 1 \, (3)} 
 }
  & 
  (I_n,I_{3n},I_{9n}) 
  &  
 \SetCell[r=3]{c}   
E_1^d \longleftarrow  
E_3^d\longleftarrow  
\circled[0.8]{$E_9^d$} 
& 
\SetCell[r=3]{c} 1 \\
  & n=-v_3(W_9(t)) & & \\
  & t/3\equiv 1\,(9) & & \\
\hline
  & \SetCell[r=2]{t} 
 {(II^*,IV^*,II)\\[4pt] $t/3\not\equiv 1 \, (9)$} &    E_1^d \longleftarrow 
      \circled[0.8]{$E_3^d$} \longrightarrow 
      E_9^d  & 1/4
 \\

 & & E_1^d \longleftarrow 
     E_3^d\longleftarrow 
     \circled[0.8]{$E_9^d$} & 3/4 \\
 \hline
  \SetCell[r=2]{c}
{ v_3(t)=1 \,, t/3 \equiv 2 \, 
(3)\\
\text{or} \\
$v_3(t)\geq 2$}
& 
 \SetCell[r=2]{c} (IV^*,II,IV) & 
  \SetCell[r=1]{c} 
 \circled[0.8]{$E_1^d$} \longrightarrow 
  E_3^d\longrightarrow 
  E_9^d 
  & 1/4  
\\
  & &
  \SetCell[r=1]{c} 
  E_1^d \longleftarrow 
  E_3^d\longleftarrow 
  \circled[0.8]{$E_9^d$}  
  & 3/4  \\
\hline
\end{tblr}
\]

The column labeled {\it prob} indicates the probability that the circled vertex is a Falting-Stevens elliptic curve in the corresponding isogeny class.



\end{document}