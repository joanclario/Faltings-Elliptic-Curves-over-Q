\documentclass[11pt]{article}
\usepackage{amsfonts,amssymb,amsmath,amsthm,latexsym,graphics,epsfig,amsfonts}
\usepackage{verbatim,enumerate,array,booktabs,color,bigstrut,prettyref,tikz-cd}
\usepackage{multirow}
\usepackage[all]{xy}
\usepackage[backref]{hyperref}
\usepackage[OT2,T1]{fontenc}
%\usepackage{ctable}
\usepackage{mathtools}

\usepackage{longtable}


\usepackage{mathtools}
\newcommand{\Mod}[1]{\ (\mathrm{mod}\ #1)}
\newcommand{\mathdash}{\relbar\mkern-8mu\relbar}
\newcommand*\circled[2][1.6]{\tikz[baseline=(char.base)]{
    \node[shape=circle, draw, inner sep=1pt, 
        minimum height={\f@size*#1},] (char) {\vphantom{WAH1g}#2};}}
\makeatother



\usepackage{tabularray}
\UseTblrLibrary{amsmath,varwidth}

\usepackage{tabularx}
\usepackage{longtable}
\usepackage{arydshln}


\newcommand\myiso{\stackrel{\mathclap{\normalfont\mbox{\small $p$}}}{-}}
\newcommand\myisot{\stackrel{\mathclap{\normalfont\mbox{\small $3$}}}{-}}

\newcommand{\pref}[1]{\prettyref{#1}}
\newrefformat{eq}{\textup{(\ref{#1})}}
\newrefformat{prty}{\textup{(\ref{#1})}}

\definecolor{mylinkcolor}{rgb}{0.8,0,0}
\definecolor{myurlcolor}{rgb}{0,0,0.8}
\definecolor{mycitecolor}{rgb}{0,0,0.8}
\hypersetup{colorlinks=true,urlcolor=myurlcolor,citecolor=mycitecolor,linkcolor=mylinkcolor,linktoc=page,breaklinks=true}

%\DeclareSymbolFont{cyrletters}{OT2}{wncyr}{m}{n}
%\DeclareMathSymbol{\Sha}{\mathalpha}{cyrletters}{"58}

\addtolength{\textwidth}{4cm} \addtolength{\hoffset}{-2cm}
\addtolength{\marginparwidth}{-2cm}

%\theoremstyle{definition}
\newtheorem{defn}{Definition}[section]
\newtheorem{definition}[defn]{Definition}
\newtheorem{claim}[defn]{Claim}

%\theoremstyle{plain}
\newtheorem{thmA}{Theorem A}
\newtheorem{thmB}{Theorem B}
\newtheorem{thm2}{Theorem}
\newtheorem{prop2}{Proposition}
\newtheorem{note}{Note}

\newtheorem{corollary}[defn]{Corollary}
\newtheorem{lemma}[defn]{Lemma}
\newtheorem{property}[defn]{Property}
\newtheorem{thm}[defn]{Theorem}
\newtheorem{theorem}[defn]{Theorem}
\newtheorem{cor}[defn]{Corollary}
\newtheorem{prop}[defn]{Proposition}
\newtheorem{proposition}[defn]{Proposition}
\newtheorem{thmnn}{Theorem}
\newtheorem{conj}[defn]{Conjecture}

\theoremstyle{definition}
\newtheorem{remarks}{Remarks}
\newtheorem{ack}{Acknowledgements}
\newtheorem{remark}[defn]{Remark}
\newtheorem{question}[defn]{Question}
\newtheorem{example}[defn]{Example}


\newcommand{\Q}{\mathbb Q}
\newcommand{\Qbar}{\overline{\Q}}
\newcommand{\Z}{\mathbb Z}

\newcommand{\modQ}{\,\text{mod}\,(\Q^)^2}

\newcommand{\mysquare}[1]{\tikz{\path[draw] (0,0) rectangle node{\tiny #1} (8pt,8pt) ;}}
\newcommand{\mycircle}[1]{\tikz{\path[draw] (0,0) circle (4pt) node{\tiny #1};}}


%------------------------------------
\newcommand{\Kd}{\operatorname{K}}
\newcommand{\kI}{\operatorname{I}}
\newcommand{\kII}{\operatorname{II}}
\newcommand{\kIII}{\operatorname{III}}
\newcommand{\kIV}{\operatorname{IV}}
%-------------------------------------



\begin{document}
\title{Type $T_8$}
\date{\today}
\maketitle
The isogeny graphs of type $T_8$ are given by
six isogenous elliptic curves:

\[ \begin{tikzcd}
E_1 \ar[dash,dr,"2"] & & & E\\
& E_2 \ar[dash,r,"2"] & E  \ar[dash,ur,"2"]  \ar[dash,dr,"2"]& \\
E    \ar[dash,ur,"2"] &  &  & E    \,.
\end{tikzcd}
\]
 

\begin{comment}
\noindent A hauptmodul for $X_0(8)$ is  
$$
t = 2^8 \displaystyle{\left(\frac{\eta(4\tau)}{\eta(\tau)}\right)^8}\,.
$$
One has
$$
\begin{tblr}{l@{\,=\,}l}
j_1= j(E_1) = j(\tau) & 
t^{-1} \cdot (t + 16)^{-1} \cdot (t^{2} + 16 t + 16)^{3}\\
j_2 = j(E_2) = j(2\tau) & 
t^{-2} \cdot (t + 16)^{-2} \cdot (t^{2} + 16 t + 256)^{3}
\\
j_4 = j(E_4) = j(4\tau) & 
t^{-4} \cdot (t + 16)^{-1} \cdot (t^{2} + 256 t + 4096)^{3}\\
j_{12} = j(E_{12}) = j(\tau+1/2) & 
\left(-1\right) \cdot (t + 16)^{-4} \cdot t^{-1} \cdot (t^{2} - 224 t + 256)^{3}\,,
\end{tblr}
$$
and the subgroup of $\operatorname{Aut} X_0(4)$ that fixes the set of vertices of the graph is
isomorphic to the symmetric group $\mathcal{S}_3$ with elements:


$$
\begin{tblr}{l@{\,=\,}lcc}
    \SetCell[c=3]{r} \text{permutation} & & & \text{order}  \\
   \operatorname{id}(t) & t  &  (j_1,j_2,j_4,j_{12}) & 1 \\
   \sigma(t) & -256/(t+16) & (j_{12},j_2,j_1,j_4) & 3 \\
   \sigma^2(t) & -16(t+16)/t & (j_4,j_2,j_{12},j_1) & 3 \\ 
   \tau(t) & 256/t & (j_4,j_2,j_4,j_{12}) & 2 \\
   \sigma \tau(t) & -(t+16) &   (j_1,j_2,j_4,j_{12})               & 2 \\
   \sigma^2 \tau(t) & -16\, t/(t+16) &  (j_{12},j_2,j_4,j_1)                 & 2 \\
\end{tblr}
$$
\end{comment}
%------------------------------------------------------
\begin{comment}

For $t$ in $\Q\setminus \{0,-16\}$, the $p$-adic valuations of the Fricke involutions applied to $t$ are:

\begin{tblr}
{cells={mode=imath},colspec=|c|c|c|c|}
\hline
p & v_p(W_2(t)) & v_p(W_3(t)) & v_p(W_6(t))  \\
\hline
\neq 2,3 & 0 & 0 & - v_p(t) \\
\hline
 3  & v_3(t+9)-v_3(t+8) & 2 + v_3(t+8)-v_3(t+9) & 2-v_3(t) \\
\hline
 2  & 3+v_2(t+9)+v_2(t+8) & v_2(t+8)-v_2(t+9) & 3-v_2(t) \\
\hline
\end{tblr}

\end{comment}
%------------------------------------------------------

\vskip 0.5truecm

We can (and do) choose Weierstrass equations for $(E_1,E_2,\dots)$ such that their signatures are:

$$
 \begin{tblr}[mode=dmath]{|c|l|}
\hline \SetCell[c=2]{c} T_8 \\ \hline

 c_4(E_1) & (t^8 - 16t^4 + 16)\\

 c_6(E_1) & (t^4 - 8)(t^8 - 16t^4 - 8)\\

 \Delta(E_1) & (t - 2)t^{4}(t + 2)(t^2 + 4)\\ \hline
 
 c_4(E_2) & (t^8 - 16t^4 + 256)\\

 c_6(E_2) & (t^4 - 32)(t^4 - 8)(t^4 + 16)\\

 \Delta(E_2) & (t - 2)^{2}t^{8}(t + 2)^{2}(t^2 + 4)^{2}\\ \hline

 c_4(E_{1,1/2}) & (t^8 - 256t^4 + 4096)\\

 c_6(E_{1,1/2}) & (t^4 - 32)(t^8 + 512t^4 - 8192)\\

 \Delta(E_{1,1/2}) & -1(t - 2)t^{16}(t + 2)(t^2 + 4)\\ \hline

 c_4(E_4) & (t^4 - 4t^3 + 8t^2 + 16t + 16)(t^4 + 4t^3 + 8t^2 - 16t + 16)\\

 c_6(E_4) & (t^2 - 4t - 4)(t^2 + 4t - 4)(t^4 + 16)(t^4 + 24t^2 + 16)\\

 \Delta(E_4) & (t - 2)^{4}t^{4}(t + 2)^{4}(t^2 + 4)^{4}\\ \hline

 c_4(E_{2,1/2}) & (t^4 - 16t^3 + 8t^2 + 64t + 16)(t^4 + 16t^3 + 8t^2 - 64t + 16)\\

 c_6(E_{2,1/2}) & (t^2 - 4t - 4)(t^2 + 4t - 4)(t^8 + 528t^6 - 4000t^4 + 8448t^2 + 256)\\

 \Delta(E_{2,1/2}) & -1(t - 2)^{2}t^{2}(t + 2)^{2}(t^2 + 4)^{8}\\ \hline

 c_4(E_8) & (t^8 + 240t^6 + 2144t^4 + 3840t^2 + 256)\\

 c_6(E_8) & (t^4 - 24t^3 + 24t^2 - 96t + 16)(t^4 + 24t^2 + 16)(t^4 + 24t^3 + 24t^2 + 96t + 16)\\

 \Delta(E_8) & (t - 2)^{8}t^{2}(t + 2)^{8}(t^2 + 4)^{2}\\ \hline

 c_4(E_{4,1/2}) & (t^8 + 240t^7 + 2160t^6 + 6720t^5 + 17504t^4 + 26880t^3 + 34560t^2 + 
15360t + 256)\\

 c_6(E_{4,1/2}) & (t^4 + 24t^3 + 24t^2 + 96t + 16)(t^8 - 528t^7 - 3984t^6 - 14784t^5 - 31648t^4 - 59136t^3 - 63744t^2 - 
33792t + 256)\\

 \Delta(E_{4,1/2}) & (t - 2)^{16}t(t + 2)^{4}(t^2 + 4)\\ \hline

 c_4(E_{16}) & (t^8 - 240t^7 + 2160t^6 - 6720t^5 + 17504t^4 - 26880t^3 + 34560t^2 - 
15360t + 256)\\

 c_6(E_{16}) & (t^4 - 24t^3 + 24t^2 - 96t + 16)(t^8 + 528t^7 - 3984t^6 + 14784t^5 - 31648t^4 + 59136t^3 - 63744t^2 + 
33792t + 256)\\

 \Delta(E_{16}) & -1(t - 2)^{4}t(t + 2)^{16}(t^2 + 4)\\ \hline

\end{tblr}
$$


\newpage

\begin{longtblr}
[caption = {$T_8$ data for $p$=2}]
{cells = {mode=imath},hlines,vlines,measure=vbox,
hline{Z} = {1-5}{0pt},
vline{1} = {Y-Z}{0pt},
colspec  = cclclccc}
%--------------------------------------
\SetCell[c=1]{c} T_8 &\SetCell[c=7]{c} p=2  & & & &  & & \\
\SetCell[c=1]{c} t & E & 
\SetCell[c=1]{c}\operatorname{sig}_2(E) & u & \Kd_2(E) & \SetCell[c=3]{c} u_2(d) & & \\
%--------------------------------------
\SetCell[r=8]{c}
    m =  v_2(t)\ge 0  
& E_1    & ( 4 , 6 , 4 m +4) & 1  &   \kI_{4m-4}^*   & 1 & & 1  \\
& E_2    & ( 4 , 6 , 8m-4 ) & 2   &   \kI_{8m-8}^*   & 1 & & 1 \\
& E_{12} & ( 4 , 6 , 16m-20 ) & 2^{2}   &   \kI_{16m-28}^*   & 1 & & 1  \\
& E_4    & ( 4 , 6 , 4m+4 ) & 2   &   \kI_{4m-4}^*   & 1 & & 1  \\
& E_{22} & ( 4 , 6 , 2m+8 ) & 2   &   \kI_{2m}^*   & 1 &  & 1  \\
& E_8    & ( 4 , 6 , 2m+8 ) & 2   &   \kI_{2m}^*   & 1 &  & 1  \\
& E_{42} & ( 4 , 6 , m+10 ) & 2   &   \kI_{m+2}^*   & 1 &  & 1  \\
& E_{16} & ( 4 , 6 , m+10 ) & 2   &   \kI_{m+2}^*   & 1 &  & 1  \\
%------------------------------------------
\SetCell[r=8]{c}
    \begin{array}{c}
     v_2(t)=1  \\[3pt]
   t/2\equiv \, (8) 
\end{array}
& E_1    & ( 0,0 , 16 m+12 ) & 1 &   \kI_{16m+4}   & 1 & & 1  \\
& E_2    & ( 0,0 , 8m+12 ) &    2^2&   \kI_{8m+4}   & 1 & & 1 \\
& E_{12} & ( 0,0 , 4m +12) & 2^2   &   \kI_{4m+4}   & 1 & & 1  \\
& E_4    & ( 0,0 , 4m +12) & 2^3    &   \kI_{4m+4}   & 1 & & 1  \\
& E_{22} & ( 0,0 , 2m +12) & 2^3    &   \kI_{2m+4}   & 1 &  & 1  \\
& E_8    & ( 0,0 , 2m +12) & 2^4    &   \kI_{2m+4}   & 1 &  & 1  \\
& E_{42} & ( 0,0 , m +12) &   2^5 &   \kI_{m+4}   & 1 &  & 1  \\
& E_{16} & ( 0,0 , m +12) &  2^4  &   \kI_{m+4}   & 1 &  & 1  \\
%------------------------------------------
\SetCell[r=8]{c}
    \begin{array}{c}
     v_2(t)=1  \\[3pt]
   t/2\equiv \, (8) 
\end{array}
& E_1    & ( 0,0 , 16 m+12 ) & 1 &   \kI_{16m+4}   & 1 & & 1  \\
& E_2    & ( 0,0 , 8m+12 ) &    2^2&   \kI_{8m+4}   & 1 & & 1 \\
& E_{12} & ( 0,0 , 4m +12) & 2^2   &   \kI_{4m+4}   & 1 & & 1  \\
& E_4    & ( 0,0 , 4m +12) & 2^3    &   \kI_{4m+4}   & 1 & & 1  \\
& E_{22} & ( 0,0 , 2m +12) & 2^3    &   \kI_{2m+4}   & 1 &  & 1  \\
& E_8    & ( 0,0 , 2m +12) & 2^4    &   \kI_{2m+4}   & 1 &  & 1  \\
& E_{42} & ( 0,0 , m +12) &   2^4 &   \kI_{m+4}   & 1 &  & 1  \\
& E_{16} & ( 0,0 , m +12) &  2^5  &   \kI_{m+4}   & 1 &  & 1  \\
%------------------------------------------
\SetCell[r=8]{c}
    -m =  v_2(t)\le 0  
& E_1    & ( 4 , 6 , 16 m+12 ) & 2^{-(2m+1)}  &   \kI_{16m+4}^*   & 1 & & 1  \\
& E_2    & ( 4 , 6 , 8m+12 ) & 2^{-(2m+1)}   &   \kI_{8m+4}^*   & 1 & & 1 \\
& E_{12} & ( 4 , 6 , 4m +12) & 2^{-(2m+1)}   &   \kI_{4m+4}^*   & 1 & & 1  \\
& E_4    & ( 4 , 6 , 4m +12) & 2^{-(2m+1)}   &   \kI_{4m+4}^*   & 1 & & 1  \\
& E_{22} & ( 4 , 6 , 2m +12) & 2^{-(2m+1)}   &   \kI_{2m+4}^*   & 1 &  & 1  \\
& E_8    & ( 4 , 6 , 2m +12) & 2^{-(2m+1)}   &   \kI_{2m+4}^*   & 1 &  & 1  \\
& E_{42} & ( 4 , 6 , m +12) & 2^{-(2m+1)}   &   \kI_{m+4}^*   & 1 &  & 1  \\
& E_{16} & ( 4 , 6 , m +12) & 2^{-(2m+1)}   &   \kI_{m+4}^*   & 1 &  & 1  \\
%--------------------------------------
 \SetCell[c=5,r=2]{c} & & & & &  d\equiv 1 &  d\equiv 2  & d\equiv 3 \\
                      & & & & & \SetCell[c=3]{c} d \Mod{4} & \\
\end{longtblr}


\end{document}



