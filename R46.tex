\documentclass[11pt]{article}

\usepackage{amsfonts,amssymb,amsmath,amsthm,latexsym,graphics,epsfig,amsfonts}
\usepackage{verbatim,enumerate,array,booktabs,color,bigstrut,prettyref,tikz-cd}
\usepackage{multirow}
\usepackage[all]{xy}
\usepackage[backref]{hyperref}
\usepackage[OT2,T1]{fontenc}
%\usepackage{ctable}
\usepackage{mathtools}

\usepackage{longtable}


\usepackage{mathtools}
\newcommand{\Mod}[1]{\ (\mathrm{mod}\ #1)}
\newcommand{\mathdash}{\relbar\mkern-8mu\relbar}
\newcommand*\circled[2][1.6]{\tikz[baseline=(char.base)]{
    \node[shape=circle, draw, inner sep=1pt, 
        minimum height={\f@size*#1},] (char) {\vphantom{WAH1g}#2};}}
\makeatother



\usepackage{tabularray}
\UseTblrLibrary{amsmath,varwidth}

\usepackage{tabularx}
\usepackage{longtable}
\usepackage{arydshln}


\newcommand\myiso{\stackrel{\mathclap{\normalfont\mbox{\small $p$}}}{-}}
\newcommand\myisot{\stackrel{\mathclap{\normalfont\mbox{\small $3$}}}{-}}

\newcommand{\pref}[1]{\prettyref{#1}}
\newrefformat{eq}{\textup{(\ref{#1})}}
\newrefformat{prty}{\textup{(\ref{#1})}}

\definecolor{mylinkcolor}{rgb}{0.8,0,0}
\definecolor{myurlcolor}{rgb}{0,0,0.8}
\definecolor{mycitecolor}{rgb}{0,0,0.8}
\hypersetup{colorlinks=true,urlcolor=myurlcolor,citecolor=mycitecolor,linkcolor=mylinkcolor,linktoc=page,breaklinks=true}

%\DeclareSymbolFont{cyrletters}{OT2}{wncyr}{m}{n}
%\DeclareMathSymbol{\Sha}{\mathalpha}{cyrletters}{"58}

\addtolength{\textwidth}{4cm} \addtolength{\hoffset}{-2cm}
\addtolength{\marginparwidth}{-2cm}

%\theoremstyle{definition}
\newtheorem{defn}{Definition}[section]
\newtheorem{definition}[defn]{Definition}
\newtheorem{claim}[defn]{Claim}

%\theoremstyle{plain}
\newtheorem{thmA}{Theorem A}
\newtheorem{thmB}{Theorem B}
\newtheorem{thm2}{Theorem}
\newtheorem{prop2}{Proposition}
\newtheorem{note}{Note}

\newtheorem{corollary}[defn]{Corollary}
\newtheorem{lemma}[defn]{Lemma}
\newtheorem{property}[defn]{Property}
\newtheorem{thm}[defn]{Theorem}
\newtheorem{theorem}[defn]{Theorem}
\newtheorem{cor}[defn]{Corollary}
\newtheorem{prop}[defn]{Proposition}
\newtheorem{proposition}[defn]{Proposition}
\newtheorem{thmnn}{Theorem}
\newtheorem{conj}[defn]{Conjecture}

\theoremstyle{definition}
\newtheorem{remarks}{Remarks}
\newtheorem{ack}{Acknowledgements}
\newtheorem{remark}[defn]{Remark}
\newtheorem{question}[defn]{Question}
\newtheorem{example}[defn]{Example}


\newcommand{\Q}{\mathbb Q}
\newcommand{\Qbar}{\overline{\Q}}
\newcommand{\Z}{\mathbb Z}

\newcommand{\modQ}{\,\text{mod}\,(\Q^)^2}

\newcommand{\mysquare}[1]{\tikz{\path[draw] (0,0) rectangle node{\tiny #1} (8pt,8pt) ;}}
\newcommand{\mycircle}[1]{\tikz{\path[draw] (0,0) circle (4pt) node{\tiny #1};}}


%------------------------------------
\newcommand{\Kd}{\operatorname{K}}
\newcommand{\kI}{\operatorname{I}}
\newcommand{\kII}{\operatorname{II}}
\newcommand{\kIII}{\operatorname{III}}
\newcommand{\kIV}{\operatorname{IV}}
%-------------------------------------


\begin{document}
\title{Type $R_4(6)$}
\maketitle



\section{Setting}
The isogeny graphs of type $R_4(6)$ are given by
four isogenous elliptic curves:

\[ \begin{tikzcd}
E_1 \arrow[dash]{r}{3} 
    \arrow[dash]{d}{2} & 
    E_3  \arrow[dash]{d}{2} \\
 E_2 \arrow[dash]{r}{3} & E_6   \,.
\end{tikzcd}
\]

\noindent A hauptmodul for $X_0(6)$ is  
$$
t = 2^3 3^2 \displaystyle{\frac{\eta(2\tau) \eta(6\tau)^5}{\eta(\tau)^5\eta(3\tau)}}\,.
$$
One has
$$
\begin{tblr}{l@{\,=\,}l}
j(E_1) = j(\tau) & 
\displaystyle{\frac{(t+6)^3(t^3+18t^2+84t+24)^3}{t(t+8)^3(t+9)^2}}\\
j(E_2) = j(2\tau) & 
\displaystyle{\frac{(t+12)^3 \left(t^3+12
   t^2+48 t+192\right)^3}{t^2
   (t+8)^6 (t+9)}}
\\
j(E_3) = j(3\tau) & 
\displaystyle{\frac{(t+6)^3 \left(t^3+18
   t^2+324
   t+1944\right)^3}{t^3 (t+8)
   (t+9)^6}}\\
j(E_6) = j(6\tau) & 
\displaystyle{\frac{(t+12)^3(t^3+252t^2+3888t+15552)^3}{t^6(t+8)^2(t+9)^3}}\,,
\end{tblr}
$$
and the Fricke involutions of $X_0(6)$ are given by:
$$
W_2(t)=-8 (t+9)/(t+8) \,,\qquad 
W_3(t)=- 9 (t+8)/(t+9)\,,\qquad
W_6(t)= 72/t\,.
$$

For $t$ in $\Q\setminus \{0,-8,-9\}$, the $p$-adic valuations of the Fricke involutions applied to $t$ are:

\begin{tblr}
{cells={mode=imath},colspec=|c|c|c|c|}
\hline
p & v_p(W_2(t)) & v_p(W_3(t)) & v_p(W_6(t))  \\
\hline
\neq 2,3 & 0 & 0 & - v_p(t) \\
\hline
 3  & v_3(t+9)-v_3(t+8) & 2 + v_3(t+8)-v_3(t+9) & 2-v_3(t) \\
\hline
 2  & 3+v_2(t+9)+v_2(t+8) & v_2(t+8)-v_2(t+9) & 3-v_2(t) \\
\hline
\end{tblr}

\vskip 0.5truecm

We can (and do) choose Weierstrass equations for $(E_1,E_2,E_3,E_6)$ such that their signatures are:


\vskip 0.3truecm

\[
\begin{tblr}{|c|l|}
\hline \SetCell[c=2]{c} R_4(6) \\ \hline
c_4(E_1) & 
(t + 6)  (t^{3} + 18\, t^{2} + 84 \, t + 24)\\
c_6(E_1) & 
(t^{2} + 12\, t + 24)  (t^{4} + 24\, t^{3} + 192\, t^{2} + 504\, t - 72)\\  
\Delta(E_1) & 
t   (t + 8)^{3} (t + 9)^{2}\\
\hline
c_4(E_2) & 
 (t + 12)  (t^{3} + 
 12\,  t^{2} + 48\, t + 192) \\
c_6(E_2) & 
(t^{2} + 12\,  t + 24) 
 (t^{4} + 24\,  t^{3} + 
192\,    t^2 - 4608) \\  
\Delta(E_2) & 
  t^{2}    (t + 8)^{6} (t + 9) \\
\hline
c_4(E_3) & 
(t + 6)  (t^{3} + 18\,   t^{2} + 324\,   t + 1944) \\
c_6(E_3) & 
 (t^{2} + 36\,   t + 216)  (t^{4} - 216\,   t^{2} - 1944\,   t - 5832)\\  
\Delta(E_3) & 
t^{3}   (t + 8)  (t + 9)^{6}\\
\hline
c_4(E_6) & 
(t + 12)  (t^{3} + 252\,   t^{2} + 3888\,   t + 15552) \\
c_6(E_6) & 
 (t^{2} + 36\,   t + 216)  (t^{4} - 504\,   t^{3} - 13824\,   t^{2} - 124416\,   t - 373248) \\  
\Delta(E_6) & 
 t^{6}  (t + 8)^{2}  (t + 9)^{3}\\
\hline
\end{tblr}
\]

\vskip 0.3truecm

\noindent and, with this choice, the isogeny graph is normalized.

%\pagebreak

With regard to the action of the Fricke involutions 
on the isogeny graph, 
it can be displayed as follows:

\begin{comment}
$$
\begin{array}{llll}
    E_1\vert\operatorname{Id}=E_1 & E_1\vert \operatorname{W_2}=E_2 &
     E_1\vert \operatorname{W_3}=E_3^{-3} &  E_1\vert \operatorname{W_6}=E_6^{-3} \\
    E_2\vert \operatorname{Id}=E_2 & E_2\vert \operatorname{W_2}=E_1 &
     E_2\vert \operatorname{W_3}=E_6^{-3} &  E_2\vert \operatorname{W_6}=E_3^{-3} \\
    E_3\vert \operatorname{Id}=E_3 & E_3\vert \operatorname{W_2}=E_6 &
     E_3\vert \operatorname{W_3}=E_1^{-3} &  E_3\vert \operatorname{W_6}=E_2^{-3} \\
    E_6\vert \operatorname{Id}=E_6 & E_6\vert \operatorname{W_2}=E_3 &
     E_6\vert \operatorname{W_3}=E_2^{-3} &  E_6\vert \operatorname{W_6}=E_1^{-3} \\
\end{array}
$$
\end{comment}

\[ 
\begin{tikzcd}
E_1 \arrow[dash]{r}{3} 
    \arrow[dash]{d}{2} & 
    E_3  \arrow[dash]{d}{2} \\
 E_2 \arrow[dash]{r}{3} & E_6   
\end{tikzcd}
\phantom{\colon W_2}
\hskip 1truecm
\begin{tikzcd}
E_2 \arrow[dash]{r}{3} 
    \arrow[dash]{d}{2} & 
    E_6  \arrow[dash]{d}{2} \\
 E_1 \arrow[dash]{r}{3} & E_3   
\end{tikzcd}
\colon W_2
\]

\[ 
\begin{tikzcd}
E_3^{-3} \arrow[dash]{r}{3} 
    \arrow[dash]{d}{2} & 
    E_1^{-3}  \arrow[dash]{d}{2} \\
 E_6^{-3} \arrow[dash]{r}{3} & E_2^{-3}   
\end{tikzcd} \colon W_3
\hskip 1truecm
\begin{tikzcd}
E_6^{-3} \arrow[dash]{r}{3} 
    \arrow[dash]{d}{2} & 
    E_2^{-3}  \arrow[dash]{d}{2} \\
 E_3^{-3} \arrow[dash]{r}{3} & E_1^{-3} 
\end{tikzcd}
\colon W_6
\]
where the  
arrows correspond to the dual isogenies.


\section{Kodaira symbols \& Pal coefficients}

\begin{longtblr}
[caption= $R_4(6)$ data for $p\neq 2${,} $3$]
{
cells={mode=imath},hlines,vlines,measure=vbox,
colspec  = cclclc}





\SetCell[c=1]{c} R_4(6) &\SetCell[c=5]{c} p\ne 2, 3  & &  u_p(d)  & \\
\SetCell[c=1]{c} t & E & 
\SetCell[c=1]{c} \operatorname{sig}_p(E) & u & \Kd_p(E) & u_p(d)\\
\SetCell[r=4]{c} 
m= v_p(t)>0 
& E_1 & (0,0,m) & 1 &  I_m & 1\\
& E_2 & (0,0,2m) & 1 & I_{2m} & 1 \\
& E_3 & (0,0,3m) & 1 & I_{3m}  & 1\\
& E_6 & (0,0,6m) & 1 & I_{6m}  & 1\\
\SetCell[r=4]{c} 
\begin{array}{c}
     v_p(t)=0  \\[3pt]
    (m = v_p(t+9)>0)  
\end{array}
& E_1 & (0,0,2m) & 1 & I_{2m}  & 1 \\
& E_2 & (0,0,m) & 1 & I_{m} & 1\\
& E_3 & (0,0,6m) & 1 & I_{6m} & 1\\
& E_6 & (0,0,3m) & 1 & I_{3m}  & 1\\
\SetCell[r=4]{c} 
\begin{array}{c}
     v_p(t)=0  \\[3pt]
    (m = v_p(t+8)>0)  
\end{array}
& E_1 & (0,0,3m) & 1 & I_{3m}   & 1\\
& E_2 & (0,0,6m) & 1 & I_{6m}  & 1\\
& E_3 & (0,0,m) & 1 & I_{m} & 1\\
& E_6 & (0,0,2m) & 1 & I_{2m}  & 1\\
\SetCell[r=4]{c} 
-m=v_p(t)<0 
& E_1 & (0,0,6m) & p^{-m} & I_{6m}  & 1\\
& E_2 & (0,0,3m) & p^{-m} & I_{3m} & 1\\
& E_3 & (0,0,2m) & p^{-m}& I_{2m} & 1\\
& E_6 & (0,0,m) & p^{-m} & I_{m} & 1\\
\end{longtblr}

\vskip 0.3truecm

\newpage


\begin{longtblr}
[caption = {$R_4(6)$ data for $p$=3}]
{cells = {mode=imath},hlines,vlines,measure=vbox,
hline{Z} = {1-5}{0pt},
vline{1} = {Y-Z}{0pt},
colspec  = cclclcc}
%--------------------------------------
\SetCell[c=1]{c} R_4(6) &\SetCell[c=6]{c} p=3  & & & & & \\
t & E & 
\SetCell[c=1]{c} \operatorname{sig}_3(E) & u & \Kd_3(E) & \SetCell[c=2]{c} u_3(d)   \\
%--------------------------------------
\SetCell[r=4]{c} m=v_3(t)> 2 
& E_1 & (2,3,m+4) & 1 & I_{m-2}^* & 3 & 1 \\
& E_2 & (2,3,2m+2) & 1 & I_{2(m-2)}^* & 3 & 1\\
& E_3 & (2,3,3m) & 3 & I_{3(m-2)}^* & 3 & 1\\
& E_6 & (2,3,6m-6) & 3 & I_{6(m-2)}^* & 3 & 1\\
\SetCell[r=4]{c}
\begin{array}{c}
     v_3(t)=2   \\[3pt]
   m=v_3(t+9)  & 
\end{array}
& E_1 & (2,3,2m+2) & 1 & I^*_{2(m-2)} & 3 & 1 \\
& E_2 & (2,3,m+4) & 1 & I^*_{m-2} & 3 & 1 \\
& E_3 & (2,3,6m-6) & 3 & I^*_{6(m-2)} & 3 & 1 \\
& E_6 & (2,3,3m) & 3 & I^*_{3(m-2)} & 3 & 1 \\
\SetCell[r=4]{c} v_3(t)=1 
& E_1 & (\geq 2,3,3) & 1 & III & 1 & 1\\
& E_2 & (\geq 2 ,3,3) & 1 & III & 1 & 1\\
& E_3 & (\geq 4,6,9) & 1 & III^* & 3 & 1\\
& E_6 & (\geq 4,6,9) & 1 & III^* & 3 & 1 \\
\SetCell[r=4]{c}
\begin{array}{c}
     v_3(t)=0   \\[3pt]
   m=v_3(t+8)>0  & 
\end{array}
& E_1 & (0,0,3m) & 1 &  I_{3m} & 1 & 1\\
& E_2 & (0,0,6m) & 1 & I_{6m} & 1 & 1\\
& E_3 & (0,0,m) & 1 & I_m & 1 & 1 \\
& E_6 & (0,0,2m) & 1 & I_{2m} & 1 & 1\\
\SetCell[r=4]{c} -m=v_3(t)<0
& E_1 & (0,0,6m) & 3^{-m} & I_{6m} & 1 & 1 \\
& E_2 & (0,0,3m) & 3^{-m} & I_{3m} & 1 & 1\\
& E_3 & (0,0,2m) & 3^{-m} & I_{2m} & 1 & 1\\
& E_6 & (0,0,m) & 3^{-m} & I_{m} & 1 & 1\\
%-------------------------------------------------
 \SetCell[c=5,r=2]{c} & & & & & d\equiv 0  & d\not\equiv 0 \\
                      & & & & & \SetCell[c=2]{c} d \Mod 3 & \\
\end{longtblr}

\newpage

\begin{longtblr}
[caption = {$R_4(6)$ data for $p$=2}]
{cells = {mode=imath},hlines,vlines,measure=vbox,
hline{Z} = {1-5}{0pt},
vline{1} = {Y-Z}{0pt},
colspec  = cclclccc}
%--------------------------------------
\SetCell[c=1]{c} R_4(6) &\SetCell[c=7]{c} p=2  & & & &  & & \\
\SetCell[c=1]{c} t & E & 
\SetCell[c=1]{c}\operatorname{sig}_2(E) & u & \Kd_2(E) & \SetCell[c=3]{c} u_2(d)   &   &    \\
%--------------------------------------
\SetCell[r=4]{c} m=v_2(t)> 3 
& E_1 & (4,6,m+9) & 1 & I_{m+1}^* & 1 & 1 & 2\\
& E_2 & (4,6,2m+6) & 2 & I_{2m-2}^*& 1 & 1 & 2 \\
& E_3 & (4,6,3m+3) & 1 & I_{3m-5}^* & 1 & 1 & 2\\
& E_6 & (4,6,6m-6) & 2 & I_{6m-14}^*& 1 & 1 & 2 \\
%--------------------------------------
\SetCell[r=4]{c} 
\begin{array}{c}
  v_2(t)= 3     \\[3pt]
  m = v_2(t+8)   
\end{array}
& E_1 & (4,6,3+3m) & 1 & I_{3m-5}^* & 1 & 1 & 2\\
& E_2 & (4,6,6m-6) & 2 & I_{6m-14}^* & 1 & 1 & 2\\
& E_3 & (4,6,m+9) & 1 & I_{m+1}^* & 1 & 1 & 2\\
& E_6 & (4,6,2m+6) & 2 & I_{2m-2}^* & 1 & 1 & 2\\
%--------------------------------------
\SetCell[r=4]{c} v_2(t)= 2 
& E_1 & (4,6,8) & 1 & I_{0}^* & 1 & 1 & 1\\
& E_2 & (\geq 4,5,4) & 2 & II & 1 & 1 & 1\\
& E_3 & (4,6,8) & 1 & I_{0}^* & 1 & 1 & 1\\
& E_6 & (\geq 4,5,4) & 2 & II & 1 & 1 & 1\\
%--------------------------------------
\SetCell[r=4]{c} v_2(t)= 1 
& E_1 & (\geq 4,5,4) & 1 & II & 1 & 1 & 1\\
& E_2 & (4,6,8) & 1 & I_{0}^* & 1 & 1 & 1\\
& E_3 & (\geq 4,5,4) & 1 & II & 1 & 1 & 1\\
& E_6 & (4,6,8) & 1 & I_{0}^* & 1 & 1 & 1\\
%--------------------------------------
\SetCell[r=4]{c}
\begin{array}{c}
  v_2(t)= 0     \\[3pt]
  m = v_2(t+9)>0   
\end{array}
%m=v_2(t+9)>0 
& E_1 & (4,6,2m+12) & 2^{-1} & I_{2m+4}^* & 1 & 1 & 2\\
& E_2 & (4,6,m+12) & 2^{-1} & I_{m+4}^* & 1 & 1 & 2\\
& E_3 & (4,6,6m+12) & 2^{-1} & I_{6m+4}^* & 1 & 1 & 2\\
& E_6 & (4,6,3m+12) & 2^{-1} & I_{3m+4}^*& 1 & 1 & 2 \\
%--------------------------------------
\SetCell[r=4]{c} -m=v_2(t)<0 
& E_1 & (4,6,6m+12) & 2^{-m-1} & I_{6m+4}^* & 1 & 1 & 2\\
& E_2 & (4,6,3m+12) & 2^{-m-1} & I_{3m+4}^* & 1 & 1 & 2\\
& E_3 & (4,6,2m+12) & 2^{-m-1} & I_{2m+4}^* & 1 & 1 & 2\\
& E_6 & (4,6,m+12) & 2^{-m-1} & I_{m+4}^* & 1 & 1 & 2\\
%----------------------------------------------
 \SetCell[c=5,r=2]{c} & & & & &  d\equiv 1 &  d\equiv 2  & d\equiv 3 \\
                      & & & & & \SetCell[c=3]{c} d \Mod{4} & \\
\end{longtblr}


\newpage

\section{Conclusion}

\begin{prop}
Let 
\[ \begin{tikzcd}
E_1 \arrow[dash]{r}{3} 
    \arrow[dash]{d}{2} & 
    E_3  \arrow[dash]{d}{2} \\
 E_2 \arrow[dash]{r}{3} & E_6   
\end{tikzcd}
\]
be a $\mathbf{Q}$-isogeny graph of type $R_4(6)$ corresponding to a given $t$ in $\mathbf{Q}\setminus \{0,-8,-9\}$ as above. 
For every square-free integer $d$, 
the probability of a vertex
to be the Faltings curve (circled)
in the twisted graph 
\[ \begin{tikzcd}
E_1^d \arrow[dash]{r}{3}
    \arrow[dash]{d}{2} & 
    E_3^d  \arrow[dash]{d}{2} \\
 E_2^d \arrow[dash]{r}{3} &    \, E_6^d
\end{tikzcd}
\]
is given by:

\begin{longtblr}{|c|c|c|c|c|}
\hline
\SetCell[c=2]{c} R_4(6) & & \SetCell[c=1]{c}\text{twisted isogeny graph} & $d$ & \SetCell[c=1]{c}\text{prob} \\
 \hline
 v_2(t)\leq 1 & v_3(t)\leq 0 &
\makecell{%
        \begin{tikzcd}[ampersand replacement=\&]
\circled[0.8]{$E_1^d$} \ar[r] 
   \ar[d]  \& 
    E_3^d  \ar[d]  \\
 E_2^d \ar[r]  \&    \, E_6^d
        \end{tikzcd}}  
& &  1 \\
%--------------------------------------
\hline
\SetCell[r=2]{c} v_2(t)\leq 1 & \SetCell[r=2]{c} v_3(t) =1
& 
\makecell{%
        \begin{tikzcd}[ampersand replacement=\&]
\circled[0.8]{$E_1^d$} \ar[r] 
   \ar[d]  \& 
    E_3^d  \ar[d]  \\
 E_2^d \ar[r]  \&    \, E_6^d
        \end{tikzcd}} 
     &d\equiv 0\,(3)   & 1/4\\ 
& &  \makecell{%
        \begin{tikzcd}[ampersand replacement=\&]
E_1^d \ar[d] 
    \& 
    \circled[0.8]{$E_3^d$} \ar[l]   \ar[d]  \\
E_2^d  \&    \, E_6^d  \ar[l] 
        \end{tikzcd}} 
&d\not\equiv 0\,(3) &  3/4 \\
%--------------------------------------
 \hline
  v_2(t)\leq 1 & v_3(t)\geq 2 &
\makecell{%
        \begin{tikzcd}[ampersand replacement=\&]
E_1^d \ar[d] 
    \& 
    \circled[0.8]{$E_3^d$} \ar[l]   \ar[d]  \\
E_2^d  \&    \, E_6^d  \ar[l] 
        \end{tikzcd}} 
& &  1 \\
%--------------------------------------
\hline
 v_2(t)\geq 2 & v_3(t)\leq 0 &
\makecell{%
        \begin{tikzcd}[ampersand replacement=\&]
E_1^d  \ar[r]
   \& 
    E_3^d   \\
 \circled[0.8]{$E_2^d$}   \ar[u]  \ar[r]   \&    \, E_6^d \ar[u]
        \end{tikzcd}}  
& &  1 \\
%--------------------------------------
\hline
\SetCell[r=2]{c} v_2(t)\ge 2 & \SetCell[r=2]{c} v_3(t)=1 
& 
\makecell{%
        \begin{tikzcd}[ampersand replacement=\&]
E_1^d  \ar[r]
   \& 
    E_3^d   \\
 \circled[0.8]{$E_2^d$}   \ar[u]  \ar[r]   \&    \, E_6^d \ar[u]
        \end{tikzcd}}    
      &d\equiv 0\,(3)  & 1/4\\ 
& &  \makecell{%
        \begin{tikzcd}[ampersand replacement=\&]
E_1^d \& 
    E_3^d  \ar[l]  \\
 E_2^d \ar[u]  \&    \circled[0.8]{$E_6^d$}\ar[l]\ar[u]
        \end{tikzcd}} 
&d\not\equiv 0\,(3) &  3/4 \\
%--------------------------------------
 \hline
 v_2(t)\geq 2 & v_3(t)\geq 2 &
\makecell{%
        \begin{tikzcd}[ampersand replacement=\&]
E_1^d \& 
    E_3^d  \ar[l]  \\
 E_2^d \ar[u]  \&    \circled[0.8]{$E_6^d$}\ar[l]\ar[u]
        \end{tikzcd}}  
& &  1 \\
\hline
\end{longtblr}

\end{prop}



\noindent{\it Proof.} From the previous tables one gets:

\vskip 0.5truecm


\begin{tblr}{cells={mode=imath},hlines,vlines,measure=vbox}
%\hline
\SetCell[c=1]{c} t &\SetCell[c=1]{c} [u(E)]  & \SetCell[c=1]{c} [u(E)(d)] & \\
\hline
 \SetCell[c=1]{c} v_2(t)\geq 2 & \SetCell[r=1]{c} (1:2:1:2) & (1:1:1:1) &    \\
\SetCell[r=1]{c} v_2(t)\leq 1 & \SetCell[r=1]{c} (1:1:1:1) & (1:1:1:1) &  \\
\hline
\SetCell[r=1]{c} v_3(t)\ge 2 & \SetCell[r=1]{c} (1:1:3:3) & (1:1:1:1) &   \\
\SetCell[r=2]{c} v_3(t)=1 & \SetCell[r=2]{c} (1:1:1:1) & (1:1:1:1) & d\not\equiv 0\,(3)   \\
&  & \SetCell[r=1]{c} (1:1:3:3) & d\equiv 0\,(3)  \\
\SetCell[r=1]{c} v_3(t)\leq 0 & \SetCell[r=1]{c} (1:1:1:1) & (1:1:1:1) &  
\end{tblr}

\vskip 0.7truecm


\begin{tblr}{cells={mode=imath},hlines,vlines,measure=vbox}
%\hline
\SetCell[c=1]{c} t &\SetCell[c=1]{c} [u(E)]  & \SetCell[c=1]{c} [u(E)(d)] & \\
\hline
\SetCell[r=2]{c} v_2(t)>2 & \SetCell[r=2]{bg=teal2,fg=white,c} (1,2,1,2) & (1,1,1,1) & d\equiv 1,2\,(4)   \\
&  & (2,2,2,2) & d\equiv 3\,(4)  \\
\SetCell[r=1]{c} v_2(t)=2 & \SetCell[r=1]{bg=teal2,fg=white,c} (1,2,1,2) & (1,1,1,1) &   \\
\SetCell[r=1]{c} v_2(t)=1 & \SetCell[r=1]{c} (1,1,1,1) & (1,1,1,1) &  \\
\SetCell[r=2]{c} \begin{array}{c} v_2(t)=0\\ v_2(t+9)>0\end{array} & \SetCell[r=2]{c} (2^{-1},2^{-1},2^{-1},2^{-1}) & (1,1,1,1) & d\equiv 1,2\,(4)  \\
&  & (2,2,2,2) & d\equiv 3\,(4)  \\
\SetCell[r=2]{c} v_2(t)<0 & \SetCell[r=2]{c} (2^{-m-1},2^{-m-1},2^{-m-1},2^{-m-1}) & (1,1,1,1) & d\equiv 1,2\,(4)   \\
&  & (2,2,2,2) & d\equiv 3\,(4)  \\
\hline
\SetCell[r=2]{c} v_3(t)\ge 2 & \SetCell[r=2]{bg=teal2,fg=white,c} (1,1,3,3) & (1,1,1,1) & d\not\equiv 0\,(3)   \\
&  & (3,3,3,3) & d\equiv 0\,(3)  \\
\SetCell[r=2]{c} v_3(t)=1 & \SetCell[r=2]{c} (1,1,1,1) & (1,1,1,1) & d\not\equiv 0\,(3)   \\
&  & \SetCell[r=1]{bg=teal2,fg=white,c} (1,1,3,3) & d\equiv 0\,(3)  \\
\SetCell[r=1]{c} v_3(t)<0 & \SetCell[r=1]{c} (3^{-m},3^{-m},3^{-m},3^{-m}) & (1,1,1,1) &    \\
\SetCell[r=1]{c} \begin{array}{c}v_3(t)=0\\ v_3(t+8)> 0\end{array} & \SetCell[r=1]{c} (1,1,1,1) & (1,1,1,1) &    \\
\hline
\SetCell[r=1]{c} v_p(t)\ge 0 &  \SetCell[c=1]{c} (1,1,1,1) & \SetCell[r=2]{c} (1,1,1,1) & \SetCell[r=2]{c}  \\
\SetCell[r=1]{c} v_p(t)=-m<0 & \SetCell[r=1]{c} (p^{-m},p^{-m},p^{-m},p^{-m}) & &   \\
\end{tblr}

\vskip 0.35truecm


\end{document}



\section{Kodaira symbols}

\begin{longtblr}
[caption= $p\operatorname{-Kodaira}(E_i)$]
{
cells={mode=imath},hlines,vlines,measure=vbox}
\hline
\SetCell[c=5]{c} p\neq 2,3 & \\
\hline
\SetCell[c=1]{c} t & E & \operatorname{sig}(E) & u & p\operatorname{-Kod}(E) \\
\hline
\SetCell[r=4]{c} 
m= v_p(t)>0 
& E_1 & (0,0,m) & 1 &  I_m \\
& E_2 & (0,0,2m) & 1 & I_{2m} \\
& E_3 & (0,0,3m) & 1 & I_{3m} \\
& E_6 & (0,0,6m) & 1 & I_{6m} \\
\hline
\SetCell[r=4]{c} 
-m=v_p(t)<0 
& E_1 & (0,0,6m) & p^{m} & I_{6m} \\
& E_2 & (0,0,3m) & p^{m} & I_{3m}\\
& E_3 & (0,0,2m) & p^{m}& I_{2m}\\
& E_6 & (0,0,m) & p^{m} & I_{m}\\
\hline
\SetCell[r=4]{c} 
m = v_p(t+9)>0
& E_1 & (0,0,2m) & 1 & I_{2m} \\
& E_2 & (0,0,m) & 1 & I_{m}\\
& E_3 & (0,0,6m) & 1 & I_{6m}\\
& E_6 & (0,0,3m) & 1 & I_{3m} \\
\hline
\SetCell[r=4]{c} 
-m = v_p(t+9)<0
& E_1 & (0,0,6 m) & p^{m} & I_{6m} \\
& E_2 & (0,0,3m) & p^{m} & I_{3m}\\
& E_3 & (0,0,2m) & p^{m}&  I_{2m}\\
& E_6 & (0,0,m) & p^{m} & I_{m}\\
\hline
\SetCell[r=4]{c} 
m = v_p(t+8)>0 
& E_1 & (0,0,3m) & 1 & I_{3m}  \\
& E_2 & (0,0,6m) & 1 & I_{6m} \\
& E_3 & (0,0,m) & 1 & I_{m}\\
& E_6 & (0,0,2m) & 1 & I_{2m} \\
\hline
\SetCell[r=4]{h} 
-m = v_p(t+8)<0 
& E_1 & (0,0,6m) & p^m & I_{6m}  \\
& E_2 & (0,0,3m) & p^m & I_{3m} \\
& E_3 & (0,0,2m) & p^m & I_{2m}\\
& E_6 & (0,0,m) & p^m & I_{m} \\
\hline
\end{longtblr}

\vskip 0.3truecm


\begin{longtblr}
[caption= $3\operatorname{-Kodaira}(E_i)$]
{cells={mode=imath},hlines,vlines,measure=vbox}
\hline
\SetCell[c=5]{c} p=3 & & & & \\
\hline
\SetCell[c=1]{c} t & E & 
\SetCell[c=1]{c}
\operatorname{sig}(E) & \SetCell[c=1]{c} u & p\operatorname{-Kod}(E) \\
\hline
\SetCell[r=4]{c} m=v_3(t)> 2 
& E_1 & (2,3,m+4) & 1 & I_{m-2}^* \\
& E_2 & (2,3,2m+2) & 1 & I_{2(m-2)}^* \\
& E_3 & (2,3,3m) & 3^{-1} & I_{3(m-2)}^* \\
& E_6 & (2,3,6m-6) & 3^{-1} & I_{6(m-2)}^* \\
\hline
\SetCell[r=4]{c} -m=v_3(t)\leq -1 
& E_1 & (0,0,6m) & 3^{m} & I_{6m} \\
& E_2 & (0,0,3m) & 3^{m} & I_{3m} \\
& E_3 & (0,0,2m) & 3^{m} & I_{2m} \\
& E_6 & (0,0,m) & 3^{m} & I_{m} \\
\hline
\SetCell[r=4]{c} v_3(t)=1 
& E_1 & (\geq 2,3,3) & 1 & III \\
& E_2 & (\geq 2 ,3,3) & 1 & III \\
& E_3 & (\geq 4,6,9) & 1 & III^* \\
& E_6 & (\geq 4,6,9) & 1 & III^*\\
\hline
\SetCell[r=4]{c}
\begin{array}{c}
     v_3(t)=2   \\[3pt]
   (m=v_3(t+9))  & 
\end{array}
& E_1 & (2,3,2m+2) & 1 & I^*_{2(m-2)} \\
& E_2 & (2,3,m+4) & 1 & I^*_{m-2} \\
& E_3 & (2,3,6m-6) & 3^{-1} & I^*_{6(m-2)}\\
& E_6 & (2,3,3m) & 3^{-1} & I^*_{3(m-2)}\\
\hline
\SetCell[r=4]{h} m=v_3(8+t)>0 
& E_1 & (0,0,3m) & 1 &  I_{3m} \\
& E_2 & (0,0,6m) & 1 & I_{6m} \\
& E_3 & (0,0,m) & 1 & I_m\\
& E_6 & (0,0,2m) & 1 & I_{2m}\\
\hline
\SetCell[r=4]{c} -m=v_3(8+t)<0 
& E_1 & (0,0,6m) &  3^m &  I_{6m} \\
& E_2 & (0,0,3m) & 3^m & I_{3m} \\
& E_3 & (0,0,2m) & 3^m & I_{2m} \\
& E_6 & (0,0,m) & 3^m & I_m\\
\hline
\end{longtblr}

\vskip 0.3truecm

\begin{longtblr}
[caption= $2\operatorname{-Kodaira}(E_i)$]
{cells={mode=imath},hlines,vlines,measure=vbox}
\hline
\SetCell[c=5]{c} p=2 & & & & \\
\hline
\SetCell[c=1]{c} t & E & 
\SetCell[c=1]{c}
\operatorname{sig}(E) & \SetCell[c=1]{c} u & p\operatorname{-Kod}(E) \\
\hline
\SetCell[r=4]{c} m=v_2(t)> 3 
& E_1 & (4,6,m+9) & 1 & I_{m+1}^* \\
& E_2 & (4,6,2m+6) & 2^{-1} & I_{2m-2}^* \\
& E_3 & (4,6,3m+3) & 1 & I_{3m-5}^* \\
& E_6 & (4,6,6m-6) & 2^{-1} & I_{6m-14}^* \\
\hline
\SetCell[r=4]{c} 
\begin{array}{c}
  v_2(t)= 3     \\[3pt]
  (m = v_2(t+8))   
\end{array}
& E_1 & (4,6,3+3m) & 1 & I_{3m-5}^* \\
& E_2 & (4,6,6m-6) & 2^{-1} & I_{6m-14}^* \\
& E_3 & (4,6,m+9) & 1 & I_{m+1}^* \\
& E_6 & (4,6,2m+6) & 2^{-1} & I_{2m-2}^* \\
\hline
\SetCell[r=4]{c} v_2(t)= 2 
& E_1 & (4,6,8) & 1 & I_{0}^* \\
& E_2 & (\geq 4,5,4) & 2^{-1} & II \\
& E_3 & (4,6,8) & 1 & I_{0}^* \\
& E_6 & (\geq 4,5,4) & 2^{-1} & II \\
\hline
\SetCell[r=4]{c} v_2(t)= 1 
& E_1 & (\geq 4,5,4) & 1 & II \\
& E_2 & (4,6,8) & 1 & I_{0}^* \\
& E_3 & (\geq 4,5,4) & 1 & II \\
& E_6 & (4,6,8) & 1 & I_{0}^* \\
\hline
\SetCell[r=4]{c} m=v_2(t+9)>0 
& E_1 & (4,6,2m+12) & 2 & I_{2m+4}^* \\
& E_2 & (4,6,m+12) & 2 & I_{m+4}^* \\
& E_3 & (4,6,6m+12) & 2 & I_{6m+4}^* \\
& E_6 & (4,6,3m+12) & 2 & I_{3m+4}^* \\
\hline
\SetCell[r=4]{c} -m=v_2(t)\leq -1 
& E_1 & (4,6,6m+12) & 2^{m+1} & I_{6m+4}^* \\
& E_2 & (4,6,3m+12) & 2^{m+1} & I_{3m+4}^* \\
& E_3 & (4,6,2m+12) & 2^{m+1} & I_{2m+4}^* \\
& E_6 & (4,6,m+12) & 2^{m+1} & I_{m+4}^* \\
\hline
\SetCell[r=4]{c} -m=v_2(t+9)<0 
& E_1 & (4,6,6m+12) & 2^{m+1} & I_{6m+4}^* \\
& E_2 & (4,6,3m+12) & 2^{m+1} & I_{3m+4}^* \\
& E_3 & (4,6,2m+12) & 2^{m+1} & I_{2m+4}^* \\
& E_6 & (4,6,m+12) & 2^{m+1} & I_{m+4}^* \\
\hline
\end{longtblr}


\newpage
\section{OLD computations}

\vskip 1truecm

\begin{tblr}{cells={mode=imath},hlines,vlines,measure=vbox}
\hline
\SetCell[c=5]{c} p=2 (NOT-YET!) & \\
\hline
\SetCell[c=1]{c} t & E & \operatorname{sig}(E) & 
\SetCell[c=1]{c} u & p\operatorname{-Kod}(E) \\
\hline
\SetCell[r=4]{c}  m+3=v_2(t)\geq 4 
& E_1 & (0,0,m) & 2^{-6} & I_{m} \\
& E_2 & (0,0,2m) & 2^{-7} & I_{2m} \\
& E_3 & (0,0,3m) & 2^{-6} & I_{3m} \\
& E_6 & (0,0,6m) & 2^{-7} & I_{6m} \\
\hline
\SetCell[r=4]{c}  
\begin{array}{c}
v_2(t)=3\\[3pt]
t = 8(1+2+2^2+\dots+2^{k-1}+0\cdot2^k+\dots)
\end{array}
& E_1 & (0,0,3k) & 2^{-6} & I_{3k} \\
& E_2 & (0,0,6k) & 2^{-7} & I_{6k} \\
& E_3 & (0,0,k) & 2^{-6} & I_{k} \\
& E_6 & (0,0,2k) & 2^{-7} & I_{2k} \\
\hline
\SetCell[r=4]{c}  v_2(t)=2
& E_1 & (4,6,8) & 2^{-5} & I_0^* \\
& E_2 & (4,5,4) & 2^{-6} &  II \\
& E_3 & (4,6,8) & 2^{-5} &  I_0^* \\
& E_6 & (4,5,4) & 2^{-6} & II \\
\hline
\SetCell[r=4]{c} 
\begin{array}{c}
v_2(t)=1\\[3pt]
v_2(t/2-3) \geq 2
\end{array}
& E_1 & (8,8,10) & 2^{-5} &  I_0^* \\
& E_2 & (6,9,14) & 2^{-6} & I_4^* \\
& E_3 & (8,8,10) & 2^{-5} & I_0^* \\
& E_6 & (6,9,14) & 2^{-6} & I_4^* \\
\hline
\SetCell[r=4]{c}  
\begin{array}{c}
v_2(t)=1\\[5pt]
\text{Let $k\geq 4$ first (even) with}\\[3pt]
v_2(t/2-(1+\displaystyle{\sum_{n=2}^{k-1} 2^n)})\geq k+1
\end{array}
& E_1 & (k+5,5,4) & 2^{-(k+10)/2} & IV  \\
& E_2 & (4,6,8) & 2^{-(k+10)/2} & IV^* \\
& E_3 & (k+5,5,4) & 2^{-(k+10)/2} & IV \\
& E_6 & (4,6,8) & 2^{-(k+10)/2} & IV^* \\
\hline
\SetCell[r=4]{c}  
\begin{array}{c}
v_2(t)=1\\[5pt]
\text{Let $k\geq 3$ first (odd) with}\\[3pt]
v_2(t/2-(1+\displaystyle{\sum_{n=2}^{k-1} 2^n)})\geq k+1
\end{array}
& E_1   & (k+7,8,10) & 2^{-(k+9)/2} & I_0^*\\
& E_2 & (6,9,14) & 2^{-(k+11)/2} & I_4 ^* \\
& E_3 & (k+7,8,10) & 2^{-(k+9)/2} & I_0^* \\
& E_6 & (6,9,14) & 2^{-(k+11)/2} & I_4^* \\
\hline
\SetCell[r=4]{c}  
\begin{array}{c}
v_2(t)=0\\[3pt]
(t-1)/2 \equiv 1 \, (2) \\[3pt]
m = v_2(5+u)
\end{array}
& E_1 & ( 0,0,2(1+m)) & 1 & I_{2(1+m)} \\
& E_2 & ( 0,0,1+m)  & 1 & I_{1+m} \\
& E_3 & ( 0,0,6(1+m))  & 1 &  I_{6(1+m)}\\
& E_6 & ( 0,0,3(1+m))  & 1 &  I_{3(1+m)} \\
\hline
\SetCell[r=4]{c}  
\begin{array}{c}
-m= v_2(t)<0
\end{array}
& E_1 & ( 4,6,6(2+m)) & 2^{-(6m+1)} & I_{4+6m} \\
& E_2 & (4,6,3(4+m))  & 2^{-(6m+1)} & I_{4+3m} \\
& E_3 & ( 4,6,2(6+m))  & 2^{-(6m+1)} &  I_{4+2m}\\
& E_6 & ( 4,6,12+m)  & 2^{-(6m+1)} &  I_{4+m} \\
\hline
\end{tblr}


\secion{OLD OLD}

\begin{tblr}{cells={mode=imath},hlines,vlines,measure=vbox}
\hline
\SetCell[c=5]{c} p=3 & & & & \\
\hline
\SetCell[c=1]{c} t & E & \operatorname{sig}(E) & u & p\operatorname{-Kod}(E) \\
\hline
\SetCell[r=4]{c} m+2=v_3(t)> 2 
& E_1 & (0,0,m) & 3^{-3} & I_{m} \\
& E_2 & (0,0,2m) & 3^{-3} & I_{2m} \\
& E_3 & (0,0,3m) & 3^{-4} & I_{3m} \\
& E_6 & (0,0,6m) & 3^{-4} & I_{6m} \\
\hline
\SetCell[r=4]{c} 
\begin{array}{c}
v_3(t)= 2 \\[3pt] 
m=v_3(t/9+1)
\end{array}
& E_1 & (0,0,2m) & 3^{-3} & I_{2m} \\
& E_2 & (0,0,m) & 3^{-3} & I_m \\
& E_3 & (0,0,6m) & 3^{-4} & I_{6m} \\
& E_6 & (0,0,3m) & 3^{-4} & I_{3m} \\
\hline
\SetCell[r=4]{c} 
\begin{array}{c}
v_3(t)=1\\[3pt]
t/3\equiv 2\, (3)
\end{array}
& E_1 & (4,6,9) & 3^{-2} & III^*  \\
& E_2 & (\geq 4,6,9) & 3^{-2} & III^*  \\
& E_3 & (2,3,3) & 3^{-3} & III \\
& E_6 & (\geq 3,3,3) & 3^{-3} & III  \\
\hline
\SetCell[r=4]{c} 
\begin{array}{c}
v_3(t)=1\\[3pt]
t/3\equiv 1,4\, (9)
\end{array}
& E_1 & (3,3,3) & 3^{-3} & III  \\
& E_2 & (2,3,3) & 3^{-3} & III  \\
& E_3 & (5,6,9) & 3^{-3} & III^* \\
& E_6 & (4,6,9) & 3^{-3} & III^*  \\
\hline
\SetCell[r=4]{c} 
\begin{array}{c}
v_3(t)=1\\[3pt]
v_3(t/3-7)>2
\end{array} 
& E_1 & (6,6,9) & 3^{-3} & III^*  \\
& E_2 & (\geq 4,6,9) & 3^{-3} & III^*  \\
& E_3 & (4,3,3) & 3^{-4} & III \\
& E_6 & (\geq 2,3,3) & 3^{-4} & III  \\
\hline
\SetCell[r=4]{c} 
\begin{array}{c}
v_3(t)=1\\[3pt]
v_3(t/3-7)=2\\[3pt]
(t/3-7)/9 \equiv 1 \, (3)
\end{array}
& E_1 & (6,6,9) & 3^{-3} & III^*  \\
& E_2 & (4,6,9) & 3^{-3} & III^*  \\
& E_3 & (4,3,3) & 3^{-4} & III \\
& E_6 & (\geq 2,3,3) & 3^{-4} & III  \\
\hline
\SetCell[r=4]{c} 
\begin{array}{c} 
v_3(t)=0\\[3pt]
t\equiv 1\, (3)
\end{array}
& E_1 & (0,0,\geq 3) & 1 & I_{3\nu}  \\
& E_2 & (0 ,0,\geq 6) & 1 & I_{6\nu}  \\
& E_3 & (0 ,0,\geq 1) & 1 & I_{\nu}  \\
& E_6 & (0 ,0,\geq 2) & 1 &   I_{2\nu} \\
\hline
\SetCell[r=4]{c} 
\begin{array}{c} 
v_3(t)=0\\[3pt]
t\equiv 2\, (3)
\end{array}
& E_1 & (0,0,0) & 1 & I_0  \\
& E_2 & (0,0,0) & 1 & I_0  \\
& E_3 & (0,0,0) & 1 & I_0  \\
& E_6 & (0,0,0) & 1 & I_0 \\
\hline
\SetCell[r=4]{c} -m=v_3(t)\leq -1 
& E_1 & (0,0,6m) & 3^{-6m} & I_{6m} \\
& E_2 & (0,0,3m) & 3^{-6m} & I_{3m} \\
& E_3 & (0,0,2m) & 3^{-6m} & I_{2m} \\
& E_6 & (0,0,m) & 3^{-6m} & I_{m} \\
\hline
\end{tblr}


\begin{tblr}{cells={mode=imath},hlines,vlines,measure=vbox}
\hline
\SetCell[r=4]{c} 
\begin{array}{c}
v_3(t)=1\\[3pt]
v_3(t/3-7)=2\\[3pt]
v_3(t/3+6)=m+1\geq 4\\[3pt]
m \text{ odd}\\
\end{array}
& E_1 & (m+2,3,3) & 3^{-(m+5)/2} & III^*  \\
& E_2 & (2,3,3) & 3^{-(m+5)/2} & III^*  \\
& E_3 & (m+4,6,9) & 3^{-(m+5)/2} & III^* \\
& E_6 & (4,6,9) & 3^{-(m+5)/2} & III^*  \\
\hline
\SetCell[r=4]{c} 
\begin{array}{c}
v_3(t)=1\\[3pt]
v_3(t/3-7)=2\\[3pt]
v_3(t/3+6)=m+1\geq 4\\[3pt]
m \text{ even} \\
\end{array}
& E_1 & (m+4,6,9) & 3^{-(m+4)/2} & III^*  \\
& E_2 & (4,6,9) & 3^{-(m+4)/2} & III^*  \\
& E_3 & (m+2,3,3) & 3^{-(m+6)/2} & III \\
& E_6 & (2,3,3) & 3^{-(m+6)/2} & III  \\
\hline
\end{tblr}
