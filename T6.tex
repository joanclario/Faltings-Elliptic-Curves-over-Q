\documentclass[11pt]{article}
\usepackage{amsfonts,amssymb,amsmath,amsthm,latexsym,graphics,epsfig,amsfonts}
\usepackage{verbatim,enumerate,array,booktabs,color,bigstrut,prettyref,tikz-cd}
\usepackage{multirow}
\usepackage[all]{xy}
\usepackage[backref]{hyperref}
\usepackage[OT2,T1]{fontenc}
%\usepackage{ctable}
\usepackage{mathtools}

\usepackage{longtable}

\usepackage{mathtools}
\newcommand{\Mod}[1]{\ (\mathrm{mod}\ #1)}
\newcommand{\mathdash}{\relbar\mkern-8mu\relbar}
\newcommand*\circled[2][1.6]{\tikz[baseline=(char.base)]{
    \node[shape=circle, draw, inner sep=1pt, 
        minimum height={\f@size*#1},] (char) {\vphantom{WAH1g}#2};}}
\makeatother



\usepackage{tabularray}
\UseTblrLibrary{amsmath,varwidth}

\usepackage{tabularx}
\usepackage{longtable}
\usepackage{arydshln}


\newcommand\myiso{\stackrel{\mathclap{\normalfont\mbox{\small $p$}}}{-}}
\newcommand\myisot{\stackrel{\mathclap{\normalfont\mbox{\small $3$}}}{-}}

\newcommand{\pref}[1]{\prettyref{#1}}
\newrefformat{eq}{\textup{(\ref{#1})}}
\newrefformat{prty}{\textup{(\ref{#1})}}

\definecolor{mylinkcolor}{rgb}{0.8,0,0}
\definecolor{myurlcolor}{rgb}{0,0,0.8}
\definecolor{mycitecolor}{rgb}{0,0,0.8}
\hypersetup{colorlinks=true,urlcolor=myurlcolor,citecolor=mycitecolor,linkcolor=mylinkcolor,linktoc=page,breaklinks=true}

%\DeclareSymbolFont{cyrletters}{OT2}{wncyr}{m}{n}
%\DeclareMathSymbol{\Sha}{\mathalpha}{cyrletters}{"58}

\addtolength{\textwidth}{4cm} \addtolength{\hoffset}{-2cm}
\addtolength{\marginparwidth}{-2cm}

%\theoremstyle{definition}
\newtheorem{defn}{Definition}[section]
\newtheorem{definition}[defn]{Definition}
\newtheorem{claim}[defn]{Claim}

%\theoremstyle{plain}
\newtheorem{thmA}{Theorem A}
\newtheorem{thmB}{Theorem B}
\newtheorem{thm2}{Theorem}
\newtheorem{prop2}{Proposition}
\newtheorem{note}{Note}

\newtheorem{corollary}[defn]{Corollary}
\newtheorem{lemma}[defn]{Lemma}
\newtheorem{property}[defn]{Property}
\newtheorem{thm}[defn]{Theorem}
\newtheorem{theorem}[defn]{Theorem}
\newtheorem{cor}[defn]{Corollary}
\newtheorem{prop}[defn]{Proposition}
\newtheorem{proposition}[defn]{Proposition}
\newtheorem{thmnn}{Theorem}
\newtheorem{conj}[defn]{Conjecture}

\theoremstyle{definition}
\newtheorem{remarks}{Remarks}
\newtheorem{ack}{Acknowledgements}
\newtheorem{remark}[defn]{Remark}
\newtheorem{question}[defn]{Question}
\newtheorem{example}[defn]{Example}


\newcommand{\Q}{\mathbb Q}
\newcommand{\Qbar}{\overline{\Q}}
\newcommand{\Z}{\mathbb Z}

\newcommand{\modQ}{\,\text{mod}\,(\Q^)^2}

\newcommand{\mysquare}[1]{\tikz{\path[draw] (0,0) rectangle node{\tiny #1} (8pt,8pt) ;}}
\newcommand{\mycircle}[1]{\tikz{\path[draw] (0,0) circle (4pt) node{\tiny #1};}}


%------------------------------------
\newcommand{\Kd}{\operatorname{K}}
\newcommand{\kI}{\operatorname{I}}
\newcommand{\kII}{\operatorname{II}}
\newcommand{\kIII}{\operatorname{III}}
\newcommand{\kIV}{\operatorname{IV}}
%-------------------------------------



\begin{document}
\title{Type $T_6$}
\date{\today}
\maketitle
\section{Setting}

The isogeny graphs of type $T_6$ are given by
six isogenous elliptic curves:

\[ \begin{tikzcd}
E_1 \ar[dash,dr,"2"] & & & E_8\\
& E_2 \ar[dash,r,"2"] & E_4  \ar[dash,ur,"2"]  \ar[dash,dr,"2"]& \\
E_{12}    \ar[dash,ur,"2"] &  &  & E_{22}    \,.
\end{tikzcd}
\]
 

\noindent A hauptmodul for $X_0(8)$ is  
$$
t = 4 + 2^5 \displaystyle{\frac{\eta(2\tau)^2 \eta(8t)^4}{\eta(\tau)^4 \eta(4\tau)^2}}\,.
$$
One has
$$
\begin{tblr}{l@{\,=\,}l}
j_1= j(E_1) = j(\tau) & \\
j_2 = j(E_2) = j(2\tau) & \\
j_4 = j(E_4) = j(4\tau) & \\
j_8 = j(E_8) = j(8\tau) & \\
j_{12} = j(E_{12}) = j(\tau+1/2) & \\
j_{22} = j(E_{22}) = j(2\tau+1/2) & 
\,.
\end{tblr}
$$
The subgroup of $\operatorname{Aut} X_0(8)$ that fixes the set of vertices of the graph is
isomorphic to the symmetric group $\mathcal{S}_3$ with elements:


$$
\begin{tblr}{l@{\,=\,}lcc}
    \SetCell[c=3]{r} \text{permutation} & & & \text{order}  \\
   \operatorname{id}(t) & t  &  (j_1,j_2,j_4,j_{12}) & 1 \\
   \sigma(t) & -256/(t+16) & (j_{12},j_2,j_1,j_4) & 3 \\
   \sigma^2(t) & -16(t+16)/t & (j_4,j_2,j_{12},j_1) & 3 \\ 
   \tau(t) & 256/t & (j_4,j_2,j_4,j_{12}) & 2 \\
   \sigma \tau(t) & -(t+16) &   (j_1,j_2,j_4,j_{12})               & 2 \\
   \sigma^2 \tau(t) & -16\, t/(t+16) &  (j_{12},j_2,j_4,j_1)                 & 2 \\
\end{tblr}
$$

%------------------------------------------------------
\begin{comment}

For $t$ in $\Q\setminus \{0,-16\}$, the $p$-adic valuations of the Fricke involutions applied to $t$ are:

\begin{tblr}
{cells={mode=imath},colspec=|c|c|c|c|}
\hline
p & v_p(W_2(t)) & v_p(W_3(t)) & v_p(W_6(t))  \\
\hline
\neq 2,3 & 0 & 0 & - v_p(t) \\
\hline
 3  & v_3(t+9)-v_3(t+8) & 2 + v_3(t+8)-v_3(t+9) & 2-v_3(t) \\
\hline
 2  & 3+v_2(t+9)+v_2(t+8) & v_2(t+8)-v_2(t+9) & 3-v_2(t) \\
\hline
\end{tblr}

\end{comment}
%------------------------------------------------------

\vskip 0.5truecm

We can (and do) choose Weierstrass equations for $(E_1,E_2,E_{12},E_4,E_8,E_{22})$ such that their signatures are:

\[
\begin{tblr}{|c|l|}
\hline \SetCell[c=2]{c} T_6 \text{ signatures}\\ \hline
c_4(E_1) & (t^4 - 16\,t^2 + 16)\\
 c_6(E_1) & (t^2 - 8)(t^4 - 16\,t^2 - 8)\\
 \Delta(E_1) & t^{2}(t - 4)(t + 4)\\ \hline
 c_4(E_2) & (t^4 - 16\,t^2 + 256)\\
 c_6(E_2) & (t^2 - 32)(t^2 - 8)(t^2 + 16)\\
 \Delta(E_2) & t^{4} (t - 4)^{2}(t + 4)^{2}\\ \hline
 c_4(E_{12}) & (t^4 - 256\,t^2 + 4096)\\
 c_6(E_{12}) & (t^2 - 32)(t^4 + 512\,t^2 - 8192)\\
 \Delta(E_{12}) & -t^{8}(t - 4)(t + 4)\\ \hline
 c_4(E_4) & (t^4 + 224\,t^2 + 256)\\
 c_6(E_4) & (t^2 - 24\,t + 16)(t^2 + 16)(t^2 + 24\,t + 16)\\
 \Delta(E_4) & t^{2} (t - 4)^{4}(t + 4)^{4}\\ \hline
 c_4(E_8) & (t^4 + 240\,t^3 + 2144\,t^2 + 3840\,t + 256)\\
 c_6(E_8) & (t^2 + 24\,t + 16)(t^4 - 528\,t^3 - 4000\,t^2 - 8448\,t + 256)\\
 \Delta(E_8) & t\, (t - 4)^{8}(t + 4)^{2}\\ \hline
 c_4(E_{22}) & (t^4 - 240\,t^3 + 2144\,t^2 - 3840\,t + 256)\\
 c_6(E_{22}) & (t^2 - 24\,t + 16)(t^4 + 528\,t^3 - 4000\,t^2 + 8448\,t + 256)\\
 \Delta(E_{22}) & -t\, (t - 4)^{2}(t + 4)^{8}\\ \hline
\end{tblr}
\]



\newpage


\begin{longtblr}
[caption = {$T_6$ data for $p\neq 2$}]
{cells = {mode=imath},hlines,vlines,measure=vbox,
hline{Z} = {1-5}{0pt},
vline{1} = {Y-Z}{0pt},
colspec  = cclclcc}
%--------------------------------------
\SetCell[c=1]{c} T_6 &\SetCell[c=6]{c} p\neq 2  & & & & & \\ t & E & 
\SetCell[c=1]{c} \operatorname{sig}_p(E) & u & \Kd_p(E) & \SetCell[c=2]{c} u_p(d)   \\
%--------------------------------------
\SetCell[r=6]{c}
    m =  v_p(t)>0  
& E_1    & ( 0 , 0 , 2m ) & 1  &  \kI_{2m}   & 1 & 1 \\
& E_2    & ( 0 , 0 , 4m ) & 1  &  \kI_{4m}   & 1 & 1 \\
& E_{12} & ( 0 , 0 , 8m ) & 1 &  \kI_{8m}   & 1 & 1 \\
& E_4    & ( 0 , 0 , 2m ) & 1  &  \kI_{2m}   & 1 & 1 \\
& E_8    & ( 0 , 0 ,  m ) & 1  &  \kI_{m}    & 1 & 1 \\
& E_{22} & ( 0 , 0 ,  m ) & 1  &  \kI_{m}    & 1 & 1 \\
%--------------------------------------
\SetCell[r=6]{c}
\begin{array}{c}
     v_p(t)=0  \\[3pt]
    m = v_p(t-4)>0 
\end{array}
& E_1    & ( 0 , 0 ,  m ) & 1  &  \kI_{m}   & 1 & 1 \\
& E_2    & ( 0 , 0 , 2m ) & 1  &  \kI_{2m}   & 1 & 1 \\
& E_{12} & ( 0 , 0 ,  m ) & 1  &  \kI_{m}   & 1 & 1 \\
& E_4    & ( 0 , 0 , 4m ) & 1  &  \kI_{4m}   & 1 & 1 \\
& E_8    & ( 0 , 0 , 8m ) & 1 &  \kI_{8m}   & 1 & 1 \\
& E_{22} & ( 0 , 0 , 2m ) & 1  &  \kI_{2m}   & 1 & 1 \\
%--------------------------------------
\SetCell[r=6]{c}
\begin{array}{c}
     v_p(t)=0  \\[3pt]
    m = v_p(t+4)>0 
\end{array}
& E_1    & ( 0 , 0 ,  m ) & 1  &   \kI_{m}   & 1 & 1 \\
& E_2    & ( 0 , 0 , 2m ) & 1  &   \kI_{2m}   & 1 & 1 \\
& E_{12} & ( 0 , 0 ,  m ) & 1  &   \kI_{m}   & 1 & 1 \\
& E_4    & ( 0 , 0 , 4m ) & 1  &   \kI_{4m}   & 1 & 1 \\
& E_8    & ( 0 , 0 , 2m ) & 1  &   \kI_{2m}   & 1 & 1 \\
& E_{22} & ( 0 , 0 , 8m ) & 1  &  \kI_{8m}   & 1 & 1 \\
%--------------------------------------
%--------------------------------------
\SetCell[r=6]{c}
    -m =  v_p(t)<0  
& E_1    & ( 0 , 0 , 8m ) & p^{-m}  &  \kI_{8m}   & 1 & 1 \\
& E_2    & ( 0 , 0 , 4m ) & p^{-m}  &   \kI_{4m}   & 1 & 1 \\
& E_{12} & ( 0 , 0 , 2m ) & p^{-m}  &   \kI_{2m}   & 1 & 1 \\
& E_4    & ( 0 , 0 , 2m ) & p^{-m}  &   \kI_{2m}   & 1 & 1 \\
& E_8    & ( 0 , 0 ,  m ) & p^{-m}  &   \kI_{m}   & 1 & 1 \\
& E_{22} & ( 0 , 0 ,  m ) & p^{-m}  &   \kI_{m}  & 1 & 1 \\
\SetCell[c=5,r=2]{c} & & & & & d\equiv 0  & d\not\equiv 0 \\
                      & & & & & \SetCell[c=2]{c} d \Mod p & \\
\end{longtblr}

\newpage

\begin{longtblr}
[caption = {$T_6$ data for $p$=2}]
{cells = {mode=imath},hlines,vlines,measure=vbox,
hline{Z} = {1-5}{0pt},
vline{1} = {Y-Z}{0pt},
colspec  = cclclccc}
%--------------------------------------
\SetCell[c=1]{c} T_6 &\SetCell[c=7]{c} p=2  & & & &  & & \\
\SetCell[c=1]{c} t & E & 
\SetCell[c=1]{c}\operatorname{sig}_2(E) & u & \Kd_2(E) & \SetCell[c=3]{c} u_2(d) & & \\
%--------------------------------------
\SetCell[r=6]{c}
    m =  v_2(t)>4  
& E_1    & ( 4 , 6 , 2m+4 ) & 1  &   \kI_{2m-4}^*   & 1 & 1& \begin{array}{c}
\text{$2$ if $m\ge 4$}\\
\text{$1$ if $m<4$}
\end{array} \\
& E_2    & ( 4 , 6 , 4m-4 ) & 2  &   \kI_{4m-12}^*   & 1 & 1 & \begin{array}{c}
\text{$2$ if $m\ge 4$}\\
\text{$1$ if $m<4$}
\end{array}  \\
& E_{12} & ( 4 , 6 , 8m-20 ) & 2^2  &   \kI_{8m-28}^*   & 1 &1 &  \begin{array}{c}
\text{$2$ if $m\ge 4$}\\
\text{$1$ if $m<4$}
\end{array} \\
& E_4    & ( 4 , 6 , 2m+4 ) & 2  &   \kI_{2m-4}^*   & 1 & 1& \begin{array}{c}
\text{$2$ if $m\ge 4$}\\
\text{$1$ if $m<4$}
\end{array}  \\
& E_8    & ( 4 , 6 , m+8 ) & 2  &   \kI_{m}^*   & 1 &  1& \begin{array}{c}
\text{$2$ if $m\ge 4$}\\
\text{$1$ if $m<4$}
\end{array}  \\
& E_{22} & ( 4 , 6 , m+8 ) & 2  &   \kI_{m}^*   & 1 & 1 & \begin{array}{c}
\text{$2$ if $m\ge 4$}\\
\text{$1$ if $m<4$}
\end{array}  \\
%------------------------------------------
\SetCell[r=6]{c}
    v_2(t)=4  
& E_1    & ( 4 , 6 , 12 ) & 1  &   \kI_4^*   & 1 & 1  &  2 \\
& E_2    & ( 4 , 6 , 12 ) & 2  &   \kI_4^*   & 1 &1   &  2 \\
& E_{12} & ( 4 , 6 , 12 ) & 2^2  &   \kI_4^*   & 1 &1   &  2 \\
& E_4    & ( 4 , 6 , 12 ) & 2  &   \kI_4^*   & 1 & 1  &  2 \\
& E_8    & ( 4 , 6 , 12 ) & 2  &   \kI_4^*   & 1 &  1 &  2 \\
& E_{22} & ( 4 , 6 , 12 ) & 2  &   \kI_4^*   & 1 &  1 &  2 \\
%------------------------------------------
\SetCell[r=6]{c}
    v_2(t)=3  
& E_1    & ( 4 , 6 , 10 ) & 1  &   \kI_2^*   & 1 & 1  &  1  \\
& E_2    & ( 4 , 6 , 8 ) & 2  &   \kI_0^*   & 1 &  1 &  1 \\
& E_{12} & ( 5 , 5 , 4 ) & 2^2  &   \kII   & 1 &  1  &  1 \\
& E_4    & ( 4 , 6 , 10 ) & 2  &   \kI_2^*   & 1 &  1  &    1\\
& E_8    & ( 4 , 6 , 11 ) & 2  &   \kI_3^*   & 1 & 1   &1    \\
& E_{22} & ( 4 , 6 , 11 ) & 2  &   \kI_3^*   & 1 &  1  & 1   \\
%----------------------------------------------
\SetCell[r=6]{c}
\begin{array}{c}
     v_2(t)=2  \\[3pt]
   t/2^2\equiv 1\, (16) \\[3pt]
   v_2(t-4)=m+5
\end{array}
& E_1    & ( 0 , 0 , m ) & 2  &   \kI_m   & 1&  2^{-1}  &  2^{-1}  \\
& E_2    & ( 0 , 0 , 2m ) & 2^2  &   \kI_{2m}   & 1 &  2^{-1}  & 2^{-1}   \\
& E_{12} & ( 0 , 0 , m ) & 2^2  &   \kI_m   & 1&  2^{-1}  &  2^{-1}  \\
& E_4    & ( 0 , 0 , 4m ) & 2^3  &   \kI_{4m}   & 1&  2^{-1}  &  2^{-1}  \\
& E_8    & ( 0 , 0 , 2m ) & 2^4  &   \kI_{2m}   & 1&  2^{-1}  &  2^{-1}  \\
& E_{22} & ( 0 , 0 , 8m ) & 2^3  &   \kI_{8m}   & 1&  2^{-1}  &  2^{-1}  \\
%--------------------------------------
\SetCell[r=6]{c}
\begin{array}{c}
     v_2(t)=2  \\[3pt]
   t/2^2\equiv 15\, (16) \\[3pt]
   v_2(t+4)=m+5
\end{array}
& E_1    & ( 0 , 0 , m ) & 2  &   \kI_m   & 1&  2^{-1}  &  2^{-1}  \\
& E_2    & ( 0 , 0 , 2m ) & 2^2  &   \kI_{2m}   & 1 &  2^{-1}  & 2^{-1}   \\
& E_{12} & ( 0 , 0 , m ) & 2^2  &   \kI_m   & 1&  2^{-1}  &  2^{-1}  \\
& E_4    & ( 0 , 0 , 4m ) & 2^3  &   \kI_{4m}   & 1&  2^{-1}  &  2^{-1}  \\
& E_8    & ( 0 , 0 , 2m ) & 2^3  &   \kI_{2m}   & 1&  2^{-1}  &  2^{-1}  \\
& E_{22} & ( 0 , 0 , 8m ) & 2^4  &   \kI_{8m}   & 1&  2^{-1}  &  2^{-1}  \\
%--------------------------------------
\SetCell[r=6]{c}
\begin{array}{c}
     v_2(t)=2  \\[3pt]
   t/2^2\equiv 7\, (16) 
\end{array}
& E_1    & ( 0 , 0 , 0 ) & 2  &   \kI_0   & 1&  2^{-1}  &  2^{-1}  \\
& E_2    & ( 0 , 0 , 0 ) & 2^2  &   \kI_0   & 1 &  2^{-1}  & 2^{-1}   \\
& E_{12} & ( 0 , 0 , 0 ) & 2^2  &   \kI_0   & 1&  2^{-1}  &  2^{-1}  \\
& E_4    & ( 0 , 0 , 0 ) & 2^3  &   \kI_0   & 1&  2^{-1}  &  2^{-1}  \\
& E_8    & ( 0 , 0 , 0 ) & 2^3  &   \kI_0   & 1&  2^{-1}  &  2^{-1}  \\
& E_{22} & ( 0 , 0 , 0 ) & 2^4  &   \kI_0   & 1&  2^{-1}  &  2^{-1}  \\
%--------------------------------------
\SetCell[r=6]{c}
\begin{array}{c}
     v_2(t)=2  \\[3pt]
   t/2^2\equiv 9\, (16) 
\end{array}
& E_1    & ( 0 , 0 , 0 ) & 2  &   \kI_0   & 1&  2^{-1}  &  2^{-1}  \\
& E_2    & ( 0 , 0 , 0 ) & 2^2  &   \kI_0   & 1 &  2^{-1}  & 2^{-1}   \\
& E_{12} & ( 0 , 0 , 0 ) & 2^2  &   \kI_0   & 1&  2^{-1}  &  2^{-1}  \\
& E_4    & ( 0 , 0 , 0 ) & 2^3  &   \kI_0   & 1&  2^{-1}  &  2^{-1}  \\
& E_8    & ( 0 , 0 , 0 ) & 2^4  &   \kI_0   & 1&  2^{-1}  &  2^{-1}  \\
& E_{22} & ( 0 , 0 , 0 ) & 2^3  &   \kI_0   & 1&  2^{-1}  &  2^{-1}  \\
%--------------------------------------
\SetCell[r=6]{c}
\begin{array}{c}
     v_2(t)=2  \\[3pt]
   t/2^2\equiv 3\, (8) 
\end{array}
& E_1    & ( 4 , 6 , 11 ) & 1  &   \kII^*   & 1&    1&  1  \\
& E_2    & ( 4 , 6 , 10 ) & 2  &   \kIII^*   & 1 &  1  &  1  \\
& E_{12} & ( 4 , 6 , 11 ) & 2  &   \kII^*   & 1&1    &  1  \\
& E_4    & ( 4 , 6 , 8 ) & 2^2  &   \kI_1^*   & 1& 1   &  1  \\
& E_8    & ( 4 , 6 , 10 ) & 2^2  &   \kIII^*   & 1& 1   &  1  \\
& E_{22} & ( 5 , 5 , 4 ) & 2^3  &   \kIII   & 1&  1  &    1\\
%--------------------------------------
\SetCell[r=6]{c}
\begin{array}{c}
     v_2(t)=2  \\[3pt]
   t/2^2\equiv 5\, (8) 
\end{array}
& E_1    & ( 4 , 6 , 11 ) & 1  &   \kII^*   & 1&    1&  1  \\
& E_2    & ( 4 , 6 , 10 ) & 2  &   \kIII^*   & 1 &  1  &  1  \\
& E_{12} & ( 4 , 6 , 11 ) & 2  &   \kII^*   & 1&1    &  1  \\
& E_4    & ( 4 , 6 , 8 ) & 2^2  &   \kI_1^*   & 1& 1   &  1  \\
& E_8    & ( 5 , 5 , 4 ) & 2^3  &   \kIII^*   & 1& 1   &  1  \\
& E_{22} & ( 4 , 6 , 10 )  & 2^2  &   \kIII   & 1&  1  &    1\\
%--------------------------------------
\SetCell[r=6]{c}
     v_2(t)=1 
& E_1    & ( 5 , 5 , 4 ) & 1  &   \kII   & 1&   1 &  1  \\
& E_2    & ( 4 , 6 , 8 ) & 1  &   \kI_0^*   & 1 &1    &   1 \\
& E_{12} & ( 4 , 6 , 10 ) & 1  &   \kI_2^*   & 1& 1   & 1   \\
& E_4    & ( 4 , 6 , 10 ) & 1  &   \kI_2^*   & 1&  1  &  1  \\
& E_8    & ( 4 , 6 , 11 ) & 1  &   \kI_3^*   & 1&  1  &  1  \\
& E_{22} & ( 4 , 6 , 11 ) & 1  &   \kI_3^*   & 1&  1  &  1  \\
%--------------------------------------
\SetCell[r=6]{c}
    v_2(t)=0  
& E_1    & ( 4 , 6 , 12 ) & 2^{-1}  &   \kI_4^*   & 1 &  1  &  2  \\
& E_2    & ( 4 , 6 , 12 ) & 2^{-1}  &   \kI_4^*   & 1 &  1  & 2   \\
& E_{12} & ( 4 , 6 , 12 ) & 2^{-1}  &   \kI_4^*   & 1 &  1  &  2  \\
& E_4    & ( 4 , 6 , 12 ) & 2^{-1}  &   \kI_4^*   & 1 &  1  &  2  \\
& E_8    & ( 4 , 6 , 12 ) & 2^{-1}  &   \kI_4^*   & 1 &  1  &  2  \\
& E_{22} & ( 4 , 6 , 12 ) & 2^{-1}  &   \kI_4^*   & 1 &  1  &  2  \\
%--------------------------------------
\SetCell[r=6]{c}
    -m=v_2(t)<0  
& E_1    & ( 4 , 6 , 8m+12 ) & 2^{-(m+1)}  &   \kI_{8m+4}^*   & 1 &    1&  2  \\
& E_2    & ( 4 , 6 , 4m+12 ) & 2^{-(m+1)}  &   \kI_{4m+4}^*   & 1 &    1&   2 \\
& E_{12} & ( 4 , 6 , 2m+12 ) & 2^{-(m+1)}  &  \kI_{2m+4}^*   & 1 &    1&  2  \\
1& E_4    & ( 4 , 6 , 2m+12 ) & 2^{-(m+1)}  &   \kI_{2m+4}^*   & 1 & 1   &  2  \\
& E_8    & ( 4 , 6 , m+12 ) & 2^{-(m+1)}  &   \kI_{m+4}^*   & 1 &  1  &  2  \\
& E_{22} & ( 4 , 6 , m+12 ) & 2^{-(m+1)}  &   \kI_{m+4}^*   & 1 &  1  &  2  \\
%--------------------------------------
 \SetCell[c=5,r=2]{c} & & & & &  d\equiv 1 &  d\equiv 2  & d\equiv 3 \\
                      & & & & & \SetCell[c=3]{c} d \Mod{4} & \\
\end{longtblr}


\newpage
\section{Conclusion}

\begin{prop}
Let 
\[ \begin{tikzcd}
E_1 \ar[dash,dr,"2"] & & & E_8\\
& E_2 \ar[dash,r,"2"] & E_4  \ar[dash,ur,"2"]  \ar[dash,dr,"2"]& \\
E_{12}    \ar[dash,ur,"2"] &  &  & E_{22}    
\end{tikzcd}
\]
be a $\mathbf{Q}$-isogeny graph of type $T_6$ corresponding to a given $t$ in $\mathbf{Q}$, $t\ne 0,\pm 4$. For every square-free integer $d$, 
the probability of a vertex
to be the Faltings curve (circled)
in the twisted isogeny graph 
\[ \begin{tikzcd}
E^d_1 \ar[dash,dr,"2"] & & & E^d_8\\
& E^d_2 \ar[dash,r,"2"] & E^d_4  \ar[dash,ur,"2"]  \ar[dash,dr,"2"]& \\
E^d_{12}    \ar[dash,ur,"2"] &  &  & E^d_{22}    
\end{tikzcd}
\]
is given by:

\newpage

\begin{longtblr}
[caption = La Pollaaa]
{cells = {mode=imath},hlines,vlines,measure=vbox,colspec=cccc}
%--------------------------------------
 T_6 & \text{twisted isogeny graph} & \text{Prob} \\
%--------------------------------------
v_2(t)\ge 3 &
\scalebox{.6}{
        \begin{tikzcd}[ampersand replacement=\&]
        E_1  \& \& \& E_8\\
\& E_2 \ar[r]\ar[ul] \& E_4  \ar[ur]  \ar[dr] \& \\
\circled[0.8]{$E_{12}$}    \ar[ur] \&  \&  \& E_{22}    
\end{tikzcd}
} & 1 \\
%--------------------------------------
\begin{array}{c}
v_2(t)=2\\
t/2^2\equiv 3\,(4)
\end{array}
&
\scalebox{.6}{
        \begin{tikzcd}[ampersand replacement=\&]
        E_1  \& \& \& \circled[0.8]{$E_8$} \ar[dl] \\
\& E_2  \ar[dl] \ar[ul] \& E_4 \ar[l]  \ar[dr] \& \\
E_{12}    \&  \&  \& E_{22}    
\end{tikzcd}
} & 1 \\
%--------------------------------------
\begin{array}{c}
v_2(t)=2\\
t/2^2\equiv 1\,(4)
\end{array} &
\scalebox{.6}{
        \begin{tikzcd}[ampersand replacement=\&]
        E_1  \& \& \& E_8\\
\& E_2 \ar[ul]\ar[dl] \& E_4 \ar[l] \ar[ur]  \& \\
E_{12}     \&  \&  \& \circled[0.8]{$E_{22}$}   \ar[ul]  
\end{tikzcd}
} & 1 \\
%--------------------------------------
v_2(t)\le 1 &
\scalebox{.6}{
        \begin{tikzcd}[ampersand replacement=\&]
        \circled[0.8]{$E_1$}  \ar[dr] \& \& \& E_8\\
\& E_2 \ar[r]\ar[dl] \& E_4  \ar[ur]  \ar[dr] \& \\
E_{12}    \&  \&  \& E_{22}    
\end{tikzcd}
} & 1 \\
%--------------------------------------
\end{longtblr}
\end{prop}



\vskip 0.35truecm

\noindent{\it Proof.} From the previous tables one gets:

\vskip 0.5truecm


\begin{tblr}{cells={mode=imath},hlines,vlines,measure=vbox}
%-------------------------------------------------
\SetCell[c=1]{c} t &\SetCell[c=1]{c} [u(E)]  & \SetCell[c=1]{c} [u(E)(d)]  & \SetCell[c=1]{c}\text{Prob}\\
%-------------------------------------------------
\SetCell[r=1]{c} v_2(t)\ge 3 & \SetCell[r=1]{c} (1:2:2^2:2:2:2) & (1:1:1:1:1:1) &   \SetCell[r=1]{c} (0,0,1,0,0,0)\\
%-------------------------------------------------
\SetCell[r=1]{c} \begin{array}{c}
v_2(t)=2\\
t/2^2\equiv 3\,(4)
\end{array}  & \SetCell[r=1]{c} (1:2:2:2^2:2^2:2^3) & (1:1:1:1:1:1) &   \SetCell[r=1]{c} (0,0,0,0,0,1)\\
%-------------------------------------------------
\SetCell[r=1]{c} \begin{array}{c}
v_2(t)=2\\
t/2^2\equiv 1\,(4)
\end{array}  & \SetCell[r=1]{c} (1:2:2:2^2:2^3:2^2) & (1:1:1:1:1:1) &   \SetCell[r=1]{c} (0,0,0,0,1,0)\\
%-------------------------------------------------
\SetCell[r=1]{c} v_2(t)\le 1 & \SetCell[r=1]{c} (1:1:1:1:1:1) & (1:1:1:1:1:1) &   \SetCell[r=1]{c} (1,0,0,0,0,0)\\
%-------------------------------------------------
\end{tblr}

\vskip 1.8truecm



\end{document}



