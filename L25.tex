



\documentclass[11pt]{article}
\usepackage{amsfonts,amssymb,amsmath,amsthm,latexsym,graphics,epsfig,amsfonts}
\usepackage{verbatim,enumerate,array,booktabs,color,bigstrut,prettyref,tikz-cd}
\usepackage{multirow}
\usepackage[all]{xy}
\usepackage[backref]{hyperref}
\usepackage[OT2,T1]{fontenc}
%\usepackage{ctable}
\usepackage{mathtools}

\usepackage{longtable}


\usepackage{mathtools}
\newcommand{\Mod}[1]{\ (\mathrm{mod}\ #1)}
\newcommand{\mathdash}{\relbar\mkern-8mu\relbar}
\newcommand*\circled[2][1.6]{\tikz[baseline=(char.base)]{
    \node[shape=circle, draw, inner sep=1pt, 
        minimum height={\f@size*#1},] (char) {\vphantom{WAH1g}#2};}}
\makeatother



\usepackage{tabularray}
\UseTblrLibrary{amsmath,varwidth}

\usepackage{tabularx}
\usepackage{longtable}
\usepackage{arydshln}


\newcommand\myiso{\stackrel{\mathclap{\normalfont\mbox{\small $p$}}}{-}}
\newcommand\myisot{\stackrel{\mathclap{\normalfont\mbox{\small $3$}}}{-}}

\newcommand{\pref}[1]{\prettyref{#1}}
\newrefformat{eq}{\textup{(\ref{#1})}}
\newrefformat{prty}{\textup{(\ref{#1})}}

\definecolor{mylinkcolor}{rgb}{0.8,0,0}
\definecolor{myurlcolor}{rgb}{0,0,0.8}
\definecolor{mycitecolor}{rgb}{0,0,0.8}
\hypersetup{colorlinks=true,urlcolor=myurlcolor,citecolor=mycitecolor,linkcolor=mylinkcolor,linktoc=page,breaklinks=true}

%\DeclareSymbolFont{cyrletters}{OT2}{wncyr}{m}{n}
%\DeclareMathSymbol{\Sha}{\mathalpha}{cyrletters}{"58}

\addtolength{\textwidth}{4cm} \addtolength{\hoffset}{-2cm}
\addtolength{\marginparwidth}{-2cm}

%\theoremstyle{definition}
\newtheorem{defn}{Definition}[section]
\newtheorem{definition}[defn]{Definition}
\newtheorem{claim}[defn]{Claim}

%\theoremstyle{plain}
\newtheorem{thmA}{Theorem A}
\newtheorem{thmB}{Theorem B}
\newtheorem{thm2}{Theorem}
\newtheorem{prop2}{Proposition}
\newtheorem{note}{Note}

\newtheorem{corollary}[defn]{Corollary}
\newtheorem{lemma}[defn]{Lemma}
\newtheorem{property}[defn]{Property}
\newtheorem{thm}[defn]{Theorem}
\newtheorem{theorem}[defn]{Theorem}
\newtheorem{cor}[defn]{Corollary}
\newtheorem{prop}[defn]{Proposition}
\newtheorem{proposition}[defn]{Proposition}
\newtheorem{thmnn}{Theorem}
\newtheorem{conj}[defn]{Conjecture}

\theoremstyle{definition}
\newtheorem{remarks}{Remarks}
\newtheorem{ack}{Acknowledgements}
\newtheorem{remark}[defn]{Remark}
\newtheorem{question}[defn]{Question}
\newtheorem{example}[defn]{Example}


\newcommand{\Q}{\mathbb Q}
\newcommand{\Qbar}{\overline{\Q}}
\newcommand{\Z}{\mathbb Z}

\newcommand{\modQ}{\,\text{mod}\,(\Q^)^2}

\newcommand{\mysquare}[1]{\tikz{\path[draw] (0,0) rectangle node{\tiny #1} (8pt,8pt) ;}}
\newcommand{\mycircle}[1]{\tikz{\path[draw] (0,0) circle (4pt) node{\tiny #1};}}


%------------------------------------
\newcommand{\Kd}{\operatorname{K}}
\newcommand{\kI}{\operatorname{I}}
\newcommand{\kII}{\operatorname{II}}
\newcommand{\kIII}{\operatorname{III}}
\newcommand{\kIV}{\operatorname{IV}}
%-------------------------------------



\begin{document}
\title{Type $L_2(5)$}
\date{\today}
\maketitle

\section{Setting}

The isogeny graphs of type $L_2(5)$ are given by
two isogenous elliptic curves:

\[ 
\begin{tikzcd}
E_1 \arrow[dash]{r}{5} & E_5   \,.
\end{tikzcd}
\]


\noindent A hauptmodule of $X_0(5)$ is  
$$t(\tau)= 5^{3} \left( \frac{\eta(5\tau)}{\eta(\tau)}\right)^{6}\,.$$ 
Letting $t=t(\tau)$, one can write
$$
\begin{tblr}{l@{\,=\,}l}
j(E_1) = j(\tau) & 
\displaystyle{\frac{\left(t^2+10 t+5\right)^3}{t}}\\[6pt]
j(E_5) = j(5\tau) & 
\displaystyle
{\frac{\left(t^2+250 t+3125\right)^3}{t^5}}\,,\\[6pt]
\end{tblr}
$$
and the Fricke involution of $X_0(5)$ is given by $W_5(t)=5^3/t $.

We can (and do) choose Weierstrass equations for $(E_1,E_5)$ with signatures:
$$
 \begin{tblr}{|c|l|}
\hline \SetCell[c=2]{c} L_2(5) \\ \hline
 c_4(E_1) & (t^2 + 10t + 5)(t^2 + 22t + 125)\\
c_6(E_1) & (t^2 + 4t - 1)(t^2 + 22t + 125)^{2}\\
 \Delta(E_1) & t(t^2 + 22t + 125)^{3}\\ \hline
 c_4(E_5) & (t^2 + 22t + 125)(t^2 + 250t + 3125)\\
 c_6(E_5) & (t^2 - 500t - 15625)(t^2 + 22t + 125)^{2}\\
 \Delta(E_5) & t^{5}(t^2 + 22t + 125)^{3}\\ \hline
\end{tblr}
$$
With regard to the action of the Fricke involution 
on the isogeny graph, 
it can be displayed as follows:
\[ 
\begin{tikzcd}
W_5\,(
E_1 \arrow[dash]{r}{5} & E_5
) = \,\, E_5^{-1} \arrow[dash]{r}{5} & E_1^{-1}   \,.
\end{tikzcd}
\]


\newpage

\section{Kodaira symbols \& Pal coefficients}


\begin{longtblr}
[caption= {$L_2(5)$ data for $p\ne 2,3,5$}]
{cells={mode=imath},hlines,vlines,measure=vbox,
hline{Z}={1-X}{0pt},
vline{1}={Y-Z}{0pt},
colspec=cclclcc}
\SetCell[c=1]{c} L_2(5) &\SetCell[c=6]{c} p\ne 2,3,5 &    & \\
\SetCell[c=1]{c} t & E & 
\SetCell[c=1]{c} \operatorname{sig}_p(E) & u & \Kd_p(E) & \SetCell[c=2]{c} u_p(d)\\
%----------------------------------------------
\SetCell[r=2]{c} 
     m= v_p(t)>0   
& E_1 & (0,0,m) & 1 & \kI_{m} & 1& 1\\
& E_5 & (0,0,5m) & 1 &  \kI_{5m} & 1& 1\\
%----------------------------------------------
\SetCell[r=2]{c} 
\begin{array}{c}
     v_{p}(t)=0  \\[3pt]
    v_{p}(t^2 + 22\,t + 125)=4m>0 
\end{array}
& E_1 & (0,2m,0) & p^m &  \kI_0  & 1 & 1\\
& E_5 & (0,2m,0) & p^m &  \kI_0  & 1 & 1\\
%----------------------------------------------
\SetCell[r=2]{c} 
\begin{array}{c}
     v_{p}(t)=0  \\[3pt]
    v_{p}(t^2 + 22\,t + 125)=4m+1>0 
\end{array}
& E_1 & (1,2m+2,3) & p^m  &  \kIII  & 1 & 1\\
& E_5 & (1,2m+2,3) & p^m  &  \kIII  & 1 & 1\\
%----------------------------------------------
\SetCell[r=2]{c} 
\begin{array}{c}
     v_{p}(t)=0  \\[3pt]
    v_{p}(t^2 + 22\,t + 125)=4m+2>0 
\end{array}
& E_1 & (2,2m+4,6) & p^m  &  \kI_0^*  & p & 1\\
& E_5 & (2,2m+4,6) & p^m  &  \kI_0^*  & p & 1\\
%----------------------------------------------
\SetCell[r=2]{c} 
\begin{array}{c}
     v_{p}(t)=0  \\[3pt]
    v_{p}(t^2 + 22\,t + 125)=4m+3>0 
\end{array}
& E_1 & (3,2m+6,9) & p^m  &  \kIII^*  & p & 1\\
& E_5 & (3,2m+6,9) & p^m  &  \kIII^*  & p & 1\\
%----------------------------------------------
\SetCell[r=2]{c} 
     -m= v_p(t)<0   
& E_1 & (0,0,5m) & p^{-m} & \kI_{5m} & 1& 1\\
& E_5 & (0,0,m) & p^{-m} &  \kI_{m} & 1& 1\\
%----------------------------------------------
 \SetCell[c=5,r=2]{c} & & & & & d\equiv 0  & d\not\equiv 0 \\
                      & & & & & \SetCell[c=2]{c} d \Mod p & \\
\end{longtblr}

\newpage

\begin{longtblr}
[caption= {$L_2(5)$ data for $p=3$}]
{cells={mode=imath},hlines,vlines,measure=vbox,
hline{Z}={1-X}{0pt},
vline{1}={Y-Z}{0pt},
colspec=cclclcc}
\SetCell[c=1]{c} L_2(5) &\SetCell[c=6]{c} p=3 &    & \\
\SetCell[c=1]{c} t & E & 
\SetCell[c=1]{c} \operatorname{sig}_3(E) & u & \Kd_3(E) & \SetCell[c=2]{c} u_3(d)\\
%----------------------------------------------
\SetCell[r=2]{c} 
     m= v_3(t)\ge 0   
& E_1 & (0,0,m) & 1 & \kI_{m} & 1& 1\\
& E_5 & (0,0,5m) & 1 &  \kI_{5m} & 1& 1\\
%----------------------------------------------
\SetCell[r=2]{c} 
     -m= v_3(t)<0   
& E_1 & (0,0,5m) & 3^{-m} & \kI_{5m} & 1& 1\\
& E_5 & (0,0,m) & 3^{-m} &  \kI_{m} & 1& 1\\
%----------------------------------------------
 \SetCell[c=5,r=2]{c} & & & & & d\equiv 0  & d\not\equiv 0 \\
                      & & & & & \SetCell[c=2]{c} d \Mod 3 & \\
\end{longtblr}

\newpage

\begin{longtblr}
[caption= {$L_2(5)$ data for $p=5$}]
{cells={mode=imath},hlines,vlines,measure=vbox,
hline{Z}={1-X}{0pt},
vline{1}={Y-Z}{0pt},
colspec=cclclcc}
\SetCell[c=1]{c} L_2(5) &\SetCell[c=6]{c} p=5 &    & \\
\SetCell[c=1]{c} t & E & 
\SetCell[c=1]{c} \operatorname{sig}_5(E) & u & \Kd_5(E) & \SetCell[c=2]{c} u_5(d)\\
%----------------------------------------------
\SetCell[r=2]{c} 
     m= v_5(t)> 3 
& E_1 & (0,0,m-3) & 5 & \kI_{m-3} & 1& 1\\
& E_5 & (0,0,5(m-3)) & 5^2 & \kI_{5(m-3)} & 1& 1\\
%----------------------------------------------
\SetCell[r=2]{c} 
\begin{array}{c}
     v_5(t)=3  \\[3pt]
    v_5(t^2 + 22t + 125)=m+3>0\\[3pt]
    m=4k
\end{array}
& E_1 & (0,\ge 0, 0) & 5^{k+1} &  \kI_0  & 1 & 1\\
& E_5 & (0,\ge 0, 0) & 5^{k+2} &  \kI_0  & 1 & 1\\
%----------------------------------------------
\SetCell[r=2]{c} 
\begin{array}{c}
     v_5(t)=3  \\[3pt]    
   v_5(t^2 + 22t + 125)=m+3>0\\[3pt]
    m=4k+1\end{array}
& E_1 & (1,\ge 2,3) & 5^{k+1} &  \kIII  & 1 & 1\\
& E_5 & (1,\ge 2,3) & 5^{k+2} &  \kIII  & 1 & 1\\
%----------------------------------------------
\SetCell[r=2]{c} 
\begin{array}{c}
     v_5(t)=3  \\[3pt]
 v_5(t^2 + 22t + 125)=m+3>0\\[3pt]
    m=4k+2\end{array}
& E_1 & (2,\ge 4,6) & 5^{k+1} &  \kI_0^*  &5  & 1\\
& E_5 & (2,\ge 4,6) & 5^{k+2} &  \kI_0^*  &5  & 1\\
%----------------------------------------------
\SetCell[r=2]{c} 
\begin{array}{c}
     v_5(t)=3  \\[3pt]
 v_5(t^2 + 22t + 125)=m+3>0\\[3pt]
    m=4k+3\end{array}
& E_1 & (3,\ge 6,9) & 5^{k+1} &  \kIII^*  & 5 & 1\\
& E_5 & (3,\ge 6,9) & 5^{k+2} &  \kIII^*  & 5 & 1\\
%----------------------------------------------
\SetCell[r=2]{c} 
     v_5(t)=2 
& E_1 & (3,4,8) & 1 & \kIV^* & 5 & 1\\
& E_5 & (2,2,4) & 5 & \kIV & 1 & 1\\
%----------------------------------------------
\SetCell[r=2]{c} 
     v_5(t)=1
& E_1 & (2,2,4) & 1 & \kIV & 1 & 1\\
& E_5 & (3,4,8) & 1 & \kIV^* & 5 & 1\\
%----------------------------------------------
\SetCell[r=2]{c} 
\begin{array}{c}
     v_5(t)=0  \\[3pt]
          t\not\equiv 3\,(5)\\[3pt]
\end{array}
& E_1 & (0,0,0) & 1 &  \kI_0  & 1 & 1\\
& E_1 & (0,0,0) & 1 &  \kI_0  & 1 & 1\\
%----------------------------------------------
\SetCell[r=2]{c} 
\begin{array}{c}
     v_5(t)=0  \\[3pt]
 %         t\equiv 3\,(5)\\[3pt]
    v_5(t^2 + 22t + 125)=4m
\end{array}
& E_1 & (0,\ge 0, 0) & 5^m &  \kI_0  & 1 & 1\\
& E_5 & (0,\ge 0, 0) & 5^m &  \kI_0  & 1 & 1\\
%----------------------------------------------
\SetCell[r=2]{c} 
\begin{array}{c}
     v_5(t)=0  \\[3pt]    
 %     t\equiv 3\,(5)\\[3pt]
    v_5(t^2 + 22t + 125)=4m+1
\end{array}
& E_1 & (1,\ge 2,3) & 5^m &  \kIII  & 1 & 1\\
& E_5 & (1,\ge 2,3) & 5^m &  \kIII  & 1 & 1\\
%----------------------------------------------
\SetCell[r=2]{c} 
\begin{array}{c}
     v_5(t)=0  \\[3pt]
%          t\equiv 3\,(5)\\[3pt]
    v_5(t^2 + 22t + 125)=4m+2
\end{array}
& E_1 & (2,\ge 4,6) & 5^m &  \kI_0^*  &5  & 1\\
& E_5 & (2,\ge 4,6) & 5^m &  \kI_0^*  &5  & 1\\
%----------------------------------------------
\SetCell[r=2]{c} 
\begin{array}{c}
     v_5(t)=0  \\[3pt]
%     t\equiv 3\,(5)\\[3pt]
    v_5(t^2 + 22t + 125)=4m+3
\end{array}
& E_1 & (3,\ge 6,9) & 5^m &  \kIII^*  & 5 & 1\\
& E_5 & (3,\ge 6,9) & 5^m &  \kIII^*  & 5 & 1\\
%----------------------------------------------
\SetCell[r=2]{c} 
     -m= v_5(t)<0   
& E_1 & (0,0,5m) & 5^{-m} & \kI_{5m} & 1& 1\\
& E_5 & (0,0,m) & 5^{-m} & \kI_{m} & 1& 1\\
%----------------------------------------------
 \SetCell[c=5,r=2]{c} & & & & & d\equiv 0  & d\not\equiv 0 \\
                      & & & & & \SetCell[c=2]{c} d \Mod 5 & \\
\end{longtblr}

\newpage




\newpage

\begin{longtblr}
[caption = {$L_2(5)$ data for $p$=2}]
{cells = {mode=imath},hlines,vlines,measure=vbox,
hline{Z} = {1-5}{0pt},
vline{1} = {Y-Z}{0pt},
colspec  = cclclccc}
%----------------------------------------------
L_2(5) & \SetCell[c=7]{c} p=2  & & & & & \\ 
t & E & \SetCell[c=1]{c} \operatorname{sig}_2(E) & u & \SetCell[c=1]{c} \Kd_2(E) & \SetCell[c=3]{c} u_2(d)  \\
%----------------------------------------------
\SetCell[r=2]{c} m=v_2(t)\ge 1 
& E_1 & (0,0,m) & 1& \kI_{m} & 1 & 2^{-1} &  2^{-1}\\
& E_1 & (0,0,5m) & 1& \kI_{5m} & 1 & 2^{-1} & 2^{-1} \\
%----------------------------------------------
\SetCell[r=2]{c} 
\begin{array}{c}
v_2(t)=0 \\[3pt]
t\equiv 1\, (4)
\end{array}
& E_1 & (6,6,6) & 1 & \kI & 1 & \text{$1^*$ or $2^*$}  & 1 \\
& E_5 & (6,6,6) & 1 & \kI & 1 & \text{$1^*$ or $2^*$} &  1 \\
 %----------------------------------------------
\SetCell[r=2]{c} 
\begin{array}{c}
v_2(t)=0 \\[3pt]
t\equiv 3\, (4)
\end{array}
& E_1 & (5,8,9) & 1& \kIII & 1 & 1& 1 \\
& E_5 & (5,8,9) & 1& \kIII & 1 & 1 & 1 \\
%----------------------------------------------
\SetCell[r=2]{c} -m=v_2(t)<0 
& E_1 & (4,6,5m+12) & 2^{-(m+1)} & I_{5m+4}^* & 1 &  1& 2\\
& E_1 & (4,6,m+12) & 2^{-(m+1)} & I_{m+4}^* & 1 & 1 & 2\\
%----------------------------------------------
 \SetCell[c=5,r=2]{c} & & & & &  d\equiv 1 &  d\equiv 2  & d\equiv 3 \\
                      & & & & & \SetCell[c=3]{c} d \Mod{4} & \\
\end{longtblr}





\newpage
\section{Conclusion}

\begin{prop}
Let 
$ 
\begin{tikzcd}
E_1 \arrow[dash]{r}{5}  & E_5 
\end{tikzcd}
$
be a $\mathbf{Q}$-isogeny graph of type $L_2(5)$ corresponding to a given $t$ in $\mathbf{Q}^*$. For every square-free integer $d$, 
the probability of a vertex
to be the Faltings curve (circled)
in the twisted isogeny graph 
$
\begin{tikzcd} 
E_1^d \arrow[dash]{r}{5}  & E_5^d 
\end{tikzcd}
$ 
is given by:

\[
\begin{tblr}{|c|c|c|c|c|}
\hline
 L_2(5) & \text{twisted isogeny graph} & d & \text{Prob} \\
\hline
 %----------------------------------------------
 \SetCell[r=1]{c} v_5(t)\ge 3 &  E_1^d \longleftarrow \circled[0.8]{$E_5^d$} & & 1 \\
\hline
 %----------------------------------------------
 \SetCell[r=2]{c} v_5(t)=2   
&E_1^d\longleftarrow  \circled[0.8]{$E_5^d$}  & d\not\equiv 0\,(5) & 5/6\\ 
&    \circled[0.8]{$E_1^d$}  \longrightarrow E_5^d &d\equiv 0\,(7) &  1/6 \\
\hline
 %----------------------------------------------

\SetCell[r=2]{c} v_5(t)=1   
& \circled[0.8]{$E_1^d$} \longrightarrow E_5^d  & d\not\equiv 0\,(5) & 5/6\\ 
&   E_1^d \longleftarrow \circled[0.8]{$E_5^d$} &d\equiv 0\,(7) &  1/6 \\
\hline
 %----------------------------------------------
 \SetCell[r=1]{c} v_5(t)\le 0 & \circled[0.8]{$E_1^d$} \longrightarrow E_5^d & & 1 \\
 %----------------------------------------------
\hline
\end{tblr}
\]



\end{prop}

\vskip 0.35truecm

\noindnet{\it Proof.} From the previous tables one gets:

\vskip 0.5truecm

\begin{tblr}{cells={mode=imath},hlines,vlines,measure=vbox}
%-------------------------------------------------
\SetCell[c=1]{c} t &\SetCell[c=1]{c} [u(E)]  & \SetCell[c=1]{c} [u(E)(d)] & \SetCell[c=1]{c} d & \SetCell[c=1]{c}\text{Prob}\\
%-------------------------------------------------
\SetCell[r=1]{c} v_5(t)\ge 3 & \SetCell[r=1]{c} (1:5) & \SetCell[r=1]{c} (1:1) &  & \SetCell[r=1]{c} (0,1)\\
%-------------------------------------------------
\SetCell[r=2]{c}  v_5(t)=2 & \SetCell[r=2]{c} (1:5) & \SetCell[r=1]{c} (1:1) & d\not\equiv 0\,(5)&\SetCell[r=2]{c} \left(\frac{5}{6},\frac{1}{6}\right) \\
& & \SetCell[r=1]{c} (5:1) & d\equiv 0\,(5)& \\
%-------------------------------------------------
\SetCell[r=2]{c}  v_5(t)=1 & \SetCell[r=2]{c} (1:1) & \SetCell[r=1]{c} (1:1) & d\not\equiv 0\,(5)&\SetCell[r=2]{c} \left(\frac{1}{6},\frac{5}{6}\right) \\
& & \SetCell[r=1]{c} (1:5) & d\equiv 0\,(5)& \\
%-------------------------------------------------
\SetCell[r=1]{c} v_5(t)\le 0 & \SetCell[r=1]{c} (1:1) & \SetCell[r=1]{c}  (1:1) &  & \SetCell[r=1]{c} (1,0)\\
%-------------------------------------------------
\end{tblr}

\end{document}

