



\documentclass[11pt]{article}
\usepackage{amsfonts,amssymb,amsmath,amsthm,latexsym,graphics,epsfig,amsfonts}
\usepackage{verbatim,enumerate,array,booktabs,color,bigstrut,prettyref,tikz-cd}
\usepackage{multirow}
\usepackage[all]{xy}
\usepackage[backref]{hyperref}
\usepackage[OT2,T1]{fontenc}
%\usepackage{ctable}
\usepackage{mathtools}

\usepackage{longtable}


\usepackage{mathtools}
\newcommand{\Mod}[1]{\ (\mathrm{mod}\ #1)}
\newcommand{\mathdash}{\relbar\mkern-8mu\relbar}
\newcommand*\circled[2][1.6]{\tikz[baseline=(char.base)]{
    \node[shape=circle, draw, inner sep=1pt, 
        minimum height={\f@size*#1},] (char) {\vphantom{WAH1g}#2};}}
\makeatother



\usepackage{tabularray}
\UseTblrLibrary{amsmath,varwidth}

\usepackage{tabularx}
\usepackage{longtable}
\usepackage{arydshln}


\newcommand\myiso{\stackrel{\mathclap{\normalfont\mbox{\small $p$}}}{-}}
\newcommand\myisot{\stackrel{\mathclap{\normalfont\mbox{\small $3$}}}{-}}

\newcommand{\pref}[1]{\prettyref{#1}}
\newrefformat{eq}{\textup{(\ref{#1})}}
\newrefformat{prty}{\textup{(\ref{#1})}}

\definecolor{mylinkcolor}{rgb}{0.8,0,0}
\definecolor{myurlcolor}{rgb}{0,0,0.8}
\definecolor{mycitecolor}{rgb}{0,0,0.8}
\hypersetup{colorlinks=true,urlcolor=myurlcolor,citecolor=mycitecolor,linkcolor=mylinkcolor,linktoc=page,breaklinks=true}

%\DeclareSymbolFont{cyrletters}{OT2}{wncyr}{m}{n}
%\DeclareMathSymbol{\Sha}{\mathalpha}{cyrletters}{"58}

\addtolength{\textwidth}{4cm} \addtolength{\hoffset}{-2cm}
\addtolength{\marginparwidth}{-2cm}

%\theoremstyle{definition}
\newtheorem{defn}{Definition}[section]
\newtheorem{definition}[defn]{Definition}
\newtheorem{claim}[defn]{Claim}

%\theoremstyle{plain}
\newtheorem{thmA}{Theorem A}
\newtheorem{thmB}{Theorem B}
\newtheorem{thm2}{Theorem}
\newtheorem{prop2}{Proposition}
\newtheorem{note}{Note}

\newtheorem{corollary}[defn]{Corollary}
\newtheorem{lemma}[defn]{Lemma}
\newtheorem{property}[defn]{Property}
\newtheorem{thm}[defn]{Theorem}
\newtheorem{theorem}[defn]{Theorem}
\newtheorem{cor}[defn]{Corollary}
\newtheorem{prop}[defn]{Proposition}
\newtheorem{proposition}[defn]{Proposition}
\newtheorem{thmnn}{Theorem}
\newtheorem{conj}[defn]{Conjecture}

\theoremstyle{definition}
\newtheorem{remarks}{Remarks}
\newtheorem{ack}{Acknowledgements}
\newtheorem{remark}[defn]{Remark}
\newtheorem{question}[defn]{Question}
\newtheorem{example}[defn]{Example}


\newcommand{\Q}{\mathbb Q}
\newcommand{\Qbar}{\overline{\Q}}
\newcommand{\Z}{\mathbb Z}

\newcommand{\modQ}{\,\text{mod}\,(\Q^)^2}

\newcommand{\mysquare}[1]{\tikz{\path[draw] (0,0) rectangle node{\tiny #1} (8pt,8pt) ;}}
\newcommand{\mycircle}[1]{\tikz{\path[draw] (0,0) circle (4pt) node{\tiny #1};}}


%------------------------------------
\newcommand{\Kd}{\operatorname{K}}
\newcommand{\kI}{\operatorname{I}}
\newcommand{\kII}{\operatorname{II}}
\newcommand{\kIII}{\operatorname{III}}
\newcommand{\kIV}{\operatorname{IV}}
%-------------------------------------



\begin{document}
\title{Type $L_2(2)$}
\date{\today}
\maketitle
\section{Setting}
The isogeny graphs of type $L_2(2)$ are given by
two isogenous elliptic curves:

\[ 
\begin{tikzcd}
E_1 \arrow[dash]{r}{2} & E_2   \,.
\end{tikzcd}
\]

\noindent A hauptmodule of $X_0(2)$ is  
$$t(\tau)= 2^{12} \left( \frac{\eta(2\tau)}{\eta(\tau)}\right)^{24}\,.$$ 
Letting $t=t(\tau)$, one can write
$$
\begin{tblr}{l@{\,=\,}l}
j(E_1) = j(\tau) & 
\displaystyle{\frac{(t + 16)^3}{t}}\\[6pt]
j(E_2) = j(2\tau) & 
\displaystyle
{\frac{(t + 256)^3}{t^2}}\,,\\[6pt]
\end{tblr}
$$
and the Fricke involution of $X_0(2)$ is given by $W_2(t)= 2^{12}/t$.
We can (and do) choose Weierstrass equations for $(E_1,E_2)$ with signatures:

\[
\begin{tblr}{|c|l|}
\hline \SetCell[c=2]{c} L_2(2) \\ \hline
 c_4(E_1) & (t + 16)(t + 64)\\
 c_6(E_1) & (t - 8)(t + 64)^{2}\\
 \Delta(E_1) & t(t + 64)^{3}\\ \hline
 c_4(E_2) & (t + 64)(t + 256)\\
 c_6(E_2) & (t - 512)(t + 64)^{2}\\
 \Delta(E_2) & t^{2}(t + 64)^{3}\\ \hline
\end{tblr}
\]
so that the corresponding isogenies are normalized.

With regard to the action of the Fricke involution 
on the isogeny graph, 
it can be displayed as follows:
\[ 
\begin{tikzcd}
W_2\,\,:\,\, E_2^{-2t} \arrow[dash]{r}{2} & E_1^{-2t}   \,.
\end{tikzcd}
\]



\newpage

\section{Kodaira symbols \& Pal coefficients}

\begin{longtblr}
[caption= {$L_2(2)$ data for $p\ne 2$}]
{cells={mode=imath},hlines,vlines,measure=vbox,
hline{Z}={1-X}{0pt},
vline{1}={Y-Z}{0pt},
colspec=cclclcc}
\SetCell[c=1]{c} L_2(2) &\SetCell[c=6]{c} p\ne 2  &    & \\
\SetCell[c=1]{c} t & E & 
\SetCell[c=1]{c} \operatorname{sig}_p(E) & u & \Kd_p(E) & \SetCell[c=2]{c} u_p(d)\\
\SetCell[r=2]{c} 
m= v_p(t)>0 
& E_1 & (0,0,m) & 1 &  \kI_m & 1& 1\\
& E_2 & (0,0,2m) & 1 &  \kI_{2m} & 1& 1 \\
\SetCell[r=2]{c} 
\begin{array}{c}
     v_p(t)=0  \\[3pt]
    0 < v_p(t+64)=m\equiv 0 \, (4) 
\end{array}
& E_1 & (0,m/2,0) & p^{m/4} &  \kI_{0}  & 1 & 1\\
& E_2 & (0,m/2,0) & p^{m/4} &  \kI_{0} & 1& 1\\
\SetCell[r=2]{c} 
\begin{array}{c}
     v_p(t)=0  \\[3pt]
       0 < v_p(t+64)=m\equiv 1 \, (4)  
\end{array}
& E_1 & (1,(m+3)/2,3) & p^{(m-1)/4} &  \kIII  & 1 & 1\\
& E_2 & (1,(m+3)/2,3) & p^{(m-1)/4} &  \kIII & 1 & 1\\
\SetCell[r=2]{c} 
\begin{array}{c}
     v_p(t)=0  \\[3pt]
     0 < v_p(t+64)=m\equiv 2 \, (4) 
\end{array}
& E_1 & (2,(m+7)/2,6) & p^{(m-2)/4} &  \kI_{0}^*  & p & 1\\
& E_2 & (2,(m+7)/2,6) & p^{(m-2)/4} &  \kI_{0}^* & p& 1\\
\SetCell[r=2]{c} 
\begin{array}{c}
     v_p(t)=0  \\[3pt]
     0 < v_p(t+64)=m\equiv 3 \, (4) 
\end{array}
& E_1 & (3,(m+11)/2,9) & p^{(m-3)/4} &  \kIII^*  & p & 1\\
& E_2 & (3,(m+11)/2,9) & p^{(m-3)/4} &  \kIII^* & p& 1\\
\SetCell[r=2]{c} 
\begin{array}{c}
     -m= v_p(t)<0  \\[3pt]
    m \text{ odd} 
\end{array}
& E_1 & (2,3,2(m+6)) & p^{-(m-1)} & \kI_{2m}^*  & p& 1\\
& E_2 & (2,3,m+6) & p^{-(m-1)} &  \kI_{m}^* & p& 1\\
\SetCell[r=2]{c} 
\begin{array}{c}
     -m= v_p(t)<0  \\[3pt]
    m \text{ even} 
\end{array}
& E_1 & (0,0,2m) & p^{-m/2} & \kI_{2m} & 1& 1\\
& E_2 & (0,0,m) & p^{-m/2} &  \kI_{m} & 1& 1\\
 \SetCell[c=5,r=2]{c} & & & & & d\equiv 0  & d\not\equiv 0 \\
                      & & & & & \SetCell[c=2]{c} d \Mod p & \\

\end{longtblr}

\newpage

\begin{longtblr}
[caption = {$L_2(2)$ data for $p$=2}]
{cells = {mode=imath},hlines,vlines,measure=vbox,
hline{Z} = {1-5}{0pt},
vline{1} = {Y-Z}{0pt},
colspec  = cclclccc}
%----------------------------------------------
L_2(2) & \SetCell[c=7]{c} p=2  & & & & & \\ 
t & E & \SetCell[c=1]{c} \operatorname{sig}_2(E) & u & \SetCell[c=1]{c} \Kd_2(E) & \SetCell[c=3]{c} u_2(d)  \\
%----------------------------------------------
\SetCell[r=2]{c} m=v_2(t)  \geq 8  
& E_1 & (6,9,m+6) & 2 & \kI_{m-4}^* & 1 & 
\begin{array}{cl} 
4^* & \text{if $m\ge 12$}\\
2^*& \text{if $m< 12$} \end{array} & 1 \\
& E_2 & (\geq 7,\geq 8, 2m-6) & 2^2 & \kI_{2m-16}^* &1  & 
\begin{array}{cl} 
4^* & \text{if $m\ge 12$}\\
2^* & \text{if $m< 12$} \end{array}
& 1\\
%----------------------------------------------
\SetCell[r=2]{c} v_2(t)=7 
& E_1 & (6,9,13) & 2 & \kII^* & 1 & 2 & 1 \\
& E_2 & (5,7,8) & 2^2 & \kIII & 1 & 1 & 1 \\
%----------------------------------------------
\SetCell[r=2]{c} 
\begin{array}{c}
v_2(t)=6 \\[3pt]
t/2^6 \equiv 3 \, (16)
\end{array}
& E_1 & (4, 7, 6)  &  2^2 & \kII &  1& 1 & 1\\
& E_2 & (6,10,12) & 2^2 & \kI_2^* & 1 & 2 & 1\\ 
%----------------------------------------------
\SetCell[r=2]{c} 
\begin{array}{c}
v_2(t)=6 \\[3pt]
t/2^6 \equiv 11 \, (16) 
\end{array}
& E_1 & (4,7,6) & 2^2 & \kIII & 1 & 1 & 1\\
& E_2 & (6,10,12) & 2^2 & \kI_3^* & 1 & 2 & 1\\ 
%----------------------------------------------
\SetCell[r=2]{c} 
\begin{array}{c}
v_2(t)=6 \\[3pt]
t/2^6 \equiv 7 \, (16) 
\end{array}
& E_1 & (5,9,9) & 2^2 & \kIII & 1 & 1 & 1\\
& E_2 & (7,12,15) & 2^2 & \kIII^* & 1 & 2 & 1\\ 
%----------------------------------------------
\SetCell[r=2]{c} 
\begin{array}{c}
v_2(t)=6 \\[3pt]
t/2^6 \equiv 1,5,9 \, (16) 
\end{array}
& E_1 & (7,11,15) & 2 & \kIII^* & 1 & 2 & 1\\
& E_2 & (5,8,9) & 2^2 & \kIII & 1 & 1 & 1\\ 
%----------------------------------------------
\SetCell[r=2]{c} v_2(t)=5 
& E_1 & (5,7,8) & 2 & \kIII & 1 & 1 & 1\\
& E_2 & (6,9,13) & 2 & \kI_2^* & 1 & 2 & 1 \\
%----------------------------------------------
\SetCell[r=2]{c} 
\begin{array}{c}
v_2(t)=4 \\[3pt]
t/2^4\equiv 1,5 \, (8)
\end{array}
& E_1 & (5,5,4) & 2 & \kIII & 1 & 1 & 1\\
& E_2 & (4,6,8) & 2 & \kI_1^* & 1 & 1 & 1 \\
%----------------------------------------------
\SetCell[r=2]{c} 
\begin{array}{c}
v_2(t)=4 \\[3pt]
t/2^4\equiv 3,7 \, (8)
\end{array}
& E_1 & (7,5,4) & 2 & \kIV & 1 & 1 & 1\\
& E_2 & (4,6,8) & 2 & \kIV^* & 1 & 1 & 1 \\
%----------------------------------------------
\SetCell[r=2]{c} v_2(t)=3 
& E_1 & (6,\geq 9,12) & 1 & \kI_2^* & 1 & 2 & 1\\
& E_2 & (6,9,15) & 1 & \kI_5^* & 1 & 2 &  1\\
%----------------------------------------------
\SetCell[r=2]{c} 
\begin{array}{c}
v_2(t)=2 \\[3pt]
t/2^2\equiv 1,5 \, (8)
\end{array}
& E_1 & (4,6,8) & 1 & \kI_0^* & 1 & 1 & 1\\
& E_2 & (4,6,10) & 1 & \kI_2^* & 1 & 1 & 1 \\
%----------------------------------------------
\SetCell[r=2]{c} 
\begin{array}{c}
v_2(t)=2 \\[3pt]
t/2^2\equiv 3,7 \, (8)
\end{array}
& E_1 & (4,6,8) & 1 & \kI_!^* & 1 &  1& 1\\
& E_2 & (4,6,10) & 1 & \kIII^* & 1 & 1 & 1 \\
%----------------------------------------------
\SetCell[r=2]{c} v_2(t)=1 
& E_1 & (6,9,16) & 2^{-1} & \kI_6^* & 1 & 2 & 1\\
& E_2 & (6,9,17) & 2^{-1} & \kI_7^* & 1 & 2 &  1\\
%----------------------------------------------
\SetCell[r=2]{c} 
\begin{array}{c}
v_2(t)=0 \\[3pt]
t\equiv 1 \, (4)
\end{array}
& E_1 & (4,6,12) & 2^{-1} & \kI_4^* & 1 & 1 & 2\\
& E_2 & (4,6,12) & 2^{-1} & \kI_4^* &1  & 1 &  2\\
%----------------------------------------------
\SetCell[r=2]{c} 
\begin{array}{c}
v_2(t)=0 \\[3pt]
t\equiv 3 \, (4)
\end{array}
& E_1 & (0,0,0) & 1 & \kI_0 & 1 & 2^{-1} & 2^{-1}\\
& E_2 & (0,0,0) & 1 & \kI_0 & 1 & 2^{-1} &  2^{-1}\\
%----------------------------------------------
\SetCell[r=2]{c} -(2m+1)=v_2(t)<0 
& E_1 & (6,9,4m+20) & 2^{-(m+2)} & I_{4m+10}^* & 1 & 4 & 1\\
& E_2 & (6,9,2m+19) & 2^{-(m+2)} & I_{2m+9}^* & 1 & 4 &  1\\
%----------------------------------------------
\SetCell[r=2]{c}
\begin{array}{c}
-2m=v_2(t)<0 \\[3pt]
2^{2m} t\equiv 3 \, (4)
\end{array}
& E_1 & (0,0,4m) & 2^{-m} & I_{4m} & 1 & 2^{-1} & 2^{-1}\\
& E_2 & (0,0,2m) & 2^{-m} & I_{2m} &1  & 2^{-1} & 2^{-1} \\
%----------------------------------------------
\SetCell[r=2]{c}
\begin{array}{c}
-2m=v_2(t)<0 \\[3pt]
2^{2m} t\equiv 1 \, (4)
\end{array}
& E_1 & (4,6,12+4m) & 2^{-(m+1)} & I_{4m+4}^* & 1 & 
1
& 2\\
& E_2 & (4,6,12+2m) & 2^{-(m+1)} & I_{2m+4}^* & 1 & 
1
& 2 \\
%----------------------------------------------
 \SetCell[c=5,r=2]{c} & & & & &  d\equiv 1 &  d\equiv 2  & d\equiv 3 \\
                      & & & & & \SetCell[c=3]{c} d \Mod{4} & \\
\end{longtblr}

\newpage

\section{Conclusion}
\begin{prop}
Let 
$ 
\begin{tikzcd}
E_1 \arrow[dash]{r}{2}  & E_2 
\end{tikzcd}
$
be a $\mathbf{Q}$-isogeny graph of type $L_2(2)$ corresponding to a given $t$ in $\mathbf{Q}^*$, $t\ne -64$. For every square-free integer $d$, 
the probability of a vertex
to be the Faltings curve (circled)
in the twisted isogeny graph 
$
\begin{tikzcd} 
E_1^d \arrow[dash]{r}{2}  & E_2^d 
\end{tikzcd}
$ 
is given by:

\[
\begin{tblr}[mode=imath]{|c|c|c|c|c|}
\hline
 L_2(2) & \text{twisted isogeny graph} & d & \text{Prob} \\
%-------------------------------------------------
\hline
 \SetCell[r=1]{c} v_2(t)\ge 8   &  E_1^d \longleftarrow  \circled[0.8]{$E_2^d$} & & 1 \\
%-------------------------------------------------
\hline
\SetCell[r=1]{c} 
 v_2(t)=7   
 &  E_1^d \longleftarrow \circled[0.8]{$E_2^d$}&d\not\equiv 0\,(2) & 2/3 \\
\cline{1-1}
\SetCell[r=1]{c}  \begin{array}{c}
 v_2(t)=6\\
 t/2^6\equiv 1(4)
 \end{array}  &   \circled[0.8]{$E_1^d$}\longrightarrow E_2^d &d\equiv 0\,(2) & 1/3 \\
%-------------------------------------------------
\hline
\SetCell[r=1]{c} 
 \begin{array}{c}
 v_2(t)=6\\
 t/2^6\equiv 3(4)
 \end{array}
 &  \circled[0.8]{$E_1^d$}\longrightarrow E_2^d   &d\not\equiv 0\,(2) & 2/3 \\
\cline{1-1}
\SetCell[r=1]{c}  v_2(t)=5     &  E_1^d \longleftarrow \circled[0.8]{$E_2^d$}  &d\equiv 0\,(2) & 1/3 \\
%-------------------------------------------------
\hline
 \SetCell[r=1]{c} v_3(t)\leq 4 & \circled[0.8]{$E_1^d$} \longrightarrow  E_2^d  & & 1 \\
%-------------------------------------------------
\hline
%-------------------------------------------------
\end{tblr}
\]



\end{prop}

\vskip 0.35truecm

\noindnet{\it Proof.} From the previous tables one gets:

\vskip 0.5truecm

\begin{tblr}{cells={mode=imath},hlines,vlines,measure=vbox}
%-------------------------------------------------
\SetCell[c=1]{c} t &\SetCell[c=1]{c} [u(E)]  & \SetCell[c=1]{c} [u(E)(d)] & \SetCell[c=1]{c} d & \SetCell[c=1]{c}\text{Prob}\\
%-------------------------------------------------
\SetCell[r=1]{c} v_2(t)\ge 8 & \SetCell[r=1]{c} (1:2) & \SetCell[r=1]{c}(1:1) &  & \SetCell[r=1]{c} (0,1)\\
%-------------------------------------------------
\SetCell[r=1]{c} v_2(t)=7 & \SetCell[r=2]{c} (1:2) & \SetCell[r=1]{c} (1:1) & d\not\equiv 0\,(2)&\SetCell[r=2]{c} \displaystyle{\left(\frac{1}{3},\frac{2}{3}\right)} \\
\SetCell[r=1]{c} \begin{array}{c}
 v_2(t)=6\\
 t/2^6\equiv 1(4)
 \end{array} & & \SetCell[r=1]{c} (2:1) & d\equiv 0\,(2)& \\
%-------------------------------------------------
 \SetCell[r=1]{c} \begin{array}{c}
 v_2(t)=6\\
 t/2^6\equiv 3(4)
 \end{array}  & \SetCell[r=2]{c} (1:1) & \SetCell[r=1]{c} (1:1) & d\not\equiv 0\,(2)&\SetCell[r=2]{c} \displaystyle{\left(\frac{2}{3},\frac{1}{3}\right)} \\
\SetCell[r=1]{c} v_2(t)=5 & & \SetCell[r=1]{c} (1:2) & d\equiv 0\,(2)& \\
%-------------------------------------------------
\SetCell[r=1]{c}  v_2(t)\le 4 &\SetCell[r=1]{c} (1:1) & \SetCell[r=1]{c}(1:1) & & \SetCell[r=1]{c}(1,0) \\
%-------------------------------------------------
\end{tblr}



\end{document}