

\documentclass[11pt]{article}
\usepackage{amsfonts,amssymb,amsmath,amsthm,latexsym,graphics,epsfig,amsfonts}
\usepackage{verbatim,enumerate,array,booktabs,color,bigstrut,prettyref,tikz-cd}
\usepackage{multirow}
\usepackage[all]{xy}
\usepackage[backref]{hyperref}
\usepackage[OT2,T1]{fontenc}
%\usepackage{ctable}
\usepackage{mathtools}

\usepackage{longtable}


\usepackage{mathtools}
\newcommand{\Mod}[1]{\ (\mathrm{mod}\ #1)}
\newcommand{\mathdash}{\relbar\mkern-8mu\relbar}
\newcommand*\circled[2][1.6]{\tikz[baseline=(char.base)]{
    \node[shape=circle, draw, inner sep=1pt, 
        minimum height={\f@size*#1},] (char) {\vphantom{WAH1g}#2};}}
\makeatother



\usepackage{tabularray}
\UseTblrLibrary{amsmath,varwidth}

\usepackage{tabularx}
\usepackage{longtable}
\usepackage{arydshln}


\newcommand\myiso{\stackrel{\mathclap{\normalfont\mbox{\small $p$}}}{-}}
\newcommand\myisot{\stackrel{\mathclap{\normalfont\mbox{\small $3$}}}{-}}

\newcommand{\pref}[1]{\prettyref{#1}}
\newrefformat{eq}{\textup{(\ref{#1})}}
\newrefformat{prty}{\textup{(\ref{#1})}}

\definecolor{mylinkcolor}{rgb}{0.8,0,0}
\definecolor{myurlcolor}{rgb}{0,0,0.8}
\definecolor{mycitecolor}{rgb}{0,0,0.8}
\hypersetup{colorlinks=true,urlcolor=myurlcolor,citecolor=mycitecolor,linkcolor=mylinkcolor,linktoc=page,breaklinks=true}

%\DeclareSymbolFont{cyrletters}{OT2}{wncyr}{m}{n}
%\DeclareMathSymbol{\Sha}{\mathalpha}{cyrletters}{"58}

\addtolength{\textwidth}{4cm} \addtolength{\hoffset}{-2cm}
\addtolength{\marginparwidth}{-2cm}

%\theoremstyle{definition}
\newtheorem{defn}{Definition}[section]
\newtheorem{definition}[defn]{Definition}
\newtheorem{claim}[defn]{Claim}

%\theoremstyle{plain}
\newtheorem{thmA}{Theorem A}
\newtheorem{thmB}{Theorem B}
\newtheorem{thm2}{Theorem}
\newtheorem{prop2}{Proposition}
\newtheorem{note}{Note}

\newtheorem{corollary}[defn]{Corollary}
\newtheorem{lemma}[defn]{Lemma}
\newtheorem{property}[defn]{Property}
\newtheorem{thm}[defn]{Theorem}
\newtheorem{theorem}[defn]{Theorem}
\newtheorem{cor}[defn]{Corollary}
\newtheorem{prop}[defn]{Proposition}
\newtheorem{proposition}[defn]{Proposition}
\newtheorem{thmnn}{Theorem}
\newtheorem{conj}[defn]{Conjecture}

\theoremstyle{definition}
\newtheorem{remarks}{Remarks}
\newtheorem{ack}{Acknowledgements}
\newtheorem{remark}[defn]{Remark}
\newtheorem{question}[defn]{Question}
\newtheorem{example}[defn]{Example}


\newcommand{\Q}{\mathbb Q}
\newcommand{\Qbar}{\overline{\Q}}
\newcommand{\Z}{\mathbb Z}

\newcommand{\modQ}{\,\text{mod}\,(\Q^)^2}

\newcommand{\mysquare}[1]{\tikz{\path[draw] (0,0) rectangle node{\tiny #1} (8pt,8pt) ;}}
\newcommand{\mycircle}[1]{\tikz{\path[draw] (0,0) circle (4pt) node{\tiny #1};}}


%------------------------------------
\newcommand{\Kd}{\operatorname{K}}
\newcommand{\kI}{\operatorname{I}}
\newcommand{\kII}{\operatorname{II}}
\newcommand{\kIII}{\operatorname{III}}
\newcommand{\kIV}{\operatorname{IV}}
%-------------------------------------



\begin{document}
\title{Type $L_2(3)$}
\date{\today}
\maketitle
\section{Setting}

The isogeny graphs of type $L_2(3)$ are given by
two isogenous elliptic curves:

\[ 
\begin{tikzcd}
E_1 \arrow[dash]{r}{3} & E_3   \,.
\end{tikzcd}
\]


\noindent A hauptmodule of $X_0(3)$ is  
$$t(\tau)= 3^{6} \left( \frac{\eta(3\tau)}{\eta(\tau)}\right)^{12}\,.$$ 
Letting $t=t(\tau)$, one can write
$$
\begin{tblr}{l@{\,=\,}l}
j(E_1) = j(\tau) & 
\displaystyle{\frac{(t+3)^3 (t+27)}{t}}\\[6pt]
j(E_3) = j(3\tau) & 
\displaystyle
{\frac{(t+27) (t+243)^3}{t^3}}\,,\\[6pt]
\end{tblr}
$$
and the Fricke involution of $X_0(3)$ is given by $W_3(t)=3^6/t $.

We can (and do) choose Weierstrass equations for $(E_1,E_3)$ with signatures:$$
 \begin{tblr}[mode=dmath]{|c|l|}
\hline \SetCell[c=2]{c} L_2(3) \\ \hline
 c_4(E_1) & (t + 3)(t + 27)\\

 c_6(E_1) & (t + 27)(t^2 + 18t - 27)\\

 \Delta(E_1) & t(t + 27)^{2}\\ \hline

 c_4(E_3) & (t + 27)(t + 243)\\

 c_6(E_3) & (t + 27)(t^2 - 486t - 19683)\\

 \Delta(E_3) & t^{3}(t + 27)^{2}\\ \hline

\end{tblr}
$$
With regard to the action of the Fricke involution 
on the isogeny graph, 
it can be displayed as follows:
\[ 
\begin{tikzcd}
W_3\,(
E_1 \arrow[dash]{r}{3} & E_3
) = \,\, E_3^{-t} \arrow[dash]{r}{3} & E_1^{-t}   \,.
\end{tikzcd}
\]
\newpage

\section{Kodaira symbols \& Pal coefficients}

\begin{longtblr}
[caption= {$L_2(3)$ data for $p\ne 2,3$}]
{cells={mode=imath},hlines,vlines,measure=vbox,
hline{Z}={1-X}{0pt},
vline{1}={Y-Z}{0pt},
colspec=cclclcc}
\SetCell[c=1]{c} L_2(3) &\SetCell[c=6]{c} p\ne 2,3  &    & \\
\SetCell[c=1]{c} t & E & 
\SetCell[c=1]{c} \operatorname{sig}_p(E) & u & \Kd_p(E) & \SetCell[c=2]{c} u_p(d)\\
%----------------------------------------------
\SetCell[r=2]{c} 
m= v_p(t)>0
& E_1 & (0,0,m) & 1 &  \kI_m & 1& 1\\
& E_3 & (0,0,3m) & 1 &  \kI_{3m} & 1& 1 \\
%----------------------------------------------
\SetCell[r=2]{c} 
\begin{array}{c}
     v_p(t)=0  \\[3pt]
    m = v_p(t+27)=6k 
\end{array}
& E_1 & (\ge 0,0,0) & p^k & \kI_0  & 1 & 1\\
& E_3 & (\ge 0,0,0) & p^k & \kI_0  & 1 & 1\\
%----------------------------------------------
\SetCell[r=2]{c} 
\begin{array}{c}
     v_p(t)=0  \\[3pt]
    m = v_p(t+27)=6k +1
\end{array}
& E_1 & (\ge 1,1,2) & p^k & \kII  & 1 & 1\\
& E_3 & (\ge 1,1,2) & p^k & \kII  &1  & 1\\
%----------------------------------------------
\SetCell[r=2]{c} 
\begin{array}{c}
     v_p(t)=0  \\[3pt]
    m = v_p(t+27)=6k +2
\end{array}
& E_1 & (\ge 2,2,4) & p^k & \kIV  & 1 & 1\\
& E_3 & (\ge 2,2,4) & p^k & \kIV  &1  & 1\\
%----------------------------------------------
\SetCell[r=2]{c} 
\begin{array}{c}
     v_p(t)=0  \\[3pt]
    m = v_p(t+27)=6k +3
\end{array}
& E_1 & (\ge 3,3,6) & p^k & \kI_0^*  & p & 1\\
& E_3 & (\ge 3,3,6) & p^k & \kI_0^*  & p & 1\\
%----------------------------------------------
\SetCell[r=2]{c} 
\begin{array}{c}
     v_p(t)=0  \\[3pt]
    m = v_p(t+27)=6k +4
\end{array}
& E_1 & (\ge 4,4,8) & p^k & \kIV^*  & p & 1\\
& E_3 & (\ge 4,4,8) & p^k & \kIV^*  & p & 1\\
%----------------------------------------------
\SetCell[r=2]{c} 
\begin{array}{c}
     v_p(t)=0  \\[3pt]
    m = v_p(t+27)=6k +5
\end{array}
& E_1 & (\ge 5,5,10) & p^k & \kII^*  & p & 1\\
& E_3 & (\ge 5,5,10) & p^k & \kII^*  & p & 1\\
%----------------------------------------------
\SetCell[r=2]{c} 
\begin{array}{c}
     -m= v_p(t)<0  \\[3pt]
    \text{$m$ even} 
\end{array}
& E_1 & (0,0,3m) & p^{-m/2} & \kI_{3m}  & 1& 1\\
& E_3 & (0,0,m) & p^{-m/2} & \kI_{m}  & 1& 1\\
%----------------------------------------------
\SetCell[r=2]{c} 
\begin{array}{c}
     -m= v_p(t)<0  \\[3pt]
    \text{$m$ odd}\\
\end{array}
& E_1 & (2,3,3m+6) & p^{-(m+1)/2} & \kI_{3m}^* & p& 1\\
& E_3 & (2,3,m+6) & p^{-(m+1)/2} &  \kI_{m}^* & p& 1\\
%----------------------------------------------
 \SetCell[c=5,r=2]{c} & & & & & d\equiv 0  & d\not\equiv 0 \\
                      & & & & & \SetCell[c=2]{c} d \Mod p & \\

\end{longtblr}



\begin{longtblr}
[caption= {$L_2(3)$ data for $p=3$}]
{cells={mode=imath},hlines,vlines,measure=vbox,
hline{Z}={1-X}{0pt},
vline{1}={Y-Z}{0pt},
colspec=cclclcc}
\SetCell[c=1]{c} L_2(3) &\SetCell[c=6]{c} p=3  &    & \\
\SetCell[c=1]{c} t & E & 
\SetCell[c=1]{c} \operatorname{sig}_3(E) & u & \Kd_3(E) & \SetCell[c=2]{c} u_3(d)\\
%----------------------------------------------
\SetCell[r=2]{c} 
m= v_3(t)\ge 6
& E_1 & (0,0,m-6) & 3 &  \kI_{m-6} & 1& 1\\
& E_3 & (0,0,3(m-6)) & 3^2 &  \kI_{3(m-6)} & 1& 1\\
%----------------------------------------------
\SetCell[r=2]{c} 
v_3(t)=5
& E_1 & (4,6,11) & 1 &  \kII^* & 3& 1\\
& E_3 & (\ge 4,6,9) & 3 &  \kIV^* & 3& 1\\
%----------------------------------------------
\SetCell[r=2]{c} 
v_3(t)=4
& E_1 & (4,6,10) & 1 &  \kIV^* & 3& 1\\
& E_3 & (3,4,6) & 3 &  \kIV & 1& 1\\
%----------------------------------------------
\SetCell[r=2]{c} 
\begin{array}{c}
v_3(t)=3\\
v_3(t+27)=3+6k\\
(t+27)/3^{3+6k}\equiv 4,5\,(9)
\end{array}
& E_1 & (4+2k,6,9) & 3^k &  \kIII^* & 3& 1\\
& E_3 & (2+2k,3,3) & 3^{k+1} &  \kIII & 1& 1\\
%----------------------------------------------
\SetCell[r=2]{c} 
\begin{array}{c}
v_3(t)=3\\
v_3(t+27)=3+6k\\
(t+27)/3^{3+6k}\not \equiv 4,5\,(9)
\end{array}
& E_1 & (4+2k,6,9) & 3^k &  \kIV^* & 3& 1\\
& E_3 & (2+2k,3,3) & 3^{k+1} &  \kII & 1& 1\\
%----------------------------------------------
\SetCell[r=2]{c} 
\begin{array}{c}
v_3(t)=3\\
v_3(t+27)=4+6k
\end{array}
& E_1 & (5+2k,7,11) & 3^k &  \kIV^* & 3& 1\\
& E_3 & (3+2k,4,5) & 3^{k+1} &  \kII & 1& 1\\
%----------------------------------------------
\SetCell[r=2]{c} 
\begin{array}{c}
v_3(t)=3\\
v_3(t+27)=5+6k
\end{array}
& E_1 & (6+2k,8,13) & 3^k &  \kII^* & 3& 1\\
& E_3 & (4+2k,5,7) & 3^{k+1} &  \kIV & 1& 1\\
%----------------------------------------------
\SetCell[r=2]{c} 
\begin{array}{c}
v_3(t)=3\\
v_3(t+27)=6+6k\\
(t+27)/3^{3+6k}\equiv 4,5\,(9)
\end{array}
& E_1 & (3+2k,3,3) & 3^{k+1} &  \kII & 1& 1\\
& E_3 & (5+2k,6,9) & 3^{k+1} &  \kIII^* & 3& 1\\
%----------------------------------------------
\SetCell[r=2]{c} 
\begin{array}{c}
v_3(t)=3\\
v_3(t+27)=6+6k\\
(t+27)/3^{3+6k}\not\equiv 4,5\,(9)
\end{array}
& E_1 & (3+2k,3,3) & 3^{k+1} &  \kII & 1& 1\\
& E_3 & (5+2k,6,9) & 3^{k+1} &  \kIV^* & 3& 1\\
%----------------------------------------------
\SetCell[r=2]{c} 
\begin{array}{c}
v_3(t)=3\\
v_3(t+27)=7+6k
\end{array}
& E_1 & (4+2k,4,5) & 3^{k+1} &  \kII & 1& 1\\
& E_3 & (6+2k,7,11) & 3^{k+1} &  \kIV^* & 3& 1\\
%----------------------------------------------
\SetCell[r=2]{c} 
\begin{array}{c}
v_3(t)=3\\
v_3(t+27)=8+6k
\end{array}
& E_1 & (5+2k,5,7) & 3^{k+1} &  \kIV & 1& 1\\
& E_3 & (7+2k,8,13) & 3^{k+1} &  \kI^* & 3& 1\\
%----------------------------------------------
\SetCell[r=2]{c} 
v_3(t)=2
& E_1 & (3,5,6) & 1 &  \kIV & 1& 1\\
& E_3 & (4,6,10) & 1 &  \kIV^* & 3& 1\\
%----------------------------------------------
\SetCell[r=2]{c} 
v_3(t)=1
& E_1 & (\ge 2,3,3) & 1 &  \kII & 1& 1\\
& E_3 & (2,3,5) & 1 &  \kIV & 1& 1\\
%----------------------------------------------
\SetCell[r=2]{c} 
v_3(t)=0
& E_1 & (0,0,0) & 1 &  \kI_0 & 1& 1\\
& E_3 & (0,0,0) & 1 &  \kI_0 & 1& 1\\
%----------------------------------------------
\SetCell[r=2]{c} 
\begin{array}{c}
     -m= v_3(t)<0  \\[3pt]
    \text{$m$ even} 
\end{array}
& E_1 & (0,0,3m) & 3^{-m/2} & \kI_{3m}  & 1& 1\\
& E_3 & (0,0,m) & 3^{-m/2} & \kI_{m}  & 1& 1\\
%----------------------------------------------
\SetCell[r=2]{c} 
\begin{array}{c}
     -m= v_3(t)<0  \\[3pt]
    \text{$m$ odd}\\
\end{array}
& E_1 & (2,3,3m+6) & 3^{-(m+1)/2} & \kI_{3m}^* & 3& 1\\
& E_3 & (2,3,m+6) & 3^{-(m+1)/2} & \kI_{m}^* & 3& 1\\
%----------------------------------------------
 \SetCell[c=5,r=2]{c} & & & & & d\equiv 0  & d\not\equiv 0 \\
                      & & & & & \SetCell[c=2]{c} d \Mod p & \\

\end{longtblr}



\newpage

\begin{longtblr}
[caption = {$L_2(3)$ data for $p$=2}]
{cells = {mode=imath},hlines,vlines,measure=vbox,
hline{Z} = {1-5}{0pt},
vline{1} = {Y-Z}{0pt},
colspec  = cclclccc}
%----------------------------------------------
L_2(3) & \SetCell[c=7]{c} p=2  & & & & & \\ 
t & E & \SetCell[c=1]{c} \operatorname{sig}_2(E) & u & \SetCell[c=1]{c} \Kd_2(E) & \SetCell[c=3]{c} u_2(d)  \\
%----------------------------------------------
\SetCell[r=2]{c} m=v_2(t)\ge 2 
& E_1 & (0,0,m) & 1 & I_{m} & 1 & 2^{-1} & 2^{-1} \\
& E_3 & (0,0,3m) & 1 & I_{3m} & 1 & 2^{-1} & 2^{-1} \\
%----------------------------------------------
\SetCell[r=2]{c} v_2(t)=1 
& E_1 & (4,6,13) & 2^{-1} & I_{5}^* & 1 & 1 &2  \\
& E_3 & (4,6,15) & 2^{-1} & I_{7}^* & 1 & 1 &  2\\
%----------------------------------------------
\SetCell[r=2]{c} 
\begin{array}{c}
v_2(t)=0 \\
t\equiv 3\,(4)
\end{array}
& E_1 & (6,9,14) & 2^{-1} & \kI_4^* & 1 & 2 & 1\\
& E_3 & (6,9,14) & 2^{-1} & \kI_4^* & 1 & 2 & 1\\
%----------------------------------------------
\SetCell[r=2]{c} 
\begin{array}{c}
v_2(t)=0 \\
t\equiv 1\,(4)\\
(t+27)/2^6\equiv 1\,(4)
\end{array}
& E_1 & (9,9,12) & 1 & \kII^* & 1 & 2 & 2\\
& E_3 & (9,9,12) & 1 & \kII^* & 1 & 2 & 2\\
%----------------------------------------------
\SetCell[r=2]{c} 
\begin{array}{c}
v_2(t)=0 \\
t\equiv 1\,(4)\\
(t+27)/2^6\equiv 3\,(4)
\end{array}
& E_1 & (5,3,0) & 2 & \kI_0 & 1 & 1 & 2^{-1}\\
& E_3 & (5,3,0) & 2 & \kI_0 & 1 & 1 & 2^{-1}\\
%----------------------------------------------
\SetCell[r=2]{c} 
\begin{array}{c}
v_2(t)=0 \\
t\equiv 1\,(4)\\
v_2(t+27)=6k>6
\end{array}
& E_1 & (\ge 4,3,0) & 2^k & I_0 & 1 &1  & 2^{-1}\\
& E_3 & (\ge 4,3,0) & 2^k & I_0 & 1 &1  & 2^{-1}\\
%----------------------------------------------
\SetCell[r=2]{c} 
\begin{array}{c}
v_2(t)=0 \\
t\equiv 1\,(4)\\
v_2(t+27)=6k+1>1
\end{array}
& E_1 & (\ge,9,10,14) & 2^{k-1} & \kII^* & 1 & 2 & 1\\
& E_3 & (\ge,9,10,14) & 2^{k-1} & \kII^* & 1 & 2 & 1\\
%----------------------------------------------
\SetCell[r=2]{c} 
\begin{array}{c}
v_2(t)=0 \\
t\equiv 1\,(4)\\
v_2(t+27)=6k+2
\end{array}
& E_1 & (\ge 4,5,4) & 2^k & \text{$\kII$ or $\kIII$ or $\kIV$}  & 2^k & 1 &1 \\
& E_3 & (\ge 4,5,4) & 2^k & \text{$\kII$ or $\kIII$ or $\kIV$}  & 2^k & 1 & 1\\
%----------------------------------------------
\SetCell[r=2]{c} 
\begin{array}{c}
v_2(t)=0 \\
t\equiv 1\,(4)\\
v_2(t+27)=6k+3
\end{array}
& E_1 & (\ge 5,6,6) & 2^k & \kII & 1 & \text{$1^*$ or $2^*$} & 1\\
& E_3 & (\ge 5,6,6) & 2^k & \kII & 1 & \text{$1^*$ or $2^*$} & 1\\
%----------------------------------------------
\SetCell[r=2]{c} 
\begin{array}{c}
v_2(t)=0 \\
t\equiv 1\,(4)\\
v_2(t+27)=6k+4
\end{array}
& E_1 & (\ge 6,7,8) & 2^k & \text{$\kI_1^*$ or $\kI_0^*$ or $\kIV^*$}  & 1 & 1 & 1\\
& E_3 & (\ge 6,7,8) & 2^k & \text{$\kI_1^*$ or $\kI_0^*$ or $\kIV^*$}  & 1 & 1 & 1\\
%----------------------------------------------
\SetCell[r=2]{c} 
\begin{array}{c}
v_2(t)=0 \\
t\equiv 1\,(4)\\
v_2(t+27)=6k+5
\end{array}
& E_1 & (\ge 7,8,10) & 2^k  & \kI_0^* & 1 & 2 & 1\\
& E_3 & (\ge 7,8,10) & 2^k  & \kI_0^* & 1 & 2 & 1\\
%----------------------------------------------
\SetCell[r=2]{c} 
\begin{array}{c}
-m=v_2(t)<0 \\
\text{$m$ odd}
\end{array}
& E_1 & (2,3,3m+6) & 2^{-(m+3)/2} & I_{3m}^* & 1 & 1 & 1\\
& E_3 & (2,3,m+6) & 2^{-(m+3)/2} & I_{m}^* & 1 & 1 & 1\\
%----------------------------------------------
\SetCell[r=2]{c}
\begin{array}{c}
-m=v_2(t)<0 \\
\text{$m$ even}\\
2^m t\equiv 1\,(4)
\end{array}
& E_1 & (4,6,3m+12) & 2^{-(m+2)/2} & I_{3m+4}^* & 1 & 1 & 2\\
& E_3 & (4,6,m+12) & 2^{-(m+2)/2} & I_{m+4}^* & 1 & 1 & 2\\
%----------------------------------------------
\SetCell[r=2]{c}
\begin{array}{c}
-m=v_2(t)<0 \\
\text{$m$ even}\\
2^m t\equiv 3\,(4)
\end{array}
& E_1 & (0,0,3m) & 2^{-m/2} & I_{3m} & 1 & 2^{-1}  & 2^{-1}\\
& E_3 & (0,0,m) & 2^{-m/2} & I_{m} & 1 & 2^{-1} & 2^{-1} \\
%----------------------------------------------
 \SetCell[c=5,r=2]{c} & & & & &  d\equiv 1 &  d\equiv 2  & d\equiv 3 \\
                      & & & & & \SetCell[c=3]{c} d \Mod{4} & \\
\end{longtblr}
\newpage

\section{Conclusion}

\begin{prop}
Let 
$ 
\begin{tikzcd}
E_1 \arrow[dash]{r}{3}  & E_3 
\end{tikzcd}
$
be a $\mathbf{Q}$-isogeny graph of type $L_2(3)$ corresponding to a given $t$ in $\mathbf{Q}^*$, $t\ne -27$. For every square-free integer $d$, 
the probability of a vertex
to be the Faltings curve (circled)
in the twisted isogeny graph 
$
\begin{tikzcd} 
E_1^d \arrow[dash]{r}{3}  & E_3^d 
\end{tikzcd}
$ 
is given by:

\[
\begin{tblr}[mode=imath]{|c|c|c|c|c|}
\hline
 L_2(3) & \text{twisted isogeny graph} & d & \text{Prob} \\
%-------------------------------------------------
\hline
 \SetCell[r=1]{c} v_3(t)\ge 5   &  E_1^d \longleftarrow  \circled[0.8]{$E_3^d$} & & 1 \\
%-------------------------------------------------
\hline
\SetCell[r=1]{c} 
 v_3(t)=4   
 &  E_1^d \longleftarrow \circled[0.8]{$E_3^d$}&d\not\equiv 0\,(3) & 3/4 \\
\cline{1-1}
\SetCell[r=1]{c}  \begin{array}{c}
 v_3(t)=3\\
 v_3(t+27)\equiv 3,4,5\,(6)
 \end{array}  &   \circled[0.8]{$E_1^d$}\longrightarrow E_3^d &d\equiv 0\,(3) & 1/4 \\
%-------------------------------------------------
\hline
\SetCell[r=1]{c} 
 \begin{array}{c}
 v_3(t)=3\\
 v_3(t+27)\equiv 0,1,2\,(6)
 \end{array}
 &  \circled[0.8]{$E_1^d$}\longrightarrow E_3^d   &d\not\equiv 0\,(3) & 3/4 \\
\cline{1-1}
\SetCell[r=1]{c}  v_3(t)=2     &  E_1^d \longleftarrow \circled[0.8]{$E_3^d$}  &d\equiv 0\,(3) & 1/4 \\
%-------------------------------------------------
\hline
 \SetCell[r=1]{c} v_3(t)\leq 1 & \circled[0.8]{$E_1^d$} \longrightarrow  E_3^d  & & 1 \\
%-------------------------------------------------
\hline
%-------------------------------------------------
\end{tblr}
\]



\end{prop}

\vskip 0.35truecm

\noindnet{\it Proof.} From the previous tables one gets:

\vskip 0.5truecm

\begin{tblr}{cells={mode=imath},hlines,vlines,measure=vbox}
%-------------------------------------------------
\SetCell[c=1]{c} t &\SetCell[c=1]{c} [u(E)]  & \SetCell[c=1]{c} [u(E)(d)] & \SetCell[c=1]{c} d & \SetCell[c=1]{c}\text{Prob}\\
%-------------------------------------------------
\SetCell[r=1]{c} v_3(t)\ge 5 & \SetCell[r=1]{c} (1:3) & \SetCell[r=1]{c}(1:1) &  & \SetCell[r=1]{c} (0,1)\\
%-------------------------------------------------
\SetCell[r=1]{c} v_3(t)=4 & \SetCell[r=2]{c} (1:3) & \SetCell[r=1]{c} (1:1) & d\not\equiv 0\,(3)&\SetCell[r=2]{c} \displaystyle{\left(\frac{1}{4},\frac{3}{4}\right)} \\
\SetCell[r=1]{c} \begin{array}{c}
 v_3(t)=3\\
 v_3(t+27)\equiv 3,4,5\,(6)
 \end{array} & & \SetCell[r=1]{c} (3:1) & d\equiv 0\,(3)& \\
%-------------------------------------------------
 \SetCell[r=1]{c} \begin{array}{c}
 v_3(t)=3\\
 v_3(t+27)\equiv 0,1,2\,(6)
 \end{array}  & \SetCell[r=2]{c} (1:1) & \SetCell[r=1]{c} (1:1) & d\not\equiv 0\,(3)&\SetCell[r=2]{c} \displaystyle{\left(\frac{3}{4},\frac{1}{4}\right)} \\
\SetCell[r=1]{c} v_3(t)=2 & & \SetCell[r=1]{c} (1:3) & d\equiv 0\,(3)& \\
%-------------------------------------------------
\SetCell[r=1]{c}  v_3(t)\le 1 &\SetCell[r=1]{c} (1:1) & \SetCell[r=1]{c}(1:1) & & \SetCell[r=1]{c}(1,0) \\
%-------------------------------------------------
\end{tblr}



\end{document}
