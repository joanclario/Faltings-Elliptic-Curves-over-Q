\documentclass[11pt]{article}
\usepackage{amsfonts,amssymb,amsmath,amsthm,latexsym,graphics,epsfig,amsfonts}
\usepackage{verbatim,enumerate,array,booktabs,color,bigstrut,prettyref,tikz-cd}
\usepackage{multirow}
\usepackage[all]{xy}
\usepackage[backref]{hyperref}
\usepackage[OT2,T1]{fontenc}
%\usepackage{ctable}
\usepackage{mathtools}

\usepackage{longtable}

\usepackage{mathtools}
\newcommand{\Mod}[1]{\ (\mathrm{mod}\ #1)}
\newcommand{\mathdash}{\relbar\mkern-8mu\relbar}
\newcommand*\circled[2][1.6]{\tikz[baseline=(char.base)]{
    \node[shape=circle, draw, inner sep=1pt, 
        minimum height={\f@size*#1},] (char) {\vphantom{WAH1g}#2};}}
\makeatother



\usepackage{tabularray}
\UseTblrLibrary{amsmath,varwidth}

\usepackage{tabularx}
\usepackage{longtable}
\usepackage{arydshln}


\newcommand\myiso{\stackrel{\mathclap{\normalfont\mbox{\small $p$}}}{-}}
\newcommand\myisot{\stackrel{\mathclap{\normalfont\mbox{\small $3$}}}{-}}

\newcommand{\pref}[1]{\prettyref{#1}}
\newrefformat{eq}{\textup{(\ref{#1})}}
\newrefformat{prty}{\textup{(\ref{#1})}}

\definecolor{mylinkcolor}{rgb}{0.8,0,0}
\definecolor{myurlcolor}{rgb}{0,0,0.8}
\definecolor{mycitecolor}{rgb}{0,0,0.8}
\hypersetup{colorlinks=true,urlcolor=myurlcolor,citecolor=mycitecolor,linkcolor=mylinkcolor,linktoc=page,breaklinks=true}

%\DeclareSymbolFont{cyrletters}{OT2}{wncyr}{m}{n}
%\DeclareMathSymbol{\Sha}{\mathalpha}{cyrletters}{"58}

\addtolength{\textwidth}{4cm} \addtolength{\hoffset}{-2cm}
\addtolength{\marginparwidth}{-2cm}

%\theoremstyle{definition}
\newtheorem{defn}{Definition}[section]
\newtheorem{definition}[defn]{Definition}
\newtheorem{claim}[defn]{Claim}

%\theoremstyle{plain}
\newtheorem{thmA}{Theorem A}
\newtheorem{thmB}{Theorem B}
\newtheorem{thm2}{Theorem}
\newtheorem{prop2}{Proposition}
\newtheorem{note}{Note}

\newtheorem{corollary}[defn]{Corollary}
\newtheorem{lemma}[defn]{Lemma}
\newtheorem{property}[defn]{Property}
\newtheorem{thm}[defn]{Theorem}
\newtheorem{theorem}[defn]{Theorem}
\newtheorem{cor}[defn]{Corollary}
\newtheorem{prop}[defn]{Proposition}
\newtheorem{proposition}[defn]{Proposition}
\newtheorem{thmnn}{Theorem}
\newtheorem{conj}[defn]{Conjecture}

\theoremstyle{definition}
\newtheorem{remarks}{Remarks}
\newtheorem{ack}{Acknowledgements}
\newtheorem{remark}[defn]{Remark}
\newtheorem{question}[defn]{Question}
\newtheorem{example}[defn]{Example}


\newcommand{\Q}{\mathbb Q}
\newcommand{\Qbar}{\overline{\Q}}
\newcommand{\Z}{\mathbb Z}

\newcommand{\modQ}{\,\text{mod}\,(\Q^)^2}

\newcommand{\mysquare}[1]{\tikz{\path[draw] (0,0) rectangle node{\tiny #1} (8pt,8pt) ;}}
\newcommand{\mycircle}[1]{\tikz{\path[draw] (0,0) circle (4pt) node{\tiny #1};}}


%------------------------------------
\newcommand{\Kd}{\operatorname{K}}
\newcommand{\kI}{\operatorname{I}}
\newcommand{\kII}{\operatorname{II}}
\newcommand{\kIII}{\operatorname{III}}
\newcommand{\kIV}{\operatorname{IV}}
%-------------------------------------



\begin{document}
\title{Type $T_8$}
\date{\today}
\maketitle

\noindent{\bf Graph.} The isogeny graphs of type $T_8$ are given by
eight isogenous elliptic curves:

\[ \begin{tikzcd}
& E_{21} \ar[dash,d,"2"] & & E_{81} \ar[dash,d,"2"] & \\
& E_{2} \ar[dash,swap,dl,"2"] \ar[dash,dr,"2"] & & E_{8} \ar[dash,dl,swap,"2"] \ar[dash,dr,"2"] & \\
 E_1 &  & E_4 \ar[dash,d,"2"] & & E_{16}  \\
 &  & E_{41} & &    \,.
\end{tikzcd}
\]

\noindent{\bf Hauptmodul.} A hauptmodul for $X_0(16)$ is  
$$
t = 2 + 2^3 \, \displaystyle{\frac{\eta(2\tau) \eta(16t)^2}{\eta(\tau)^2 \eta(8\tau)}}\,.
$$

\noindent{\bf $j$-invariants.}
One has
$$
\begin{tblr}{l}
j(E_1) = j(\tau) = \\ 
t^{-4} \cdot (t - 2)^{-1} \cdot (t + 2)^{-1} \cdot (t^{2} + 4)^{-1} \cdot (t^{8} - 16 t^{4} + 16)^{3}
\\[5pt]
j(E_2) = j(2\tau)= \\ 
t^{-8} \cdot (t - 2)^{-2} \cdot (t + 2)^{-2} \cdot (t^{2} + 4)^{-2} \cdot (t^{8} - 16 t^{4} + 256)^{3}
\\[5pt]
j(E_{21}) = j(\tau+1/2) =\\ 
\left(-1\right) \cdot t^{-16} \cdot (t - 2)^{-1} \cdot (t + 2)^{-1} \cdot (t^{2} + 4)^{-1} \cdot (t^{8} - 256 t^{4} + 4096)^{3}
\\[5pt]
j(E_4) = j(4\tau) =\\ (t - 2)^{-4} \cdot t^{-4} \cdot (t + 2)^{-4} \cdot (t^{2} + 4)^{-4} \cdot (t^{4} - 4 t^{3} + 8 t^{2} + 16 t + 16)^{3} \cdot 
\\ (t^{4} + 4 t^{3} + 8 t^{2} - 16 t + 16)^{3}
\\[5pt]
j(E_{41}) = j(2\tau+1/2) =\\
\left(-1\right) \cdot (t - 2)^{-2} \cdot t^{-2} \cdot (t + 2)^{-2} \cdot (t^{2} + 4)^{-8} \cdot (t^{4} - 16 t^{3} + 8 t^{2} + 64 t + 16)^{3} \cdot 
\\ (t^{4} + 16 t^{3} + 8 t^{2} - 64 t + 16)^{3}
\\[5pt]
j(E_8) = j(8\tau) = \\ 
(t - 2)^{-8} \cdot (t + 2)^{-8} \cdot t^{-2} \cdot (t^{2} + 4)^{-2} \cdot (t^{8} + 240 t^{6} + 2144 t^{4} + 3840 t^{2} + 256)^{3}
\\[5pt]
j(E_{81}) = j(4\tau+1/2) = \\ 
\left(-1\right) \cdot (t + 2)^{-16} \cdot (t - 2)^{-4} \cdot t^{-1} \cdot (t^{2} + 4)^{-1} \cdot \\
(t^{8} - 240 t^{7} + 2160 t^{6} - 6720 t^{5} + 17504 t^{4} - 26880 t^{3} + 34560 t^{2} - 15360 t + 256)^{3}
\\[5pt]
j(E_{16}) = j(16\tau)= \\
(t - 2)^{-16} \cdot (t + 2)^{-4} \cdot t^{-1} \cdot (t^{2} + 4)^{-1} \cdot \\
(t^{8} + 240 t^{7} + 2160 t^{6} + 6720 t^{5} + 17504 t^{4} + 26880 t^{3} + 34560 t^{2} + 15360 t + 256)^{3}
\,.
\end{tblr}
$$

\vskip 0.3truecm

\noindent{\bf Automorphisms.}
The subgroup of $\operatorname{Aut} X_0(16)$ that fixes the set of vertices of the graph is
isomorphic to the dihedral group of order $8$  with elements:

$$
t \mapsto
\pm t, \pm 4/t, \pm 2(t-2)/(t+2), \pm 2(t+2)/(t-2)\,.
$$




\newpage


\noindent{\bf Signatures.}
We can (and do) choose Weierstrass equations for the elliptic curves so that their signatures are:

\[
\begin{tblr}{|c|l|}
\hline \SetCell[c=2]{c} T_8 \text{ signatures}\\ \hline
c_4(E_1) & (t^{8} - 16 t^{4} + 16)  \\ 
c_6(E_1) & (t^{4} - 8) \cdot (t^{8} - 16 t^{4} - 8)  \\ 
\Delta(E_1) & (t - 2) \cdot (t + 2) \cdot t^{4} \cdot (t^{2} + 4)  \\  \hline
% --------------
c_4(E_2) & (t^{8} - 16 t^{4} + 256)  \\ 
c_6(E_2) & (t^{4} - 32) \cdot (t^{4} - 8) \cdot (t^{4} + 16)  \\ 
\Delta(E_2) & (t - 2)^{2} \cdot (t + 2)^{2} \cdot t^{8} \cdot (t^{2} + 4)^{2}  \\  \hline
% --------------
c_4(E_{21}) & (t^{8} - 256 t^{4} + 4096)  \\ 
c_6(E_{21}) & (t^{4} - 32) \cdot (t^{8} + 512 t^{4} - 8192)  \\ 
\Delta(E_{21}) & \left(-1\right) \cdot (t - 2) \cdot (t + 2) \cdot t^{16} \cdot (t^{2} + 4)  \\  \hline
% --------------
c_4(E_4) & (t^{4} - 4 t^{3} + 8 t^{2} + 16 t + 16) \cdot (t^{4} + 4 t^{3} + 8 t^{2} - 16 t + 16)  \\ 
c_6(E_4) & (t^{2} - 4 t - 4) \cdot (t^{2} + 4 t - 4) \cdot (t^{4} + 16) \cdot (t^{4} + 24 t^{2} + 16)  \\ 
\Delta(E_4) & (t - 2)^{4} \cdot t^{4} \cdot (t + 2)^{4} \cdot (t^{2} + 4)^{4}  \\  \hline
% --------------
c_4(E_{41}) & (t^{4} - 16 t^{3} + 8 t^{2} + 64 t + 16) \cdot (t^{4} + 16 t^{3} + 8 t^{2} - 64 t + 16)  \\ 
c_6(E_{41}) & (t^{2} - 4 t - 4) \cdot (t^{2} + 4 t - 4) \cdot (t^{8} + 528 t^{6} - 4000 t^{4} + 8448 t^{2} + 256)  \\ 
\Delta(E_{41}) & \left(-1\right) \cdot (t - 2)^{2} \cdot t^{2} \cdot (t + 2)^{2} \cdot (t^{2} + 4)^{8}  \\  \hline
% --------------
c_4(E_{8}) & (t^{8} + 240 t^{6} + 2144 t^{4} + 3840 t^{2} + 256)  \\ 
c_6(E_{8}) & (t^{4} - 24 t^{3} + 24 t^{2} - 96 t + 16) \cdot (t^{4} + 24 t^{2} + 16) \cdot (t^{4} + 24 t^{3} + 24 t^{2} + 96 t + 16)  \\ 
\Delta(E_{8}) & t^{2} \cdot (t - 2)^{8} \cdot (t + 2)^{8} \cdot (t^{2} + 4)^{2}  \\  \hline
% --------------
c_4(E_{81}) & (t^{8} - 240 t^{7} + 2160 t^{6} - 6720 t^{5} + 17504 t^{4} - 26880 t^{3} + 34560 t^{2} - 15360 t + 256)  \\ 
c_6(E_{81}) & (t^{4} - 24 t^{3} + 24 t^{2} - 96 t + 16) \cdot (t^{8} + 528 t^{7} - 3984 t^{6} + 14784 t^{5} - 31648 t^{4} + 59136 t^{3} - 63744 t^{2} + 33792 t + 256)  \\ 
\Delta(E_{81}) & \left(-1\right) \cdot t \cdot (t - 2)^{4} \cdot (t + 2)^{16} \cdot (t^{2} + 4)  \\  \hline
% --------------
c_4(E_{16}) & (t^{8} + 240 t^{7} + 2160 t^{6} + 6720 t^{5} + 17504 t^{4} + 26880 t^{3} + 34560 t^{2} + 15360 t + 256)  \\ 
c_6(E_{16}) & (t^{4} + 24 t^{3} + 24 t^{2} + 96 t + 16) \cdot (t^{8} - 528 t^{7} - 3984 t^{6} - 14784 t^{5} - 31648 t^{4} - 59136 t^{3} - 63744 t^{2} - 33792 t + 256)  \\ 
\Delta(E_{16}) & t \cdot (t + 2)^{4} \cdot (t - 2)^{16} \cdot (t^{2} + 4)  \\  \hline
% --------------
\end{tblr}
\]


\newpage

\noindent{\bf Action of Aut on the graph.}

$$
\begin{tblr}{l@{\,=\,}lcc}
 \hline
 \SetCell[c=2]{l} \text{automorphism }  & &
 \SetCell[c=1]{c} \text{permutation}  &  \text{order}  \\
 \hline
   \operatorname{id}(t) & t  &  (\,) & 1 \\
    \sigma(t) & 2 (t - 2)/(t + 2) & 
   (j_1,j_{81},j_{21},j_{16}) (j_{2},j_{8}) & 4 \\
   \sigma^2(t) & -4/t & (j_1, j_{21})(j_{81},j_{16}) & 2 \\
    \sigma^3(t) & -2 (t + 2)/(t - 2) &   (j_1,j_{16},j_{21},j_{81})(j_2,j_8)               & 4 \\
    \tau(t)  & -t  &  (j_{81},j_{16}) & 2 \\
    \sigma \tau(t) & 2 (t + 2)/(t - 2) & (j_1,j_{16})(j_2,j_8)(j_{21},j_{81}) & 2 \\
    \sigma^2 \tau(t)  & 4/t & (j_1, j_{21}) & 2 \\
  \sigma^3 \tau(t) & -2 (t - 2)/(t + 2) & 
   (j_1,j_{81})(j_2,j_{8}) (j_{21},j_{16}) & 2 \\ 
 \hline
\end{tblr}
$$

\vskip 0.5truecm

$$
\begin{tblr}{|l|c|}
 \hline
 \SetCell[c=2]{c} \text{Automorphism action on the graph}  &   \\
 \hline
\operatorname{id} & (\,) \\
\sigma & (E_1,E_{81},E_{21},E_{16})^{\otimes {-1}} (E_{2},E_{8})^{\otimes {-1}} (E_4)^{\otimes {-1}} (E_{41})^{\otimes {-1}} \\
\sigma^2 & (E_1,E_{21}) (E_{81},E_{16})  \\
\sigma^3 & (E_1,E_{16},E_{21},E_{81})^{\otimes {-1}} (E_{2},E_{8})^{\otimes {-1}} \\
 \tau & (E_{81},E_{16})  \\
\sigma \tau & (E_{1},E_{16})^{\otimes {-1}}
(E_{2},E_{8})^{\otimes {-1}}
(E_{21},E_{81})^{\otimes {-1}} (E_4)^{\otimes {-1}} (E_{41})^{\otimes {-1}} \\
 \sigma^2\tau & (E_{1},E_{21})  \\
 \sigma^3 \tau & (E_{1},E_{81})^{\otimes {-1}}
(E_{2},E_{8})^{\otimes {-1}}
(E_{21},E_{16})^{\otimes {-1}} (E_4)^{\otimes {-1}} (E_{41})^{\otimes {-1}} \\
\hline
\end{tblr}
$$




\newpage


\noindent{\bf Kodaira symbols.}

\begin{longtblr}
[caption = {$T_8$ data for $p\neq 2$}]
{cells = {mode=imath},hlines,vlines,measure=vbox,
hline{Z} = {1-5}{0pt},
vline{1} = {Y-Z}{0pt},
colspec  = cclclcc}
%--------------------------------------
\SetCell[c=1]{c} T_8 &\SetCell[c=6]{c} p\neq 2  & & & & & \\ t & E & 
\SetCell[c=1]{c} \operatorname{sig}_p(E) & u & \Kd_p(E) & \SetCell[c=2]{c} u_p(d)   \\
%--------------------------------------
\SetCell[r=8]{c}
    m=v_p(t)>0  
& E_1    & ( 0 , 0 , 4m ) & 1  &   \kI_{4m}   & 1 & 1  \\
& E_2    & ( 0 , 0 , 8m ) & 1  &   \kI_{8m}   & 1 & 1  \\
& E_{21} & ( 0 , 0 , 16m ) & 1  &  \kI_{16m}   & 1 & 1  \\
& E_4    & ( 0 , 0 , 4m ) & 1  &   \kI_{4m}   & 1 & 1  \\
& E_{41} & ( 0 , 0 , 2m ) & 1  &   \kI_{2m}   & 1 & 1  \\
& E_8    & ( 0 , 0 , 2m ) & 1  &   \kI_{2m}   & 1 & 1  \\
& E_{81} & ( 0 , 0 , m ) & 1  &   \kI_{m}   & 1 & 1  \\
& E_{16} & ( 0 , 0 , m ) & 1  &   \kI_{m}   & 1 & 1  \\
%--------------------------------------
\SetCell[r=8]{c}
    \begin{array}{c}
    v_p(t)=0 \\[3pt]
    m = v_p(t-2) >0
    \end{array}
& E_1    & ( 0 , 0 , m ) & 1  &   \kI_m   & 1 & 1  \\
& E_2    & ( 0 , 0 , 2m ) & 1  &  \kI_{2m}   & 1 & 1  \\
& E_{21} & ( 0 , 0 , m ) & 1  &  \kI_m   & 1 & 1  \\
& E_4    & ( 0 , 0 , 4m ) & 1  &  \kI_{4m}   & 1 & 1  \\
& E_{41} & ( 0 , 0 , 2m ) & 1  &  \kI_{2m}   & 1 & 1  \\
& E_8    & ( 0 , 0 , 8m ) & 1  &  \kI_{8m}   & 1 & 1  \\
& E_{81} & ( 0 , 0 , 4m ) & 1  &  \kI_{4m}   & 1 & 1 \\
& E_{16} & ( 0 , 0 , 16m ) & 1  &  \kI_{16m}   & 1 & 1  \\
%--------------------------------------
\SetCell[r=8]{c}
    \begin{array}{c}
    v_p(t)=0 \\[3pt]
    m = v_p(t+2) >0
    \end{array}
& E_1    & ( 0 , 0 , m ) & 1  &   \kI_m   & 1 & 1  \\
& E_2    & ( 0 , 0 , 2m ) & 1  &  \kI_{2m}   & 1 & 1  \\
& E_{21} & ( 0 , 0 , m ) & 1  &  \kI_m   & 1 & 1  \\
& E_4    & ( 0 , 0 , 4m ) & 1  &  \kI_{4m}   & 1 & 1  \\
& E_{41} & ( 0 , 0 , 2m ) & 1  &  \kI_{2m}   & 1 & 1  \\
& E_8    & ( 0 , 0 , 8m ) & 1  &  \kI_{8m}   & 1 & 1  \\
& E_{81} & ( 0 , 0 , 16m ) & 1  &  \kI_{16m}   & 1 & 1 \\
& E_{16} & ( 0 , 0 , 4m ) & 1  &  \kI_{4m}   & 1 & 1  \\
%--------------------------------------
\SetCell[r=8]{c}
    \begin{array}{c}
    v_p(t)=0 \\[3pt]
    m = v_p(t^2+4) >0
    \end{array}
& E_1    & ( 0 , 0 , m ) & 1  &   \kI_m   & 1 & 1  \\
& E_2    & ( 0 , 0 , 2m ) & 1  &  \kI_{2m}   & 1 & 1  \\
& E_{21} & ( 0 , 0 , m ) & 1  &  \kI_m   & 1 & 1 \\
& E_4    & ( 0 , 0 , 4m ) & 1  &  \kI_{4m}   & 1 & 1  \\
& E_{41} & ( 0 , 0 , 8m ) & 1  &  \kI_{8m}   & 1 & 1  \\
& E_8    & ( 0 , 0 , 2m ) & 1  &  \kI_{2m}   & 1 & 1  \\
& E_{81} & ( 0 , 0 , m ) & 1  &  \kI_{m}   & 1 & 1  \\
& E_{16} & ( 0 , 0 , m ) & 1  &  \kI_{m}   & 1 & 1  \\
%--------------------------------------
\SetCell[r=8]{c}
    m=v_p(t)<0  
& E_1    & ( 0 , 0 , 16m ) & 1  &   \kI_{16m}   & 1 & 1  \\
& E_2    & ( 0 , 0 , 8m ) & 1  &   \kI_{8m}   & 1 & 1  \\
& E_{21} & ( 0 , 0 , 4m ) & 1  &  \kI_{4m}   & 1 & 1  \\
& E_4    & ( 0 , 0 , 4m ) & 1  &   \kI_{4m}   & 1 & 1  \\
& E_{41} & ( 0 , 0 , 2m ) & 1  &   \kI_{2m}   & 1 & 1  \\
& E_8    & ( 0 , 0 , 2m ) & 1  &   \kI_{2m}   & 1 & 1  \\
& E_{81} & ( 0 , 0 , m ) & 1  &   \kI_{m}   & 1 & 1  \\
& E_{16} & ( 0 , 0 , m ) & 1  &   \kI_{m}   & 1 & 1  \\
%--------------------------------------
\SetCell[c=5,r=2]{c} & & & & & d\equiv 0  & d\not\equiv 0 \\
                      & & & & & \SetCell[c=2]{c} d \Mod p & \\
\end{longtblr}



\newpage


\begin{longtblr}
[caption = {$T_8$ data for $p$=2}]
{cells = {mode=imath},hlines,vlines,measure=vbox,
hline{Z} = {1-5}{0pt},
vline{1} = {Y-Z}{0pt},
colspec  = cclclccc}
%--------------------------------------
\SetCell[c=1]{c} T_8 &\SetCell[c=7]{c} p=2  & & & &  & & \\
\SetCell[c=1]{c} t & E & 
\SetCell[c=1]{c}\operatorname{sig}_2(E) & u & \Kd_2(E) & \SetCell[c=3]{c} u_2(d) & & \\
%--------------------------------------
\SetCell[r=8]{c}
    m=v_2(t)>2  
& E_1    & ( 4 , 6 , 4m+4 ) & 1  &   \kI_{4m-4}^*   & 1 & 1 & 
\begin{array}{cl} 
2^* & \text{if $m\ge 2$}\\
1^* & \text{if $m<2$}
\end{array} \\
& E_2    & ( 4 , 6 , 8m-4 ) & 2  &   \kI_{8m-12}^*   & 1 & 1 & \begin{array}{cl} 
2^* & \text{if $m\ge 2$}\\
1^* & \text{if $m<2$}
\end{array} \\
& E_{21} & ( 4 , 6 , 16m-20 ) & 2^2  &   \kI_{16m-28}^*   & 1 & 1 & \begin{array}{cl} 
2^* & \text{if $m\ge 2$}\\
1^* & \text{if $m<2$}
\end{array} \\
& E_4    & ( 4 , 6 , 4m+4 ) & 2  &   \kI_{4m-4}^*   & 1 & 1 & \begin{array}{cl} 
2^* & \text{if $m\ge 2$}\\
1^* & \text{if $m<2$}
\end{array} \\
& E_{41} & ( 4 , 6 , 2m+8 ) & 2  &   \kI_{2m}^*   & 1 & 1 & \begin{array}{cl} 
2^* & \text{if $m\ge 2$}\\
1^* & \text{if $m<2$}
\end{array} \\
& E_{8} & ( 4 , 6 , 2m+8 ) & 2  &   \kI_{2m}^*   & 1 & 1 & \begin{array}{cl} 
2^* & \text{if $m\ge 2$}\\
1^* & \text{if $m<2$}
\end{array} \\
& E_{81} & ( 4 , 6 , m+10 ) & 2  &   \kI_{m+2}^*   & 1 & 1 & \begin{array}{cl} 
2^* & \text{if $m\ge 2$}\\
1^* & \text{if $m<2$}
\end{array} \\
& E_{16} & ( 4 , 6 , m+10 ) & 2  &   \kI_{m+2}^*   & 1 & 1 & \begin{array}{cl} 
2^* & \text{if $m\ge 2$}\\
1^* & \text{if $m<2$}
\end{array} \\
%--------------------------------------
\SetCell[r=8]{c}
    v_2(t)=2  
& E_1    & ( 4 , 6 , 12 ) & 1  &   \kI_4^*   & 1 & 1 & 2 \\
& E_2    & ( 4 , 6 , 12 ) & 2  &   \kI_4^*   & 1 & 1 & 2 \\
& E_{21} & ( 4 , 6 , 12 ) & 4  &   \kI_4^*   & 1 & 1 & 2 \\
& E_4    & ( 4 , 6 , 12 ) & 2  &   \kI_4^*   & 1 & 1 & 2 \\
& E_{41} & ( 4 , 6 , 12 ) & 2  &   \kI_4^*   & 1 & 1 & 2 \\
& E_8    & ( 4 , 6 , 12 ) & 2  &   \kI_4^*   & 1 & 1 & 2 \\
& E_{81} & ( 4 , 6 , 12 ) & 2  &   \kI_4^*   & 1 & 1 & 2 \\
& E_{16} & ( 4 , 6 , 12 ) & 2  &   \kI_4^*   & 1 & 1 & 2 \\
%--------------------------------------
\SetCell[r=8]{c}
\begin{array}{c}
     v_2(t)=1   \\[3pt]
     t/2\equiv 1 \, (4)  \\[3pt]
    m = v_2(t^2-4)  \\[3pt]
    n = v_2(t^2+4)  
\end{array}
& E_1    & ( 0 , 0 , m+n-8 ) & 2  &   \kI_{m-6}   & 1 & 2^{-1} & 2^{-1} \\
& E_2    & ( 0 , 0 , 2(m+n-8) ) & 4  &   \kI_{2(m+n)-16}   & 1 & 2^{-1} & 2^{-1} \\
& E_{21} & ( 0 , 0 , m+n-8 ) & 4  &   \kI_{m+n-8}   & 1 & 2^{-1} & 2^{-1} \\
& E_4    & ( 0 , 0 , 4(m+n-8) ) & 8  &   \kI_{4(m+n-8)}   & 1 & 2^{-1} & 2^{-1} \\
& E_{41} & ( 0 , 0 , 2(m+n-8) ) & 8  &   \kI_{2(m+n-8)}   & 1 & 2^{-1} & 2^{-1} \\
& E_8    & ( 0 , 0 , 8(m+n-8) ) & 16  &   \kI_{8(m+n-8)}   & 1 & 2^{-1} & 2^{-1} \\
& E_{81} & ( 0 , 0 , 4(m+n-8) ) & 16  &   \kI_{4(m+n-8)}   & 1 & 2^{-1} & 2^{-1} \\
& E_{16} & ( 0 , 0 , 16(m+n-8) ) & 32  &   \kI_{16(m+n-8)}   & 1 & 2^{-1} & 2^{-1} \\
%--------------------------------------
\SetCell[r=8]{c}
\begin{array}{c}
     v_2(t)=1   \\[3pt]
    t/2\equiv 3 \, (4)  \\[3pt]
    m = v_2(t^2-4)   \\[3pt]
    n = v_2(t^2+4)    
\end{array}
& E_1    & ( 0 , 0 , m+n-8 ) & 2  &   \kI_{m-6}   & 1 & 2^{-1} & 2^{-1} \\
& E_2    & ( 0 , 0 , 2(m+n-8) ) & 4  &   \kI_{2(m+n)-16}   & 1 & 2^{-1} & 2^{-1} \\
& E_{21} & ( 0 , 0 , m+n-8 ) & 4  &   \kI_{m+n-8}   & 1 & 2^{-1} & 2^{-1} \\
& E_4    & ( 0 , 0 , 4(m+n-8) ) & 8  &   \kI_{4(m+n-8)}   & 1 & 2^{-1} & 2^{-1} \\
& E_{41} & ( 0 , 0 , 2(m+n-8) ) & 8  &   \kI_{2(m+n-8)}   & 1 & 2^{-1} & 2^{-1} \\
& E_8    & ( 0 , 0 , 8(m+n-8) ) & 16  &   \kI_{8(m+n-8)}   & 1 & 2^{-1} & 2^{-1} \\
& E_{81} & ( 0 , 0 , 16(m+n-8) ) & 32  &   \kI_{16(m+n-8)}   & 1 & 2^{-1} & 2^{-1} \\
& E_{16} & ( 0 , 0 , 4(m+n-8) ) & 16  &   \kI_{4(m+n-8)}   & 1 & 2^{-1} & 2^{-1} \\
%--------------------------------------
%--------------------------------------
\SetCell[r=8]{c}
    v_2(t)=0  
& E_1    & ( 4 , 6 , 12 ) & 1/2  &   \kI_4^*   & 1 & 1 & 2 \\
& E_2    & ( 4 , 6 , 12 ) & 1/2  &   \kI_4^*   & 1 & 1 & 2 \\
& E_{21} & ( 4 , 6 , 12 ) & 1/2  &   \kI_4^*   & 1 & 1 & 2 \\
& E_4    & ( 4 , 6 , 12 ) & 1/2  &   \kI_4^*   & 1 & 1 & 2 \\
& E_{41} & ( 4 , 6 , 12 ) & 1/2  &   \kI_4^*   & 1 & 1 & 2 \\
& E_8    & ( 4 , 6 , 12 ) & 1/2  &   \kI_4^*   & 1 & 1 & 2 \\
& E_{81} & ( 4 , 6 , 12 ) & 1/2  &   \kI_4^*   & 1 & 1 & 2 \\
& E_{16} & ( 4 , 6 , 12 ) & 1/2  &   \kI_4^*   & 1 & 1 & 2 \\
%--------------------------------------
\SetCell[r=8]{c}
    v_p(t)=-1 
& E_1    & ( 4 , 6 , 28 ) & 2^{-3}  &   \kI_{20}^*   & 1 & 1 & 2 \\
& E_2    & ( 4 , 6 , 20 ) & 2^{-3}  &   \kI_{12}^*   & 1 & 1 & 2 \\
& E_{21} & ( 4 , 6 , 16 ) & 2^{-3}  &   \kI_{8}^*   & 1 & 1 & 2 \\
& E_4    & ( 4 , 6 , 16 ) & 2^{-3}  &   \kI_{8}^*   & 1 & 1 & 2 \\
& E_{41} & ( 4 , 6 , 14 ) & 2^{-3}  &   \kI_{6}^*   & 1 & 1 & 2 \\
& E_8    & ( 4 , 6 , 14 ) & 2^{-3}  &   \kI_{6}^*   & 1 & 1 & 2 \\
& E_{81} & ( 4 , 6 , 13 ) & 2^{-3}  &   \kI_{5}^*   & 1 & 1 & 2 \\
& E_{16} & ( 4 , 6 , 13 ) & 2^{-3}  &   \kI_{5}^*   & 1 & 1 & 2 \\
%--------------------------------------
\SetCell[r=8]{c}
    -m= v_2(t)<-1
& E_1    & ( 4 , 6 , 12+16 m ) & 2^{-2m-1}  &   \kI_{4+16 m}^*   & 1 & 1 & 2 \\
& E_2    & ( 4 , 6 , 12+8 m ) & 2^{-2m-1}  &   \kI_{4+8 m}^*   & 1& 1 & 2 \\
& E_{21} & ( 4 , 6 , 12+4m ) & 2^{-2m-1}  &   \kI_{4+4 m}^*   & 1& 1 & 2 \\
& E_4    & ( 4 , 6 , 12+4m ) & 2^{-2m-1}  &  \kI_{4+4 m}^*   & 1& 1 & 2 \\
& E_{41} & ( 4 , 6 , 12+2m ) & 2^{-2m-1}  &   \kI_{4+2 m}^*   & 1 & 1 & 2 \\
& E_8    & ( 4 , 6 , 12+2m ) & 2^{-2m-1}  &   \kI_{4+2 m}^*   & 1 & 1 & 2 \\
& E_{81} & ( 4 , 6 , 12+m ) & 2^{-2m-1}  &   \kI_{4+ m}^*   & 1 & 1 & 2 \\
& E_{16} & ( 4 , 6 , 12+2m ) & 2^{-2m-1}  &   \kI_{4+ m}^*   & 1 & 1 & 2 \\
%--------------------------------------
 \SetCell[c=5,r=2]{c} & & & & &  d\equiv 1 &  d\equiv 2  & d\equiv 3 \\
                      & & & & & \SetCell[c=3]{c} d \Mod{4} & \\
\end{longtblr}


\newpage
\section{Conclusion}

\begin{prop}
Let 
\[ \begin{tikzcd}
& E_{21} \ar[dash,d,"2"] & & E_{81} \ar[dash,d,"2"] & \\
& E_{2} \ar[dash,swap,dl,"2"] \ar[dash,dr,"2"] & & E_{8} \ar[dash,dl,swap,"2"] \ar[dash,dr,"2"] & \\
 E_1 &  & E_4 \ar[dash,d,"2"] & & E_{16}  \\
 &  & E_{41} & &    \,.
\end{tikzcd}
\]
be a $\mathbf{Q}$-isogeny graph of type $T_8$ corresponding to a given $t$ in $\mathbf{Q}$, $t\ne 0,\pm 2$. For every square-free integer $d$, 
the probability of a vertex
to be the Faltings curve (circled)
in the twisted isogeny graph 
\[ \begin{tikzcd}
& E^d_{21} \ar[dash,d,"2"] & & E^d_{81} \ar[dash,d,"2"] & \\
& E^d_{2} \ar[dash,swap,dl,"2"] \ar[dash,dr,"2"] & & E^d_{8} \ar[dash,dl,swap,"2"] \ar[dash,dr,"2"] & \\
 E^d_1 &  & E^d_4 \ar[dash,d,"2"] & & E^d_{16}  \\
 &  & E^d_{41} & &    \,.
\end{tikzcd}
\]
is given by:

\newpage

\begin{longtblr}
[caption = ]
{cells = {mode=imath},hlines,vlines,measure=vbox,colspec=cccc}
%--------------------------------------
 T_8 & \text{twisted isogeny graph} & \text{Prob} \\
%--------------------------------------
v_2(t)\ge 2 &
\scalebox{.6}{
        \begin{tikzcd}[ampersand replacement=\&]
\&  \circled[0.8]{$E_{21}$} \ar[d] \& \& E_{81}  \& \\
\& E_{2} \ar[dl] \ar[dr] \& \& E_{8} \ar[dr]\ar[u] \& \\
 E_1 \&  \& E_4 \ar[d]  \ar[ur]\& \& E_{16}  \\
 \&  \& E_{41} \& \&   
\end{tikzcd}
} & 1 \\
%--------------------------------------
\begin{array}{c}
v_2(t)=1\\
t/2^2\equiv 3\,(4)
\end{array}
&
\scalebox{.6}{
        \begin{tikzcd}[ampersand replacement=\&]
\& E_{21}  \& \& \circled[0.8]{$E_{81}$} \ar[d] \& \\
\& E_{2} \ar[u]\ar[dl] \& \&  E_{8} \ar[dl] \ar[dr] \& \\
 E_1 \&  \& \ar[ul] E_4 \ar[d] \& \& E_{16}  \\
 \&  \& E_{41} \& \&   
\end{tikzcd}
} & 1 \\
%--------------------------------------
\begin{array}{c}
v_2(t)=1\\
t/2^2\equiv 1\,(4)
\end{array} &
\scalebox{.6}{
        \begin{tikzcd}[ampersand replacement=\&]
\& E_{21} \ar[d] \& \& E_{81} \ar[d] \& \\
\& E_{2} \ar[dl] \ar[dr] \& \& E_{8} \ar[dl] \ar[dr] \& \\
 E_1 \&  \& E_4 \ar[d] \& \&  \circled[0.8]{$E_{16}$}  \\
 \&  \& E_{41} \& \&   
\end{tikzcd}
} & 1 \\
%--------------------------------------
v_2(t)\le 0 &
\scalebox{.6}{
        \begin{tikzcd}[ampersand replacement=\&]
\& E_{21} \ar[d] \& \& E_{81} \ar[d] \& \\
\& E_{2} \ar[dl] \ar[dr] \& \& E_{8} \ar[dl] \ar[dr] \& \\
  \circled[0.8]{$E_1$} \&  \& E_4 \ar[d] \& \& E_{16}  \\
 \&  \& E_{41} \& \&  
\end{tikzcd}
} & 1 \\
%--------------------------------------
\end{longtblr}
\end{prop}



\vskip 0.35truecm

\noindent{\it Proof.} From the previous tables one gets:

\vskip 0.5truecm


\begin{tblr}{cells={mode=imath},hlines,vlines,measure=vbox}
%-------------------------------------------------
\SetCell[c=1]{c} t &\SetCell[c=1]{c} [u(E)]  & \SetCell[c=1]{c} [u(E)(d)]  & \SetCell[c=1]{c}\text{Prob}\\
%-------------------------------------------------
\SetCell[r=1]{c} v_2(t)\ge 2 & \SetCell[r=1]{c} (1:2:2^2:2:2:2:2:2) & (1:1:1:1:1:1:1:1) &   \SetCell[r=1]{c} (0,0,1,0,0,0,0,0)\\
%-------------------------------------------------
\SetCell[r=1]{c} \begin{array}{c}
v_2(t)=1\\
t/2\equiv 3\,(4)
\end{array}  & \SetCell[r=1]{c} (1:2:2:2^2:2^2:2^3:2^4:2^3) & (1:1:1:1:1:1:1:1) &   \SetCell[r=1]{c} (0,0,0,0,0,0,1,0)\\
%-------------------------------------------------
\SetCell[r=1]{c} \begin{array}{c}
v_2(t)=1\\
t/2\equiv 1\,(4)
\end{array}  & \SetCell[r=1]{c} (1:2:2:2^2:2^2:2^3:2^3:2^4) & (1:1:1:1:1:1:1:1) &   \SetCell[r=1]{c} (0,0,0,0,0,0,0,1)\\
%-------------------------------------------------
\SetCell[r=1]{c} v_2(t)\le 0 & \SetCell[r=1]{c} (1:1:1:1:1:1:1:1) & (1:1:1:1:1:1:1:1) &   \SetCell[r=1]{c} (1,0,0,0,0,0,0,0)\\
%-------------------------------------------------
\end{tblr}

\vskip 1.8truecm





\end{document}