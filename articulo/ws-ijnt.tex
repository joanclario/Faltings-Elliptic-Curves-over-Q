%%%%%%%%%%%%%%%%%%%%%%%%%%%%%%%%%%%%%%%%%%%%%%%%%%%%%%%%%%%%%%%%%%%%%%%%%%%%
%% Trim Size: 9.75in x 6.5in
%% Text Area: 8in (include Runningheads) x 5in
%% ws-ijnt.tex   :   10-10-2007
%% Tex file to use with ws-ijnt.cls written in Latex2E.
%% The content, structure, format and layout of this style file is the
%% property of World Scientific Publishing Co. Pte. Ltd.
%% Copyright 1995, 2002 by World Scientific Publishing Co.
%% All rights are reserved.
%%%%%%%%%%%%%%%%%%%%%%%%%%%%%%%%%%%%%%%%%%%%%%%%%%%%%%%%%%%%%%%%%%%%%%%%%%%%

\documentclass{ws-ijnt}

% ------------------------------------------------------------------
%\usepackage{amsfonts,amssymb,amsmath,amsthm,latexsym,graphics,epsfig,amsfonts}
%\usepackage{verbatim,enumerate,array,booktabs,color,bigstrut,prettyref,tikz-cd}
%\usepackage{multirow}
%\usepackage[backref]{hyperref}
%\usepackage[OT2,T1]{fontenc}
% ------------------------------------------------------------------
\usepackage[all]{xy}
\usepackage[all]{tikz-cd}
\usepackage{mathtools}
\usepackage{tabularray}
\UseTblrLibrary{amsmath,varwidth}
\usepackage{tabularx}
\usepackage{longtable}
\usepackage{arydshln}
% ------------------------------------------------------------------
\newcommand{\Mod}[1]{\ (\mathrm{mod}\ #1)}
\newcommand{\mathdash}{\relbar\mkern-8mu\relbar}
\newcommand*\circled[2][1.6]{\tikz[baseline=(char.base)]{
    \node[shape=circle, draw, inner sep=1pt, 
        minimum height={\f@size*#1},] (char) {\vphantom{WAH1g}#2};}}
\makeatother
\newcommand\myiso{\stackrel{\mathclap{\normalfont\mbox{\small $p$}}}{-}}
\newcommand\myisot{\stackrel{\mathclap{\normalfont\mbox{\small $3$}}}{-}}
\addtolength{\textwidth}{4cm} \addtolength{\hoffset}{-2cm}
\addtolength{\marginparwidth}{-2cm}
\newcommand{\mysquare}[1]{\tikz{\path[draw] (0,0) rectangle node{\tiny #1} (8pt,8pt) ;}}
\newcommand{\mycircle}[1]{\tikz{\path[draw] (0,0) circle (4pt) node{\tiny #1};}}
%------------------------------------
\newcommand{\Kd}{\operatorname{K}}
\newcommand{\kI}{\operatorname{I}}
\newcommand{\kII}{\operatorname{II}}
\newcommand{\kIII}{\operatorname{III}}
\newcommand{\kIV}{\operatorname{IV}}
\newcommand{\Q}{\mathbb Q}
\newcommand{\Qbar}{\overline{\Q}}
\newcommand{\Z}{\mathbb Z}
\newcommand{\modQ}{\,\text{mod}\,(\Q^)^2}
%-------------------------------------

\begin{document}

\markboth{Enrique González-Jiménez and Joan-C. Lario}
{Faltings elliptic curves in twisted $\Q$-isogeny classes}

%%%%%%%%%%%%%%%%%%%%% Publisher's Area please ignore %%%%%%%%%%%%%%%
%
\catchline{}{}{}{}{}
%
%%%%%%%%%%%%%%%%%%%%%%%%%%%%%%%%%%%%%%%%%%%%%%%%%%%%%%%%%%%%%%%%%%%%

\title{
Faltings elliptic curves in twisted $\mathbf{Q}$-isogeny classes
%\footnote{No footnte so far.}
}

\author{Enrique González-Jiménez
% \footnote{Madrid}
}

\address{
University Department, University Name, Address\\
City, State ZIP/Zone,Country\,
%\footnote{State completely without
%abbreviations, the affiliation and mailing address, %including country.
%Typeset in 8 pt italic.}
\\
\email{author\_id@domain\_name
%\footnote{Typeset author's e-mail address in 8pt italic.}
}}

\author{Joan-C. Lario}

\address{Departament de Matemàtiques, 
Univrersitat Politècnica de Catalunya \\
Barcelona, Catalunya\\
joan.carles.lario@upc.edu }

\maketitle

\begin{history}
\received{(Day Month Year)}
\accepted{(Day Month Year)}
\end{history}

\begin{abstract}
Let $G$ be the isogeny graph attached an elliptic curve over $\Q$: a vertex for every elliptic curve defined over $\Q$ in the isogeny class, and edges in correspondence with the  prime degree rational isogenies 
between them. Stevens shows that there is a unique elliptic curve in $G$ with minimal Faltings height. 
We call  this curve the Faltings elliptic curve in $G$.

For every  square-free integer $d$, we consider the graph~$G^d$
attached to the twisted elliptic curves in $G$ 
by the quadratic character of $\Q(\sqrt{d})$.
It turns out that 
$G$ and $G^d$ are canonically isomorphic as abstract graphs
(the isomorphism identifies the vertices with equal $j$-invariant.)
In this paper we discuss the probability distribution of a vertex to be the Faltings elliptic curve in $G^d$ as $|d|$ grows to infinity. In many cases, this probability depends on the $p$-adic valuations of rational values of modular functions.
\end{abstract}

\keywords{Elliptic curves; modular curves; Faltings height.}

\ccode{Mathematics Subject Classification 2010: 11G05, 11G07, 14H52}

%-----------------------------

\section{Introduction}	

Every isogeny class of elliptic curves over $\mathbf{Q}$ has a distinguished elliptic curve.
First, Mazur and Swinnerton-Dyer \cite{MS} proposed the so-called {\it strong} curve which is an optimal quotient of the Jacobian of the modular curve $X_0(M)$, where $M$ is the conductor of the isogeny class.
Later, Stevens \cite{S} suggested that it is better to consider the elliptic curve which is an optimal quotient of the Jacobian of the modular curve $X_1(M)$. In both cases the Manin constant plays a role, and the Stevens proposal seems to be more intrinsically arithmetic due to the intervention of Néron models, étale isogenies, and Faltings heights. We define the Faltings elliptic curve as the one with minimal Faltings height in the isogeny class.

For an elliptic curve $E$ over $\mathbf{Q}$, we shall denote by $\mathcal{E}$ its $\mathbf{Q}$-isogeny class, and by $G=G({\mathcal E})$ the isogeny graph attached to it,  labeled by the prime degrees of the isogenies.
There are 26 possible types of labeled isogeny graphs associated to elliptic curves defined over $\mathbf{Q}$ that are classified in the following types:
$$
\begin{tblr}{l}
L_1 \\
L_2(p) 
\text{ for $p$ in $\{ 2,3,5,7,11,13,17,19,37,43,67,163\}$} \\
L_3(9), L_3(25), L_4, T_4, T_6, T_8 \\
R_4(N) \text{ for $N$ in $\{ 6,10,14,15,21\}$} \\
R_6, S_8\,.
\end{tblr}
$$
The subindexs denote the number of vertexs, the letter indicates the shape of the isogeny graph, and the level relates to the maximal isogeny degree of a path 
in the graph (abscence of level means isogenies of degree $2$ or~$3$).

To get this classification, Mazur and Goldfeld dealt with the case of rational isogenies of elliptic curves defined over~$\mathbf{Q}$ of prime degree \cite{M}. The complete classification of the degrees of rational isogenies of elliptic curves defined over~$\mathbf{Q}$, for prime or composite degree, was completed due to work of Fricke, Kenku, Klein, Kubert, Ligozat, Mazur and Ogg, among others. In particular, the work of
Kenku shows that there are at most 8 elliptic curves in each isogeny class over $\mathbf{Q}$. We refer to Chiloyan--Lozano-Robledo \cite{C} for more details on the classification and its proof.

For every square-free integer $d$, we denote by $E^d$ the quadratic twist of $E$ by $\mathbf{Q}(\sqrt{d})$, and also by $G^d$ the graph attached to the $\mathbf{Q}$-isogeny class of $E^d$.
It tuns out that $G$ and $G^d$ are canonically isomorphic as labeled graphs (the isomorphism identifies the vertices with equal $j$-invariant.) Our aim is to discuss the probability distribution of a vertex to be the Faltings elliptic curve in $G^d$ as $|d|$ grows up to infinity. 
As we shall see, in many cases, this probability depends on the $p$-valuations of rational values of modular functions.

The plan of the paper is as follows. In Section 2 we fix the terminology, review some general facts, and describe the strategy. In Sections 3 and 4 we present the results for the $\mathbf{Q}$-isogeny graphs $G({\mathcal E})$ of types $L_3(9)$ and $T_4$, respectively, with detailed proofs. Section 5 contains the results for the remaining genus zero cases of isogeny graphs without including the proofs since they can be obtained {\it mutatis mutandis}. We refer to github BLA-BLA for more details and related information for these cases. Finally, Section~6 is devoted to the sporadic cases arising from modular curves of genus greater than zero with non-cuspidal rational points.

%-----------------------------

\section{Terminology, general facts, and strategy}

Let $E/\mathbf{Q}$ be an elliptic curve, given by a Weierstrass equation
\begin{equation}
E\colon y^2+a_{1}x+a_{3}=x^3+a_{2}x+a_{4}x+a_{6}\,.
\label{Weiers_equation}
\end{equation}
Throughout we shall make use of the usual {\it formulaire} for the constants $a_i$, $b_i$, $c_i$ attached to $E$ (see \cite{Sil}, Chapter III.1). In particular, we shall consider:  

\begin{equation}
\begin{tblr}{ll}
\Delta & \text{discriminant of $E$} \\
j & \text{$j$-invariant of $E$} \\
\omega=
\displaystyle{\frac{dx}{2y+a_1x+a_3}}
& \text{regular differential of $E$}\\
\Lambda = \lambda 
\langle 1, \tau \rangle 
& \text{period lattice of $E$, with $\operatorname{Im}(\tau)>0$.}
\end{tblr}
\label{Weiers}
\end{equation}
Another Weierstrass equation for $E$ can be obtained by the change:
\begin{equation}
\begin{tblr}{l@{\,=\,}l}
x & u^2 x'+ r \\
y & u^3y'+u^2sx'+t
\end{tblr}
\label{Weiers_change}
\end{equation}
with $u,r,s,t$ in $\mathbf{Q}$, $u\neq 0$. 
The new Weierstrass equation satisfies:
\begin{equation}
u^{12}\Delta' = \Delta\,,\qquad
j' = j
\,,\qquad
u^4 c_4' = c_4
\,,\qquad
u^6 c_6' = c_6
\,,\qquad
u^{-1} \omega' = \omega
\,,\qquad
u^{-1} \Lambda' = \Lambda \,.
\label{formula_change}
\end{equation}
The Tate's algorithm produces a Weierstrass change giving rise to a global minimal equation of $E$. 
This global minimal equation has integer coefficients~$a_i$ and minimal discriminant $\Delta_{\operatorname{min}}$. Also, it describes the special fibers of the Néron model of $E$ at every prime $p$ using the so-called $p$-Kodaira symbols:
$\kI_0$, $\kI_\nu$, $\kII$, $\kIII$, $\kIV$, $\kI_0^*$, $\kI_\nu^*$, $\kIV^*$, $\kIII^*$, $\kII^*$. We shall make use of the Faltings height of the elliptic curve:
\begin{equation}
h(E) = -\frac{1}{2} \log (
\operatorname{vol}(\Lambda_{\operatorname{min}}))
\label{Faltings}
\end{equation}
where $\Lambda_{\operatorname{min}}$ denotes the Néron lattice attached to $E$; that is, the period lattice attached to a global minimal Weierstrass equation of $E$.

Given  an isogeny $\varphi\colon  E_1\longrightarrow
E_2$ between two  elliptic curves defined over the rationals 
 \begin{equation}   
E_i\colon y^2+a_{1,i}x+a_{3,i}=x^3+a_{2,i}x+a_{4,i}x+a_{6,i}
\label{Weiers_isog}
\end{equation}
     one has
\begin{equation}
\varphi^*(dx/(2y+a_{1,2}x+a_{3,2})) = c\, \cdot
dx/(2y+a_{1,1}x+a_{3,1})\,,
\label{c}
\end{equation}
where the constant $c$ (in the base field $\mathbf{Q}$) is called the scaling factor of the isogeny. 
An isogeny is said to be normalized if $c=1$. In what follows, we shall be using the following well known facts:

\vskip 0.3truecm

\noindent (1)
If $\varphi\colon  E_1\longrightarrow
E_2$ denotes a prime degree isogeny  between elliptic curves given by global minimal equations, then either  $\varphi$ 
or its dual isogeny $\hat\varphi$ is normalized (and the other is not.) \textcolor{red}{Is this true????}

\vskip 0.3truecm

\noindent (2) In every $\mathbf{Q}$-isogeny class, there is a unique elliptic curve with minimal Faltings height. WE shall call this curve the Faltings elliptic curve in the isogeny class.

\vskip 0.3truecm

\noindent (3) If $\varphi\colon  E_1\longrightarrow
E_2$ denotes a prime degree isogeny between elliptic curves with period lattices 
$\Lambda_1=\lambda\langle 1, \tau \rangle$ and 
$\Lambda_2=\lambda\langle 1, p\tau \rangle$, then
$\varphi^*(\omega_2)= p\,\omega_1$. 

\vskip 0.3truecm

\noindent (4) If $\varphi\colon  E_1\longrightarrow
E_2$ denotes a prime degree isogeny between elliptic curves with period lattices 
$\Lambda_1=\lambda\langle 1, \tau \rangle$ and 
$\Lambda_2=\lambda\langle 1, (\tau+k)/p \rangle$ with $k$ any integer, then
$\varphi^*(\omega_2)= \omega_1$ (\textcolor{red}{or $\varphi^*(\omega_2)= 1/p \omega_1$??????}). 

\vskip 0.3truecm

We refer to \cite{DD} for (1), to \cite{S} for (2), and (3) and (4) follow from an easy computation. 
When we need to make emphasis, we shall mark normalized isogenies with arrows. In such cases, the labeled isogeny graphs $G$ will be considered also as oriented graphs. In other words, 
$E \mathdash E'$ will denote an isogeny which is not necessarily normalized, and 
$E \longrightarrow E'$ means that the isogeny is normalized.

Our strategy to deal with the 26 possible isogeny graphs is a follows: first we distinguish between infinite and finite types, depending on the cardinality of the set of 
rational points on the corresponding modular curve that parameterizes the graphs of that type. For the infinite cases, we parameterize the isogeny graphs $G(\mathcal{E})$ in terms of pairs $(t,d)$, where $t$ runs the rational values of a Hauptmodul for the specific modular curve and $d$ runs over the square-free integers. 

First of all, we fix signatures $(c_4,c_6,\Delta)$ for every vertex $E$ depending on the rational value $t$ and in such a way that the corresponding isogenies are normalized. This can be accomplished by means of the Vélu procedure~\cite{V}. Let $\Lambda_E$ denote the lattice attached to every elliptic curve $E$ in the isogeny class $\mathcal{E}$. Then, in terms of the $p$-adic valuations of $t$ we are able to compute the vector of rational scaling factors 
\begin{equation}
{\bf u} = [ u_E \colon E \in \mathcal{E}]
\label{u}
\end{equation}
that it produces he different Néron lattices of each elliptic curve in the isogeny class $\mathcal{E}$.
To this end, we make an extensive use of the version of Tate's algorithm given in \cite{Papa}. 
At this point, $u_E \Lambda_E$ is the Néron lattice attached to $E$. In view of our ultimate purposes, we only need to consider the vector ${\bf u}$ as a projective point. 

Next, for every square-free integer $d$, we 
need also to compute the vector of rationals
\begin{equation}
{\bf u}(d) = [ u_E(d) \colon E \in \mathcal{E}]
\label{ud}
\end{equation}
such that it produces the different Néron lattices of each elliptic curve $E^d$ in the twisted isogeny class $\mathcal{E}^d$. More precisely, one has that 
\begin{equation}
\displaystyle{\frac{u_E(d)}{\sqrt{d}} u_E \Lambda_E}
\label{Neron_d}
\end{equation}
is the Néron lattice of the quadratic twist $E^d$.
To compute the vector ${\bf u}(d)$ we make use of 
\cite{Pal}. Again, we only need to consider
the vector ${\bf u}(d)$ as a projective point.
Finally, it only remains to compute all 
the volumes of Néron lattices for $E^d$ and select the greatest. This corresponds to the elliptic curve in $\mathcal{E}^d$ with minimal Faltings height. 
As for the finite cases, we proceed likewise. 
First, one needs to make the list of the finite number of graphs up to twist and then discuss the volumes of Néron lattices with regard to the square-free parameter $d$. 

In both cases (finite or infinite), in order to compute the probability of a vertex to be the Faltings curve in the isogeny class $\mathcal{E}^d$ as $|d|$ grows up to infinity, we apply the following result.

\begin{proposition}
\label{analyticNT}
Let $a$ and $q$ integers such that $\operatorname{gcd}(a,q)=1$. Then, the counting function $\pi_{a,q}^{\operatorname{{s-f}}}(x)$ for the 
number of square-free integers $d\leq x$ that are congruent to 
$a$ mod $q$ satisfies:
\begin{equation}
\displaystyle{
\pi_{a,q}^{\operatorname{{s-f}}}(x) = 
\sum_{\substack{ n\leq x \\ n \equiv a (q)}}
  \mu^2(n) = 
\frac{1}{q} 
\prod_{p|q} \frac{1}{1-p^{-2}}
\frac{6}{\pi^2} x + O(\sqrt{x q} \log q) }
\end{equation}
where $p|q$ are the prime divisors of $q$ 
(or $p=0$ when $q=1$) and $\mu(n)$ denotes the Moebius function.
\end{proposition}

\begin{proof}
    \textcolor{red}{To be included?}
\end{proof}

\section{Type $L_3(9)$}

The isogeny graphs of type $L_3(9)$ are given by
three isogenous elliptic curves:

\begin{equation}
\begin{tikzcd}
E_1 \arrow[dash]{r}{3} & E_3  \arrow[dash]{r}{3} & E_9   \,.
\end{tikzcd}
\label{L39}
\end{equation}

\noindent To parameterize all of them, we are going to 
use the hauptmodul of $X_0(9)$ given by  
\begin{equation}
t(\tau)= 3^3 \left( \frac{\eta(9\tau)}{\eta(\tau)}\right)^3\,.
\label{HauptX09}
\end{equation}
Letting $t=t(\tau)$, one can write

\begin{equation}
\begin{tblr}{l@{\,=\,}l}
j(E_1) = j(\tau) & 
\displaystyle{\frac{(t + 3)^{3}(t^{3} + 9 \, t^{2} + 27 \, t + 3)^{3}}{t(t^{2} + 9 \, t + 27)}}\\[6pt]
j(E_3) = j(3\tau) & 
\displaystyle{\frac{(t + 3)^{3}(t + 9)^{3}}{t^3(t^{2} + 9\,  t + 27)^{3} }}
\\[6pt]
j(E_9) = j(9\tau) & 
\displaystyle{\frac{ (t + 9)^{3}(t^{3} + 243 \, t^{2} + 2187\,  t + 6561)^{3}}{t^9(t^{2} + 9\,  t + 27) }}\,,\\[6pt]
\end{tblr}
\label{L39}
\end{equation}
and the Fricke involution of $X_0(9)$ is given by $W_9(t)= 3^3/t$.
By using the Velú procedure, we can (and do) choose Weierstrass equations for $(E_1,E_3,E_9)$ with signatures:

%\[
\begin{longtblr}
[caption=  $L_3(9)$ signatures]
{cells={mode=imath},
hlines,
vlines,
hline{1,2,5,8,11}={1pt,solid},
vline{1,3}={1-10}{1pt,solid},
measure=vbox,
colspec=ll}
%\begin{tblr}[
%  caption = {Keys for Vlines}
%  ]{|l@{\,=\,}l|}
%\begin{tblr}{|c|l|}
\SetCell[c=2]{c} L_3(9) \\ 
c_4(E_1) & 
(t + 3)  (t^{3} + 9\,  t^{2} + 27\,  t + 3)\\
c_6(E_1) & 
t^{6} + 18\,  t^{5} + 135\,  t^{4} + 504 \, t^{3} + 891\,  t^{2} + 486\,  t - 27\\  
\Delta(E_1) & 
t  (t^{2} + 9\,  t + 27)\\
%----------------------------------
c_4(E_3) & 
(t + 3)  (t + 9) (t^{2} + 27) \\
c_6(E_3) & 
(t^{2} - 27) (t^{4} + 18\,  t^{3} + 162\,  t^{2} + 486\,  t + 729) \\  
\Delta(E_3) & 
t^{3}  (t^{2} + 9 \, t + 27)^{3} \\
%----------------------------------
c_4(E_9) & 
(t + 9)  (t^{3} + 243\,  t^{2} + 2187\,  t + 6561) \\
c_6(E_9) & 
t^{6} - 486\,  t^{5} - 24057\,  t^{4} - 367416\,  t^{3} - 2657205\,  t^{2} - 9565938\,  t - 14348907\\  
\Delta(E_9) & 
 t^{9}  (t^{2} + 9\,  t + 27)\\
\end{longtblr}
%\]

\vskip 0.5truecm

\noindent and, with this choice, the isogeny graph is normalized. The involution $W_9$ acts on the
isogeny graph of type $L_3(9)$ as:

\begin{equation}
\begin{tikzcd}
E_9^{-3} \arrow[dash]{r}{3} & E_3^{-3}  \arrow[dash]{r}{3} & E_1^{-3}  \,.
\end{tikzcd}
\end{equation}

\noindent 
For every prime $p$, the signatures, the scalar $u$ for the Weierstrass change to get the $p$-minimal model, the Kodaira symbols at $p$, and Pal values $u_p(p)$ can be read form the following tables:

\vskip 0.3truecm

\begin{longtblr}[
caption= {$L_3(9)$ data for $p\neq 2${,} $3$},
label = {L39p},
]
{cells={mode=imath},
hlines,
hline{1,3,6,9,12}={1.2pt,solid},
vlines,
measure=vbox,
colspec=cclclccc}
%-------------------------------------------------
L_3(9) & \SetCell[c=5]{c} p\ne 2,3 & & & & \\
t & E & \SetCell[c=1]{c} \operatorname{sig}_p(E) & u & \Kd_p(E) & u_p(d) \\
%-------------------------------------------------
\SetCell[r=3]{c} m=v_p(t)> 0 
& E_1 & (0,0,m) & 1 & \kI_m & 1\\
& E_3 & (0,0,3m) & 1 & \kI_{3m}& 1 \\
& E_9 & (0,0,9m) & 1 & \kI_{9m} & 1\\
%-------------------------------------------------
\SetCell[r=3]{c} 
\begin{array}{c}
v_p(t)=0  \\[6pt]
m=v_p(t^2+9\,t+27) \geq 0 
\end{array}
& E_1 & (0,0,m) & 1 & \kI_{m} & 1 \\
& E_3 & (0,0,3m) & 1 & \kI_{3m} & 1 \\
& E_9 & (0,0,m) & 1 & \kI_{m} & 1 \\
%-------------------------------------------------
\SetCell[r=3]{c} -m=v_p(t) < 0 
& E_1 & (0,0,9m) & p^{-m} & \kI_{9m} & 1  \\
& E_3 & (0,0,3m) & p^{-m} & \kI_{3m} & 1 \\
& E_9 & (0,0,m) & p^{-m} & \kI_{m} & 1 \\
\end{longtblr}

\vskip 0.75truecm

\begin{longtblr}
[caption= {$L_3(9)$ data for $p$=3},
label = {L393},]
{cells={mode=imath},
hlines,
vlines,
hline{1,3,6,9,12,15}={1.2pt,solid},
measure=vbox,
hline{Z}={1-X}{0pt},
vline{1}={Y-Z}{0pt},
colspec=cclclcc,
rowhead=2}
%-------------------------------------------------
L_3(9) &\SetCell[c=6]{c} p=3  & & & & & \\
t & E & \SetCell[c=1]{c} \operatorname{sig}_3(E) & u & \Kd_3(E) & \SetCell[c=2]{c} u_3(d)  & \\
%-------------------------------------------------
\SetCell[r=3]{c} m=v_3(t)\ge 3 
& E_1 & (2,3,m+3) & 1 & {\kI}^*_{m-3} & 3 & 1\\
& E_3 & (2,3,3m-3) & 3^ & {\kI}^*_{3(m-3)} & 3 & 1 \\
& E_9 & (2,3,9m-21) & 3^{2} & {\kI}^*_{9(m-3)} & 3 & 1 \\
%-------------------------------------------------
\SetCell[r=3]{c} 
     v_3(t)=2   
& E_1 & (2,3,5) & 1 & \kIV  & 1 & 1\\
& E_3 & (\ge 2,3,3) & 3 & \kII  & 1 & 1\\
& E_9 & (\ge 4,6,9) & 3 & {\kIV}^* & 3 & 1 \\
%-------------------------------------------------
\SetCell[r=3]{c} 
     v_3(t)=1  
& E_1 & (\ge 2,3,3) & 1 & \kII  & 1 & 1\\
& E_3 & (\ge 4,6,9) & 1 & {\kIV}^* & 3 & 1 \\
& E_9 & (4,6,11) & 1 & {\kII}^*  & 3 & 1 \pagebreak \\
%-------------------------------------------------
\pagebreak
\SetCell[r=3]{c} 
    -m=v_3(t)\le 0   
& E_1 & (0,0,9m) & 3^{-m} & \kI_{9m}  & 1 & 1 \\
& E_3 & (0,0,3m) & 3^{-m} & \kI_{3m}  & 1 & 1 \\
& E_9 & (0,0,m) & 3^{-m} & \kI_{m}  & 1 & 1 \\
%-------------------------------------------------
 \SetCell[c=5,r=2]{c} & & & & & d\equiv 0  & d\not\equiv 0 \\
                      & & & & & \SetCell[c=2]{c} d \Mod 3 & \\
\end{longtblr}

\vskip 0.75truecm

\begin{longtblr}
[caption = {$L_3(9)$ data for $p$=2},
label = {L392},]
{cells = {mode=imath},
hlines,
hline{1,3,6,9,12}={1.2pt,solid},
vlines,
measure=vbox,
hline{Z} = {1-5}{0pt},
vline{1} = {Y-Z}{0pt},
colspec  = cclclccc}
%----------------------------------------------
L_3(9) & \SetCell[c=7]{c} p=2  & & & & & \\ 
t & E & \SetCell[c=1]{c} \operatorname{sig}_2(E) & u & \SetCell[c=1]{c} \Kd_2(E) & \SetCell[c=3]{c} u_2(d)  \\
%----------------------------------------------
\SetCell[r=3]{c} m=v_2(t)>0 
& E_1 & (4,6,m+12) & 2^{-1} & \kI_{m+4}^* & 1 & 1 & 2 \\
& E_3 & (4,6,3m+12) & 2^{-1} & \kI_{3m+4}^* & 1 & 1 & 2\\
& E_9 & (4,6,9m+12) & 2^{-1} & \kI_{9m+4}^*& 1 & 1 & 2\\
%----------------------------------------------
\SetCell[r=3]{c} v_2(t)=0 
& E_1 & (\geq 8,9,12) & 2^{-1} & \kII^* & 1 & 2 & 2 \\
& E_3 & (\geq 8,9,12) & 2^{-1} & \kII^* & 1 & 2 & 2\\
& E_9 & (\geq 8,9,12) & 2^{-1} & \kII^*& 1 & 2 & 2\\
%----------------------------------------------
\SetCell[r=3]{c} -m=v_2(t)<0 
& E_1 & (4,6,9m+12) & 2^{-m-1} & \kI_{9m+4}^* & 1 & 1 & 2\\
& E_3 & (4,6,3m+12) & 2^{-m-1}  & \kI_{3m+4}^*& 1 & 1 & 2\\
& E_9 & (4,6,m+12) & 2^{-m-1}  & \kI_{m+4}^*& 1 & 1 & 2 \\
%----------------------------------------------
 \SetCell[c=5,r=2]{c} & & & & &  d\equiv 1 &  d\equiv 2  & d\equiv 3 \\
                      & & & & & \SetCell[c=3]{c} d \Mod{4} & \\
\end{longtblr}

\begin{prop}
Let 
$ 
\begin{tikzcd}
E_1 \arrow[dash]{r}{3}  & E_3 \arrow[dash]{r}{3} & E_9 
\end{tikzcd}
$
be a $\mathbf{Q}$-isogeny graph of type $L_3(9)$ corresponding to a given $t$ in $\mathbf{Q}^*$. For every square-free integer $d$, 
the probability of a vertex
to be the Faltings curve (circled)
in the twisted isogeny graph 
$
\begin{tikzcd} 
E_1^d \arrow[dash]{r}{3}  & E_3^d\arrow[dash]{r}{3} & E_9^d 
\end{tikzcd}
$ 
is given by:

\pagebreak


\begin{longtblr}
[caption = Faltings curves for $L_3(9)$]
{cells = {mode=imath},
hlines,
hline{1,2,3,5,7,8}={1.2pt,solid},
vlines,
measure=vbox,
colspec  = cccc}
 L_3(9) & \text{twisted isogeny graph} & d & \text{Prob} \\
 \SetCell[r=1]{c} v_3(t)\leq 0 & \circled[0.8]{$E_1^d$} \longrightarrow E_3^d\longrightarrow E_9^d  & & 1 \\
\SetCell[r=2]{c} v_3(t)=1   
& \circled[0.8]{$E_1^d$} \longrightarrow E_3^d\longrightarrow E_9^d & d\not\equiv 0\,(3) & 3/4\\ 
&  E_1^d \longleftarrow \circled[0.8]{$E_3^d$} \longrightarrow E_9^d &d\equiv 0\,(3) &  1/4 \\
\SetCell[r=2]{c}
v_3(t)=2   
 &  E_1^d \longleftarrow \circled[0.8]{$E_3^d$} \longrightarrow E_9^d &d\not\equiv 0\,(3) & 3/4 \\
 &  E_1^d \longleftarrow E_3^d\longleftarrow \circled[0.8]{$E_9^d$} &d\equiv 0\,(3) & 1/4 \\
 \SetCell[r=1]{c} v_3(t)\ge 3   &  E_1^d \longleftarrow  E_3^d\longleftarrow  \circled[0.8]{$E_9^d$} & & 1 \\
\end{longtblr}

\end{prop}

\vskip 0.35truecm

\begin{proof}
From Tables \ref{L39p}, \ref{L393}, \ref{L392} 
one gets the (projective) vectors 
${\bf u}=[u(E)]$ and ${\bf u}(d)=[u(E)(d)]$ as explained in Section~2:

\vskip 0.5truecm

\begin{longtblr}
[caption = The vectors $[u(E)]$ and $[u(E)(d)]$]
{cells={mode=imath},
hlines,
vlines,
hline{1,2,3,5,7,8}={1.2pt,solid},
measure=vbox,
colspec  = cccc}
%-------------------------------------------------
\SetCell[c=1]{c} t &\SetCell[c=1]{c} [u(E)]  & \SetCell[c=1]{c} [u(E)(d)] &  \SetCell[c=1]{c} d \\
%-------------------------------------------------
\SetCell[r=1]{c} v_3(t)\le 0 & \SetCell[r=1]{c} (1:1:1) & (1:1:1) & \\
%-------------------------------------------------
\SetCell[r=2]{c} v_3(t)=1 & \SetCell[r=2]{c} (1:1:1) & (1:1:1) & d\not\equiv 0\,(3) \\ 
& & \SetCell[r=1]{c} (1:3:3) & d\equiv 0\,(3) \\
%-------------------------------------------------
\SetCell[r=2]{c}  v_3(t)=2 & \SetCell[r=2]{c} (1:3:3) & (1:1:1) & d\not\equiv 0\,(3) \\
& & \SetCell[r=1]{c} (1:1:3) & d\equiv 0\,(3) \\
%-------------------------------------------------
v_3(t)\ge 3 & (1:3:3^{2}) & (1:1:1) &  \\
%-------------------------------------------------
\end{longtblr} 

\vskip 0.4truecm

Let us consider the case $v_3(t)=2$ and $d\equiv 0 \, (3)$. The other cases can be dealt analogously. Since the signatures are taken in such a way that the isogenies are normalized,  the period lattices of $(E_1,E_3,E_9)$
are
\begin{equation}
\displaystyle{
\Lambda_1 = \lambda \langle 1, \tau \rangle \,,\qquad
\Lambda_3 = \frac{1}{3}\lambda \langle 1, 3\tau \rangle \,,\qquad
\Lambda_9 = \frac{1}{9} \lambda \langle 1, 9\tau \rangle \,.}   
\end{equation}
for some $\lambda,\tau\in \mathbf{C}$ and $\operatorname{Im}(\tau)>0$. 
From the column $[u(E)]$ 
we get the Néron lattices of $(E_1,E_3,E_9)$:
\begin{equation}
\displaystyle{
\Lambda_1 = \lambda \langle 1, \tau \rangle \,,\qquad
\Lambda_3 = \lambda \langle 1, 3\tau \rangle \,,\qquad
\Lambda_9 = \frac{1}{3} \lambda \langle 1, 9\tau \rangle \,.}   
\end{equation}
For every square-free integer $d$ with $d\equiv 0 \, (3)$, the Néron
lattices of $(E_1^d,E_3^d,E_9^d)$ are:
\begin{equation}
\displaystyle{
\Lambda_1^d = \frac{1}{\sqrt{d}} \lambda \langle 1, \tau \rangle \,,\qquad
\Lambda_3^d = \frac{1}{\sqrt{d}}  \lambda \langle 1, 3\tau \rangle \,,\qquad
\Lambda_9^d = \frac{1}{\sqrt{d}} \lambda \langle 1, 9\tau \rangle \,.}   
\end{equation}
The Néron volumes of $(E_1^d,E_3^d,E_9^d)$ are 
$v = |\lambda|^2/|d| \operatorname{Im}(\tau)$, $3v$, and $9v$, respectively. Hence, in these cases $E_9$ has minimum Faltings height. By Proposition\ref{analyticNT}, we get that
the probability of a square-free integer $d$ to be $d\equiv 0 \, (3)$ is $1-(\frac{3}{8}+\frac{3}{8})= 1/4$ as wished.
\end{proof} 

\section{Type $T_4$}

\section{Other genus zero types}

\section{Sporadic types}

\section{Tables}

Tables should be inserted in the text as close to the point of
reference as possible. Some space should be left above and below
the table.

Tables should be numbered sequentially in the text in Arabic
numerals. Captions are to be centralized above the tables.  Typeset
table text and captions in 8 pt Roman with baselineskip of 10 pt.

\begin{table}[ht]
\tbl{Comparison of acoustic for frequencies for piston-cylinder problem.}
{\begin{tabular}{@{}cccc@{}} \toprule
Piston mass & Analytical frequency & TRIA6-$S_1$ model &
\% Error \\
& (Rad/s) & (Rad/s) \\ \colrule
1.0\hphantom{00} & \hphantom{0}281.0 & \hphantom{0}280.81 & 0.07 \\
0.1\hphantom{00} & \hphantom{0}876.0 & \hphantom{0}875.74 & 0.03 \\
0.01\hphantom{0} & 2441.0 & 2441.0\hphantom{0} & 0.0\hphantom{0} \\
0.001 & 4130.0 & 4129.3\hphantom{0} & 0.16\\ \botrule
\end{tabular}}
\end{table}

If tables extend over to a second page, the continuation of
the table should be preceded by a caption,
e.g.~``Table~2. (Continued)''.


\section*{References}

References are to be listed in alphabetical order of the author's name
and cited in the text in Arabic numerals within square brackets.
They can be referred to indirectly,
e.g.~``$\ldots$ in the statement \cite{2}.'' or used directly,
e.g.~``$\ldots$ see [2] for examples.'' List references using the
style shown in the following examples. For journal names, use the
standard abbreviations.  Typeset references in 9 pt Roman.

\begin{thebibliography}{99}
%---------------------------------------------------
\bibitem{C} G. Chiloyan, and A. Lozano-Robledo, A classification of isogeny-torsion graphs of 
{$\Q$}-isogeny classes of elliptic curves, 
{\it Trans. London Math. Soc.} 
{\bf 8} (2021) 1--34.
%---------------------------------------------------
\bibitem{DD}
T. Dokchitser, and V. Dokchitser,
Local invariants of isogenous elliptic curves,
{\it Transactions of the American Mathematical Society} {\bf 367}6 (2015), 4339--4358.
%---------------------------------------------------
\bibitem{MS} B. Mazur, and P. Swinnerton-Dyer, 
Arithmetic of Weil Curves, {\it  Inventiones mathematicae} {\bf 25} (1974) 1--62.
%---------------------------------------------------
\bibitem{M}
B. Mazur, D. Goldfeld, 
Rational isogenies of prime degree, {\it Invent Math} 
{\bf 44} (1978) 129-–162. 
%---------------------------------------------------
\bibitem{Pal}
V. Pal, with and appendix by A.Agashe,
Periods of quadratic twists of ellipitc curves
{\it Proceedings of the American Mathematical Society} {\bf 140} 5 (2012) 1513–-1525.
%---------------------------------------------------
\bibitem{Papa}
I. Papadopoulos, 
On Néron’s classification of elliptic curves of residue characteristics 2 and 3. (Sur la classification de Néron des courbes elliptiques en caractéristique résiduelle 2 et 3.) (French),
{\it J. Number Theory} 
{\bf 44}(2)
(1993), 119--152.
%---------------------------------------------------
\bibitem{Sil}	
J. H. Silverman,
{\it The arithmetic of elliptic curves} (Graduate texts in mathematics 106, Springer 1986).
%---------------------------------------------------
\bibitem{S}
G. Stevens, 
Stickelberger elements and modular parametrizations of elliptic curves, {\it Inventiones mathematicae} {\bf 98}(1) (1989) 
75--106. 
%---------------------------------------------------
\bibitem{V}
J. Vélu, Isogénies entre courbes elliptiques, 
{\it C.R. Acad. Sc. Paris, S ́erie A.} {\bf 273}
(1971)
238–-241.
%---------------------------------------------------
\end{thebibliography}

\end{document} 

%---------------------------------------------------
\bibitem{1} A. Alekseev, A. Malkin and E. Meinrenken, Lie group valued
moment maps, {\it J. Differential Geom.} {\bf 48}(3) (1998) 445--495.
%---------------------------------------------------
\bibitem{2} F. Bouchut, On zero pressure gas dynamics, in
{\it Advances in Kinetic Theory and Computing}, ed.~B. Perthame,
Ser. Adv. Math. Appl. Sci., Vol. 22 (World Scientific, 1994),
pp.~171--190.
%---------------------------------------------------
\bibitem{3} S. K. Godunov and E. Romenskii, {\it Elements of Continuum
Mechanics and Conservation Laws} (Kluwer Academic/Plenum Publishers,
2003).
%---------------------------------------------------
\bibitem{4} J. Li and T. Zhang, On the initial-value problem for
zero-pressure gas dynamics, in {\it Proc. 7th Intl. Conf. on Hyperbolic
Problems}, ed. R. Jeltsch (Birkhauser Verlag, 1998), pp.~629--640.
%---------------------------------------------------
