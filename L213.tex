



\documentclass[11pt]{article}
\usepackage{amsfonts,amssymb,amsmath,amsthm,latexsym,graphics,epsfig,amsfonts}
\usepackage{verbatim,enumerate,array,booktabs,color,bigstrut,prettyref,tikz-cd}
\usepackage{multirow}
\usepackage[all]{xy}
\usepackage[backref]{hyperref}
\usepackage[OT2,T1]{fontenc}
%\usepackage{ctable}
\usepackage{mathtools}

\usepackage{longtable}


\usepackage{mathtools}
\newcommand{\Mod}[1]{\ (\mathrm{mod}\ #1)}
\newcommand{\mathdash}{\relbar\mkern-8mu\relbar}
\newcommand*\circled[2][1.6]{\tikz[baseline=(char.base)]{
    \node[shape=circle, draw, inner sep=1pt, 
        minimum height={\f@size*#1},] (char) {\vphantom{WAH1g}#2};}}
\makeatother



\usepackage{tabularray}
\UseTblrLibrary{amsmath,varwidth}

\usepackage{tabularx}
\usepackage{longtable}
\usepackage{arydshln}


\newcommand\myiso{\stackrel{\mathclap{\normalfont\mbox{\small $p$}}}{-}}
\newcommand\myisot{\stackrel{\mathclap{\normalfont\mbox{\small $3$}}}{-}}

\newcommand{\pref}[1]{\prettyref{#1}}
\newrefformat{eq}{\textup{(\ref{#1})}}
\newrefformat{prty}{\textup{(\ref{#1})}}

\definecolor{mylinkcolor}{rgb}{0.8,0,0}
\definecolor{myurlcolor}{rgb}{0,0,0.8}
\definecolor{mycitecolor}{rgb}{0,0,0.8}
\hypersetup{colorlinks=true,urlcolor=myurlcolor,citecolor=mycitecolor,linkcolor=mylinkcolor,linktoc=page,breaklinks=true}

%\DeclareSymbolFont{cyrletters}{OT2}{wncyr}{m}{n}
%\DeclareMathSymbol{\Sha}{\mathalpha}{cyrletters}{"58}

\addtolength{\textwidth}{4cm} \addtolength{\hoffset}{-2cm}
\addtolength{\marginparwidth}{-2cm}

%\theoremstyle{definition}
\newtheorem{defn}{Definition}[section]
\newtheorem{definition}[defn]{Definition}
\newtheorem{claim}[defn]{Claim}

%\theoremstyle{plain}
\newtheorem{thmA}{Theorem A}
\newtheorem{thmB}{Theorem B}
\newtheorem{thm2}{Theorem}
\newtheorem{prop2}{Proposition}
\newtheorem{note}{Note}

\newtheorem{corollary}[defn]{Corollary}
\newtheorem{lemma}[defn]{Lemma}
\newtheorem{property}[defn]{Property}
\newtheorem{thm}[defn]{Theorem}
\newtheorem{theorem}[defn]{Theorem}
\newtheorem{cor}[defn]{Corollary}
\newtheorem{prop}[defn]{Proposition}
\newtheorem{proposition}[defn]{Proposition}
\newtheorem{thmnn}{Theorem}
\newtheorem{conj}[defn]{Conjecture}

\theoremstyle{definition}
\newtheorem{remarks}{Remarks}
\newtheorem{ack}{Acknowledgements}
\newtheorem{remark}[defn]{Remark}
\newtheorem{question}[defn]{Question}
\newtheorem{example}[defn]{Example}


\newcommand{\Q}{\mathbb Q}
\newcommand{\Qbar}{\overline{\Q}}
\newcommand{\Z}{\mathbb Z}

\newcommand{\modQ}{\,\text{mod}\,(\Q^)^2}

\newcommand{\mysquare}[1]{\tikz{\path[draw] (0,0) rectangle node{\tiny #1} (8pt,8pt) ;}}
\newcommand{\mycircle}[1]{\tikz{\path[draw] (0,0) circle (4pt) node{\tiny #1};}}


%------------------------------------
\newcommand{\Kd}{\operatorname{K}}
\newcommand{\kI}{\operatorname{I}}
\newcommand{\kII}{\operatorname{II}}
\newcommand{\kIII}{\operatorname{III}}
\newcommand{\kIV}{\operatorname{IV}}
%-------------------------------------



\begin{document}
\title{Type $L_2(13)$}
\date{\today}
\maketitle

\section{Setting}

The isogeny graphs of type $L_2(13)$ are given by
two isogenous elliptic curves:

\[ 
\begin{tikzcd}
E_1 \arrow[dash]{r}{13} & E_{13}   \,.
\end{tikzcd}
\]


\noindent A hauptmodule of $X_0(13)$ is  
$$t(\tau)= 13 \left( \frac{\eta(13\tau)}{\eta(\tau)}\right)^{2}\,.$$ 
Letting $t=t(\tau)$, one can write
$$
\begin{tblr}{l@{\,=\,}l}
j(E_1) = j(\tau) & 
\displaystyle{\frac{\left(t^2+5 t+13\right) \left(t^4+7 t^3+20 t^2+19 t+1\right)^3}{t}}\\[6pt]
j(E_{13}) = j(13\tau) & 
\displaystyle
{\frac{\left(t^2+5 t+13\right) \left(t^4+247 t^3+3380 t^2+15379 t+28561\right)^3}{t^{13}}}\,,\\[6pt]
\end{tblr}
$$
and the Fricke involution of $X_0(13)$ is given by $W_{13}(t)=13/t $.

We can (and do) choose Weierstrass equations for $(E_1,E_{13})$ with signatures:
$$
 \begin{tblr}{|c|l|}
\hline \SetCell[c=2]{c} L_2(13) \\ \hline
 c_4(E_1) & (t^2 + 5t + 13)(t^2 + 6t + 13)(t^4 + 7t^3 + 20t^2 + 19t + 1)\\

 c_6(E_1) & (t^2 + 5t + 13)(t^2 + 6t + 13)^{2}(t^6 + 10t^5 + 46t^4 + 108t^3 + 122t^2 + 38t - 1)\\

 \Delta(E_1) & t(t^2 + 5t + 13)^{2}(t^2 + 6t + 13)^{3}\\ \hline

 c_4(E_{13}) & (t^2 + 5t + 13)(t^2 + 6t + 13)(t^4 + 247t^3 + 3380t^2 + 15379t + 28561)\\

 c_6(E_{13}) & (t^2 + 5t + 13)(t^2 + 6t + 13)^{2}(t^6 - 494t^5 - 20618t^4 - 237276t^3 - 1313806t^2 - 3712930t - 4826809)\\

 \Delta(E_{13}) & t^{13}(t^2 + 5t + 13)^{2}(t^2 + 6t + 13)^{3}\\ \hline

\end{tblr}
$$
With regard to the action of the Fricke involution 
on the isogeny graph, 
it can be displayed as follows:
\[ 
\begin{tikzcd}
W_{13}\,\,:\,\, E_{13}^{-13} \arrow[dash]{r}{13} & E_1^{-13}   \,.
\end{tikzcd}
\]


\newpage

\section{Kodaira symbols \& Pal coefficients}


\begin{longtblr}
[caption= {$L_2(13)$ data for $p\ne 2,3,13$}]
{cells={mode=imath},hlines,vlines,measure=vbox,
hline{Z}={1-X}{0pt},
vline{1}={Y-Z}{0pt},
colspec=cclclcc}
\SetCell[c=1]{c} L_2(13) &\SetCell[c=6]{c} p\ne 2,3,13 &    & \\
\SetCell[c=1]{c} t & E & 
\SetCell[c=1]{c} \operatorname{sig}_p(E) & u & \Kd_p(E) & \SetCell[c=2]{c} u_p(d)\\
%----------------------------------------------
\SetCell[r=2]{c} 
     m= v_p(t)>0   
& E_1 & (0,0,m) & 1 & \kI_{m} & 1& 1\\
& E_{13} & (0,0,13m) & 1 &  \kI_{13m} & 1& 1\\
%----------------------------------------------
\SetCell[r=2]{c} 
\begin{array}{c}
     v_{p}(t)=0  \\[3pt]
    v_{p}(t^2 + 5t + 13)=6m>0 
\end{array}
& E_1 & (2m,0,0) & p^m &  \kI_0  & 1 & 1\\
& E_{13} & (2m,0,0) & p^m &  \kI_0 & 1 & 1\\
%----------------------------------------------
\SetCell[r=2]{c} 
\begin{array}{c}
     v_{p}(t)=0  \\[3pt]
    v_{p}(t^2 + 5t + 13)=6m+1>0 
\end{array}
& E_1 & (2m+1,1,2) & p^m  &  \kII  & 1 & 1\\
& E_{13} & (2m+1,1,2) & p^m  &  \kII & 1 & 1\\
%----------------------------------------------
\SetCell[r=2]{c} 
\begin{array}{c}
     v_{p}(t)=0  \\[3pt]
    v_{p}(t^2 + 5t + 13)=6m+2>0 
\end{array}
& E_1 & (2m+2,2,4) & p^m  &  \kIV  & 1 & 1\\
& E_{13} & (2m+2,2,4) &p^m   &  \kIV & 1 & 1\\
%----------------------------------------------
\SetCell[r=2]{c} 
\begin{array}{c}
     v_{p}(t)=0  \\[3pt]
    v_{p}(t^2 + 5t + 13)=6m+3>0 
\end{array}
& E_1 & (2m+3,3,6) & p^m  &  \kI_0^*  & p & 1\\
& E_{13} & (2m+3,3,6) & p^m  &  \kI_0^* & p & 1\\
%----------------------------------------------
\SetCell[r=2]{c} 
\begin{array}{c}
     v_{13}(t)=0  \\[3pt]
    v_{p}(t^2 + 5t + 13)=6m+4>0 
\end{array}
& E_1 & (2m+4,4,8) & p^m  &  \kIV^*  & p & 1\\
& E_{13} & (2m+4,4,8) & p^m  &  \kIV^* & p & 1\\
%----------------------------------------------
\SetCell[r=2]{c} 
\begin{array}{c}
     v_{p}(t)=0  \\[3pt]
    v_{p}(t^2 + 5t + 13)=6m+5>0 
\end{array}
& E_1 & (2m+5,5,10) & p^m  &  \kII^*  & p & 1\\
& E_{13} & (2m+5,5,10) & p^m  &  \kII^* & p & 1\\
%----------------------------------------------
\SetCell[r=2]{c} 
\begin{array}{c}
     v_{p}(t)=0  \\[3pt]
    v_{p}(t^2 + 6t + 13)=4m>0 
\end{array}
& E_1 & (0,2m,0) & p^m &  \kI_0  & 1 & 1\\
& E_{13} & (0,2m,0) & p^m &  \kI_0 & 1 & 1\\
%----------------------------------------------
\SetCell[r=2]{c} 
\begin{array}{c}
     v_{p}(t)=0  \\[3pt]
    v_{p}(t^2 + 6t + 13)=4m+1>0 
\end{array}
& E_1 & (1,2m+2,3) & p^m  &  \kIII  & 1 & 1\\
& E_{13} & (1,2m+2,3) & p^m  &  \kIII & 1 & 1\\
%----------------------------------------------
\SetCell[r=2]{c} 
\begin{array}{c}
     v_{p}(t)=0  \\[3pt]
    v_{p}(t^2 + 6t + 13)=4m+2>0 
\end{array}
& E_1 & (2,2m+4,6) & p^m  &  \kI_0^*  & p & 1\\
& E_{13} & (2,2m+4,6) &p^m   &  \kI_0^* & p & 1\\
%----------------------------------------------
\SetCell[r=2]{c} 
\begin{array}{c}
     v_{p}(t)=0  \\[3pt]
    v_{p}(t^2 + 6t + 13)=4m+3>0 
\end{array}
& E_1 & (3,2m+6,9) & p^m  &  \kIII^*  & p & 1\\
& E_{13} & (3,2m+6,9) & p^m  &  \kIII^* & p & 1\\
%----------------------------------------------
\SetCell[r=2]{c} 
     -m= v_p(t)<0   
& E_1 & (0,0,13m) & p^{-2m} & \kI_{7m} & 1& 1\\
& E_{13} & (0,0,m) & p^{-2m} &  \kI_{m} & 1& 1\\
%----------------------------------------------
 \SetCell[c=5,r=2]{c} & & & & & d\equiv 0  & d\not\equiv 0 \\
                      & & & & & \SetCell[c=2]{c} d \Mod p & \\
\end{longtblr}

\newpage

\begin{longtblr}
[caption= {$L_2(13)$ data for $p=3$}]
{cells={mode=imath},hlines,vlines,measure=vbox,
hline{Z}={1-X}{0pt},
vline{1}={Y-Z}{0pt},
colspec=cclclcc}
\SetCell[c=1]{c} L_2(13) &\SetCell[c=6]{c} p=3 &    & \\
\SetCell[c=1]{c} t & E & 
\SetCell[c=1]{c} \operatorname{sig}_3(E) & u & \Kd_3(E) & \SetCell[c=2]{c} u_3(d)\\
%----------------------------------------------
\SetCell[r=2]{c} 
     m= v_3(t)>0   
& E_1 & (0,0,m) & 1 & \kI_{m} & 1& 1\\
& E_{13} & (0,0,13m) & 1 &  \kI_{13m} & 1& 1\\
%----------------------------------------------
\SetCell[r=2]{c} 
\begin{array}{c}
     v_3(t)=0  \\[3pt]
   t\not \equiv 2,5,8\,(9) 
\end{array}
& E_1 & (0,\ge 0,0) & 1 & \kI_0   & 1 & 1\\
& E_{13} & (0,\ge 0,0) & 1 & \kI_0   & 1 & 1\\
%----------------------------------------------
\SetCell[r=2]{c} 
\begin{array}{c}
     v_3(t)=0  \\[3pt]
   t \equiv 5,8\,(9) 
\end{array}
& E_1 & (2,3,4) & 1 & \kII   & 1 & 1\\
& E_{13} & (2,3,4) & 1 & \kII   & 1 & 1\\
%----------------------------------------------
\SetCell[r=2]{c} 
\begin{array}{c}
     v_3(t)=0  \\[3pt]
   t\equiv 2,20\,(27) 
\end{array}
& E_1 & (3,5,6) & 1 & \kIV   & 1 & 1\\
& E_{13} & (3,5,6) & 1 & \kIV   & 1 & 1\\
%----------------------------------------------
\SetCell[r=2]{c} 
\begin{array}{c}
     v_3(t)=0  \\[3pt]
   t \equiv 11\,(27) 
\end{array}
& E_1 & (3,\ge 6,6) & 1 & \kI_0^*   & 3 & 1\\
& E_{13} & (3,\ge 6,6) & 1 & \kI_0^*   & 3 & 1\\
%----------------------------------------------
\SetCell[r=2]{c} 
     -m= v_3(t)<0   
& E_1 & (0,0,13m) & 3^{-2m} & \kI_{13m} & 1& 1\\
& E_{13} & (0,0,m) & 3^{-2m} &  \kI_{m} & 1& 1\\
%----------------------------------------------
 \SetCell[c=5,r=2]{c} & & & & & d\equiv 0  & d\not\equiv 0 \\
                      & & & & & \SetCell[c=2]{c} d \Mod 3 & \\
\end{longtblr}

\newpage

\begin{longtblr}
[caption= {$L_2(13)$ data for $p=13$}]
{cells={mode=imath},hlines,vlines,measure=vbox,
hline{Z}={1-X}{0pt},
vline{1}={Y-Z}{0pt},
colspec=cclclcc}
\SetCell[c=1]{c} L_2(13) &\SetCell[c=6]{c} p=13 &    & \\
\SetCell[c=1]{c} t & E & 
\SetCell[c=1]{c} \operatorname{sig}_{13}(E) & u & \Kd_{13}(E) & \SetCell[c=2]{c} u_{13}(d)\\
%----------------------------------------------
\SetCell[r=2]{c} 
     m= v_{13}(t)\ge 2 
& E_1 & (2,3,m+5) & 1 & \kI_{m-1}^* & 13& 1\\
& E_{13} & (2,3,13m-7) & 13 &  \kI_{13(m-1)}^* & 13& 1\\
%----------------------------------------------
\SetCell[r=2]{c} 
\begin{array}{c}
     v_{13}(t)=1  \\[3pt]
    t/13\not\equiv 2,5\,(13)
\end{array}
& E_1 & (2,3,6) & 1&  \kI_0^*  & 13 & 1\\
& E_{13} & (2,3,6) & 13 &  \kI_0^* & 13 & 1\\
%----------------------------------------------
\SetCell[r=2]{c} 
\begin{array}{c}
     v_{13}(t)=1  \\[3pt]
     t/13\equiv 2\,(13)\\[3pt]
    v_{13}(t^2 + 6t + 13)=4m
\end{array}
& E_1 & (1,\ge 1,3) & 13^m &  \kIII  & 1 & 1\\
& E_{13} & (1,\ge 1,3) & 13^{m+1} & \kIII  & 1 & 1\\
%----------------------------------------------
\SetCell[r=2]{c} 
\begin{array}{c}
     v_{13}(t)=1  \\[3pt]
     t/13\equiv 2\,(13)\\[3pt]
    v_{13}(t^2 + 6t + 13)=4m+1
\end{array}
& E_1 & (2,\ge 3,6) & 13^m &  \kI_0^*  & 13 & 1\\
& E_{13} & (2,\ge 4,6) & 13^{m+1} &  \kI_0^*  & 13 & 1\\
%----------------------------------------------
\SetCell[r=2]{c} 
\begin{array}{c}
     v_{13}(t)=1  \\[3pt]
     t/13\equiv 2\,(13)\\[3pt]
    v_{13}(t^2 + 6t + 13)=4m+2
\end{array}
& E_1 & (3,\ge 5,9) & 13^m &  \kIII^*  & 13 & 1\\
& E_{13} &(3,\ge 6,9) & 13^{m+1} &  \kIII^*  & 13 & 1\\

%----------------------------------------------
\SetCell[r=2]{c} 
\begin{array}{c}
     v_{13}(t)=1  \\[3pt]
     t/13\equiv 2\,(13)\\[3pt]
    v_{13}(t^2 + 6t + 13)=4m+3
\end{array}
& E_1 & (0,\ge 1,0) & 13^{m+1} &  \kI_0  & 1 & 1\\
& E_{13} & (0,\ge 2,0) & 13^{m+2} & \kI_0  & 1 & 1\\
%----------------------------------------------
\SetCell[r=2]{c} 
\begin{array}{c}
     v_{13}(t)=1  \\[3pt]
     t/13\equiv 5\,(13)\\[3pt]
    v_{13}(t^2 + 5t + 13)=6m
\end{array}
& E_1 & (\ge 2,2,4) & 13^{m} &  \kIV  & 1 & 1\\
& E_{13} &  (\ge 2,2,4) & 13^{m+1} &  \kIV  & 1 & 1\\
%----------------------------------------------
\SetCell[r=2]{c} 
\begin{array}{c}
     v_{13}(t)=1  \\[3pt]
     t/13\equiv 5\,(13)\\[3pt]
    v_{13}(t^2 + 5t + 13)=6m+1
\end{array}
& E_1 & (\ge 2,3,6) & 13^{m} &  \kI_0^*  & 13 & 1\\
& E_{13} & (\ge 3,3,6) & 13^{m+1} &  \kI_0^*  & 13 & 1\\
%----------------------------------------------
\SetCell[r=2]{c} 
\begin{array}{c}
     v_{13}(t)=1  \\[3pt]
     t/13\equiv 5\,(13)\\[3pt]
    v_{13}(t^2 + 5t + 13)=6m+2
\end{array}
& E_1 & (\ge 3,4,8) & 13^{m} &  \kIV^*  & 13 & 1\\
& E_{13} & (\ge 4,4,8) & 13^{m+1} &  \kIV^*  & 13 & 1\\
%----------------------------------------------
\SetCell[r=2]{c} 
\begin{array}{c}
     v_{13}(t)=1  \\[3pt]
     t/13\equiv 5\,(13)\\[3pt]
    v_{13}(t^2 + 5t + 13)=6m+3
\end{array}
& E_1 & (\ge 4,5,10) & 13^{m} &  \kII^*  & 13 & 1\\
& E_{13} & (\ge 5,5,10) & 13^{m+1} &  \kII^*  & 13 & 1\\
%----------------------------------------------
\SetCell[r=2]{c} 
\begin{array}{c}
     v_{13}(t)=1  \\[3pt]
     t/13\equiv 5\,(13)\\[3pt]
    v_{13}(t^2 + 5t + 13)=6m+4
\end{array}
& E_1 & (\ge 0,0,0) & 13^{m+1} &  \kI_0  & 1 & 1\\
& E_{13} & (\ge 2,0,0) & 13^{m+2} &  \kI_0  & 1 & 1\\
%----------------------------------------------
\SetCell[r=2]{c} \begin{array}{c}
     v_{13}(t)=1  \\[3pt]
     t/13\equiv 5\,(13)\\[3pt]
    v_{13}(t^2 + 5t + 13)=6m+5
\end{array}
& E_1 & (\ge 2,1,2) & 13^{m+1} &  \kII  & 1 & 1\\
& E_{13} & (\ge 3,1,2) & 13^{m+2} &  \kII  & 1 & 1\\
%----------------------------------------------
\SetCell[r=2]{c} 
\begin{array}{c}
     v_{13}(t)=0  \\[3pt]
    t\not\equiv 7,8\,(13)
\end{array}
& E_1 & (0,0,0) & 1&  \kI_0  & 1 & 1\\
& E_1 & (0, 0,0) & 1&  \kI_0  & 1 & 1\\
%----------------------------------------------
\SetCell[r=2]{c} 
\begin{array}{c}
     v_{13}(t)=0  \\[3pt]
     t\equiv 7\,(13)\\[3pt]
    v_{13}(t^2 + 6t + 13)=4m
\end{array}
& E_1 & (0,\ge 1,0) & 13^{m} &  \kI_0  & 1 & 1\\
& E_{13} & (0,\ge 0,0) & 13^{m} &  \kI_0  & 1 & 1\\
%----------------------------------------------
\SetCell[r=2]{c} 
\begin{array}{c}
     v_{13}(t)=0  \\[3pt]
     t\equiv 7\,(13)\\[3pt]
    v_{13}(t^2 + 6t + 13)=4m+1
\end{array}
& E_1 & (1,\ge 3,3) & 13^m &  \kIII  & 1 & 1\\
& E_{13} & (1,\ge 2,3) & 13^m &  \kIII  & 1 & 1\\
%----------------------------------------------
\SetCell[r=2]{c} 
\begin{array}{c}
     v_{13}(t)=0  \\[3pt]
     t\equiv 7\,(13)\\[3pt]
    v_{13}(t^2 + 6t + 13)=4m+2
\end{array}
& E_1 & (2,\ge 5,6) & 13^m &  \kI_0^*  & 13 & 1\\
& E_{13} & (2,\ge 4,6) & 13^m &  \kI_0^*  & 13 & 1\\
%----------------------------------------------
\SetCell[r=2]{c} 
\begin{array}{c}
     v_{13}(t)=0  \\[3pt]
     t\equiv 7\,(13)\\[3pt]
    v_{13}(t^2 + 6t + 13)=4m+3
\end{array}
& E_1 & (3,\ge 7,9) & 13^m &  \kIII^*  & 13 & 1\\
& E_{13} & (3,\ge 6,9) & 13^m &  \kIII^*  & 13 & 1\\
%----------------------------------------------
\SetCell[r=2]{c} 
\begin{array}{c}
     v_{13}(t)=0  \\[3pt]
     t/13\equiv 8\,(13)\\[3pt]
    v_{13}(t^2 + 5t + 13)=6m
\end{array}
& E_1 & (\ge 1,0,0) & 13^{m} &  \kI_0  & 1 & 1\\
& E_{13} & (\ge 0,0,0) & 13^{m} &  \kI_0  & 1 & 1\\
%----------------------------------------------
\SetCell[r=2]{c} 
\begin{array}{c}
     v_{13}(t)=0  \\[3pt]
     t/13\equiv 8\,(13)\\[3pt]
    v_{13}(t^2 + 5t + 13)=6m+1
\end{array}
& E_1 & (\ge 2,1,2) & 13^{m} &  \kII  & 1 & 1\\
& E_{13} & (\ge 1,1,2) & 13^{m} &  \kII  & 1 & 1\\
%----------------------------------------------
\SetCell[r=2]{c} 
\begin{array}{c}
     v_{13}(t)=0  \\[3pt]
     t/13\equiv 8\,(13)\\[3pt]
    v_{13}(t^2 + 5t + 13)=6m+2
\end{array}
& E_1 & (\ge 3,2,4) & 13^{m} &  \kIV  & 1 & 1\\
& E_{13} & (\ge 2,2,4) & 13^{m} &  \kIV  & 1 & 1\\
%----------------------------------------------
\SetCell[r=2]{c} 
\begin{array}{c}
     v_{13}(t)=0  \\[3pt]
     t/13\equiv 8\,(13)\\[3pt]
    v_{13}(t^2 + 5t + 13)=6m+3
\end{array}
& E_1 & (\ge 4,3,6) & 13^{m} &  \kI_0^*  & 13 & 1\\
& E_1 & (\ge 3,3,6) & 13^{m} &  \kI_0^*  & 13 & 1\\
%----------------------------------------------
\SetCell[r=2]{c} 
\begin{array}{c}
     v_{13}(t)=0  \\[3pt]
     t/13\equiv 8\,(13)\\[3pt]
    v_{13}(t^2 + 5t + 13)=6m+4
\end{array}
& E_1 & (\ge 5,4,8) & 13^{m} &  \kIV^*  & 13 & 1\\
& E_{13} & (\ge 4,4,8) & 13^{m} &  \kIV^*  & 13 & 1\\
%----------------------------------------------
\SetCell[r=2]{c} \begin{array}{c}
     v_{13}(t)=0  \\[3pt]
     t/13\equiv 8\,(13)\\[3pt]
    v_{13}(t^2 + 5t + 13)=6m+5
\end{array}
& E_1 & (\ge 6,5,10) & 13^{m} &  \kII^*  & 13 & 1\\
& E_{13} & (\ge 5,5,10) & 13^{m} &  \kII^*  & 13 & 1\\
%----------------------------------------------
\SetCell[r=2]{c} 
     -m= v_{13}(t)<0   
& E_1 & (0,0,13m) & 13^{-2m} & \kI_{13m} & 1& 1\\
& E_{13} & (0,0,m) & 13^{-2m} &  \kI_{m} & 1& 1\\
%----------------------------------------------
 \SetCell[c=5,r=2]{c} & & & & & d\equiv 0  & d\not\equiv 0 \\
                      & & & & & \SetCell[c=2]{c} d \Mod 7 & \\
\end{longtblr}

\newpage

\begin{longtblr}
[caption = {$L_2(13)$ data for $p$=2}]
{cells = {mode=imath},hlines,vlines,measure=vbox,
hline{Z} = {1-5}{0pt},
vline{1} = {Y-Z}{0pt},
colspec  = cclclccc}
%----------------------------------------------
L_2(13) & \SetCell[c=7]{c} p=2  & & & & & \\ 
t & E & \SetCell[c=1]{c} \operatorname{sig}_2(E) & u & \SetCell[c=1]{c} \Kd_2(E) & \SetCell[c=3]{c} u_2(d)  \\
%----------------------------------------------
\SetCell[r=2]{c} m=v_2(t)\ge 2 
& E_1 & (0,0,m) & 1 &\kI_{m} & 1 &  2^{-1}& 2^{-1} \\
& E_{13} & (0,0,13m) & 1 &\kI_{13m} & 1 & 2^{-1} &  2^{-1}\\
%----------------------------------------------
\SetCell[r=2]{c} v_2(t)=1 
& E_1 & (4,6,13) & 2^{-1} &\kI_{5}^* & 1 & 1 & 2 \\
& E_{13} & (4,6,25) & 2^{-1} &\kI_{17}^* & 1 & 1 & 2 \\
%----------------------------------------------
\SetCell[r=2]{c} 
\begin{array}{c}
v_2(t)=0 \\
t\equiv 1\,(4)
\end{array}
& E_1 & (6,6,6) & 1 &\kII& 1 &  \text{$1^*$ or $2^*$}& 1 \\
& E_{13} & (6,6,6) & 1 & \kII& 1 &  \text{$1^*$ or $2^*$} & 1 \\
%----------------------------------------------
\SetCell[r=2]{c} 
\begin{array}{c}
v_2(t)=0 \\
t\equiv 3\,(4)
\end{array}
& E_1 & (5,8,9) & 1 & \KIII& 1 & 1 & 1 \\
& E_{13} & (5,8,9) & 1 & \kIII& 1 & 1 & 1 \\
%----------------------------------------------
\SetCell[r=2]{c} v_2(t)=-1 
& E_1 & (0,0,13) & 2^{-2} &\kI_{13} & 1 & 2^{-1} & 2^{-1}  \\
& E_{13} & (0,0,1) & 2^{-2} &\kI_{1} & 1 & 2^{-1} & 2^{-1} \\
%----------------------------------------------
\SetCell[r=2]{c} -m=v_2(t)\le -2 
& E_1 & (4,6,13m+12) & 2^{-(2m+1)} &\kI_{13m+4}^* & 1 &  1& 2\\
& E_{13} & (4,6,m+12) & 2^{-(2m+1)} &\kI_{m+4}^* & 1 &  1 &  2\\
%----------------------------------------------
 \SetCell[c=5,r=2]{c} & & & & &  d\equiv 1 &  d\equiv 2  & d\equiv 3 \\
                      & & & & & \SetCell[c=3]{c} d \Mod{4} & \\
\end{longtblr}





\newpage
\section{Conclusion}

\begin{prop}
Let 
$ 
\begin{tikzcd}
E_1 \arrow[dash]{r}{13}  & E_{13} 
\end{tikzcd}
$
be a $\mathbf{Q}$-isogeny graph of type $L_2(13)$ corresponding to a given $t$ in $\mathbf{Q}^*$. For every square-free integer $d$, 
the probability of a vertex
to be the Faltings curve (circled)
in the twisted isogeny graph 
$
\begin{tikzcd} 
E_1^d \arrow[dash]{r}{13}  & E_{13}^d 
\end{tikzcd}
$ 
is given by:

\[
\begin{tblr}{|c|c|c|c|c|}
\hline
 L_2(13) & \text{twisted isogeny graph}  & \text{Prob} \\
\hline
 \SetCell[r=1]{c} v_{13}(t)>0 &  E_1^d \longleftarrow \circled[0.8]{$E_{13}^d$}  & 1 \\
\hline
 \SetCell[r=1]{c} v_{13}(t)\le 0 & \circled[0.8]{$E_1^d$} \longrightarrow E_{13}^d & 1 \\
\hline
\end{tblr}
\]



\end{prop}

\vskip 0.35truecm

\noindnet{\it Proof.} From the previous tables one gets:

\vskip 0.5truecm

\begin{tblr}{cells={mode=imath},hlines,vlines,measure=vbox}
%-------------------------------------------------
\SetCell[c=1]{c} t &\SetCell[c=1]{c} [u(E)]  & \SetCell[c=1]{c} [u(E)(d)]  & \SetCell[c=1]{c}\text{Prob}\\
%-------------------------------------------------
\SetCell[r=1]{c} v_{13}(t)>0 & \SetCell[r=1]{c} (1:13) & \SetCell[r=1]{c} (1:1) &  \SetCell[r=1]{c} (0,1)\\
%-------------------------------------------------
\SetCell[r=1]{c} v_{13}(t)\le 0 & \SetCell[r=1]{c} (1:1) & \SetCell[r=1]{c}  (1:1)   & \SetCell[r=1]{c} (1,0)\\
%-------------------------------------------------
\end{tblr}

\end{document}

